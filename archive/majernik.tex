\section{Introduction}
\label{mintroduction}

Jeffrey conditioning is a method of update (recommended first by
Richard Jeffrey in \scite{7}{jeffrey65}{}) which generalizes standard
conditioning and operates in probability kinematics where evidence is
uncertain ($P(E)\neq{}1$). Sometimes, when we reason inductively,
outcomes that are observed have entailment relationships with
partitions of the possibility space that pose challenges that Jeffrey
conditioning cannot meet. As we will see, it is not difficult to
resolve these challenges by generalizing Jeffrey conditioning. There
are claims in the literature that the principle of maximum entropy,
from now on \textsc{pme}, conflicts with this generalization. I will
show under which conditions this conflict obtains. Since proponents of
\textsc{pme} are unlikely to subscribe to these conditions, the
position of \textsc{pme} in the larger debate over inductive logic and
reasoning is not undermined.

In Section \ref{juppme}, I will introduce the obverse Majern{\'\i}k
problem and sketch how it ties in with two natural generalizations of
Jeffrey conditioning: Wagner conditioning and the \textsc{pme}. In
Section \ref{jc}, I will introduce Jeffrey conditioning in a notation
that will later help us to solve the obverse Majern{\'\i}k problem. In
Section \ref{wc}, I will introduce Wagner conditioning and show how it
naturally generalizes Jeffrey conditioning. In Section
\ref{Generalization}, I will show that \textsc{pme} does so as well
under conditions that are straightforward to accept for proponents of
\textsc{pme}. This solves the obverse Majern{\'\i}k problem and makes
Wagner conditioning unnecessary as a generalization of Jeffrey
conditioning, since the \textsc{pme} seamlessly incorporates it. The
conclusion in Section \ref{Conclusion} summarizes my claims and
briefly refers to epistemological consequences. An appendix gives
proofs how \textsc{pme} generalizes standard conditioning and Jeffrey
conditioning, providing a template for a simplified proof of the claim
in the body of the paper.

\section{Jeffrey's Updating Principle and the Principle of Maximum Entropy}
\label{juppme}

In his paper \qeins{Marginal Probability Distribution Determined by
  the Maximum Entropy Method} (see \scite{7}{majernik00}{}),
Vladim{\'\i}r Majern{\'\i}k asks the following question: If we had two
partitions of an event space and knew all the conditional
probabilities (any conditional probability of one event in the first
partition conditional on another event in the second partition), would
we be able to calculate the marginal probabilities for the two
partitions? The answer is yes, if we commit ourselves to \textsc{pme}:

\begin{quotex}
  [\textsc{pme}] Keep the information entropy of your probability
  distribution maximal within the constraints that the evidence
  provides (in the synchronic case), or your cross-entropy minimal (in
  the diachronic case).
\end{quotex}

For Majern{\'\i}k's question, \textsc{pme} provides us with a unique
and plausible answer (see Majern{\'\i}k's paper). We may also be
interested in the obverse question: if the marginal probabilities of
the two partitions were given, would we similarly be able to calculate
the conditional probabilities? The answer is yes: given \textsc{pme},
Theorems 2.2.1. and 2.6.5. in \scite{7}{coverthomas06}{} reveal that
the joint probabilities are the product of the marginal probabilities
(see also \scite{7}{debbahmueller05}{}). Once the joint probabilities
and the marginal probabilities are available, it is trivial to
calculate the conditional probabilities.

It is important to note that these joint probabilities do not
legislate independence, even though they allow it. M{\'e}rouane Debbah
and Ralf M{\"u}ller correctly describe these joint probabilities as a
model with as many degrees of freedom as possible, which leaves free
degrees for correlation to exist or not (see
\scite{8}{debbahmueller05}{1674}). This avoids the introduction of
unjustified information corresponding to the simple intuition behind
\textsc{pme}: when updating your probabilities, waste no useful
information and do not gain information unless the evidence compels
you to gain it (see \scite{8}{fraassenetal86}{376};
\scite{7}{zellner88}{}; \scite{7}{jaynes88}{}; and
\scite{8}{palmiericiuonzo13}{186}). The principle comes with its own
formal apparatus, not unlike probability theory itself: Shannon's
information entropy (see \scite{7}{shannon48}{}), the Kullback-Leibler
divergence (see \scite{7}{kullbackleibler51}{};
\scite{7}{kullback59}{}; \scite{8}{guiasu77}{308ff}; and
\scite{8}{seidenfeld86}{262ff}), the use of Lagrange multipliers (see
\scite{8}{coverthomas06}{409ff}; \scite{8}{guiasu77}{327f}; and
\scite{8}{seidenfeld86}{281}), and the log-inverse relationship
between information and probability (see \scite{7}{kampe67}{};
\scite{7}{ingardenurbanik62}{}; \scite{7}{khinchin57}{}; and
\scite{7}{kolmogorov68}{}). 

There is an older problem by Carl Wagner (\scite{11}{wagner92}{})
which can be cast in similar terms as Majern{\'\i}k's. If we were
given some of the marginal probabilities in an updating problem as
well as some logical relationships between the two partitions, would
we be able to calculate the remaining marginal probabilities? This
problem is best understood by example (see Wagner's \emph{Linguist}
problem in section \ref{wc}). Wagner solves it using a natural
generalization of Jeffrey conditioning, which I will call Wagner
conditioning. It is not based on \textsc{pme}, but on what I call
Jeffrey's updating principle, or \textsc{jup} for short:

\begin{quotex}
  [\textsc{jup}] In a diachronic updating process, keep the ratio of
  probabilities constant as long as they are unaffected by the
  constraints that the evidence poses.
\end{quotex}

As is the case for \textsc{pme}, there is a debate whether updating on
evidence by rational agents is bound by \textsc{jup} (for a defence
see \scite{7}{teller73}{}; for detractors see
\scite{7}{howsonfranklin94}{}). Our interest in this paper is the
relationship between \textsc{pme} and \textsc{jup}, both of which are
updating principles. Wagner contends that his natural generalization
of Jeffrey conditioning, based on \textsc{jup}, contradicts
\textsc{pme}. Among formal epistemologists, there is a widespread view
that, while \textsc{pme} is a generalization of Jeffrey conditioning,
it is an inappropriate updating method in certain cases and does not
enjoy the generality of Jeffrey conditioning. Wagner's claims support
this view inasmuch as Wagner conditioning is based on the relatively
plausible \textsc{jup} and naturally generalizes Jeffrey conditioning,
but according to Wagner it contradicts \textsc{pme}, which gives wrong
results in these cases.

This paper resists Wagner's conclusions and shows that \textsc{pme}
generalizes both Jeffrey conditioning and Wagner conditioning,
providing a much more integrated approach to probability updating.
This integrated approach also gives a coherent answer to the obverse
Majern{\'\i}k problem posed above.

\section{Jeffrey Conditioning}
\label{jc}

Richard Jeffrey proposes an updating method for cases in which the
evidence is uncertain, generalizing standard probabilistic
conditioning. I will present this method in unusual notation,
anticipating using my notation to solve Wagner's \emph{Linguist}
problem and to give a general solution for the obverse Majern{\'\i}k
problem. Let $\Omega$ be a finite event space and
$\{\theta_{j}\}_{j=1,\ldots,n}$ a partition of $\Omega$. Let $\kappa$
be an $m\times{}n$ matrix for which each column contains exactly one
$1$, otherwise $0$. Let $P=P_{\mbox{\tiny{prior}}}$ and
$\hat{P}=P_{\mbox{\tiny{posterior}}}$. Then
$\{\omega_{i}\}_{i=1,\ldots,m}$, for which

\begin{equation}
  \label{eq:m1}
  \omega_{i}=\bigcup_{j=1,\dots,n}\theta^{*}_{ij},
\end{equation}

{\noindent}is likewise a partition of $\Omega$ (the $\omega$ are
basically a more coarsely grained partition than the $\theta$).
$\theta^{*}_{ij}=\emptyset$ if $\kappa_{ij}=0$,
$\theta^{*}_{ij}=\theta_{j}$ otherwise. Let $\beta$ be the vector of
prior probabilities for $\{\theta_{j}\}_{j=1,\ldots,n}
(P(\theta_{j})=\beta_{j})$ and $\hat{\beta}$ the vector of posterior
probabilities $(\hat{P}(\theta_{j})=\hat{\beta}_{j})$; likewise for
$\alpha$ and $\hat{\alpha}$ corresponding to the prior and posterior
probabilities for $\{\omega_{i}\}_{i=1,\ldots,m}$, respectively.

A Jeffrey-type problem is when $\beta$ and $\hat{\alpha}$ are given
and we are looking for $\hat{\beta}$. A mathematically more concise
characterization of a Jeffrey-type problem is the triple
$(\kappa,\beta,\hat{\alpha})$. The solution, using Jeffrey
conditioning, is

\begin{equation}
  \label{eq:m2}
  \hat{\beta_{j}}=\beta_{j}\sum_{i=1}^{n}\frac{\kappa_{ij}\hat{\alpha_{i}}}{\sum_{l=1}^{m}\kappa_{il}\beta_{l}}\mbox{ for all }j=1,\ldots,n.
\end{equation}

{\noindent}The notation is more complicated than it needs to be for Jeffrey
conditioning. In Section \ref{Generalization}, however, I will take
full advantage of it to present a generalization where the
$\omega_{i}$ do not range over the $\theta_{j}$. In the meantime, here
is an example to illustrate (\ref{eq:m2}).

\begin{quotex}
  A token is pulled from a bag containing 3 yellow tokens, 2 blue
  tokens, and 1 purple token. You are colour blind and cannot
  distinguish between the blue and the purple token when you see it.
  When the token is pulled, it is shown to you in poor lighting and
  then obscured again. You come to the conclusion based on your
  observation that the probability that the pulled token is yellow is
  $1/3$ and that the probability that the pulled token is blue or
  purple is $2/3$. What is your updated probability that the pulled
  token is blue?
\end{quotex}

{\noindent}Let $P(\mbox{blue})$ be the prior subjective probability
that the pulled token is blue and $\hat{P}(\mbox{blue})$ the
respective posterior subjective probability. Jeffrey conditioning,
based on \textsc{jup} (which mandates, for example, that
$\hat{P}(\mbox{blue}|\mbox{blue or}\mbox{
  purple})=P(\mbox{blue}|\mbox{blue or purple})$) recommends

\begin{align}
  \label{eq:jcs}
&\hat{P}(\mbox{blue})&=&\hat{P}(\mbox{blue}|\mbox{blue or purple})\hat{P}(\mbox{blue or
  purple})+\notag \\
&&&\hat{P}(\mbox{blue}|\mbox{neither blue nor
  purple})\hat{P}(\mbox{neither blue nor purple})\notag \\
&&=&P(\mbox{blue}|\mbox{blue or purple})\hat{P}(\mbox{blue or
  purple})=4/9
\end{align}

{\noindent}In the notation of (\ref{eq:m2}), the example is calculated
with $\beta=(1/2,1/3,1/6)^{\top},\hat{\alpha}=(1/3,2/3)^{\top}$,

\begin{equation}
  \label{eq:kappa}
  \kappa=\left[
  \begin{array}{ccc}
    1 & 0 & 0 \\
    0 & 1 & 1
  \end{array}\right]
\end{equation}

{\noindent}and yields the same result as (\ref{eq:jcs}) with
$\hat{\beta}_{2}=4/9$.

\section{Wagner Conditioning}
\label{wc}

Carl Wagner uses \textsc{jup} (explained in more detail in
\scite{7}{wagner02}{}) to solve a problem which cannot be solved by
Jeffrey conditioning. Here is the narrative (call this the
\emph{Linguist} problem):

\begin{quotex}
  You encounter the native of a certain foreign country and wonder
  whether he is a Catholic northerner ($\theta_{1}$), a Catholic
  southerner ($\theta_{2}$), a Protestant northerner ($\theta_{3}$),
  or a Protestant southerner ($\theta_{4}$). Your prior probability
  $p$ over these possibilities (based, say, on population statistics
  and the judgment that it is reasonable to regard this individual as
  a random representative of his country) is given by
  $p(\theta_{1})=0.2,p(\theta_{2})=0.3,p(\theta_{3})=0.4,\mbox{ and
  }p(\theta_{4})=0.1$. The individual now utters a phrase in his
  native tongue which, due to the aural similarity of the phrases in
  question, might be a traditional Catholic piety ($\omega_{1}$), an
  epithet uncomplimentary to Protestants ($\omega_{2}$), an innocuous
  southern regionalism ($\omega_{3}$), or a slang expression used
  throughout the country in question ($\omega_{4}$). After reflecting
  on the matter you assign subjective probabilities
  $u(\omega_{1})=0.4,u(\omega_{2})=0.3,u(\omega_{3})=0.2,\mbox{ and
  }u(\omega_{4})=0.1$ to these alternatives. In the light of this new
  evidence how should you revise $p$? (See \scite{8}{wagner92}{252};
  and \scite{8}{spohn12}{197}.)
\end{quotex}

Let us call a problem of this type a Wagner-type problem. It is an
instance of the more general obverse Majern{\'\i}k problem where partitions
are given with logical relationships between them as well as some
marginal probabilities. Wagner-type problems seek as a solution
missing marginals, while obverse Majern{\'\i}k problems seek the
conditional probabilities as well, both of which I will eventually
provide using \textsc{pme}.

Wagner's solution for such problems (from now on Wagner conditioning)
rests on \textsc{jup} and a formal apparatus established by Arthur
Dempster (see \scite{7}{dempster67}{}), which is quite different from
our notational approach. Wagner legitimately calls his solution a
\qeins{natural generalization of Jeffrey conditioning}
\scite{3}{wagner92}{250}. There is, however, another natural
generalization of Jeffrey conditioning, E.T. Jaynes' principle of
maximum entropy (see \scite{7}{jaynes57a}{}). \textsc{pme} does not rest on
\textsc{jup}, but rather claims that one should keep one's entropy
maximal within the constraints that the evidence provides (in the
synchronic case) and one's cross-entropy minimal (in the diachronic
case).

It is important to distinguish between type I and type II prior
probabilities. The former precede any information at all (so-called
ignorance priors). The latter are simply prior relative to posterior
probabilities in probability kinematics. They may themselves be
posterior probabilities with respect to an earlier instance of
probability kinematics. Although Jaynes' original claims are concerned
with type I prior probabilities, this paper works on the assumptions
of Jaynes' later work focusing on type II prior probabilities. Some
distinguish between \textsc{maxent}, the synchronic rule, and
\emph{Infomin}, the diachronic rule. The understanding here is that
both operate on type II prior probabilities: \textsc{maxent} considers
uniform prior probabilities (however this uniformity may have arisen)
and a set of synchronic constraints on them; \emph{Infomin}, in a more
standard sense of updating, considers type II prior probabilities that
are not necessarily uniform and updates them given evidence
represented as new (diachronic) constraints on acceptable posterior
probability distributions. Some say that \textsc{maxent} and
\emph{Infomin} contradict each other, but I disagree and maintain that
they are compatible. I will have to defer this problem to future work
(for what this may look like see \scite{7}{wagner02}{}).

One advantage of \textsc{pme} is that it works on the wide domain of
updating problems where the evidence corresponds to an affine
constraint (for affine constraints see \scite{7}{csiszar67}{}; for problems with
evidence not in the form of affine constraints see \scite{7}{paris06}{}).
Updating problems where standard conditioning and Jeffrey conditioning
are applicable are a subset of this domain. Some partial information
cases (using the moment(s) of a distribution as evidence), such as Bas
van Fraassen's \emph{Judy Benjamin} problem and Jaynes' \emph{Brandeis
  Dice} problem, are not amenable to either standard conditioning or
Jeffrey conditioning. \textsc{pme} generalizes Jeffrey conditioning
(and, a fortiori, standard conditioning) and therefore absorbs
\textsc{jup} on the more narrow domain of problems that we can solve
using Jeffrey conditioning (for a proof see the appendix, although it
can also be gleaned from \scite{7}{catichagiffin06}{}).

Wagner's contention is that on the wider domain of problems where we
must use Wagner conditioning (and which he does not cast in terms of
affine constraints), \textsc{jup} and \textsc{pme} contradict each
other. We are now in the awkward position of being confronted with two
plausible intuitions, \textsc{jup} and \textsc{pme}, and it appears
that we have to let one of them go. Wagner adduces other conceptual
problems for \textsc{pme} (see \scite{7}{seidenfeld86}{};
\scite{7}{friedmanshimony71}{}; \scite{7}{skyrms87}{};
\scite{7}{uffink95}{}; \scite{8}{walley91}{270}; and
\scite{8}{halpern03}{107}) to reinforce his conclusion that
\textsc{pme} is not a principle on which we should rely in general.

\section{A Natural Generalization of Jeffrey and Wagner Conditioning}
\label{Generalization}

In order to show how \textsc{pme} generalizes Jeffrey conditioning (in
the appendix) and Wagner conditioning to boot, I use the notation that
I have already introduced for Jeffrey conditioning. We can
characterize Wagner-type problems analogously to Jeffrey-type problems
by a triple $(\kappa,\beta,\hat{\alpha})$.
$\{\theta_{j}\}_{j=1,\ldots,n}$ and $\{\omega_{i}\}_{i=1,\ldots,m}$
now refer to independent partitions of $\Omega$, i.e.\ (\ref{eq:m1})
need not be true. Besides the marginal probabilities
$P(\theta_{j})=\beta_{j}, \hat{P}(\theta_{j})=\hat{\beta}_{j},
P(\omega_{i})=\alpha_{i},\hat{P}(\omega_{i})=\hat{\alpha}_{i}$, we
therefore also have joint probabilities
$\mu_{ij}=P(\omega_{i}\cap\theta_{j})$ and
$\hat{\mu}_{ij}=\hat{P}(\omega_{i}\cap\theta_{j})$.

Given the specific nature of Wagner-type problems, there are a few
constraints on the triple $(\kappa,\beta,\hat{\alpha})$. The last row
$(\mu_{mj})_{j=1,\ldots,n}$ is special because it represents the
probability of $\omega_{m}$, which is the negation of the events
deemed possible after the observation. In the \emph{Linguist} problem,
for example, $\omega_{5}$ is the event (initially highly likely, but
impossible after the observation of the native's utterance) that the
native does not make any of the four utterances. The native may have,
after all, uttered a typical Buddhist phrase, asked where the nearest
bathroom was, complimented your fedora, or chosen to be silent.
$\kappa$ will have all $1$s in the last row. Let
$\hat{\kappa}_{ij}=\kappa_{ij}$ for $i=1,\ldots,m-1$ and
$j=1,\ldots,n$; and $\hat{\kappa}_{mj}=0$ for $j=1,\ldots,n$.
$\hat{\kappa}$ equals $\kappa$ except that its last row are all $0$s,
and $\hat{\alpha}_{m}=0$. Otherwise the $0$s are distributed over
$\kappa$ (and equally over $\hat{\kappa}$) so that no row and no
column has all $0$s, representing the logical relationships between
the $\omega_{i}$s and the $\theta_{j}$s ($\kappa_{ij}=0$ if and only
if $\hat{P}(\omega_{i}\cap\theta_{j})=\mu_{ij}=0$). We set
$P(\omega_{m})=x$ ($\hat{P}(\omega_{m})=0$), where $x$ depends on the
specific prior knowledge. Fortunately, the value of $x$ cancels out
nicely and will play no further role. For convenience, we define

\begin{equation}
\label{eq:zeta}
\zeta=(0,\ldots,0,1)^{\top}
\end{equation}

{\noindent}with $\zeta_{m}=1$ and $\zeta_{i}=0$ for $i\neq{}m$.

The best way to visualize such a problem is by providing the joint
probability matrix $M=(\mu_{ij})$ together with the marginals $\alpha$
and $\beta$ in the last column/row, here for example as for the
\emph{Linguist} problem with $m=5$ and $n=4$ (note that this is not
the matrix $M$, which is $m\times{}n$, but $M$ expanded with the
marginals in improper matrix notation):

\begin{equation}
  \label{eq:m3}
      \left[
      \begin{array}{ccccc}
        \mu_{11} & \mu_{12} & 0 & 0 & \alpha_{1} \\
        \mu_{21} & \mu_{22} & 0 & 0 & \alpha_{2} \\
        0 & \mu_{32} & 0 & \mu_{34} & \alpha_{3} \\
        \mu_{41} & \mu_{42} & \mu_{43} & \mu_{44} & \alpha_{4} \\
        \mu_{51} & \mu_{52} & \mu_{53} & \mu_{54} & x \\
        \beta_{1} & \beta_{2} & \beta_{3} & \beta_{4} & 1.00
      \end{array}
\right].
\end{equation}

{\noindent}The $\mu_{ij}\neq{}0$ where $\kappa_{ij}=1$. Ditto, mutatis mutandis,
for $\hat{M},\hat{\alpha},\hat{\beta}$. To make this a little less
abstract, Wagner's \emph{Linguist} problem is characterized by the
triple $(\kappa,\beta,\hat{\alpha})$,

\begin{equation}
  \label{eq:m4}
  \kappa=\left[
  \begin{array}{cccc}
    1 & 1 & 0 & 0 \\
    1 & 1 & 0 & 0 \\
    0 & 1 & 0 & 1 \\
    1 & 1 & 1 & 1 \\
    1 & 1 & 1 & 1
  \end{array}
\right]\mbox{ and }
  \hat{\kappa}=\left[
  \begin{array}{cccc}
    1 & 1 & 0 & 0 \\
    1 & 1 & 0 & 0 \\
    0 & 1 & 0 & 1 \\
    1 & 1 & 1 & 1 \\
    0 & 0 & 0 & 0
  \end{array}
\right]
\end{equation}

\begin{equation}
  \label{eq:m5}
  \beta=(0.2,0.3,0.4,0.1)^{\top}\mbox{ and }\hat{\alpha}=(0.4,0.3,0.2,0.1,0)^{\top}.
\end{equation}

Wagner's solution, based on \textsc{jup}, is

\begin{equation}
  \label{eq:m6}
  \hat{\beta_{j}}=\beta_{j}\sum_{i=1}^{m-1}\frac{\hat{\kappa}_{ij}\hat{\alpha_{i}}}{\sum_{\hat{\kappa}_{il}=1}\beta_{l}}\mbox{ for all }j=1,\ldots,n.
\end{equation}

{\noindent}In numbers,

\begin{equation}
  \label{eq:m7}
  \hat{\beta_{j}}=(0.3,0.6,0.04,0.06)^{\top}.
\end{equation}

{\noindent}The posterior probability that the native encountered by
the linguist is a northerner, for example, is 34\%. Wagner's notation
is completely different and never specifies or provides the joint
probabilities, but I hope the reader appreciates both the analogy to
(\ref{eq:m2}) underlined by this notation as well as its efficiency in
delivering a correct \textsc{pme} solution for us. The solution that
Wagner attributes to \textsc{pme} is misleading because of Wagner's
Dempsterian setup which does not take into account that proponents of
\textsc{pme} are likely to be proponents of the classical Bayesian
position that type II prior probabilities are specified and
determinate once the agent attends to the events in question. Some
Bayesians in the current discussion explicitly disavow this
requirement for (possibly retrospective) determinacy (especially James
Joyce in \scite{10}{joyce10}{} and other papers). Proponents of
\textsc{pme} (a proper subset of Bayesians), however, are unlikely to
follow Joyce---if they did, they would indeed have to address Wagner's
example to show that their allegiances to \textsc{pme} and to
indeterminacy are compatible.

That (\ref{eq:m6}) follows from \textsc{jup} is well-documented in
Wagner's paper. For the \textsc{pme} solution for this problem, I
will not use (\ref{eq:m6}) or \textsc{jup}, but maximize the entropy
for the joint probability matrix $M$ and then minimize the
cross-entropy between the prior probability matrix $M$ and the
posterior probability matrix $\hat{M}$. The \textsc{pme} solution,
despite its seemingly different ancestry in principle, formal method,
and assumptions, agrees with (\ref{eq:m6}). This completes our
argument.

What follows may only be accessible to \textsc{pme} cognoscenti, since
it involves the Lagrange multiplier method (see
\scite{8}{guiasu77}{327ff}; and \scite{8}{jaynes78}{224}). Others may
read the conclusion and find a sketch for an easier, but much less
rigorous proof in the appendix. To maximize the Shannon entropy of $M$
and minimize the Kullback-Leibler divergence between $\hat{M}$ and
$M$, consider the Lagrangian functions:

\begin{flalign}
\label{eq:m8}
& \Lambda(\mu_{ij},\xi)= & \notag \\
& \sum_{\kappa_{ij}=1}\mu_{ij}\log{}\mu_{ij}+\sum_{j=1}^{n}\xi_{j}\left(\beta_{j}-\sum_{\kappa_{kj}=1}\mu_{kj}\right)+ & \notag \\
& \lambda_{m}\left(x-\sum_{j=1}^{n}\mu_{mj}\right) &
\end{flalign}

and

\begin{flalign}
\label{eq:m9}
& \hat{\Lambda}(\hat{\mu}_{ij},\hat{\lambda})= & \notag \\
& \sum_{\hat{\kappa}_{ij}=1}\hat{\mu}_{ij}\log{}\frac{\hat{\mu}_{ij}}{\mu_{ij}}+\sum_{i=1}^{m}\hat{\lambda}_{i}\left(\hat{\alpha}_{i}-\sum_{\hat{\kappa}_{il}=1}\hat{\mu}_{il}\right). &
\end{flalign}

{\noindent}For the optimization, we set the partial derivatives to
$0$, which results in

\begin{equation}
  \label{eq:m10}
  M=rs^{\top}\circ\kappa
\end{equation}

\begin{equation}
  \label{eq:m11}
  \hat{M}=\hat{r}s^{\top}\circ\hat{\kappa}
\end{equation}

\begin{equation}
  \label{eq:m12}
  \beta=S\kappa^{\top}r
\end{equation}

\begin{equation}
  \label{eq:m13}
  \hat{\alpha}=\hat{R}\kappa{}s
\end{equation}

{\noindent}where
$r_{i}=e^{\zeta_{i}\lambda_{m}},s_{j}=e^{-1-\xi_{j}},\hat{r}_{i}=e^{-1-\hat{\lambda}_{i}}$
represent factors arising from the Lagrange multiplier method ($\zeta$
was defined in (\ref{eq:zeta})). The
operator $\circ$ is the entry-wise Hadamard product in linear algebra.
$r,s,\hat{r}$ are the vectors containing the
$r_{i},s_{j},\hat{r}_{i}$, respectively. $R,S,\hat{R}$ are the
diagonal matrices with
$R_{il}=r_{i}\delta_{il},S_{kj}=s_{j}\delta_{kj},\hat{R}_{il}=\hat{r}_{i}\delta_{il}$
($\delta$ is Kronecker delta).

Note that 

\begin{equation}
  \label{eq:m14}
  \frac{\beta_{j}}{\sum_{\hat{\kappa}_{il}=1}\beta_{l}}=\frac{s_{j}}{\sum_{\hat{\kappa}_{il}=1}s_{l}}\mbox{ for all }(i,j)\in\{1,\ldots,m-1\}\times\{1,\ldots,n\}.
\end{equation}

{\noindent}(\ref{eq:m13}) implies

\begin{equation}
  \label{eq:m15}
  \hat{r}_{i}=\frac{\hat{\alpha_{i}}}{\sum_{\hat{\kappa}_{il}=1}s_{l}}\mbox{ for all }i=1,\ldots,m-1.
\end{equation}

{\noindent}Consequently,

\begin{equation}
  \label{eq:m16}
  \hat{\beta}_{j}=s_{j}\sum_{i=1}^{m-1}\frac{\hat{\kappa}_{ij}\hat{\alpha_{i}}}{\sum_{\kappa_{il}=1}s_{l}}\mbox{ for all }j=1,\ldots,n.
\end{equation}

{\noindent}(\ref{eq:m16}) gives us the same solution as (\ref{eq:m6}),
taking into account (\ref{eq:m14}). Therefore, Wagner conditioning and
\textsc{pme} agree.

\section{Conclusion}
\label{Conclusion}

Wagner-type problems (but not obverse Majern{\'\i}k-type problems) can
be solved using \textsc{jup} and Wagner's ad hoc method. Obverse
Majern{\'\i}k-type problems, and therefore all Wagner-type problems,
can also be solved using \textsc{pme} and its established and
integrated formal method. What at first blush looks like serendipitous
coincidence, namely that the two approaches deliver the same result,
reveals that \textsc{jup} is safely incorporated in \textsc{pme}. Not
to gain information where such information gain is unwarranted and to
process all the available and relevant information is the intuition at
the foundation of \textsc{pme}. My results show that this more
fundamental intuition generalizes the more specific intuition that
ratios of probabilities should remain constant unless they are
affected by observation or evidence. Wagner's argument that
\textsc{pme} conflicts with \textsc{jup} is ineffective because it
rests on assumptions that proponents of \textsc{pme} naturally reject.