\section{Introduction}
\label{pintroduction}

Standard conditioning in Bayesian probability theory gives us a
relatively well-accepted tool to update on the observation of an
event. Jeffrey conditioning provides another tool which updates
probability distributions (or densities, from now on omitted) given
uncertain evidence. Jeffrey conditioning generalizes standard
conditioning. Evidence can be viewed as imposing a constraint on
acceptable probability distributions, often one with which the prior
probability distribution is inconsistent. If it is a conditional which
constitutes this constraint, standard conditioning and Jeffrey
conditioning do not always apply. Carl Wagner presents such a case
(see \scite{7}{wagner92}{}) together with a solution based on a
plausible intuition. We will call this intuition (W). Wagner's (W)
solution, or Wagner conditioning, in its turn generalizes Jeffrey
conditioning.

Twenty years earlier, E.T. Jaynes had already proposed a
generalization of Jeffrey conditioning, the principle of maximum
entropy (M). This generalization is more sweeping than Wagner's and
includes partial information cases (using the moment(s) of a
distribution as evidence), such as Bas van Fraassen's \emph{Judy
  Benjamin} problem and Jaynes' own \emph{Brandeis Dice} problem. It
uses information theory to suggest that one should (a) always choose
prior probabilities which are minimally informative, and (b) update to
the probability distribution which is minimally informative relative
to the prior probability distribution while obeying the constraints
imposed by the observation or the evidence. Again, there was a
plausible intuition at work, but (M) soon ran into counter-examples
(e.g.\ \emph{Judy Benjamin}, see van Fraassen, 1981) and conceptual
difficulties (e.g.\ Abner Shimony's Lagrange multiplier problem, see
\scite{7}{friedmanshimony71}{}; or more recently, Joseph Halpern's and
Peter Gr{\"u}nwald's Coarsening at Random, see
\scite{7}{gruenwaldhalpern03}{}).

The question for Wagner was therefore whether his generalization (W)
agreed with (M) or not. Wagner found that it did not. Wagner then used
his method not only to present a \qeins{natural generalization of
  Jeffrey conditioning} (see \scite{8}{wagner92}{250}), but also to
deepen criticism of (M). I will show that (M) not only generalizes
Jeffrey conditioning (as is well known, for a formal proof see
\scite{7}{catichagiffin06}{}) but also Wagner conditioning. Wagner's
intuition (W) is plausible, and his method works. His derivation of a
disagreement with (M), however, is conceptually more complex than he
assumed. Below, we will show that (M) and (W) are consistent given
(L). (L) is what I call the Laplacean principle which requires a
rational agent, besides other standard Bayesian commitments, to hold
sharp credences with respect to well-defined events under
consideration. (I), which is inconsistent with (L) and which some
Bayesians accept, allows a rational agent to have indeterminate or
imprecise credences (see \scite{7}{ellsberg61}{}; \scite{7}{levi85}{};
\scite{7}{walley91}{}; and \scite{7}{joyce10}{}).

\medskip

\begin{tabular}{|c|c|c|c|c|l|}\hline
(M) & (W) & (I) & (L) & & \\ \hline
$\bullet$ & $\bullet$ &  & & $\times$ & according to Wagner's article \\ \hline
$\bullet$ & $\bullet$ &  & & \checkmark & according to this article \\ \hline
& & $\bullet$ & $\bullet$ & $\times$ & disagree over permitting mushy credences \\ \hline
$\bullet$ & $\bullet$ & $\bullet$ & & $\times$ & formally shown in Wagner's article \\ \hline
$\bullet$ & $\bullet$ & & $\bullet$ & \checkmark & formally shown in this article \\ \hline 
\end{tabular}

\medskip

While Wagner is welcome to deny (L), my sense is that advocates of (M)
usually accept it because they are already to the moderate right of
Sandy L. Zabell's spectrum between left-wing dadaists and right-wing
totalitarians (see \scite{8}{zabell05}{27}; Zabell's representative of
right-wing totalitarianism is E.T. Jaynes). If there were an advocate
of (M) sympathetic to (I), Wagner's result would indeed force her to
choose. Wagner's criticism of (M) is misplaced, however, since it
rests on a hidden assumption that someone who believes in (M) would
tend not to hold. Wagner does not give an independent argument for
(I). This paper shows how elegantly (M) generalizes not only standard
conditioning and Jeffrey conditioning but also Wagner conditioning,
once we accept (L).

A tempered and differentiated account of (M) (contrasted with Jaynes'
right-wing earlier version) is not only largely immune to criticisms,
but often illuminates the problems that the criticisms pose (for
example in the \emph{Judy Benjamin} case see \scite{7}{lukits14}{}).
This account rests on principles such as (L) and a reasonable
interpretation of what we mean by objectivity. To make this more
clear, and before we launch into the formalities of generalizing
Wagner conditioning by using (M), let us articulate (L) and (M). (L)
is what I call the Laplacean principle and in addition to standard
Bayesian commitments states that a rational agent assigns a
determinate precise probability to a well-defined event under
consideration (for a defence of (L) against (I) see
\scite{7}{white10}{}; and \scite{7}{elga10}{}).

To avoid excessive apriorism (see \scite{7}{seidenfeld79}{414}), (L)
does not require that a rational agent has probabilities assigned to
all events in an event space, only that, once an event has been
brought to attention, and sometimes retrospectively, the rational
agent is able to assign a sharp probability. Newton did not need to
have a prior probability for Einstein's theory in order to have a
posterior probability for his theory of gravity.

(L) also does not require objectivity in the sense that all rational
agents must agree in their probability distributions if they have the
same information. It is important to distinguish between type I and
type II prior probabilities. The former precede any information at all
(so-called ignorance priors). The latter are simply prior relative to
posterior probabilities in probability kinematics. They may themselves
be posterior probabilities with respect to an earlier instance of
probability kinematics. 

The case for objectivity in probability kinematics, where prior
probabilities are of type II, is consistent with and dependent on a
subjectivist interpretation of probabilities, making for some
terminological confusion. Interpretations of the evidence and how it
is to be cast in terms of formal constraints may vary. Once we agree
on a prior distribution (type II), however, and on a set of formal
constraints representing our evidence, (M) claims that posterior
probabilities follow mechanically. Just as is the case in deductive
logic, we may come to a tentative and voluntary agreement on an
interpretation, a set of rules and presuppositions and then go part of
the way together. To standard Bayesian commitments and (L), (M) adds

\begin{quotex}
  Update type II prior distributions under formalized constraints in
  accordance with information theory and a commitment to keep the
  entropy maximal, if constraints are synchronic, and the
  cross-entropy minimal, if they are diachronic.
\end{quotex}

This corresponds to the intuition that we ought not to gain
information where the additional information is not warranted by the
evidence. Some want to drive a wedge between the synchronic rule to
keep the entropy maximal (\textsc{maxent}) and the diachronic rule to
keep the cross-entropy minimal (\emph{Infomin}) (for this objection
see \scite{8}{walley91}{270f}).

Here is a brief excursion to dispel this worry. Consider a bag with
blue, red, and green tokens. You know that ($C'$) at least 50\% of the
tokens are blue. Then you learn that ($C''$) at most 20\% of the tokens
are red. The synchronic norm \textsc{maxent}, on the one hand, ignores
the diachronic dimension and prescribes the probability distribution
which has the maximum entropy and obeys both ($C'$) and ($C''$). The
three-dimensional vector containing the probabilities for blue, red,
and green is $(\frac{1}{2},\frac{1}{5},\frac{3}{10})$. \emph{Infomin},
on the other hand, processes ($C'$) and ($C''$) sequentially, taking in its
second step $(\frac{1}{2},\frac{1}{4},\frac{1}{4})$ as its prior
probability distribution and then diachronically updating to
$(\frac{8}{15},\frac{1}{5},\frac{4}{15})$.

The information provided in a problem calling for \textsc{maxent} and
the information provided in a problem calling for \emph{Infomin} is
different, as temporal relations and their implications for dependence
between variables clearly matter. In the above case, we might have
relevantly received information ($C''$) before ($C'$) (\qnull{before} may
be understood logically rather than temporally) so that \emph{Infomin}
updates in its last step $(\frac{2}{5},\frac{1}{5},\frac{2}{5})$ to
$(\frac{1}{2},\frac{1}{6},\frac{1}{3})$. Even if ($C'$) and ($C''$) are
received in a definite order, the problem may be phrased in a way that
indicates independence between the two constraints. In this case,
\textsc{maxent} is the appropriate norm to use. \emph{Infomin} does
not assume such independence and therefore processes the two pieces of
information separately. Disagreement arises when observations are
interpreted differently, not because \textsc{maxent} and
\emph{Infomin} are inconsistent with each other. In the following I
will assume that \textsc{maxent} and \emph{Infomin} are compatible and
part of the toolkit at the disposal of (M), the principle of maximum
entropy.

Returning now to the issue of updating on conditionals, some advocates
of (M) may find (L) too weak in its claims, but none think it is too
strong. Once (L) is assumed, however, Wagner's diagnosis of
disagreement between (W) and (M) fails. Moreover, (M) and (L) together
seamlessly generalize Wagner conditioning. In the remainder of this
paper I will provide a sketch of a formal proof for this claim. A
welcome side-effect of reinstating harmony between (M) and (W) is that
it provides an inverse procedure to Vladim{\'\i}r Majern{\'\i}k's
method of finding marginals based on given conditional probabilities
(see \scite{7}{majernik00}{}; and \scite{7}{lukits15}{}).

\section{Wagner's Natural Generalization of Jeffrey Conditioning}
\label{NatGen}

Wagner claims that he has found a relatively common case of
probability kinematics in which (M) delivers the wrong result so that
we must develop an ad hoc generalization of Jeffrey conditioning. This
is best explained by using Wagner's example, the \emph{Linguist}
problem.

\begin{quotex}
  You encounter the native of a certain foreign country and wonder
  whether he is a Catholic northerner ($\theta_{1}$), a Catholic
  southerner ($\theta_{2}$), a Protestant northerner ($\theta_{3}$),
  or a Protestant southerner ($\theta_{4}$). Your prior probability
  $p$ over these possibilities (based, say, on population statistics
  and the judgment that it is reasonable to regard this individual as
  a random representative of his country) is given by
  $p(\theta_{1})=0.2,p(\theta_{2})=0.3,p(\theta_{3})=0.4,\mbox{ and
  }p(\theta_{4})=0.1$. The individual now utters a phrase in his
  native tongue which, due to the aural similarity of the phrases in
  question, might be a traditional Catholic piety ($\omega_{1}$), an
  epithet uncomplimentary to Protestants ($\omega_{2}$), an innocuous
  southern regionalism ($\omega_{3}$), or a slang expression used
  throughout the country in question ($\omega_{4}$). After reflecting
  on the matter you assign subjective probabilities
  $u(\omega_{1})=0.4,u(\omega_{2})=0.3,u(\omega_{3})=0.2,\mbox{ and
  }u(\omega_{4})=0.1$ to these alternatives. In the light of this new
  evidence how should you revise $p$? (See \scite{8}{wagner92}{252}
  and \scite{8}{spohn12}{197}.)
\end{quotex}

Let
$\Theta=\{\theta_{i}:i=1,\ldots,4\},\Omega=\{\omega_{i}:i=1,\ldots,4\}$.
Let $\Gamma:\Omega\rightarrow{}2^{\Theta}-\{\emptyset\}$ be the
function which maps $\omega$ to $\Gamma(\omega)$, the narrowest event
in $\Theta$ entailed by the outcome $\omega\in\Omega$. Here are two
definitions that take advantage of the apparatus established by Arthur
Dempster (see \scite{7}{dempster67}{}). We will need $m$ and $b$ to
articulate Wagner's (W) solution for \emph{Linguist} type problems.

\begin{equation}
  \mbox{For all }E\subseteq{}\Theta, m(E)=u(\{\omega\in\Omega:\Gamma(\omega)=E\})\label{eq:mof}.
\end{equation}

\begin{equation}
  \mbox{For all }E\subseteq{}\Theta, b(E)=\sum_{H\subseteq{}E}m(H)=u(\{\omega\in\Omega:\Gamma(\omega)\subseteq{}E\})\label{eq:bof}.
\end{equation}

Let $Q$ be the posterior joint probability measure on
$\Theta\times\Omega$, and $Q_{\Theta}$ the marginalization of $Q$ to
$\Theta$, $Q_{\Omega}$ the marginalization of $Q$ to $\Omega$.
Wagner plausibly suggests that $Q$ is compatible with $u$ and $\Gamma$
if and only if

\begin{equation}
  \label{eq:entail}
  \mbox{for all }\theta\in\Theta\mbox{ and for all
  }\omega\in\Omega,\theta\notin\Gamma(\omega)\mbox{ implies that }Q(\theta,\omega)=0
\end{equation}

and

\begin{equation}
  \label{eq:marg}
  Q_{\Omega}=u.
\end{equation}

The two conditions (\ref{eq:entail}) and (\ref{eq:marg}), however, are
not sufficient to identify a \qeins{uniquely acceptable revision of a
  prior} \scite{2}{wagner92}{250}. Wagner's proposal includes a third
condition, which extends Jeffrey's rule to the situation at hand. We
will call it (W). To articulate the condition, we need some more
definitions. For all $E\subseteq{}\Theta$, let
$E_{\bigstar}=\{\omega\in\Omega:\Gamma(\omega)=E\}$, so that
$m(E)=u(E_{\bigstar})$. For all $A\subseteq\Theta$ and all
$B\subseteq\Omega$, let $\mbox{``A''}=A\times\Omega$ and
$\mbox{``B''}=\Theta\times{}B$, so that
$Q(\mbox{``A''})=Q_{\Theta}(A)$ for all $A\subseteq\Theta$ and
$Q(\mbox{``B''})=Q_{\Omega}(B)$ for all $B\subseteq\Omega$. Let also
$\mathcal{E}=\{E\subseteq\Theta:m(E)>0\}$ be the family of evidentiary
focal elements.

According to Wagner only those $Q$ satisfying the condition

\begin{equation}
  \label{eq:wagn}
  \mbox{for all }A\subseteq\Theta\mbox{ and for all }E\in\mathcal{E},Q(\mbox{``A''}|\mbox{``E$_{\bigstar}$''})=p(A|E)
\end{equation}

are eligible candidates for updated joint probabilities in
\emph{Linguist} type problems. 
To adopt (\ref{eq:wagn}), says Wagner, is to make sure
that the total impact of the occurrence of the event $E_{\bigstar}$ is
to preclude the occurrence of any outcome $\theta\notin{}E$, and that,
within $E$, $p$ remains operative in the assessment of relative
uncertainties (see \scite{8}{wagner92}{250}). While conditions
(\ref{eq:entail}), (\ref{eq:marg}) and (\ref{eq:wagn}) may admit an
infinite number of joint probability distributions on
$\Theta\times\Omega$, their marginalizations to $\Theta$ are identical
and give us the desired posterior probability, expressible by the
formula

\begin{equation}
  \label{eq:qofa}
  q(A)=\sum_{E\in\mathcal{E}}m(E)p(A|E).
\end{equation}

So far we are in agreement with Wagner. Wagner's scathing verdict
about (M) towards the end of his article, however, is not really a
verdict about (M) in the Laplacean tradition but about the curious
conjunction of (M) and (I):

\begin{quotex}
  Students of maximum entropy approaches to probability revision may
  [\ldots] wonder if the probability measure defined by our formula
  (\ref{eq:qofa}) similarly minimizes [the Kullback-Leibler
  information number] $D_{\textsc{kl}}(q,p)$ over all probability
  measures $q$ bounded below by $b$. The answer is negative [\ldots]
  convinced by Skyrms, among others, that \textsc{maxent} is not a
  tenable updating rule, we are undisturbed by this fact. Indeed, we
  take it as additional evidence against \textsc{maxent} that
  (\ref{eq:qofa}), firmly grounded on [\ldots] a considered judgment
  that (\ref{eq:wagn}) holds, might violate \textsc{maxent} [\ldots]
  the fact that Jeffrey's rule coincides with \textsc{maxent} is
  simply a misleading fluke, put in its proper perspective by the
  natural generalization of Jeffrey conditioning described in
  this paper. [References to formulas and notation modified.]
  \scite{3}{wagner92}{255}
\end{quotex}

In the next section, we will contrast what Wagner considers to be the
solution of (M) for this problem, \qnull{Wagner's (M) solution,} and
Wagner's solution presented in this section, \qnull{Wagner's (W)
  solution,} and show, in much greater detail than Wagner does, why
Wagner's (M) solution misrepresents (M).

\section{Wagner's (M) Solution}
\label{WagnersMSolution}

Wagner's (M) solution assumes the constraint that $b$ must act as a
lower bound for the posterior probability. Consider
$E_{12}=\{\theta_{1}\vee\theta_{2}\}$. Because both $\omega_{1}$ and
$\omega_{2}$ entail $E_{12}$, according to (\ref{eq:bof}),
$b(E_{12})=0.70$. It makes sense to consider it a constraint that the
posterior probability for $E_{12}$ must be at least $b(E_{12})$. Then
we choose from all probability distributions fulfilling the constraint
the one which is closest to the prior probability distribution, using
the Kullback-Leibler divergence.

Wagner applies this idea to the marginal probability distribution on
$\Theta$. He does not provide the numbers, but refers to simpler
examples to make his point that (M) does not generally agree with his
solution. To aid the discussion, I want to populate Wagner's claim for
the \emph{Linguist} problem with numbers. Using proposition 1.29 in
Dimitri Bertsekas' book \emph{Constrained Optimization and Lagrange
  Multiplier Methods} (see \scite{8}{bertsekas82}{71}) and some
non-trivial calculations, Wagner's (M) solution for the
\emph{Linguist} problem (indexed $Q_{wm}$) is

\begin{equation}
  \label{eq:p13}
  \tilde{\beta}=(Q_{wm}(\theta_{j}))^{\intercal}=(0.30,0.45,0.10,0.15)^{\intercal}.
\end{equation}

A brief remark about notation: I will use $\alpha$ for vectors
expressing $\omega_{i}$ probabilities and $\beta$ for vectors
expressing $\theta_{j}$ probabilities. I will use a tilde as in
$\tilde{\beta}$ or a hat as in $\hat{\beta}$ for posteriors, while
priors remain without such ornamentation. The tilde is used for
Wagner's (M) solution (which, as we will see, is incorrect) and the
hat for the correct solution (both (W) and (M)).

The cross-entropy between $\tilde{\beta}$ and the prior

\begin{equation}
  \label{eq:p14}
  \beta=(P(\theta_{j}))^{\intercal}=(0.20,0.30,0.40,0.10)^{\intercal}
\end{equation}

is indeed significantly smaller than the cross-entropy between
Wagner's (W) solution 

\begin{equation}
  \label{eq:p15}
  \hat{\beta}=(Q(\theta_{j}))^{\intercal}=(0.30,0.60,0.04,0.06)^{\intercal}
\end{equation}

and the prior $\beta$ ($0.0823$ compared to $0.4148$). For the
cross-entropy, we use the Kullback-Leibler Divergence

\begin{equation}
  \label{eq:kl}
  D_{\textsc{kl}}(q,p)=\sum_{j}q(\theta_{j})\log_{2}\frac{q(\theta_{j})}{p(\theta_{j})}.
\end{equation}

From the perspective of an (M) advocate, there are only two
explanations for this difference in cross-entropy. Either Wagner's (W)
solution illegitimately uses information not contained in the problem,
or Wagner's (M) solution has failed to include information that is
contained in the problem. I will simplify the \emph{Linguist} problem
in order to show that the latter is the case.

\begin{quotex}
  The \emph{Simplified Linguist Problem.} Imagine the native is either
  Protestant or Catholic (50:50). Further imagine that the utterance
  of the native either entails that the native is a Protestant (60\%)
  or provides no information about the religious affiliation of the
  native (40\%).
\end{quotex}

Using (\ref{eq:qofa}), the posterior probability distribution is 80:20
(Wagner's (W) solution and, surely, the correct solution). Using $b$
as a lower bound and (M), Wagner's (M) solution for this radically
simplified problem is 60:40, clearly a more entropic solution than
Wagner's (W) solution. The problem, as we will show, is that Wagner's
(M) solution does not take into account (L), which an (M) advocate
would naturally accept.

For a Laplacean, the prior joint probability distribution on
$\Theta\times\Omega$ is not left unspecified for the calculation of
the posteriors. Before the native makes the utterance, the event space
is unspecified with respect to $\Omega$. After the utterance, however,
the event space is defined (or brought to attention) and populated by
prior probabilities according to (L). That this happens
retrospectively may or may not be a problem: Bayes' theorem is
frequently used retrospectively, for example when the anomalous
precession of Mercury's perihelion, discovered in the mid-1800s, was
used to confirm Albert Einstein's General Theory of Relativity in
1915. I shall bracket for now that this procedure is controversial and
refer the reader to the literature on Old Evidence.

Ariel Caticha and Adom Giffin make the following
appeal:

\begin{quotex}
  Bayes' theorem requires that $P(\omega,\theta)$ be defined and that
  assertions such as \qzwei{$\omega$ \emph{and} $\theta$} be
  meaningful; the relevant space is neither $\Omega$ nor $\Theta$ but
  the product $\Omega\times\Theta$ [notation modified]
  \scite{3}{catichagiffin06}{9}
\end{quotex}

Following (L) we shall populate the joint probability matrix $P$ on
$\Omega\times\Theta$, which is a perfect task for \textsc{maxent}, as
updating the joint probability $P$ to $Q$ on $\Omega\times\Theta$ will
be a perfect task for \emph{Infomin}. For the \emph{Simplified
  Linguist Problem,} this procedure gives us the correct result,
agreeing with Wagner's (W) solution (80:20). 

There is a more general theorem which incorporates Wagner's (W) method
into Laplacean realism and \textsc{maxent} orthodoxy. The proof of
this theorem will be in a more technical companion paper, but its
validity is confirmed by how well it works for the \emph{Linguist}
problem (as well as the \emph{Simplified Linguist Problem}).

\section{The Linguist}
\label{TheLinguist}

The \emph{Linguist} problem is a specific case of a more general
Wagner-type problem characterized by two vectors and one matrix
$(\beta,\hat{\alpha},\kappa)$ (the dimensions are $n$, $m$, and
$m\times{}n$, respectively). The first vector, $\beta$, represents
the marginal prior probability $P(\theta_{j})$. For the \emph{Linguist
problem},

\begin{equation}
  \label{eq:p1}
  \beta=(0.2,0.3,0.4,0.1)^{\intercal}.
\end{equation}

The second vector, $\hat{\alpha}$, represents the marginal posterior
probability $Q(\omega_{i})$. For the \emph{Linguist problem},

\begin{equation}
  \label{eq:p2}
  \hat{\alpha}=(0.4,0.3,0.2,0.1,0)^{\intercal}.
\end{equation}

Whereas Wagner only considers four dimensions, corresponding to the
four utterances of the native, we have to add a fifth dimension
corresponding to the case in which the native does not make any of
those utterances, i.e.\
$\omega_{5}=\urcorner(\bigvee_{i=1,\ldots,4}\omega_{i})$. Presumably,
the prior probability of $\omega_{5}$ is very high, nearly 1 (the
native may have uttered a typical Buddhist phrase, asked where the
nearest bathroom was, complimented your fedora, or chosen to be
silent, as a commenter pointed out to me). By the principle of
regularity, however, it does not equal 1 (for a defence of the
principle of regularity, that one should not assign probability 0 to
any possibility, see \scite{7}{edwardsetal63}{}). The posterior
probability is $0$, as the \emph{Linguist} problem specifies that one
of the four possibilities was uttered by the native.
$\hat{\alpha}_{m}$ is therefore always $0$ for Wagner-type problems.

The matrix $\kappa$ represents the logical relationships between the
$\theta_{j}$'s and the $\omega_{i}$'s. In Wagner-type problems, the
conditionals imply that some of the joint probabilities are zero. The
observation of $\omega_{i}$ for $i=1,\ldots,m-1$ implies that the last
row of $\kappa$, which consists of $1$'s, becomes a row of $0$'s in
the posterior representation $\hat{\kappa}$ of these relationships.
Thus,

\begin{equation}
  \label{eq:p3}
  \kappa=\left[
  \begin{array}{cccc}
    1 & 1 & 0 & 0 \\
    1 & 1 & 0 & 0 \\
    0 & 1 & 0 & 1 \\
    1 & 1 & 1 & 1 \\
    1 & 1 & 1 & 1
  \end{array}
\right]\mbox{ and }
  \hat{\kappa}=\left[
  \begin{array}{cccc}
    1 & 1 & 0 & 0 \\
    1 & 1 & 0 & 0 \\
    0 & 1 & 0 & 1 \\
    1 & 1 & 1 & 1 \\
    0 & 0 & 0 & 0
  \end{array}
\right].
\end{equation}

The triple $(\beta,\hat{\alpha},\kappa)$ corresponds to Wagner's
conditions (\ref{eq:entail}) (dictating the zero joint probabilities
or $\kappa$), (\ref{eq:marg}) (dictating the marginal probabilities
$\hat{\alpha}$ or $Q(\omega_{i})$), and (\ref{eq:qofa}) (dictating the
marginal probabilities $\beta$ or $P(\theta_{j})$). The marginal prior
probabilities
$\alpha=(P(\omega_{1}),\ldots,P(\omega_{m})^{\intercal})$ and
posterior probabilities $\hat{\beta}=(Q(\theta_{1}),\ldots,$
$Q(\theta_{n})^{\intercal})$ are unknown. We do not need to know
$\alpha$, but the point of the exercise is to determine $\hat{\beta}$.
According to (W), $\hat{\beta}=(0.3,0.6,0.04,0.06)$.

According to (M), we use Lagrange multipliers and first maximize the
entropy of $M$, the joint prior probability matrix; then we use
Lagrange multipliers again to minimize the cross-entropy from $M$ to
the joint posterior probability matrix $\hat{M}$. The situation can be
visualized like this for the \emph{Linguist} problem:

\begin{equation}
  \label{eq:p5}
      \left[
      \begin{array}{ccccc}
        m_{11} & m_{12} & 0 & 0 & \alpha_{1} \\
        m_{21} & m_{22} & 0 & 0 & \alpha_{2} \\
        0 & m_{32} & 0 & m_{34} & \alpha_{3} \\
        m_{41} & m_{42} & m_{43} & m_{44} & \alpha_{4} \\
        m_{51} & m_{52} & m_{53} & m_{54} & \alpha_{5} \\
        \beta_{1} & \beta_{2} & \beta_{3} & \beta_{4} & 1.00
      \end{array}
\right]
\end{equation}

where the last column and the last row are the row and column sums of
$M=(m_{ij})$. Similarly for the posterior joint probability matrix
$\hat{M}=(\hat{m}_{ij})$

\begin{equation}
  \label{eq:p6}
      \left[
      \begin{array}{ccccc}
        \hat{m}_{11} & \hat{m}_{12} & 0 & 0 & \hat{\alpha}_{1} \\
        \hat{m}_{21} & \hat{m}_{22} & 0 & 0 & \hat{\alpha}_{2} \\
        0 & \hat{m}_{32} & 0 & \hat{m}_{34} & \hat{\alpha}_{3} \\
        \hat{m}_{41} & \hat{m}_{42} & \hat{m}_{43} & \hat{m}_{44} & \hat{\alpha}_{4} \\
        0 & 0 & 0 & 0 & \hat{\alpha}_{5} \\
        \hat{\beta}_{1} & \hat{\beta}_{2} & \hat{\beta}_{3} & \hat{\beta}_{4} & 1.00
      \end{array}
\right].
\end{equation}

Wagner's (W) solution $\hat{\beta}$ solves the system of equations
yielded by the Lagrange multiplier method (but not Wagner's (M)
solution $\tilde{\beta}$). The mathematically detailed general proof
is complex and will be published elsewhere in connection with the
inverse Majern{\'\i}k problem cited above. However, the interested
reader can easily solve this problem for the special case of the
\emph{Simplified Linguist.} Because maximum entropy and minimum
cross-entropy solutions are unique (see \scite{7}{shorejohnson80}{}),
(M) agrees with (W). To get there, we have assumed (L), namely that
the joint probability matrices are populated by determinate
probabilities. Wagner ostensibly disagrees with (L) and follows Peter
Walley's recommendation in \emph{Statistical Reasoning with Imprecise
  Probabilities} that marginals do not determine a unique product (see
\scite{8}{walley91}{456}), as he represents the joint probability
matrix $\hat{M}$ like this (visualized here with the marginals), :

\begin{equation}
  \label{eq:p8}
      \left[
      \begin{array}{ccccc}
        ? & ? & 0 & 0 & \hat{\alpha}_{1}=0.4 \\
        ? & ? & 0 & 0 & \hat{\alpha}_{2}=0.3 \\
        0 & ? & 0 & ? & \hat{\alpha}_{3}=0.2 \\
        ? & ? & ? & ? & \hat{\alpha}_{4}=0.1 \\
        \hat{\beta}_{1}=0.3 & \hat{\beta}_{2}=0.6 & \hat{\beta}_{3}=0.04 & \hat{\beta}_{4}=0.06 & 1.00
      \end{array}
\right].
\end{equation}

The posterior probability that the native encountered by the linguist
is a northerner, for example, is 34\%. (L) in conjunction with (M), by
contrast, provides the joint probability matrix in full without
lacunae.

\begin{equation}
  \label{eq:p11}
      \left[
      \begin{array}{ccccc}
        0.16 & 0.24 & 0 & 0 & \hat{\alpha}_{1}=0.4 \\
        0.12 & 0.18 & 0 & 0 & \hat{\alpha}_{2}=0.3 \\
        0 & 0.15 & 0 & 0.05 & \hat{\alpha}_{3}=0.2 \\
        0.02 & 0.03 & 0.04 & 0.01 & \hat{\alpha}_{4}=0.1 \\
        \hat{\beta}_{1}=0.3 & \hat{\beta}_{2}=0.6 & \hat{\beta}_{3}=0.04 & \hat{\beta}_{4}=0.06 & 1.00
      \end{array}
\right]
\end{equation}

We have not formally demonstrated that for all Wagner-type problems
$(\beta,\hat{\alpha},\kappa)$, the correct (M) solution (versus
Wagner's deficient (M) solution) agrees with Wagner's (W) solution,
although we have established a useful framework and demonstrated the
agreement for the \emph{Linguist} problem. The technical companion
paper will accomplish the more general proof. As Vladim{\'\i}r
Majern{\'\i}k has shown how to derive marginal probabilities from
conditional probabilities using (M) (see \scite{7}{majernik00}{}), we
will inversely show how to derive conditional probabilities (i.e.\ the
joint probability matrices) from the marginal probabilities and
logical relationships provided in Wagner-type problems. This technical
result together with the claim established in the present paper that
Wagner's intuition (W) is consistent with (M), given (L), underlines
the formal and conceptual virtue of (M).