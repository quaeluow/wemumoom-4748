% This is NOT the official version, but a seriously cut down version for APA
% Pacific 2014. If you make corrections here, they must be made in sonk.tex as
% well if they are meant to endure.

\documentclass[11pt]{article}
\usepackage{october}
% For BJPS
% \hyphenpenalty=10000
% \hbadness=10000

\begin{document}
% For BJPS
% \raggedright
% \doublespacing
\newcounter{expls}

\title{Semantics of Not Knowing}
\author{Stefan Lukits}
\date{}
\maketitle
% \doublespacing

% \begin{abstract} 
%   {\noindent}
% \end{abstract}

\section{Synopsis} 
\label{Synopsis}

Traditional Bayesian epistemology used to include the requirement for
a rational agent to hold a sharp credence function. It has recently
become popular to drop this requirement. There are now Bayesian
theories which permit a rational agent to hold indeterminate credal
states based on incomplete or ambiguous evidence. I will refer to
Bayesians who continue to adhere to the classical theory of sharp
credences for rational agents as \qnull{Laplaceans} (e.g.\ Adam Elga
and Roger White). I will refer to Bayesians who do not believe in the
requirement for a rational agent to hold sharp credences as
\qnull{Booleans} (e.g.\ Peter Walley and James Joyce; see Boole, 1854,
chapters 16--21, for alternative methods to the ones suggested by
Laplace which result in imprecise epistemic probabilities).

After describing the appeal of indeterminacy and showing how
contemporary Laplacean objections fail, I will point to more serious
failings of indeterminacy in semantic terms and show how a proper
semantics of not knowing, which we could also call a semantics of
partial belief, solves the problems for sharp credences that Booleans
address by introducing indeterminate credal states. There is a sense
in which, by linking knowledge of chances to its reflection in
credences, Booleans seek to reconcile traditional knowledge
epistemology concerned with full belief and formal epistemology
concerned with partial belief. There are other more recent
reconciliation projects (see Spohn, 2012; and Moss, 2013), but if my
paper is correct then the Boolean approach will not contribute to this
reconciliation because it mixes full belief and partial belief
metaphors in ways that are semantically problematic.

A sharp credence, as much as the term suggests precision and a measure
of certainty, is a representation of an epistemic state chracterized
by uncertainty and lack of information. Importantly, it does not
represent the evidence which informs the epistemic state, and it makes
no claim of such a representation. Indeterminate credal states are
often lauded as doing much better representing uncertainty together
with the evidence that constrains it, but they can no more give an
adequate representation of evidence than sharp credences. This paper
is concerned with the semantic legitimacy of having a $0.5$ sharp
credence in heads for a coin of whose bias we are completely ignorant;
for a coin whose fairness is supported by a lot of evidence; and even
for a coin about whose bias we know to be either 1/3 or 2/3 for heads.

One potential Boolean claim is that agents who use indeterminate
credal states do better than Laplaceans when they bet on the truth of
events for which they have varying degrees of evidence. Peter Walley
gives an example where a Laplacean does much worse at predicting
soccer games than Boolean peers. I show that the result is due to an
unreasonable restriction on the betting behaviour of the Laplacean.
Once this restriction is lifted, Laplaceans do just as well as
Booleans, except that they are not tangled in the semantic problem of
the double task, where indeterminate credal states are supposed to
reflect both the uncertainty of an agent and other properties of her
evidence.

I will present several examples where this double task stretches
indeterminate credal states to the limits of plausibility, for example
when they need to aggregate expert opinion or account for dilation.
Joyce's idea that credences can represent balance, weight, and
specificity of the evidence is inconsistent with the use of
indeterminacy (and Joyce himself, in response to the dilation problem,
gives the argument why this is the case). The implicit Boolean claim
that certain properties of the evidence (its ambiguity, its
completeness, conflicts within it) can be recovered from indeterminate
credal states is inconsistent with an effective Boolean answer to the
dilation problem.

The Laplacean approach of assigning subjective probabilities to
partitions of the event space (e.g.\ objective chances) and then
aggregating them by David Lewis's summation formula into a single
precise credence function is semantically tidy and shares many of the
formal virtues of Boolean theories. If the bad taste about numerical
precision in our fuzzy, nebulous world lingers, I can point to
philosophical projects in other domains where the concepts we use are
sharply bounded, even though our ability to conceive of those sharp
boundaries or know them is limited, for example Timothy Williamson's
work on vagueness and knowledge as a mental state.

\section{Motivation for Indeterminate Credal States} 
\label{MotivationForIndeterminateCredalStates}

Booleans claim that it is rational for an agent to hold indeterminate
credal states rather than sharp credences, representing uncertainty.
They are Bayesians in all other respects and defend Bayesian
epistemology, proposing that it is better off without the requirement
for sharp credences. Laplaceans, by contrast, require sharp credences
for rational agents. We want to motivate indeterminacy as forcefully
as possible so that the reader will see (a) the appeal of the Boolean
approach, (b) the insufficiency of the critical response, and (c) the
need for careful articulation of the Laplacean approach that can
address the concerns which motivate some to resort to indeterminacy.
Finally, (d) the undesirable semantics of the Boolean approach,
documented both conceptually and by example, will lead to the
conclusion that a Laplacean approach is the more promising
alternative.

Let a \textsc{coin} be a Bernoulli generator that produces successes
and failures with probability $p$ for success, labeled $H$, and $1-p$
for failure, labeled $T$. Physical coins may serve as examples, if we
are willing to set aside that most of them are approximately fair.
Imagine three \textsc{coin}s for which we have evidence that
\textsc{coin}$_{I}$ is fair, \textsc{coin}$_{II}$ has an unknown bias,
and \textsc{coin}$_{III}$ has as bias either $p=1/3$ or $p=2/3$. The
Laplacean approach permits a sharp $0.5$ credence in $H$ for a
rational agent in all three cases. A Boolean approach wants to see the
difference in the evidential situation reflected in a rational agent's
credal state and for example at least permit, as credence in $H$,
$\{x|x=0.5\}$ for \textsc{coin}$_{I}$, $\{x|0\leq{}x\leq{}1\}$ for
\textsc{coin}$_{II}$, and $\{x|1/3\leq{}x\leq{}2/3\}$ or $\{1/3,2/3\}$
for \textsc{coin}$_{III}$. Here is a list of reasons for the Boolean
position.

\begin{enumerate}[(A)]
\item The greatest emphasis motivating indeterminacy rests on lack of
  evidence or conflicting evidence and the assumption that single
  probability measures (sharp credences) do not represent such
  evidence as well as credal states composed by sets of probability
  measures (indeterminate credal states).
\item The preference structure of a rational agent may be incomplete
  so that representation theorems do not yield single probability
  measures to represent such incomplete structures.
\item There are more technical and paper-specific reasons, such as
  Thomas Augustin's attempt to mediate between the minimax pessimism
  of objectivists and the Bayesian optimism of subjectivists using
  interval probability (see \scite{8}{augustin03}{35f}); Alan
  H{\'a}jek and Michael Smithson's belief that there may be
  objectively indeterminate chances in the physical world (see
  \scite{8}{hajeksmithson12}{33}); and Jake Chandler's claim that
  \qeins{the sharp model is at odds with a trio of plausible
    propositions regarding agnosticism} \scite{2}{chandler14}{4}.
\end{enumerate}

This paper mostly addresses (A), while taking (B) seriously as well
and pointing towards solutions for it. I am leaving (C) to more
specific responses to the issues presented in the cited articles, and
for the remainder of this section I am adding a reason (D) that is
poorly documented in the literature.

One motivation for permitting a rational agent to have a range of
probability measures rather than mandating her to hold a single one as
her credence is that she would do better making decisions and
accepting advantageous bets. Walley conducted an experiment in which
Boolean participants did significantly better than Laplacean
participants, betting on soccer games played in the Soccer World Cup
1982 in Spain (see \scite{7}{walley91}{}, Appendix F).\tbd{check
  reference} 

I replicated the experiment using two computer players with
rudimentary artificial intelligence and made them specify betting
parameters (previsions) for games played in the Soccer World Cup 2014
in Brazil. I used the Poisson distribution (which is an excellent
predictor for the outcome of soccer matches) and the FIFA ranking to
simulate millions of counterfactual World Cup results and their
associated bets, using Walley's evaluation method. The Boolean player
had a slight but systematic advantage. In section
\ref{WalleysWorldCupWoes}, I will provide an explanation and show how
it undermines any support the experiment might give to the Boolean
position.

\section{Dilation} 
\label{Dilation}

Roger White introduces a problem for indeterminate credal states,
which I will call White's dilation problem, claiming that
indeterminate credal states lead to unacceptable doxastic scenarios
involving dilation. In dilation, there is a widening of the
indeterminate credal state upon updating instead of a narrowing.
Sometimes this widening is extreme, from a maximally precise credal
state to a vacuous credal state, based on little information (see
\scite{7}{white10}{} for examples). White's objection, even though I
think it fails as it stands, triggers semantic concessions on the part
of Booleans defending their position which will be important to my
semantic criticism of indeterminacy.

In Joyce's response to White, Lewis' indadmissibility criterion as
well as two significant semantic concessions to White show why the
indeterminate credal states give us the right result. I agree with
Joyce: dilation is what you would expect (1) if credences do not
adequately represent evidence (the same indeterminate credal state can
reflect different evidential situations); and (2) if indeterminate
credal states do not reflect knowledge claims about objective chances
(Joyce rejects White's Chance Grounding Thesis CGT, see
\scite{8}{joyce10}{289}). Dilation from a sharp $0.5$ to an
indeterminate $[0.01,0.99]$ or $\{0.01,0.99\}$ (depending on whether
convexity is required) is unproblematic in the following example,
although the example already prefigures that there is something odd
about the Boolean semantic approach. The example licences a 99:1 bet
for one of the colours, but this is a problem that arises out of
Boolean semantics without dilation, which we will address again in
example \ref{monkey}.

\addtocounter{expls}{1}

\begin{quotex}
  \textbf{Example \arabic{expls}: \addtocounter{expls}{1} Dilating
    Urns.} You draw from an urn with 200 balls (100 red, 100 black) and
  receive the information that the urn actually had two chambers,
  one with 99 red balls and 1 black ball, the other with 1 red ball
  and 99 black balls.
\end{quotex}

If one were to be committed to the principle of regularity, that all
states of the world considered possible have positive probability (for
a defence see \scite{7}{edwardsetal63}{}); and to the solution of
Henry Kyburg's lottery paradox, that what is rationally accepted
should have probability 1 (for a defence of this principle see
\scite{7}{douvenwilliamson06}{}); and the CGT, that one's spread of
credence should cover the range of possible chance hypotheses left
open by evidence; then one's indeterminate credal state is always
vacuous. Booleans must deny at least one of the premises to avoid the
conclusion. Joyce denies the CGT, but then he continues to make
implicit use of it when he repeatedly complains that sharp credences
\qeins{ignore a vast number of possibilities that are consistent with
  [the] evidence} (for example in \scite{2}{joyce05}{170}).

When updating dilates the credal state, it appears that the prior
credal state was in some sense incorrect and did not properly reflect
the state of the world, even though it properly reflected the
epistemic state of the agent. Such a divergence between the proper
reflections of epistemic state and state of the world undermines the
subjective interpretation of probabilities at the centre of Bayesian
epistemology. An indeterminate credal state semantically blurs the
line between traditional full belief epistemology and formal partial
belief epistemology (for semantically more intelligible attempts at
reconciliation between the two see \scite{7}{moss13}{}, although Moss
is committed to the Boolean approach, which may weaken her case; and
\scite{7}{spohn12}{}, especially chapter 10).

White's objection fails because dilation for indeterminate credences
is in principle not any more surprising than a piece of information
that increases the Shannon entropy of a sharp credence. It is true for
both sharp and indeterminate credences that information can make us
less certain about things. We will turn to the semantic fallout of the
Boolean response to White's dilation problem now and show how it
brings into relief the virtues of sharp credences in responding to the
forceful motivations for indeterminate credal states while making
those indeterminate credal states look semantically otiose. Then we
will show in the last section how sharp credences have an elegant
solution for being outperformed by indeterminate credal states in
betting scenarios, and there we will rest our case.

\section{Semantics of Partial Belief} 
\label{SemanticsOfPartialBelief}

At the heart of my project is the proper semantic distinction between
evidence, information, and partial belief. Both sharp credences and
indeterminate credal states, within a Bayesian framework, try to
represent the uncertainty of an agent with respect to the truth of a
proposition. Indeterminate credal states have a greater ambition and
therefore a double task: they also claim to represent properties of
the evidence, such as its weight, conflict between its constituents,
or its ambiguity.

The Laplacean approach uses partial beliefs about how a parameter is
distributed and then applies Lewis' summation formula (see
\scite{8}{lewis81}{266f}) to integrate over them and condense them to
a sharp credence. Walley comments on this \qeins{reduction} in his
section on Bayesian second order probabilities (see
\scite{8}{walley91}{258f}), but he mistakenly represents the Laplacean
approach as a second order approach, as if the probability
distributions that are summarized by Lewis' formula are of the same
kind as the resulting credences. They are not. They are objective
chances or other partitions of the event space and the subjective
probabilities that are associated with them. It is the Boolean
approach which has elements of a second order approach and thus makes
itself vulnerable to regress problems by adding another dimension of
uncertainty to a parameter (the credence) which already represents
uncertainty.

One of Joyce's complaints is that a sharp credence of $0.5$ for a
\textsc{coin} contains too much information if there is little or no
evidence that the \textsc{coin} is fair. This complaint, of course, is
only effective if we make a credence say something about the evidence.
Joyce himself, however, admits that indeterminate credal states cannot
represent the evidence without violating the reflection principle due
to White's dilation problem. He is quite clear that the same
indeterminate credal state can represent different evidential
scenarios (see, for example, \scite{8}{joyce10}{302}).

Sharp credences have one task: to represent epistemic uncertainty and
serve as a tool for updating, inference, and decision-making. They
cannot fulfill this task without continued reference to the evidence
which operates in the background. To use an analogy, credences are not
sufficient statistics with respect to updating, inference, and
decision-making. What is remarkable about Joyce's response to White's
dilation problem is that Joyce recognizes that indeterminate credal
states are not sufficient statistics either. But this means that they
fail at the double task which has been imposed on them: to represent
both epistemic uncertainty and the evidence.

In the following, I will provide a few examples where it becomes clear
that indeterminate credal states have difficulty representing
uncertainty because they are tangled in a double task which they
cannot fulfill.

\begin{quotex}
  \textbf{Example \arabic{expls}: \addtocounter{expls}{1} Aggregating
    Expert Opinion.} You have no information whether it will rain
  tomorrow ($R$) or not except the predictions of two weather
  forecasters. One of them forecasts 0.3 on channel GPY, the other 0.6
  on channel QCT. You consider the QCT forecaster to be significantly
  more reliable, based on past experience.
\end{quotex}

An indeterminate credal state corresponding to this situation may be
$[0.3,0.6]$ (see \scite{8}{walley91}{214}), but it will have a
difficult time representing the difference in reliability of the
experts. A sharp credence of $P(R)=0.55$, for example, does the right
thing. Beliefs about objective chances make little sense in many
situations where we have credences, since it is doubtful even in the
case of rain tomorrow that there is an urn of nature from which balls
are drawn (see \scite{8}{white10}{171}). What is really at play is a
complex interaction between epistemic states (for example, experts
evaluating meteorological data) and the evidence which influences
them.

\begin{quotex}
  \textbf{Example \arabic{expls}:
    \addtocounter{expls}{1}\label{monkey} Precise Credences.} Your
  sharp credence for rain tomorrow, based on the expert opinion of
  channel GPY and channel QCT (you have no other information) is
  $0.55$. Is it reasonable, considering how little evidence you have,
  to reject the belief that the chance of rain tomorrow is $0.54$ or
  $0.56$; or to prefer a 54.9 cent bet on rain to a 45.1 cent bet on
  no rain?
\end{quotex}

The first question, of course, is confused, but in instructive ways. A
sharp credence rejects no hypothesis about objective chances (unlike
an indeterminate credal state). It often has a subjective probability
distribution operating in the background, over which it integrates to
yield the sharp credence. The subjective probability distribution $P$
is condensed by Lewis' summation formula to a sharp credence $C$,
without being reduced to it:

\begin{equation}
  \label{eq:s2}
  C(R)=\int_{0}^{1}\zeta{}P(\pi(R)=\zeta)\,d\zeta
\end{equation}

Lewis' 1981 paper \qeins{A Subjectivist's Guide to Objective Chance}
addresses the question what the relationship between $\pi$, $P$, and
$C$ is. The point is that we have properly separated the semantic
dimensions and that the Laplacean approach is not a second order
probability approach. The partial belief epistemology deals with sharp
credences and how they represent uncertainty and serve as a tool in
inference, updating, and decision making; while Lewis' Humean
speculations and his interpretation of the principal principle cover
the relationship between subjective probabilities and objective
chance.

The second question is also instructive: why would we prefer a 54.9
cent bet on rain to a 45.1 cent bet on no rain, given that we do not
possess the power of descrimination between these two bets? The answer
to this question ties in with the issue of incomplete preference
structure referenced above as motiviation (B) for indeterminate credal
states. 

Via representation theorems, preferences may very well conceptually
precede an agent's probability and utility functions, but that does
not mean that we cannot inform the axioms we use for a rational
agent's preferences by undesirable consequences downstream.
Completeness may sound like an unreasonable imposition at the outset,
but if incompleteness has unwelcome semantic consequences for
credences, it is not illegitimate to revisit the issue. Timothy
Williamson goes through this exercise with vague concepts, showing
that all upstream logical solutions to the problem fail and that it
has to be solved downstream with an epistemic solution (see
\scite{7}{williamson96}{}). 

Vague concepts, like sharp credences, are sharply bounded, but not in
a way that is luminous to the agent (for anti-luminosity see chapter 4
in \scite{7}{williamson00}{}). Anti-luminosity answers the original
question: the rational agent prefers the 54.9 cent bet on rain to a
45.1 cent bet on no rain based on her sharp credence without being in
a position to have this preference necessarily or have it based on
physical or psychological ability (for the analogous claim about
knowledge see \scite{8}{williamson00}{95}).

In a way, advocates of indeterminacy have solved this problem for us.
There is strong agreement among most of them that the issue of
indeterminacy for credences is not an issue of elicitation. The appeal
of preferences is that we can elicit them more easily than assessments
of probability and utility functions. The indeterminacy issue has been
raised to the probability level (or moved downstream) by indeterminacy
advocates themselves who feel justifiably uncomfortable with an
interpretation of their theory in behaviourist terms. So it shall be
solved there, and this paper makes an appeal to reject indeterminacy
on this level. Ironically, Isaac Levi seems to agree with me on this
point: when he talks about indeterminacy, it proceeds from the level
of probability judgment to preferences, not the other way around (see
\scite{8}{levi81}{533}).

\begin{quotex}
  \textbf{Example \arabic{expls}: \addtocounter{expls}{1} Jaynes'
    Monkeys.} Let urn $A$ contain 4 balls, two red and two black. A
  monkey randomly fills urn $B$ from urn $A$ with two balls. We draw
  from urn $B$. (See \scite{8}{jaynesbretthorst03}{160}.)
\end{quotex}

The sharp credence of drawing a red ball is $0.5$, following Lewis'
summation formula for the different combinations of balls in urn $B$.
I find this solution more intuitive in terms of further inference,
decision making, and betting behaviour than a credal state of
$\{0,1/2,1\}$ or $[0,1]$ (depending on the convexity requirement),
since the indeterminate credal state would licence an exorbitant bet
in favour of one colour.

\begin{quotex}
  \textbf{Example \arabic{expls}: \addtocounter{expls}{1} Three
    Prisoners.} The Three Prisoners problem is well-documented (see
  \scite{8}{mosteller87}{28}). Let the three prisoners be
  $X_{1},X_{2},X_{3}$ and the warden tell $X_{1}$ that $X_{3}$ will be
  executed.
\end{quotex}

Peter Walley maintains that for the Monty Hall problem and the Three
Prisoners problem, the probabilities of a rational agent should dilate
rather than settle on the commonly accepted solutions. Consider the
three prisoners problem. There is a compelling case for standard
conditioning and the result that the chances of pardon for prisoner
$X_{1}$ are unchanged after the update (see
\scite{8}{lukits14}{1421f}). Walley's dilated solution would give
prisoner $X_{1}$ hope on the doubtful possibility (and unfounded
assumption) that the warden might prefer to provide $X_{3}$'s name in
case prisoner $X_{1}$ was pardoned.

Booleans charge that sharp credences often reflect independence of
variables where such independence is unwarranted. Booleans prefer to
dilate over the possible dependence relationships (independence
included). White's dilation problem is an instance of this. The
fallacy in the argument for indeterminate credal states, illustrated
by the Three Prisoners problem, is that the probabilistic independence
of sharp credences does not imply independence of variables (the
converse is correct), but only that it is unknown whether there is
dependence, and if yes, whether it is correlation or inverse
correlation.

\begin{quotex}
  \textbf{Example \arabic{expls}: \addtocounter{expls}{1} Wagner's
    Linguist.} A linguist hears the utterance of a native and
  concludes that the native cannot be part of certain population
  groups, depending on what the utterance means. The linguist is
  uncertain between some options about the meaning of the utterance.
  (For full details see \scite{8}{wagner92}{252}; and
  \scite{8}{spohn12}{197}.)
\end{quotex}

The mathematician Carl Wagner proposed a natural generalization of
Jeffrey Conditioning for his Linguist example (see
\scite{7}{wagner92}{}). Since the principle of maximum entropy is
already a generalization of Jeffrey Conditioning, the question
naturally arises whether the two generalizations agree. Wagner makes
the case that they do not agree and deduces that the principle of
maximum entropy is sometimes an inappropriate updating mechanism, in
line with many earlier criticisms of the principle of maximum entropy
(see van Fraassen, 1981; \scite{7}{shimony85}{};
\scite{7}{skyrms87updating}{}; and, later on,
\scite{7}{grovehalpern97}{}). 

What is interesting about this case is that Wagner uses indeterminate
credal states for his deduction, so that even if you agree with his
natural generalization of Jeffrey Conditioning (which I find
plausible), the inconsistency with the principle of maximum entropy
can only be inferred assuming indeterminate credal states. On the
assumption of sharp credences Wagner's generalization of Jeffrey
conditioning perfectly accords with the principle of maximum entropy.
Wagner's argument, instead of undermining the principle of maximum
entropy, just shows that indeterminate credal states are as wedded to
rejecting the claims of the principle of maximum entropy as the
principle of maximum entropy is wedded to sharp credences.

\section{Evidence Differentials and Cushioning Credences} 
\label{WalleysWorldCupWoes}

After I found out that agents with indeterminate credal states do
better betting on soccer games (see section
\ref{MotivationForIndeterminateCredalStates}), I let player $X$ (who
uses sharp credences) and player $Y$ (who uses indeterminate credal
states) play a more basic betting game and used Walley's pay off
scheme (see \scite{8}{walley91}{632}) to settle the bets. The
simulation results show that player $Y$ does better for $n>2$ while
player $X$ does better for $n=2$. A defence of sharp credences for
rational agents needs to have an explanation why, for $n>2$, player
$Y$ does better. We will call it partial belief cushioning, which is
based on an evidence differential between the bettors.

Compulsory betting is a problem for indeterminate credal states, as
Booleans have to find a way to decide without receiving instructions
from the credal state. Booleans have addressed this point extensively
(see for example \scite{8}{joyce10}{311ff}; for an opponent's view of
this see \scite{8}{elga10}{6ff}). The problem for sharp credences
arises when bets are noncompulsory, for then the data above suggests
that agents holding indeterminate credal states sometimes do better.
Bets are sometimes offered by better informed or potentially better
informed bookies. In this case, even an agent with sharp credences
must cushion her credences and is better off by rejecting bets that
look attractive in terms of her partial beliefs. If an agent does not
cushion her partial beliefs (whether they are sharp or indeterminate),
she will incur a loss in the long run. Since cushioning is permitted
in Walley's experimental setup (the bets are noncompulsory), Laplacean
agents should also have access to it and then no longer do worse than
Boolean agents.

The rational agent with a sharp credence has resources at her disposal
to use just as much differentiation with respect to accepting and
rejecting bets as the agent with indeterminate credal states. Often
(if she is able to and especially if the bets are offered to her by a
better-informed agent), she will reject both of two complementary
bets, even when they are fair. In 2013, the U.S. Commodity Futures
Trading Commission shut down the online prediction market Intrade,
even though it was offering fair bets, because of expert arbitrageurs.

Any advantage that the agent with an indeterminate credal state has
can be counteracted based on distribution over partial beliefs with
respect to all possibilities. The agent with indeterminate credal
states suffers from a semantic mixing of metaphors between evidential
and epistemic dimensions that puts her at a real disadvantage in terms
of understanding the sources and consequences of her knowledge and her
uncertainties.

\section{References} 
\label{References}

\nocite{boole54} 
\nocite{fraassen81} 
\bibliographystyle{ChicagoReedweb} 
\bibliography{bib-7293}

\end{document} 
