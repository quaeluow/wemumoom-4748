% TBD: 

% (1) What exactly is player $X$'s evidence differential compared to
% player $Y$ -- the analytic expression may help.

% (2) Talk about Second Order probabilities, especially with reference
% to hierarchical nesting, see Second order in collator, 

\documentclass[11pt]{article}
\usepackage{october}
% For BJPS
% \hyphenpenalty=10000
% \hbadness=10000

\begin{document}
% For BJPS
% \raggedright
% \doublespacing

\title{Semantics of Not Knowing}
\author{Stefan Lukits}
\date{}
\maketitle
% \doublespacing

% \begin{abstract} 
%   {\noindent}
% \end{abstract}

\section{Introduction}
\label{Introduction}

Traditional Bayesian epistemology used to include the requirement for
a rational agent to hold a sharp credence function. In the last twenty
years or so, it has become increasingly popular to drop this
requirement. There is now a well-articulated Bayesian theory which
permits a rational agent to hold indeterminate credal states based on
incomplete or ambiguous evidence. The consensus is that these
credences are not only not sharp because it does not seem reasonable
to elicit precise probability assignments from an agent, even if such
an assignment may operate in the background; but that it is in some
situations more rational to hold a set of credence functions, also
called a credal state, rather than one sharp credence function to
represent a rational partial belief. 

There is some terminological confusion around the adjectives
\qnull{imprecise,} \qnull{indeterminate,} and \qnull{mushy} credences.
In the following, I will exclusively refer to indeterminate credences
or credal states and mean by them a credal state (a set of credence
functions, which some proponents of indeterminacy require to be
convex) which it may be rational for an agent to hold within an
otherwise orthodox Bayesian framework. I will refer to Bayesians who
continue to adhere to the classical theory of sharp credences for
rational agents as \qnull{Laplaceans} (e.g.\ Adam Elga and Roger
White), while I will call Bayesians who do not believe in the
requirement for a rational agent to hold sharp credences
\qnull{Booleans} (e.g.\ Peter Walley and James Joyce; see
\scite{7}{boole54}{}, chapters 16--21, for alternative methods to the
ones suggested by Laplace which resulted in imprecise epistemic
probabilities).

% This paper will examine two objections to indeterminacy, one by Adam
% Elga and one by Roger White, the former claiming that rational agents
% who use indeterminate credal states make themselves vulnerable to sure
% loss, and the latter that indeterminate credal states lead to
% unacceptable doxastic scenarios involving dilation. I will show how
% both of these objections fail (most of the work towards this end has
% already been done by Joyce, see \scite{7}{joyce10}{}; White's
% work, even though I think the objections fail as they stand, has a
% strong influence on my own objections).

After describing the appeal of indeterminacy and showing how
contemporary Laplacean objections fail, I will point to more serious
failings of indeterminacy in semantic terms and show how a proper
semantics of not knowing, which we could also call a semantics of
partial belief, solves the problems indeterminacy seeks to address and
maintains the traditional Bayesian adherence to sharp credences for
rational agents. The title of my paper, the semantics of not knowing,
suggests that I see a more pronounced separation between traditional
knowledge epistemology addressing full belief and formal epistemology
being more concerned with partial belief. There is a sense in which
indeterminacy seeks to soften the separation by linking knowledge of
chances to its reflection in credences. There are recent projects
reconciling traditional full belief epistemology and formal partial
belief epistemology (see \scite{7}{spohn12}{}; and
\scite{7}{moss13}{}), but if my paper is on the right track
indeterminacy will not contribute to this reconciliation because it
mixes full belief and partial belief metaphors in ways that are
semantically imprecise and unhelpful.

The underlying idea of my criticism is that sharp credences run into a
genuine conceptual crisis when confronted with intuitions about
betting behaviour, completeness for preferences, and how reasonable it
is to demand precision from agents, even if they are rational.
Indeterminacy appears to provide an elegant solution to overcome this
crisis as well as a powerful formal theory to continue what many find
so compelling about the Bayesian legacy. In the face of these virtues,
I maintain that indeterminacy adds an unnecessary layer to the
semantics of partial belief which has in its wake counter-intuitive
consequences and solves nothing that cannot be solved by a carefully
articulated version of the classical Bayesian commitment to sharp
credences.

A sharp credence, as much as the term suggests precision and a measure
of certainty and being informed, is a representation of uncertainty
and lack of information. Importantly, it does not represent the
evidence which undergirds it and makes no claim of such a
representation. I will demonstrate how indeterminate credal states are
on the one hand often lauded as doing much better representing
uncertainty together with the evidence that constrains it, while on
the other hand indeterminate credal states can no more represent
evidence than sharp credences. Neither determinate nor indeterminate
credal states are \qnull{sufficient statistics} with respect to
evidence. Indeterminate credal states try and fail, determinate
credences do not pretend. Thus it is legitimate to have a 0.5 sharp
credence for a coin of whose bias we are completely ignorant, for a
coin whose fairness is supported by a lot of evidence, and even for a
coin about whose bias we know that it is either 1/3 or 2/3. The main
body of the paper will address this issue in detail.

One potential Boolean claim is that agents who use indeterminate
credal states do better than Laplaceans when they bet on the truth of
events for which they have varying degrees of evidence. Peter Walley
gives an example where a Laplacean does much worse at predicting
Soccer World Cup games than Boolean peers who use upper and lower
previsions. Upper and lower previsions are indeterminate credal states
for which bets may either be rejected or accepted if they fall within
the margin of indeterminacy. First, I will show on a much more general
level how Walley's claims are justified in practice (bettors using
upper and lower previsions do better than bettors using linear
previsions, i.e.\ sharp credences). Then I will explain why this is
the case, how a Laplacean can protect herself against this
disadvantage by drawing proper distinctions between credence and
evidence, and how indeterminacy emerges as the loser when the contest
is about clarity in one's semantics. Again, the problem will be mixed
metaphors: the advocate of indeterminacy mixes semantic levels that
ought for good reasons to remain separate. Indeterminacy imposes a
double task on credences (representing both uncertainty and available
evidence) that they cannot coherently fulfill.

I will present several examples where this double task stretches
indeterminate credal states to the limits of plausibility, for example
when they need to aggregate expert opinion and assess evidence
differentials or when they need to account for dilation. There is no
doubt that the precision of sharp credences in situations where
evidence is extremely thin or non-existent, ambiguous, or incomplete
leaves a bad taste in everyone's mouth. My proposal is to address
these problems by semantic clarity about what in detail it is that
credences represent and what they do not represent and do not pretend
to represent. For betting behaviour and decision problems, this
clarity will eventually work in favour of the agent because she can
properly distinguish which part of her decision calculus is based on
the weight of the evidence and which part on the balance of the
evidence.

The weight of the evidence can be factored into the balance and
determine decisions, but the balance cannot represent the weight.
Joyce's idea that credences can represent balance, weight, and
specificity of the evidence is inconsistent with the use of
indeterminacy (and Joyce himself, in response to White's dilation
problem, gives the argument why this is the case). The road from
evidence via information to credences is a one-way road. The implicit
Boolean claim that evidence can be recovered from indeterminate credal
states is vulnerable to White's dilation problem, and so it is
rejected by Joyce with semantic implications which we will
investigate more closely.

% One major problem for indeterminacy is that after a decision has been
% made on the basis of indeterminate credal states, the weight of the
% evidence can no longer be recovered from them. The agent can, of
% course, always go back to her evidence stored in memory, but then it
% was unnecessary in the first place to mix its weight into her
% credences.

% TBD Leave vagueness and Williamson to the discussion of Yang Liu's point
The Laplacean approach of assigning subjective probabilities to
distributions and then aggregating them by David Lewis's summation
formula into a single precise credence function is semantically clean
and shares many of the formal virtues of theories preferring
indeterminate credal states. If the bad taste about numerical
precision in our fuzzy, nebulous world lingers, we can point to
philosophical projects in other domains where the concepts we use are
sharply bounded, even though our ability to conceive of those sharp
boundaries or know them is limited, for example Timothy Williamson's
work on vagueness and knowledge as a mental state.

\section{Motivation for Indeterminate Credal States} 
\label{MotivationForIndeterminateCredal States}

We want to motivate indeterminacy for the credences of a rational
agent, independent of how they are elicited, as forcefully as possible
so that the reader will see (a) the appeal of such indeterminacy, (b)
the insufficiency of the critical response, and (c) the need for
careful articulation of the Laplacean approach that mandates a
rational agent to hold sharp credences together with an explanation of
how this careful articulation addresses the concerns which motivate
some to resort to indeterminacy.

When textual criticism compares variant readings of ancient texts in
order to recover the most probable reading of the original, it follows
a well-accepted rule not to accept the easiest reading, i.e.\ the one
that flows most naturally with the surrounding text, but the (often
more difficult) reading which best explains the variants in the
manuscripts. A similar dynamic is at work here: motivating
indeterminacy will make clear that acceptance of indeterminacy is the
easiest reading and on the surface best responds to intuitions we have
about the credences of a rational agent. We will address, however, the
ways in which indeterminacy itself violates intuitions, especially
those for semantic clarity and simplicity (the latter viewed in terms
of reducing hierarchical nesting of uncertainty versus tidily
separating the evidential and epistemic dimension of partial beliefs).
The Laplacean approach, which does not permit indeterminacy, explains
why this is the case and, despite being the more difficult reading,
ultimately yields a more integrated package that accords better with
our intuitions and requirements for a coherent formal and semantic
theory.

Our conclusion is that a rational agent is best off in terms of her
own goals when she entertains sharp credences with respect to
propositions about events that come her way. Whether this is advisable
for human or machine intelligence is a different kettle of fish. My
topic is the logic of partial beliefs, and I readily admit that such a
logic may be computationally intractable or, given finite resources,
be an irrational way of keeping track of beliefs.

% TBD: Condense Ellsberg for APA
Despite this disclaimer, we begin with an example that motivates
indeterminate credal states based on human betting behaviour. It is
nonetheless forceful. For the Ellsberg paradox (see
\scite{8}{ellsberg61}{650ff}), you  have two urns, both containing 100
red and black balls. Urn I contains an unknown ratio of the two
colours, whereas the ratio for Urn II is 50:50. In experiments,
subjects usually prefer to bet on Red$_{II}$ rather than Red$_{I}$ and
Black$_{II}$ rather than Black$_{I}$. If these preferences stem from
sharp subjective probabilities, then 

\begin{equation}
  \label{eq:s1}
  1=P(R_{II})+P(B_{II})>P(R_{I})+P(B_{I})=1
\end{equation}

leads to a violation of probability axioms by reductio. A slight
variation shows that such preferences also violate L.J. Savage's Sure
Thing principle, without the possibly doubtful numerical mapping from
preferences to subjective probabilities (see
\scite{8}{ellsberg61}{653f}). The intuition behind the Ellsberg
paradox remains a strong motivation for indeterminate credal states, often
expressed in terms of biased coins.

Let a \textsc{coin} be a Bernoulli generator that produces successes
and failures with probability $p$ for successes, labeled $H$, and
$1-p$ for failures, labeled $T$. Physical coins may serve as examples
if we are willing to set aside that most of them are approximately
fair. Imagine three \textsc{coin}s for which we have evidence that
\textsc{coin}$_{I}$ is fair, \textsc{coin}$_{II}$ has an unknown bias,
and \textsc{coin}$_{III}$ has either $p=1/3$ or $p=2/3$. The Laplacean
approach, on the one hand, would permit a sharp $0.5$ credence in $H$
for a rational agent in all three cases. A Boolean approach, on the
other hand, wants to see the difference in the evidential situation
reflected in a rational agent's credal state and for example at least
permit, as credence in $H$, $\{x|x=0.5\}$ for \textsc{coin}$_{I}$,
$\{x|0\leq{}x\leq{}1\}$ for \textsc{coin}$_{II}$, and
$\{x|1/3\leq{}x\leq{}2/3\}$ or $\{1/3,2/3\}$ for
\textsc{coin}$_{III}$.

Here are a few concise statements by Booleans:

\begin{quotex}
  [A] refusal to make a determinate probability judgment does not
  derive from a lack of clarity about one's credal state. To the
  contrary, it may derive from a very clear and cool judgment that on
  the basis of the available evidence, making a numerically
  determinate judgment would be unwarranted and arbitrary.
  \scite{3}{levi85}{395}
\end{quotex}

\begin{quotex}
  If there is little evidence concerning [a claim,] then beliefs about
  [that claim] should be indeterminate, and probability models
  imprecise, to reflect the lack of information. We regard this as the
  most important source of imprecision. \scite{3}{walley91}{212--213}
\end{quotex}

\begin{quotex}
  Imprecise probabilities and related concepts [{\ldots}] provide a
  powerful language which is able to reflect the partial nature of the
  knowledge suitably and to express the amount of ambiguity
  adequately. \scite{3}{augustin03}{34}
\end{quotex}

\begin{quotex}
  As sophisticated Bayesians like Isaac Levi (1980), Richard Jeffrey
  (1983), Mark Kaplan (1996), have long recognized, the proper
  response to symmetrically ambiguous or incomplete evidence is not to
  assign probabilities symmetrically, but to refrain from assigning
  precise probabilities at all. Indefiniteness in the evidence is
  reflected not in the values of any single credence function, but in
  the spread of values across the family of all credence functions
  that the evidence does not exclude. This is why modern Bayesians
  represent credal states using sets of credence functions. It is not
  just that sharp degrees of belief are psychologically unrealistic
  (though they are). Imprecise credences have a clear epistemological
  motivation: they are the proper response to unspecific evidence.
  \scite{3}{joyce05}{170f}
\end{quotex}

\begin{quotex}
  Precise degrees of belief are the wrong response to the sorts of
  evidence that we typically receive [{\ldots}] since the data we
  receive is often incomplete, imprecise or equivocal, the
  epistemically right response is often to have opinions that are
  similarly incomplete, imprecise or equivocal.
  \scite{3}{joyce10}{283}
\end{quotex}

% \begin{quotex}
%   The semantic value of a sentence is a set of probability measures,
%   and an assertion expresses the advice that your credence
%   distribution be among the members of that set. \scite{3}{moss13}{4}
% \end{quotex}

% \begin{quotex}
%   It hardly seems a requirement of rationality that belief be precise
%   (and preferences complete); surely imprecise belief (and
%   corresponding incomplete preferences) are at least rationally
%   permissible. \scite{3}{bradleysteele13}{2}
% \end{quotex}

% On the opposite side of the fence are the following:

% \begin{quotex}
%   The principle of maximum entropy is not an oracle telling which
%   prediction must be right; it is a rule for inductive reasoning that
%   tells us which predictions are more strongly indicated by our
%   present information'' \scite{3}{jaynesbretthorst03}{370}
% \end{quotex}

% \begin{quotex}
%   Obviously ignorance is no basis for a belief concerning contingent
%   physical conditions. But it is not at all out of the question that
%   your ignorance puts constraints on what your degrees of confidence
%   should be. \scite{3}{white10}{163}
% \end{quotex}

In summary, here is a list of reasons why credal states for ideally
rational agents do not need to be representable by single probability
measures relative to the appropriate corpora of knowledge, as Rudolf
Carnap and E.T. Jaynes require (see \scite{8}{levi81}{533}). 

\begin{enumerate}[(A)]
\item By far the greatest emphasis in motivation for indeterminacy
  rests on lack of evidence or conflicting evidence and the assumption
  that single probability measures do not represent such evidence as
  well as credal states composed by sets of probability measures.
\item The preference structure of a rational agent may be incomplete
  so that representation theorems do not yield single probability
  measures to represent such incomplete structures.
\item There are more technical and paper-specific reasons, such as
  Thomas Augustin's attempt to mediate between the minimax pessimism
  of objectivists and the Bayesian optimism of subjectivists using
  interval probability (see \scite{8}{augustin03}{35f}); Alan
  H{\'a}jek and Michael Smithson's belief that there may be
  objectively indeterminate chances in the physical world (see
  \scite{8}{hajeksmithson12}{33}); and Jake Chandler's claim that
  \qeins{the sharp model is at odds with a trio of plausible
    propositions regarding agnosticism} \scite{2}{chandler14}{4}.
\end{enumerate}

This paper mostly addresses (A), while taking (B) seriously as well
and pointing towards solutions for it. 
% I thank Yang Liu and Christopher French for bringing (B) to my
% attention. 
I am leaving (C) to more specific responses to the issues presented in
the cited articles, and for the remainder of this section I am adding
a reason (D) that is poorly documented in the literature.

One motivation for permitting a rational agent to have a range of
probability measures rather than mandating her to hold a single one as
her credence is that she would do better making decisions and
accepting advantageous bets. Even though advocates of indeterminate
credal states seldom raise this issue, I find it worth pursuing.
Before we do this, it is necessary to make brief reference to the
variety of versions which accommodate indeterminacy as we have
sketched it: upper and lower probabilities, choquet capacities, belief
and possibility functions, coherent lower previsions, sets of
probability measures, partial preference orderings, and sets of
desirable gambles (the list is from \scite{8}{walley00}{126}). The
proliferation of versions, however, is neither a problem in principle
as the debate is vigorous and useful which of them best captures
requirements for rationality, nor in practice because the versions
agree on large parts of the terrain (see \scite{8}{levi85}{390}; and
\scite{8}{walley91}{50f}).

For our purposes here, we accept Peter Walley's theory of upper and
lower previsions, which is motivated by coherence on the one hand and
avoidance of sure losses on the other hand (see Walley's 1991 book
\emph{Statistical Reasoning with Imprecise Probabilities}). Upper and
lower previsions have a behavioural interpretation as maximum buying
prices and minimum selling prices for gambles where two bets can both
be rejected if they offer a reward of \$1 for a price of $x$ in case
of $H$ and a reward of \$1 for a price of $1-x$ in case of $T$ as long
as $x$ falls between the upper and lower prevision.

In the \textsc{coin}$_{II}$ example, if the agent's lower prevision is
$1/3$ and the upper prevision $2/3$, she would accept a bet paying \$1
if $H$ (or $T$) for thirty cents and reject such a bet for seventy
cents, but if the price of the bet is between one and two thirds of a
dollar it would be rational for her to either accept or reject the
bet. Walley conducted an experiment with 17 participants, who had
various levels of understanding Bayesian theory, asking them for upper
and lower probabilities entering into bets about the games played in
the Soccer World Cup 1982 in Spain. One of the participants, a
\qeins{dogmatic Bayesian lecturer} (see \scite{8}{walley91}{633}),
only used single sharp subjective probabilities, while the others used
intervals. The dogmatic Bayesian lecturer finished a distant last when
the bets were evaluated. Walley admits that this result may have been
due to the lecturer's eccentric assessments (for the game between the
Soviet Union and Brazil, for example, his distribution between a Win,
a Draw, and a Loss was 10-80-10).

I was curious what a more rigorous treatment of the comparison between
linear previsions (when upper and lower previsions coincide) and
interval previsions would yield. I equipped two computer players with
some rudimentary artificial intelligence and made them specify
previsions for games played in the Soccer World Cup 2014 in Brazil,
ordering player $X$ to specify linear previsions and player $Y$ to
specify upper and lower previsions (acknowledgements to open source
software making this possible and making it inexpensive, especially
emacs lisp, perl, R statistics, and octave). I used the Poisson
distribution (which is an excellent predictor for the outcome of
soccer matches) and the FIFA ranking to simulate millions of
counterfactual World Cup results and their associated bets, using
Walley's evaluation method. 

Player $Y$ had a slight but systematic advantage. Her expected gain
for a whole tournament was approximately \$2 (very little compared to
the stakes, as one can see by considering the standard deviation of
approximately \$46 for this expected gain, i.e. 67\% of player $Y$'s
overall gain from a tournament was between -\$44 and +\$48). The
surprising fact that she did consistently better than player $X$
remains in need of explanation. In section \ref{WalleysWorldCupWoes},
I will provide an explanation and show how it can support rejecting
indeterminate credal states for rational agents, counter-intuitive as
that may sound.

\section{Arbitrage Opportunities and Dilation} 
\label{ArbitrageOpportunitiesAndDilation}

From a collection of arguments against indeterminacy I draw on two
that sound particularly compelling and that in the final analysis
fail. One by Adam Elga (see \scite{7}{elga10}{}) charges indeterminacy
with making a rational agent vulnerable to sure loss and ends in the
statement: \qeins{Perfectly rational agents always have perfectly
  sharp probabilities} \scite{2}{elga10}{1}.

The other by Roger White (see \scite{7}{white10}{}) is initially more
semantically minded with an analysis of the Chance Grounding Thesis
(CGT), of which I will also make use in my criticism of indeterminacy.
For the remainder of White's paper, however, the focus moves to
dilation, which according to White saddles the indeterminacy approach
with unacceptable counter-intuitive results. Joyce has reponded in
favour of indeterminacy both to Elga and to White and in my view has
successfully defended his position.

For Elga, the issue is that while it is rational for an agent with an
upper and lower provision to reject both an $x$ bet for $H$ (shorthand
for \qnull{receiving \$1 if $H$ is true for the price of $x$}) and a
$1-x$ bet against $H$ (or, equivalently, for $T$), if $x$ is between
the two previsions; it should also be rational to accept both bets if
they are an $x+\delta$ bet for $H$ and a $1-x+\delta$ bet for $T$,
where $\delta$ is chosen small enough to keep the prices of the bets
within the previsions. These bets will lead to a sure loss and an
arbitrage opportunity for bookies against our supposedly rational
agent.

There is no reason to reinvent the wheel here: both Jake Chandler and
Joyce address Elga's objection, and while Chandler's defence of
indeterminacy against Elga does not persuade me (see
\scite{8}{chandler14}{10}), Joyce's defence does (see
\scite{8}{joyce10}{314}). Similarly, for White's objection, I am
satisfied with Joyce's response. Joyce deals with White's objection in
detail in his 2010 paper (for a concise summary see
\scite{8}{bradleysteele13}{13}). As opposed to Elga's objection,
White's objection has far-reaching semantic implications, and so we
will look at it more closely.

You have two Bernoulli Generators, \textsc{coin}$_{a}$ and
\textsc{coin}$_{b}$. You have good evidence that \textsc{coin}$_{a}$
is fair and no evidence about the bias of \textsc{coin}$_{b}$.
Furthermore, the two generators are not necessarily independent. Their
results could be 100\% correlated or anticorrelated, they may be
independent or the correlation could be anywhere between the two
extremes. You toss \textsc{coin}$_{a}$. Without looking at the result,
your credence in $H$ is a sharp $0.5$, even if you are open to
indeterminate credal states, because you have good evidence for the
fairness of \textsc{coin}$_{a}$. Then you toss \textsc{coin}$_{b}$.
This time you look at the result and the moment you learn it, your
credence in $H$ for \textsc{coin}$_{a}$ dilates from a sharp $0.5$ to
the vacuous credal state covering the whole interval $[0,1]$ (provided
that this was your credence in $H$ for \textsc{coin}$_{b}$, as
stipulated). Even though you have just received information (the
result of \textsc{coin}$_{b}$'s toss), your credence in $H$ for
\textsc{coin}$_{a}$ dilates. Usually, we would expect more information
to sharpen our credal states (see Walley's \qeins{the more information
  the more precision} principle and his response to this problem in
\scite{8}{walley91}{207 and 299}). 

White did not discover the phenomenon of dilation (besides Walley's
comments cited in previous the paragraph see also the detailed study
in \scite{7}{seidenfeldwasserman93}{}), but he was able to find
examples where the consequences appear grotesque, especially in his
bonbon case (see \scite{8}{white10}{183}).

\begin{quotex}
  Having noodled about this puzzle on and off for some time, I
  discovered that the general phenomenon of dilation is old news. Some
  statisticians and philosophers have studied how the phenomenon
  arises in other cases and appear to have taken it in their stride.
  This is not a reductio but a result, they might say. I want to
  suggest that the present case brings out particularly forcefully how
  bizarre this phenomenon is, at least in the present case where we
  are assuming evidential symmetry between $p$ and not-$p$.
  \scite{3}{white10}{177}
\end{quotex}

White's claim is also that dilation contradicts Bas van Fraassen's
reflection principle (see \scite{7}{vanfraassen84}{}). Joyce's
response uses two lines of defence: the results are not
counter-intuitive upon closer inspection and Lewis's indadmissibility
criterion as well as two significant semantic concessions to White
show why the indeterminate credal states give us the right result.
This response is intriguing in several ways. It articulates just the
right response to White: yes, dilation is what you would expect if
credences do not represent evidence (the same indeterminate credal
state can reflect different evidential situations) and if the CGT is
not a necessary consequence of the indeterminacy approach.

These are just the kinds of semantic questions we will address in the next
section. In the meantime, I agree with Joyce and with White's own words that
\qeins{this is not a reductio but a result.} The possible dependence between
\textsc{coin}$_{a}$ and \textsc{coin}$_{b}$ and the unknown bias of
\textsc{coin}$_{b}$ dilate our credences in the result of tossing
\textsc{coin}$_{a}$. The same would happen if we were drawing from an urn with
200 balls (100 red, 100 black) and received the information that the urn
actually had two chambers, one with 99 red balls and 1 black ball, the other
with 1 red ball and 99 black balls. Our credence would dilate, given this
additional piece of information, from a sharp $0.5$ to an indeterminate
$[0.01,0.99]$ without much mystery (the indeterminate credence would licence
an unreasonable 99:1 bet, but that is part of the semantic criticism below). 

The problem with dilation is a problem that proponents of
indeterminacy have created for themselves, but they also have the
resources to extract themselves from it, albeit only at the cost of
opening up semantic issues which turn the case against them. A sharp
credence, on the one hand, constrains partial beliefs in objective
chances by Lewis's summation formula (which we will provide in the
next section). No objective chance is excluded by it and any updating
will merely change the partial beliefs, but no full beliefs.
Indeterminate credal states, on the other hand, by giving ranges of
acceptable objective chances suggest that there is a full belief that
the objective chance does not lie outside what is indicated by the
indeterminate credal state. 

If one were to be committed to the principle of regularity, that all
states of the world considered possible have positive probability (for
a defence see \scite{7}{edwardsetal63}{}); and to the solution of
Henry Kyburg's lottery paradox, that what is rationally accepted
should have probability 1 (for a defence of this principle see
\scite{7}{douvenwilliamson06}{}); and to White's CGT, that one's
spread of credence should cover the range of possible chance
hypotheses left open by evidence; then one's indeterminate credal
state must always be vacuous. Booleans must deny at least one of the
premises to avoid the conclusion. Joyce, for example, denies White's
CGT (see \scite{8}{joyce10}{289}), but then he continues to make
implicit use of it when he repeatedly complains that sharp credences
\qeins{ignore a vast number of possibilities that are consistent with
  [the] evidence} (for example in \scite{2}{joyce05}{170}). 

When updating dilates the credal state, it appears that the prior
credal state was in some sense incorrect and did not properly reflect
the state of the world, even though it properly reflected the
epistemic state of the agent. Such a divergence between the proper
reflections of epistemic state and state of the world undermines the
subjective interpretation of probabilities at the heart of Bayesian
epistemology. Credences are not knowledge claims about the world, but
represent uncertainty in the agent's epistemic state. Yet an
indeterminate credal state sits uncomfortably and blurs the line
between the two (for two rigorous attempts to reconcile traditional
full belief epistemology and formal partial belief epistemology see
\scite{7}{moss13}{}; and \scite{7}{spohn12}{}, especially chapter 10).

Matthias Perth, an Austrian civil servant, observes the Bavarian king
at the Congress of Vienna and writes in his diary that the king
\qeins{appears to be a man between 45 and 47 years old} (see
\texttt{http://www.das-perth-projekt.at}). If Perth then learns that
the king was 49 years old, he must revise, not just update, his
earlier judgment. The appropriate formal instrument is belief
revision, not probability update, and whoever wants to use it leaves
what Levi calls the ample bosom of Mother Bayes (see
\scite{8}{levi85}{392}), unless there is a substantial reconciliation
project between formal and traditional epistemology operating in the
background, which I do not see in the literature defending
indeterminacy. If Perth had wanted to express a sharp credence, he
would have said, \qeins{my best guess is that the king is 46 years
  old,} and the information that the king was 49 would have triggered
the appropriate update, without any revision of full beliefs.

White's objection fails because dilation for indeterminate credences
is in principle not any more surprising than a piece of information
that increases the Shannon entropy of a sharp credence. It is true for
both sharp and indeterminate credences that information can make us
less certain about things. Once proponents of indeterminate credal
states have brought their house in order to accommodate White's
objection, however, they open wide the door on semantic problems. We
finally get to inquire what kind of coherence there is in defending
indeterminacy when it neither fulfills the promise of representing
evidence nor the promise of reconciling traditional full belief
\qnull{knowledge} epistemology and Bayesian partial belief
epistemology as outlined in the CGT, but only adds another
hierarchical layer of uncertainty to a numerical quantity (a sharp
credence) whose job it already is to represent uncertainty, thus
unnecessarily introducing regress problems. We will turn to these
semantic considerations now, show how they display the virtues of
sharp credences in responding to the forceful motivations for
indeterminate credal states while making those indeterminate credal
states look semantically otiose. Then we will show in the last section
how sharp credences have an elegant solution for being outperformed by
indeterminate credal states in betting scenarios, and there we will
rest our case.

\section{Semantics of Partial Belief} 
\label{SemanticsOfPartialBelief}

At the heart of my project is a proper semantic distinction between
evidence, information, and partial belief. Both sharp credences and
indeterminate credal states, within a Bayesian framework, try to
represent the uncertainty of an agent with respect to the truth of a
proposition. Indeterminate credal states have a greater ambition: they
also claim to represent properties of the evidence, such as its
weight, conflict between its constituents, or its ambiguity. 

Indeterminate credal states are suggestive of a measurement that wants
to represent numerically the mass of an object and then also make
claims about its density. With sharp credences, the semantic roles of
evidence, information, and uncertainty are appropriately
differentiated. Rational decision-making, inference, and betting
behaviour are based on sharp credences together with the evidence that
is at its foundation. Information represents evidence, and sharp
credences represent uncertainty. Measurement is in any case a
misleading analogy for credences. Measurements are always imprecise.
Epistemic credences, however, are not measurements, especially not of
objective chances. They represent uncertainty. They are more like
logical truth values than they are like measurements.
\scite{8}{levi85}{407}, makes this category mistake, whereas
\scite{8}{walley91}{249} is clear on the difference.

It is a slippery affair to determine what evidence is, which I will
leave to others. My claim is that a rational agent is someone who can
distill information from evidence which places numerically precise
constraints on relatively prior probability distributions, which then
can be updated to form posterior probability distributions and the
credences associated with them. Note that relatively prior probability
distributions are not ignorance priors or non-informative priors,
which I would call absolutely prior probability distributions. I have
no answers where absolutely prior probability distributions come from,
how they are justified, or in what sense they are objective. I am not
concerned whether all rational agents, if they have the same evidence,
should arrive at the same credal states; or even if they should all
update a given relatively prior credal state to the same posterior
credal state, if they have the same evidence. Sometimes there may be
different ways to translate or interpret evidence into information.

It is important not to confuse the claim that it is reasonable to hold
both $X$ and $Y$ with the claim that it is reasonable to hold either
$X$ (without $Y$) or $Y$ (without $X$). It is the reasonableness of
holding $X$ and $Y$ concurrently that is controversial, not the
reasonableness of holding $Y$ (without holding $X$) when it is
reasonable to hold $X$. We will later talk about anti-luminosity, the
fact that a rational agent may not be able to distinguish
psychologically between a 54.9 cent bet on an event and a 45.1 bet on
its negation. She must reject one of them not to incur sure loss, so
proponents of indeterminacy suggest that she choose one of them freely
without being constrained by her credal state or reject both of them.
I claim that a sharp credence will make a recommendation between the
two so that only one of the bets is rational given her particular
credence, but that does not mean that another sharp credence which
would give a different recommendation may not also be rational for her
to have.

I am sympathetic to the viewpoint that once a rational agent has a
relatively prior credal state and has formalized her evidence in terms
of information, then the probability distributions forming her
posterior credal state should be unique. Joyce, with his
\qnull{committee member} approach, shows how this kind of updating can
be done for indeterminate credal states (see \scite{8}{joyce10}{288};
also \scite{8}{bradleysteele13}{6}).

If you are willing to follow me so far (assuming the Bayesian core
commitments of credal states as representing the epistemic state of
the agent in terms of subjective probabilities, which are updated
using standard conditioning where it can be applied), there are two
approaches you can take towards the indeterminacy question, both of
them involving two levels. On the one hand, you can have partial
beliefs about how a parameter is distributed and then use Lewis's
summation formula (see \scite{8}{lewis81}{266f}) to integrate over
them and condense them to a sharp credence. On the other hand, you can
try to represent your uncertainty about the distribution of the
parameter by an indeterminate credal state. I want to show that the
first approach can be brought in line with the forceful motivations we
listed in the previous section to introduce indeterminate credal
states, as long as we do not require that a sharp credence represent
the evidence as well as the epistemic state of uncertainty in the
agent.

One of Joyce's complaints, for example, is that a sharp credence of
$0.5$ for a \textsc{coin} contains too much information if there is
little or no evidence that the \textsc{coin} is fair. This complaint,
of course, is only effective if we make a credence say something about
the evidence. Joyce himself, however, admits that indeterminate credal
states cannot represent the evidence without violating the reflection
principle due to White's dilation problem. He is quite clear that the
same indeterminate credal state can represent different evidential
scenarios (see, for example, \scite{8}{joyce10}{302}). 

In any case, Walley's and Joyce's claim that indeterminate credal
states are less informative than sharp credences has no foundation in
information theory (see \scite{8}{walley91}{34}; and
\scite{8}{joyce10}{311} for examples, but this attitude is passim). To
compare indeterminate credal states and sharp credences
informationally, we would need a non-additive set function obeying
Shannon's axioms for information. This is a non-trivial task. I have
not succeeded solving it, but I am not at all convinced that it will
result in an information measure which assigns more information to a
sharp credence such as $\{0.5\}$ than to an indeterminate credal state
such as $\{x|1/3\leq{}x\leq{}2/3\}$ for a binomial random variable.

Augustin recognizes the problem of inadequate representation long
before Joyce, with specific reference to indeterminate credal states:
\qeins{The imprecise posterior does no longer contain all the relevant
  information to produce optimal decisions. Inference and decision do
  not coincide any more} \scite{2}{augustin03}{41} (see also an
example for inadequate representation of evidence by indeterminate
credal states in \scite{8}{bradleysteele13}{16}). Indeterminate credal
states fare no better than sharp credences, except perhaps for the
problem that they unhelpfully mimic saying something about the
evidence that is much better said elsewhere.

Not only can we align sharp credences with the motivations to
introduce indeterminate credal states, we can also show that
indeterminate credal states perform worse semantically because they
mix evidential and epistemic metaphors in deleterious ways. Sharp
credences have one task: to represent epistemic uncertainty and serve
as a tool for updating, inference, and decision-making. They cannot
fulfill this task without continued reference to the evidence which
operates in the background. To use an analogy, credences are not
sufficient statistics with respect to updating, inference, and
decision-making. What is remarkable about Joyce's response to White's
dilation problem is that Joyce recognizes that indeterminate credal
states are not sufficient statistics either. But this means that they
fail at the double task which has been imposed on them: to represent
both epistemic uncertainty and the evidence.

\begin{quotex}
  \textbf{Example 1: Aggregating Expert Opinion} You have no
  information whether it will rain tomorrow $R$ or not except the
  predictions of two weather forecasters. One of them forecasts 0.3 on
  channel GPY, the other 0.6 on channel QCT. You consider the QCT
  forecaster to be significantly more reliable, based on past
  experience.
\end{quotex}

An indeterminate credal state corresponding to this situation may be
$[0.3,0.6]$ (see \scite{8}{walley91}{214}), but it will have a
difficult time representing the difference in reliability of the
experts. A sharp credence of $P(R)=0.55$, for example, does the right
thing. Such a credence says nothing about any beliefs that the
objective chance of $R$ is $x\in{}I$ or restricted to $X\subseteq{}I$
(where $I$ is the unit interval), but it accurately reflects the
degree of uncertainty that the rational agent has over the various
possibilities. Beliefs about objective chances make little sense in
many situations where we have credences, since it is doubtful even in
the case of rain tomorrow that there is an urn of nature from which
balls are drawn. What is really at play is a complex interaction
between epistemic states (for example, experts evaluating
meteorological data) and the evidence which influences them. 

% (White has some fun with the urn of nature idea and its involvement in
% the CGT in \scite{8}{white10}{171}, featuring a big urn with all
% formal epistemologists in it, which is vigorously shaken and Branden
% Fitelson picked out.)

A sharp credence is usually associated with distributions over
chances, while an indeterminate credal state puts chances in sets
where they all have an equal voice. This may also be at the bottom of
Susanna Rinard's objection (see \scite{8}{white10}{184}) that Joyce's
committee members are all equally enfranchised and so it is not clear
how extremists among them could not always be replaced by even greater
extremists even after updating on evidence which should serve to
consolidate indeterminacy. Joyce has a satisfactory response to this
objection (see \scite{8}{joyce10}{291}), but I do not see how the
response addresses the problem of aggregating expert opinion. More
generally, the two levels for sharp credences, representation of
uncertainty and distributions over chances, tidily differentiate
between the epistemic and the evidential dimension; indeterminate
credal states, on the other hand, just slap another level of
uncertainty on top of the uncertainty that is already expressed in the
partial belief and thus do not make the appropriate semantic
distinctions.

As we will see in the next example, it is an advantage of sharp
credences that they do not exclude objective chances, even extreme
ones, because they are fully committed to partial belief and do not
suggest, as indeterminate credences do, that there is full belief
knowledge that the objective chance is a member of a proper subset of
the possibilities.

\begin{quotex}
  \textbf{Example 2: How Precise Can a Rational Agent Reasonably Be}
  Your sharp credence for rain tomorrow, based on the expert opinion
  of channel GPY and channel QCT (you have no other information) is
  $0.55$. Is it reasonable, considering how little evidence you have,
  to reject the belief that the chance of rain tomorrow is $0.54$ or
  $0.56$; or to prefer a 54.9 cent bet on rain to a 45.1 cent bet on
  no rain?
\end{quotex}

The first question, of course, is confused, but in instructive ways (a
flagrant display of this confusion is found in
\scite{8}{hajeksmithson12}{38f}, and their doctor and time of the day
analogy). A sharp credence rejects no hypothesis about objective
chances (unlike an indeterminate credal state). It probably has a
distribution of either objective chances or subjective probabilities
operating in the background, over which it integrates to yield the
sharp credence (it would do likewise in H{\'a}jek and Smithson's
example for the prognosis of the doctor or the time of the day,
without any problems). This subjective probability distribution may
look like this:

\begin{tabular}{|lcr|} \hline
$P(\pi(R)=0.00)$ & = & $0.0001$ \\ \hline
$P(\pi(R)=0.01)$ & = & $0.0003$ \\ \hline
$P(\pi(R)=0.02)$ & = & $0.0007$ \\ \hline
$\ldots$ & & $\ldots$ \\ \hline
$P(\pi(R)=0.30)$ & = & $0.0015$ \\ \hline
$P(\pi(R)=0.31)$ & = & $0.0016$ \\ \hline
$\ldots$ & & $\ldots$ \\ \hline
$P(\pi(R)=0.54)$ & = & $0.031$ \\ \hline
$P(\pi(R)=0.55)$ & = & $0.032$ \\ \hline
$P(\pi(R)=0.56)$ & = & $0.054$ \\ \hline
$\ldots$ & & $\ldots$ \\ \hline
\end{tabular}

This probability distribution is condensed by Lewis's summation
formula to a sharp credence:

\begin{equation}
  \label{eq:s2}
  P(R)=\int_{0}^{1}\zeta{}P(\pi(R)=\zeta)\,d\zeta
\end{equation}

There are more intriguing semantic questions here: what is $\pi(R)$,
the objective chance that it rains tomorrow, and how do we get to use
$P(\pi(R)=\zeta)$ in our calculation of $P(R)$ without begging the
question. Lewis's 1981 paper \qeins{A Subjectivist's Guide to
  Objective Chance} remains the gold standard in addressing these
questions, and I will no longer pursue them here. The point is that we
have properly separated the semantic dimensions: the partial belief
epistemology deals with sharp credences and how they represent
uncertainty and serve as tools in inference, updating, and decision
making; while Lewis's Humean speculations and his interpretation of
the principal principle cover the relationship between subjective
probabilities and objective chance.

Indeterminate credal states, by contrast, mix these semantic
dimensions so that in the end we get a muddle where a superficial
reading of indeterminacy suddenly follows a converse principal
principle of sorts, namely that objective chances are constrained by
the credences of a rational agent (Lewis actually talks about such a
converse, but in completely different and epistemologically more
intelligible terms, see \scite{8}{lewis81}{289}). Sharp credences are
more, not less, permissive with respect to objective chances operating
externally (compared to the internal epistemic state of the agent,
which the credence reflects). By the principle of regularity and in
keeping with statistical practice, all objective chances as possible
states of the world are given positive probabilities, even though they
may be very small. Indeterminate credal states, on the other hand, mix
partial belief epistemology with full belief epistemology and
presumably exclude objective chances which lie outside the credal
state from consideration because they are fully known not to hold (see
\scite{8}{levi81}{540}, \qeins{inference derives credal probability
  from knowledge of the chances of possible outcomes}).

The second question is also instructive: why would we prefer a 54.9
cent bet on rain to a 45.1 cent bet on no rain, given that we do not
possess the power of descrimination between these two bets? The answer
to this question ties in with the issue of incomplete preference
structure referenced above as motiviation (B) for indeterminate credal
states. 

\begin{quotex}
  It hardly seems a requirement of rationality that belief be precise
  (and preferences complete); surely imprecise belief (and
  corresponding incomplete preferences) are at least rationally
  permissible. \scite{3}{bradleysteele13}{2}
\end{quotex}

In personal communication, Yang Liu at Columbia University posed this
problem to me more forcefully: the development of representation
theorems beginning with Frank Ramsey (followed by increasingly more
compelling representation theorems in \scite{7}{savage54}{}; and
\scite{7}{jeffrey65}{}; and numerous other variants in contemporary
literature) puts the horse before the cart and bases probability and
utility functions of an agent on her preferences, not the other way
around. Once completeness as an axiom for the preferences of an agent
is jettisoned, indeterminacy follows automatically. Indeterminacy may
thus be a natural consequence of the proper way to think about
credences in terms of the preferences that underlie them.

In response, preferences may very well logically and psychologically
precede an agent's probability and utility functions, but that does
not mean that we cannot inform the axioms we use for a rational
agent's preferences by undesirable consequences downstream.
Completeness may sound like an unreasonable imposition at the outset,
but if incompleteness has unwelcome semantic consequences for
credences, it is not illegitimate to revisit the issue. Timothy
Williamson goes through this exercise with vague concepts, showing
that all upstream logical solutions to the problem fail and that it
has to be solved downstream with an epistemic solution (see
\scite{7}{williamson96}{}). Vague concepts, like sharp credences, are
sharply bounded, but not in a way that is luminous to the agent (for
anti-luminosity see chapter 4 in \scite{7}{williamson00}{}).
Anti-luminosity answers the original question: the rational agent
prefers the 54.9 cent bet on rain to a 45.1 cent bet on no rain based
on her sharp credence without being in a position to have this
preference necessarily or have it based on physical or psychological
ability (for the analogous claim about knowledge see
\scite{8}{williamson00}{95}).

In a way, advocates of indeterminacy have solved this problem for us.
There is strong agreement among most of them that the issue of
determinacy for credences is not an issue of elicitation (sometimes
the term \qnull{indeterminacy} is used instead of \qnull{imprecision}
to underline this difference; see \scite{8}{levi85}{395}, but also
Walley and Joyce passim, even though Walley prefers to use the term
\qnull{imprecision,} strongly in the sense, however, that they are not
an elicitation issue). The appeal of preferences is that we can elicit
them more easily than assessments of probability and utility
functions. 

The indeterminacy issue has been raised to the probability level (or
moved downstream) by indeterminacy advocates themselves who feel
justifiably uncomfortable with an interpretation of their theory in
behaviourist terms. So it shall be solved there, and this paper makes
an appeal to reject indeterminacy on this level. The solution then has
to be carried upstream (or lowered to the logically more basic level
of preferences), where we recognize that completeness for preferences
is after all a desirable axiom for rationality. Ironically, Isaac Levi
seems to agree with me on this point: when he talks about
indeterminacy, it proceeds from the level of probability judgment to
preferences, not the other way around (see \scite{8}{levi81}{533}).

At the end of this section, I want to give a few more examples where
indeterminate credal states give results that are counter-intuitive.
(1) E.T. Jaynes describes an experiment with monkeys filling an urn
randomly with balls from another urn, for which sampling provides no
information and so makes updating vacuous (see
\scite{8}{jaynesbretthorst03}{160}). Here is a variant of this
experiment for which a sharp credence provides a more compelling
result than the associated indeterminate credal state: Let urn $A$
contain 4 balls, two red and two black. A monkey randomly fills urn
$B$ from urn $A$ with two balls. We draw from urn $B$. The sharp
credence of drawing a red ball is $0.5$, following Lewis's summation
formula for the different combinations of balls in urn $B$. I find
this solution more intuitive in terms of further inference, decision
making, and betting behaviour than a credal state of $\{0,1/2,1\}$,
since this indeterminate credal state would licence an exorbitant bet
in favour of one colour, for example one that costs \$9,999 and pays
\$10,000 if red is drawn and nothing if black is drawn.
%  (the scenario can be tweaked,
% of course, to make the indeterminate credal state
% $\{\delta,1/2,1-\delta\}$ for very small $\delta$).

(2) Peter Walley maintains that for the Monty Hall problem and the
Three Prisoners problem, the probabilities of a rational agent should
dilate rather than settle on the commonly accepted solutions. Consider
the three prisoners problem. Prisoner $X_{1}$ knows that two out of
three prisoners ($X_{1},X_{2},X_{3}$) will be executed and one of them
pardoned. He asks the warden of the prison to tell him the name of
another prisoner who will be executed, hoping to gain knowledge about
his own fate. When the warden tells him that $X_{3}$ will be executed,
$X_{1}$ erroneously updates his probability of pardon from $1/3$ to
$1/2$, since either $X_{1}$ or $X_{2}$ will be spared. I will not go
into any detail here, but I think there is a compelling case here for
standard conditioning and the result that the chances of pardon for
prisoner $X_{1}$ are unchanged after the update (see
\scite{8}{lukits14}{1421f}). Walley's dilated solution would give
prisoner $X_{1}$ hope on the doubtful possibility (and unfounded
assumption) that the warden might prefer to provide $X_{3}$'s name in
case prisoner $X_{1}$ was pardoned.

This example brings an interesting issue to the forefront. Sharp
credences often reflect independence of variables where such
independence is unwarranted. Proponents of indeterminacy (more
specifically, detractors of the principle of indifference or the
principle of maximum entropy, principles which are used to generate
sharp credences for rational agents) tend to point this out gleefully.
They prefer to dilate over the possible dependence relationships
(independence included). White's dilation problem is an instance of
this. I detect a fallacy in the argument for indeterminate credal
states, illustrated by the three prisoners problem. The probabilistic
independence of sharp credences does not imply independence of
variables, but only that it is unknown whether there is dependence,
and if yes, whether it is correlation or inverse correlation. 

The fallacy is that it is illegitimate to conclude from probabilistic
independence to causal independence, even though the converse
conclusion is legitimate. In the Three Prisoners problem, there is no
evidence about the degree or the direction of the dependence, and so
prisoner $X_{1}$ should take no comfort in the information that he
receives. The prisoner's probabilities will reflect probabilistic
independence, but make no claims about causal independence. Walley has
terribly unkind things to say about sharp credences and their ability
to respond to evidence (for example that their \qeins{inferences
  rarely conform to evidence}, see \scite{8}{walley91}{396}), but in
this case it appears to me that they outperform the indeterminate
approach.

(3) A while ago, the mathematician Carl Wagner published a paper in
which he proposed a natural generalization of Jeffrey Conditioning
(see \scite{7}{wagner92}{}). Since the principle of maximum entropy is
already a generalization of Jeffrey Conditioning, the question
naturally arises whether the two generalizations agree. Wagner makes
the case that they do not agree and deduces that the principle of
maximum entropy is sometimes an inappropriate updating mechanism, in
line with many earlier criticisms of the principle of maximum entropy
(see \scite{7}{fraassen81}{}; \scite{7}{shimony85}{};
\scite{7}{skyrms87updating}{}; and, later on,
\scite{7}{grovehalpern97}{}). What is interesting about this case is
that Wagner uses indeterminate credal states for his deduction, so
that even if you agree with his natural generalization of Jeffrey
Conditioning (which I find plausible), the inconsistency with the
principle of maximum entropy can only be inferred assuming
indeterminate credal states. Wagner is unaware of this, and I am
showing in another paper (in process) how on the assumption of sharp
credences Wagner's generalization of Jeffrey conditioning perfectly
aligns with the principle of maximum entropy.

This, of course, will not convince a proponent of indeterminate
credences, since they are already unlikely to believe in the general
applicability of the principle of maximum entropy (just as Wagner's
argument is unlikely to convince a proponent of the principle of
maximum entropy, since they are more likely to reject indeterminate
credal states). The battle lines are clearly drawn. Wagner's argument,
instead of undermining the principle of maximum entropy, just shows
that indeterminate credal states are as wedded to rejecting the claims
of the principle of maximum entropy as the principle of maximum
entropy is wedded to sharp credences. 

Being sympathetic to the principle of maximum entropy, I want to
point out, however, that endorsement of indeterminate credal states
therefore implies that there are situations of probability update in
which the posterior probability distribution is more informative than
(in my opinion) it ought to be. Indeterminate credences violate the
relatively natural intuition that we should not gain information from
evidence when a less informative updated probability will do the job
of responding to the evidence. This is not a strong argument in favour
of sharp credences. The principle of maximum of entropy has received a
thorough bashing in the last thirty years. I consider it to be much
easier to convince someone to reject indeterminate credal states on
independent (semantic) grounds than to convince them to give the
principle of maximum entropy a second chance. But the section on
semantics comes to an end here, and we want to proceed to the
intriguing issue of who does better in betting situations:
indeterminate credal states or sharp credences.

\section{Evidence Differentials and Cushioning Credences} 
\label{WalleysWorldCupWoes}

I have given away the answer already in the introduction:
indeterminate credal states do better. It is surprising that, except
for a rudimentary allusion to this in Walley's book, no proponent of
indeterminate credal states has caught on to this yet. After I found
out that agents with indeterminate credal states do better betting on
soccer games, I let player $X$ (who uses sharp credences) and player
$Y$ (who uses indeterminate credal states) play a more basic betting
game. An $n$-sided die is rolled (by the computer). The die is fair,
unbeknownst to the players. Their bets are randomly and uniformly
drawn from the simplex for which the probabilities attributed to the
$n$ results add up to 1. Player $Y$ also surrounds her credences with
an imprecision uniformly drawn from the interval $(0,y)$. I used
Walley's pay off scheme (see \scite{8}{walley91}{632}) to settle the
bets.

Here is an example: let $n=2$, so the die is a fair \textsc{coin}.
$X$'s and $Y$'s bets are randomly and uniformly drawn from the line
segment from $(0,1)$ to $(1,0)$ (these are two-dimensional Cartesian
coordinates), the two-dimensional simplex (for higher $n$, the simplex
is a pentatope generalized for $n$ dimensions with side length
$2^{1/2}$). The bets may be $(0.21,0.79)$ for player $X$ and
$(0.35\pm{}0.11,0.65\pm{}0.11)$ for player $Y$, where
the indeterminacy $\pm{}0.11$ is also randomly and uniformly
drawn from the imprecision interval $(0,y)\subseteq(0,1)$. The first
bet is on $H$, and player $Y$ is willing to pay $22.5$ cents for it,
while $X$ is willing to pay $77.5$ cents against it. The second bet is
on $T$ (if $n>2$, there will not be the same symmetry as in the
\textsc{coin} case between the two bets), for which player $X$ is
willing to pay $77.5$ cents, and against which player $Y$ is willing
to pay $22.5$ cents. Each bet pays \$1 if successful. Often, $Y$'s
credal state will overlap with $X$'s sharp credence so that there will
not be a bet.

Here is a table of the results. Each result is based on 100,000 die
rolls. The second column shows the mean gain for player $X$, the third
column shows the standard deviation, the fourth column shows the
percentage of bets which are called off. The table is for $y=1$.

\begin{tabular}{|l|r|r|r|}
  \hline
$n=2$ & 0.14 & 18.8 & 25.2 \\ \hline
$n=3$ & -1.67 & 13.7 & 39.1 \\ \hline
$n=4$ & -1.73 & 10.4 & 48.1 \\ \hline
$n=5$ & -1.39 & 8.2 & 54.5 \\ \hline
$n=6$ & -1.22 & 6.6 & 59.1 \\ \hline
$n=7$ & -1.01 & 5.5 & 62.8 \\ \hline
\end{tabular}

Here are the results for a player $Y$ who is not as generous with
indeterminacy, $y=0.3$ (note that fewer bets are called off).

\begin{tabular}{|l|r|r|r|}
  \hline
$n=2$ & 5.82 & 28.0 & 0.0 \\ \hline
$n=3$ & -0.55 & 21.8 & 0.0 \\ \hline
$n=4$ & -2.32 & 17.0 & 0.0 \\ \hline
$n=5$ & -2.89 & 13.7 & 0.4 \\ \hline
$n=6$ & -2.74 & 11.3 & 1.4 \\ \hline
$n=7$ & -2.52 & 9.6 & 3.2 \\ \hline
\end{tabular}

We should get similar results if we do this analytically instead of
using computer simulation. I will pursue this further for the final
version of the paper. The math is not complicated, but unwieldy. The
following expression yields the expected gain for player $X$:

\begin{eqnarray}
  \label{eq:s3}
EX =
\frac{p_{x}}{n}\left(\sum_{j=0}^{n-1}\int_{0}^{1}\int_{0}^{x}\int_{0}^{x-y}\sum_{k=0}^{n-1}g(x,y,s,k,j)\,ds\,d\upsilon(y)\,d\xi(x)\right)+
\notag \\
\frac{p_{y}}{n}\left(\sum_{j=0}^{n-1}\int_{0}^{1}\int_{0}^{1}\int_{0}^{y-s}\sum_{k=0}^{n-1}g(x,y,s,k,j)\,d\xi(x)\,d\upsilon(y)\,ds\right)
\end{eqnarray}

where $p_{x}$ and $p_{y}$ are the respective probabilities that $X$
and $Y$ win a bet and $g$ is $X$'s gain given $X$'s credence $x$ and
$Y$'s credal state $y\pm{}s$ on roll $j$, given result $k$. $\xi(x)$
and $\upsilon(y)$ are the distributions of the credences given our
method of simplex point picking (for these distributions, one must use
the Cayley-Menger Determinant to find out the volume of generalized
pentatopes involved). The computer simulation clearly shows that
except for $n=2$, the \textsc{coin} toss, player $Y$ does better. A
defence of sharp credences for rational agents needs to have an
explanation for this. We will call it partial belief cushioning, which
is based on an evidence differential between the bettors.

In many decision-making context, we do not have the luxury of calling
off the bet. We have to decide one way or another. This is a problem
for indeterminate credal states, as their proponents have to find a
way to decide without receiving clear instructions from the credal
state. Proponents have addressed this point extensively (see for
example \scite{8}{joyce10}{311ff}; for an opponent's view of this see
\scite{8}{elga10}{6ff}). The problem for sharp credences arises when
bets are noncompulsory, for then the data above suggest that agents
holding indeterminate credal states do better. Often, decision-making
happens as betting vis-{\`a}-vis uninformed nature or opponents which
are at least as uninformed as the rational agent. Sometimes, however,
bets are offered by better informed or potentially better informed
bookies. In this case, even an agent with sharp credences must cushion
her credences and is better off by rejecting bets that look attractive
in terms of her partial beliefs. 

Here are a few examples: even if I have little evidence on which to
base my opinion, someone may force me to either buy Coca Cola shares
or short them, and so I have to have a share price $p$ in mind that I
consider fair. I will buy Coca Cola shares for less than $p$, and
short them for more than $p$, if forced to do one or the other. This
does not mean that it is now reasonable for me to go (not forced by
anyone) and buy Coca Cola shares for $p$. It may not even be
reasonable to go (not forced by anyone) and buy Coca Cola share for
$p-\delta$ with $\delta{}>0$. 

It may in fact be quite unreasonable, since there are many players who
have much better evidence than I do and will exploit my ignorance. I
suspect that most lay investors in the stock market make this mistake:
even though they buy and sell stock at prices that seem reasonable to
them, professional investors are much better and faster at exploiting
arbitrage opportunities and more subtle regularities. If indices rise,
lay investors will make a little less than their professional
counterparts; and when they fall, lay investors lose a lot more. In
sum, unless there is sustained growth and everybody wins, lay
investors lose in the long term.

A case in point is the U.S. Commodity Futures Trading Commission's
crackdown on the online prediction market Intrade. Intrade offered
fair bets for or against events of public significance, such as
election results or other political events which had clear yes-or-no
outcomes. Even though the bets were all fair and Intrade only received
a small commission on all bets, and even though Intrade's predictions
were remarkably accurate, the potential for professional arbitrageurs
was too great and the CFTC shut Intrade down (see
\texttt{https://www.intrade.com}). 

None of this, however, should stand in the way of holding a sharp
credence or having in mind a fair share price for a company, even if
the evidence is dim. The evidence determines for a rational agent the
partial beliefs over possible states of the world operating in the
background of these sharp credences and fair share prices. The better
the evidence, the more pointed the distributions of these partial
beliefs will be and the more willing the rational agent will be to
enter a bet. The mathematical decision rule will be based on the
underlying distribution of the partial beliefs, not only on the sharp
credence. As we have stated before, the sharp credence is not a
sufficient statistic for decision-making, inference, or betting
behaviour; and neither is an indeterminate credal state. If a rational
agent perceives an evidence differential and lends some belief to the
proposition that the bet is offered by someone who has more evidence
about the outcome of an event than she does, then it is likely that
the rational agent will update her sharp credence, as she would do if
she were informed of another source of expert opinion.

In conclusion, the rational agent with a sharp credence has resources
at her disposal to use just as much differentiation with respect to
accepting and rejecting bets as the agent with indeterminate credal
states. Often (if she is able to and especially if the bets are
offered to her by a better-informed agent), she will reject both of
two complementary bets, even when they are fair. On the one hand, any
advantage that the agent with an indeterminate credal state has
towards her can be counteracted based on her distribution over partial
beliefs that she has with respect to all possibilities. On the other
hand, the agent with indeterminate credal states suffers from a
semantic mixing of metaphors between evidential and epistemic
dimensions that puts her at a real disadvantage in terms of
understanding the sources and consequences of her knowledge and
uncertainties.

\section{References} 
\label{References}

% \nocite{*} 
\bibliographystyle{ChicagoReedweb} 
\bibliography{bib-7293}

\end{document} 
