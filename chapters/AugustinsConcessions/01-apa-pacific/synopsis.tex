%%%%%%%%%%%%%%%%%%%%%%%%%%%%%%%%%%%%%%%%%%%%%%%%%%%%%%%%%%%%%%%%%%%%%%%%%%%%%
% This is NOT the official file. The official text of the synopsis is
% now in sonk-apa.tex
%%%%%%%%%%%%%%%%%%%%%%%%%%%%%%%%%%%%%%%%%%%%%%%%%%%%%%%%%%%%%%%%%%%%%%%%%%%%%

\documentclass[11pt]{article}
\usepackage{october}

\begin{document}

\title{Synopsis for ``Semantics of Not Knowing''}
\author{Stefan Lukits}
\date{}
\maketitle

Traditional Bayesian epistemology used to include the requirement for
a rational agent to hold a sharp credence function. It has recently
become popular to drop this requirement. There are now Bayesian
theories which permit a rational agent to hold indeterminate credal
states based on incomplete or ambiguous evidence. I will refer to
Bayesians who continue to adhere to the classical theory of sharp
credences for rational agents as \qnull{Laplaceans} (e.g.\ Adam Elga
and Roger White). I will refer ot Bayesians who do not believe in the
requirement for a rational agent to hold sharp credences as
\qnull{Booleans} (e.g.\ Peter Walley and James Joyce; see Boole, 1854,
chapters 16--21, for alternative methods to the ones suggested by
Laplace which resulted in imprecise epistemic probabilities).

After describing the appeal of indeterminacy and showing how
contemporary Laplacean objections fail, I will point to more serious
failings of indeterminacy in semantic terms and show how a proper
semantics of not knowing, which we could also call a semantics of
partial belief, solves the problems for sharp credences that Booleans
seek to address by abolition. There is a sense in which, by linking
knowledge of chances to its reflection in credences, Booleans seek to
reconcile traditional knowledge epistemology concerned with full
belief and formal epistemology concerned with partial belief. There
are other more recent reconciliation projects (see Spohn, 2012; and
Moss, 2013), but if my paper is correct then the Boolean approach will
not contribute to this reconciliation because it mixes full belief and
partial belief metaphors in ways that are semantically problematic.

A sharp credence, as much as the term suggests precision and a measure
of certainty, is a representation of an epistemic state chracterized
by uncertainty and lack of information. Importantly, it does not
represent the evidence which informs the epistemic state, and it makes
no claim of such a representation. Indeterminate credal states are
often lauded as doing much better representing uncertainty together
with the evidence that constrains it, but they can no more give an
adequate representation of evidence than sharp credences. Thus it is
legitimate to have a $0.5$ sharp credence in heads for a coin of whose
bias we are completely ignorant; for a coin whose fairness is
supported by a lot of evidence; and even for a coin about whose bias
we know that it is either 1/3 or 2/3 for heads.

One potential Boolean claim is that agents who use indeterminate
credal states do better than Laplaceans when they bet on the truth of
events for which they have varying degrees of evidence. Peter Walley
gives an example where a Laplacean does much worse at predicting
soccer games than Boolean peers. I show that the result is due to an
unreasonable restriction on the betting behaviour of the Laplacean.
Once this restriction is lifted, Laplaceans do just as well as
Booleans, except that they are not tangled in the semantic problem of
the double task, where indeterminate credal states are supposed to
reflect both the uncertainty of an agent and other properties of her
evidence.

I will present several examples where this double task stretches
indeterminate credal states to the limits of plausibility, for example
when they need to aggregate expert opinion or account for dilation.
Joyce's idea that credences can represent balance, weight, and
specificity of the evidence is inconsistent with the use of
indeterminacy (and Joyce himself, in response to the dilation problem,
gives the argument why this is the case). The implicit Boolean claim
that evidence can be recovered from indeterminate credal states is
inconsistent with an effective Boolean answer to the dilation problem.

The Laplacean approach of assigning subjective probabilities to
distributions and then aggregating them by David Lewis's summation
formula into a single precise credence function is semantically tidy
and shares many of the formal virtues of Boolean theories. If the bad
taste about numerical precision in our fuzzy, nebulous world lingers,
I can point to philosophical projects in other domains where the
concepts we use are sharply bounded, even though our ability to
conceive of those sharp boundaries or know them is limited, for
example Timothy Williamson's work on vagueness and knowledge as a
mental state.

\end{document} 
