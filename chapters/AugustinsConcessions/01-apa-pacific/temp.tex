The first question is confused in instructive ways. A sharp credence
rejects no hypothesis about objective chances (unlike an instate). It
often has a subjective probability distribution operating in the
background, over which it integrates to yield the sharp credence. The
subjective probability distribution is condensed by Lewis' summation
formula to a sharp credence, without being reduced to it. Lewis' 1981
paper \qeins{A Subjectivist's Guide to Objective Chance} addresses the
question of the relationship between credence, subjective probability,
and objective chance. A formal epistemologist tries to separate
properly the semantic dimensions. and that the Laplacean approach is
not a second order probability approach. The partial belief
epistemology deals with sharp credences and how they represent
uncertainty and serve as a tool in inference, updating, and decision
making; while Lewis' Humean speculations and his interpretation of the
principal principle cover the relationship between subjective
probabilities and objective chance.
