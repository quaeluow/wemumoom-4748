% This is the official version. For shorter versions, see directory
% diff.

% http://tinyurl.com/nrhqdxg

% What exactly is Linda's evidence differential compared to
% Betsy -- the analytic expression may help.

\documentclass[11pt]{article}
\usepackage{october}
% For BJPS
% \hyphenpenalty=10000
% \hbadness=10000

\begin{document}
% For BJPS
% \raggedright
% \onehalfspacing
% \doublespacing

\title{Augustin's Concessions: A Problem for Indeterminate Credal States}
\author{Stefan Lukits}
\date{}
\maketitle

The claim is that rational agents are subject to a norm requiring
sharp credences. I defend this claim in spite of the initially
promising features of indeterminate credal states (from now on
instates) to address problems which sharp credences have as they
reflect the evidence at the foundation of a doxastic state.
Traditionally, Bayesians have maintained that a rational agent, when
holding a credence, holds a sharp credence. It has recently become
popular to drop the requirement for credence functions to be sharp.
There are now Bayesians who permit a rational agent to hold instates
based on incomplete or ambiguous evidence. I will refer to Bayesians
who continue to adhere to the classical theory of sharp credences for
rational agents as \qnull{Laplaceans} (e.g.\ Adam Elga and Roger
White). I will refer to Bayesians who do not believe that a rational
agent's credences are sharp as \qnull{Booleans} (e.g.\ Peter Walley
and James Joyce).\fcut{1}

I will exclusively refer to indeterminate credal states (abbreviated
\qnull{instates}, sometimes terminology such as \qnull{imprecise} or
\qnull{mushy} credences is used as well) and mean by them a set of
sharp credence functions (which some Booleans require to be convex)
which it may be rational for an agent to hold within an otherwise
orthodox Bayesian framework.\bcut{1}

When we first hear of the advantages of instates, two of them sound
particularly persuasive.

\begin{itemize}
\item \textsc{range} Instates represent the possibility range for
  objective chances.
\item \textsc{incomplete} Instates represent incompleteness or ambiguity
  of the evidence.\bcut{3}
\end{itemize}

Here are some examples. Let a \textit{coin}$_{\mbox{\tiny{x}}}$ be a
Bernoulli generator that produces successes and failures with
probability $p_{\mbox{\tiny{x}}}$ for success, labeled
$H_{\mbox{\tiny{x}}}$, and $1-p_{\mbox{\tiny{x}}}$ for failure,
labeled $T_{\mbox{\tiny{x}}}$. Physical coins may serve as Bernoulli
generators, if we are willing to set aside that most of them are
approximately fair.

\begin{quotex}
  \beispiel{Range}\label{ex:range} Bob has two Bernoulli Generators in
  his lab, \textit{coin}$_{\mbox{\tiny{i}}}$ and
  \textit{coin}$_{\mbox{\tiny{ii}}}$. Bob has a database of
  \textit{coin}$_{\mbox{\tiny{i}}}$ results and concludes on excellent
  evidence that \textit{coin}$_{\mbox{\tiny{i}}}$ is fair. Bob has no
  evidence about the bias of \textit{coin}$_{\mbox{\tiny{ii}}}$. As a
  Boolean, Bob assumes a sharp credence of $\{0.5\}$ for
  $H_{\mbox{\tiny{i}}}$ and an indeterminate credal state of $[0,1]$
  for $H_{\mbox{\tiny{ii}}}$. He feels bad for Larry, his Laplacean
  colleague, who cannot distinguish between the two cases and who must
  assign a sharp credence of $\{0.5\}$ for both $H_{\mbox{\tiny{i}}}$
  and $H_{\mbox{\tiny{ii}}}$.
\end{quotex}

\begin{quotex}
  \beispiel{Incomplete}\label{ex:incomp} Bob has another Bernoulli
  Generator, \textit{coin}$_{\mbox{\tiny{iii}}}$, in his lab. His
  graduate student has submitted \textit{coin}$_{\mbox{\tiny{iii}}}$
  to countless experiments and emails Bob the resulting bias, but
  fails to include whether the bias of $2/3$ is in favour of
  $H_{\mbox{\tiny{iii}}}$ or in favour of $T_{\mbox{\tiny{iii}}}$. As
  a Boolean, Bob assumes an indeterminate credal state of $[1/3,2/3]$
  (or $\{1/3,2/3\}$, depending on whether convexity is required) for
  $H_{\mbox{\tiny{iii}}}$. He feels bad for Larry who must assign a
  sharp credence of $\{0.5\}$ for $H_{\mbox{\tiny{iii}}}$ when Larry
  concurrently knows that his credence gets the bias wrong.
\end{quotex}

Against the force of \textsc{range} and \textsc{incomplete}, I
maintain that the Laplacean approach of assigning subjective
probabilities to partitions of the event space (e.g.\ objective
chances) and then aggregating them by David Lewis' summation formula
into a single precise credence function is conceptually tidy and
shares many of the formal virtues of Boolean theories. To put it
provocatively, this paper defends a $0.5$ sharp credence in heads in
all three cases: for a coin of whose bias we are completely ignorant;
for a coin whose fairness is supported by a lot of evidence; and even
for a coin about whose bias we know that it is either 1/3 or 2/3 for
heads.

\end{document} 
