% This is the official version. For shorter versions, see directory
% diff.

% http://tinyurl.com/nrhqdxg

% What exactly is Linda's evidence differential compared to
% Betsy -- the analytic expression may help.

\documentclass[11pt]{article}
\usepackage{october}
% For BJPS
% \hyphenpenalty=10000
% \hbadness=10000

\begin{document}
% For BJPS
% \raggedright
\onehalfspacing
% \doublespacing

\title{Augustin's Concessions: A Problem for Indeterminate Credal States}
\author{Stefan Lukits}
\date{}
\maketitle

Wordcount: 2985 (Monterey Online PDF Word Count Tool, excluding
bibliography and math symbols)

% \begin{abstract} 
% \end{abstract}

\section{Introduction}
\label{Introduction}

The claim is that rational agents are subject to a norm requiring
sharp credences. I defend this claim in spite of the initially
promising features of indeterminate credal states (from now on
instates) to address problems which sharp credences have as they
reflect the evidence at the foundation of a doxastic state.
Traditionally, Bayesians have maintained that a rational agent, when
holding a credence, holds a sharp credence. It has recently become
popular to drop the requirement for credence functions to be sharp.
There are now Bayesians who permit a rational agent to hold instates
based on incomplete or ambiguous evidence. I will refer to Bayesians
who continue to adhere to the classical theory of sharp credences for
rational agents as \qnull{Laplaceans} (e.g.\ Adam Elga and Roger
White). I will refer to Bayesians who do not believe that a rational
agent's credences are sharp as \qnull{Booleans} (e.g.\ Peter Walley
and James Joyce).\fcut{1}

I will exclusively refer to indeterminate credal states (abbreviated
\qnull{instates}, sometimes terminology such as \qnull{imprecise} or
\qnull{mushy} credences is used as well) and mean by them a set of
sharp credence functions (which some Booleans require to be convex)
which it may be rational for an agent to hold within an otherwise
orthodox Bayesian framework.\bcut{1}

When we first hear of the advantages of instates, two of them sound
particularly persuasive.

\begin{itemize}
\item \textsc{range} Instates represent the possibility range for
  objective chances.
\item \textsc{incomplete} Instates represent incompleteness or ambiguity
  of the evidence.\bcut{3}
\end{itemize}

Here are some examples. Let a \textit{coin}$_{\mbox{\tiny{x}}}$ be a
Bernoulli generator that produces successes and failures with
probability $p_{\mbox{\tiny{x}}}$ for success, labeled
$H_{\mbox{\tiny{x}}}$, and $1-p_{\mbox{\tiny{x}}}$ for failure,
labeled $T_{\mbox{\tiny{x}}}$. Physical coins may serve as Bernoulli
generators, if we are willing to set aside that most of them are
approximately fair.

\begin{quotex}
  \beispiel{Range}\label{ex:range} Bob has two Bernoulli Generators in
  his lab, \textit{coin}$_{\mbox{\tiny{i}}}$ and
  \textit{coin}$_{\mbox{\tiny{ii}}}$. Bob has a database of
  \textit{coin}$_{\mbox{\tiny{i}}}$ results and concludes on excellent
  evidence that \textit{coin}$_{\mbox{\tiny{i}}}$ is fair. Bob has no
  evidence about the bias of \textit{coin}$_{\mbox{\tiny{ii}}}$. As a
  Boolean, Bob assumes a sharp credence of $\{0.5\}$ for
  $H_{\mbox{\tiny{i}}}$ and an indeterminate credal state of $[0,1]$
  for $H_{\mbox{\tiny{ii}}}$. He feels bad for Larry, his Laplacean
  colleague, who cannot distinguish between the two cases and who must
  assign a sharp credence of $\{0.5\}$ for both $H_{\mbox{\tiny{i}}}$
  and $H_{\mbox{\tiny{ii}}}$.
\end{quotex}

\begin{quotex}
  \beispiel{Incomplete}\label{ex:incomp} Bob has another Bernoulli
  Generator, \textit{coin}$_{\mbox{\tiny{iii}}}$, in his lab. His
  graduate student has submitted \textit{coin}$_{\mbox{\tiny{iii}}}$
  to countless experiments and emails Bob the resulting bias, but
  fails to include whether the bias of $2/3$ is in favour of
  $H_{\mbox{\tiny{iii}}}$ or in favour of $T_{\mbox{\tiny{iii}}}$. As
  a Boolean, Bob assumes an indeterminate credal state of $[1/3,2/3]$
  (or $\{1/3,2/3\}$, depending on whether convexity is required) for
  $H_{\mbox{\tiny{iii}}}$. He feels bad for Larry who must assign a
  sharp credence of $\{0.5\}$ for $H_{\mbox{\tiny{iii}}}$ when Larry
  concurrently knows that his credence gets the bias wrong.
\end{quotex}

Against the force of \textsc{range} and \textsc{incomplete}, I
maintain that the Laplacean approach of assigning subjective
probabilities to partitions of the event space (e.g.\ objective
chances) and then aggregating them by David Lewis' summation formula
(see \scite{8}{lewis81}{266f}) into a single precise credence function
is conceptually tidy and shares many of the formal virtues of Boolean
theories. To put it provocatively, this paper defends a $0.5$ sharp
credence in heads in all three cases: for a coin of whose bias we are
completely ignorant; for a coin whose fairness is supported by a lot
of evidence; and even for a coin about whose bias we know that it is
either 1/3 or 2/3 for heads.

\section{Augustin's Concessions}
\label{AugustinsConcessions}

Here are two potential problems for Booleans:

\begin{itemize}
\item \textsc{dilation} Instates are vulnerable to dilation.
\item \textsc{obtuse} Instates do not permit learning.
\end{itemize}

Again, these are best explained by examples. First, here is an example
for \textsc{dilation} (see \scite{8}{white10}{175f} and
\scite{8}{joyce10}{296f}).

\begin{quotex}
  \beispiel{Dilation}\label{ex:dilation} Larry has two Bernoulli
  Generators, \textit{coin}$_{\mbox{\tiny{iv}}}$ and
  \textit{coin}$_{\mbox{\tiny{v}}}$. He has excellent evidence that
  \textit{coin}$_{\mbox{\tiny{iv}}}$ is fair and no evidence about the
  bias of \textit{coin}$_{\mbox{\tiny{v}}}$. Larry's graduate student
  independently tosses both \textit{coin}$_{\mbox{\tiny{iv}}}$ and
  \textit{coin}$_{\mbox{\tiny{v}}}$. Then she tells Larry whether the
  two tosses are correlated or not
  ($H_{\mbox{\tiny{iv}}}\equiv{}H_{\mbox{\tiny{v}}}$ or
  $H_{\mbox{\tiny{iv}}}\equiv{}T_{\mbox{\tiny{v}}}$, where
  $X\equiv{}Y$ means
  $(X\wedge{}Y)\vee(\urcorner{}X\wedge\urcorner{}Y)$). Larry, who has
  a sharp credence for $H_{\mbox{\tiny{v}}}$, takes this information
  in stride, but he feels bad for Bob, whose credence in
  $H_{\mbox{\tiny{iv}}}$ dilates to $[0.1]$ even though Bob shares
  Larry's excellent evidence that \textit{coin}$_{\mbox{\tiny{iv}}}$
  is fair.
\end{quotex}

Here is why Bob's credence in $H_{\mbox{\tiny{iv}}}$ must dilate. His
credence in $H_{\mbox{\tiny{v}}}$ is $[0,1]$, by stipulation. Let
$c(X)$ be the set of sharp credences representing Bob's instate, for
example $c(H_{\mbox{\tiny{v}}})=[0,1]$. Then

\begin{equation}
  \label{eq:d1}
  c(H_{\mbox{\tiny{iv}}}\equiv{}H_{\mbox{\tiny{v}}})=c(H_{\mbox{\tiny{iv}}}\equiv{}T_{\mbox{\tiny{v}}})=\{0.5\}
\end{equation}

because the tosses are independent and
$c(H_{\mbox{\tiny{iv}}})=\{0.5\}$ by stipulation. Next,

\begin{equation}
  \label{eq:d2}
  c(H_{\mbox{\tiny{iv}}}|H_{\mbox{\tiny{iv}}}\equiv{}H_{\mbox{\tiny{v}}})=c(H_{\mbox{\tiny{v}}}|H_{\mbox{\tiny{iv}}}\equiv{}H_{\mbox{\tiny{v}}})
\end{equation}

where $c(X|Y)$ is the updated instate after finding out $Y$. Booleans
accept (\ref{eq:d2}) because they are Bayesians and update by standard
conditioning. Therefore,

\begin{align}
  \label{eq:d3}
  &c(H_{\mbox{\tiny{iv}}}|H_{\mbox{\tiny{iv}}}\equiv{}H_{\mbox{\tiny{v}}})=c(H_{\mbox{\tiny{v}}}|H_{\mbox{\tiny{iv}}}\equiv{}H_{\mbox{\tiny{v}}})=\frac{c(H_{\mbox{\tiny{iv}}})c(H_{\mbox{\tiny{v}}})}{c(H_{\mbox{\tiny{iv}}})c(H_{\mbox{\tiny{v}}})+c(T_{\mbox{\tiny{iv}}})c(T_{\mbox{\tiny{v}}})} \notag \\
  &=c(H_{\mbox{\tiny{v}}})=[0,1].
\end{align}

Bob's updated instate for $H_{\mbox{\tiny{iv}}}$ has dilated from
$\{0.5\}$ to $[0,1]$.

This does not sound like a knock-down argument against Booleans (it is
investigated in detail in \scite{7}{seidenfeldwasserman93}{}), but
Roger White uses it to derive implications from instates which are
worrisome (see especially his chocolate example in
\scite{8}{white10}{183}).

Second, here is an example for \textsc{obtuse} (see Susanna Rinard's
objection cited in \scite{8}{white10}{84} and addressed in
\scite{8}{joyce10}{290f}). It presumes Joyce's supervaluationist
semantics of instates (see \scite{8}{joyce10}{288}; and
\scite{7}{rinard15}{}), for which Joyce uses the helpful metaphor of
committee members, each of whom holds a sharp credence. The instate
consists then of the set of sharp credences from each committee
member: for the purposes of updating, for example, each committee
member updates as if she were holding a sharp credence. The aggregate
of the committee members' updated sharp credences forms the updated
instate. Supervaluationist semantics also permits comparisons, when
for example a partial belief in $X$ is stronger than a partial belief
in $Y$ because all committee members have sharp credences in $X$ which
exceed all the sharp credences held by committee members with respect
to $Y$.

\begin{quotex}
  \beispiel{Obtuse}\label{ex:obtuse} Bob has a Bernoulli
  Generator in his lab, \textit{coin}$_{\mbox{\tiny{vi}}}$, of whose
  bias he knows nothing and which he submits to experiments. At first,
  Bob's instate for $H_{\mbox{\tiny{vi}}}$ is $[0.1]$. After a few
  experiments, it looks like \textit{coin}$_{\mbox{\tiny{vi}}}$ is
  fair. However, as committee members crowd into the centre and update
  their sharp credences to something closer to $0.5$, they are
  replaced by extremists on the fringes. The instate remains at
  $[0,1]$. 
\end{quotex}

Joyce, an authoritative Boolean voice, has defended instates against
\textsc{dilation} and \textsc{obtuse}, making Augustin's concessions
(AC1) and (AC2). I am naming them after Thomas Augustin, who has some
priority over Joyce in the matter.

\begin{description}
\item[{\bf (AC1)}] Credences do not adequately represent a doxastic
  state. The same instate can reflect different doxastic states.
\item[{\bf (AC2)}] Instates do not represent knowledge claims about
  objective chances. White's \emph{Chance Grounding Thesis} is not an
  appropriate characterization of the Boolean position.
\end{description}

(AC1) and (AC2) are both necessary and sufficient to resolve
\textsc{dilation} and \textsc{obtuse} for instates. I will address
this in more detail in a moment. Indeterminacy imposes a double task
on credences (representing both uncertainty and available evidence)
that they cannot coherently fulfill. I will present several examples
where this double task stretches instates to the limits of
plausibility. Joyce's idea that credences can represent balance,
weight, and specificity of the evidence (in \scite{7}{joyce05}{}) is
inconsistent with the use of indeterminacy. Joyce himself, in response
to \textsc{dilation} and \textsc{obtuse}, gives the argument why this
is the case (see \scite{8}{joyce10}{290ff} for \textsc{obtuse}; and
\scite{8}{joyce10}{296ff} for \textsc{dilation}).\bcut{12} Let us look
more closely at how (AC1) and (AC2) protect the Boolean position from
\textsc{dilation} and \textsc{obtuse}.

\subsection{Augustin's Concession (AC1)}
\label{jj1}

(AC1) says that credences do not adequately represent a doxastic state.
The same instate can reflect different doxastic states.

Augustin recognizes the problem of inadequate representation before
Joyce, with specific reference to instates: \qeins{The imprecise
  posterior does no longer contain all the relevant information to
  produce optimal decisions. Inference and decision do not coincide
  any more} \scite{2}{augustin03}{41} (see also an example for
inadequate representation of evidence by instates in
\scite{8}{bradleysteele13}{16}). Joyce rejects the notion that
identical instates encode identical beliefs by giving a simple
example:

\begin{quotex}
  \beispiel{Three-Sided Die}\label{ex:die} Let $\mathcal{C}'$ and
  $\mathcal{C}''$ be sets of credence functions defined on a partition
  $\{X,Y,Z\}$ corresponding to the result of a roll of a three
  sided-die. $\mathcal{C}'$ contains all credence functions $c$ for
  which $c(Z)\geq{}1/2$. $\mathcal{C}''$ contains all credence
  functions $c$ for which $c(X)=c(Y)$ (see \scite{8}{joyce10}{294}).
\end{quotex}

$\mathcal{C}'$ and $\mathcal{C}''$ represent the same instates, but
they differ in the doxastic states that they encode. The doxastic
state corresponding to $\mathcal{C}'$ regards $X$ and $Y$ as
equiprobable, the doxastic state corresponding to $\mathcal{C}''$ does
not. Joyce's contention is that Example \ref{ex:dilation} shares
features with Example \ref{ex:die} in the sense that
$H_{\mbox{\tiny{iv}}}\equiv{}H_{\mbox{\tiny{v}}}$ is inadmissible
evidence so that the Principal Principle does not hold. To unpack this
claim, note that the problem with \textsc{dilation} in Example
\ref{ex:dilation} is that on the surface we consider
$H_{\mbox{\tiny{iv}}}\equiv{}H_{\mbox{\tiny{v}}}$ to be admissible so
that Lewis' Principal Principle holds: (*)
$H_{\mbox{\tiny{iv}}}\equiv{}H_{\mbox{\tiny{v}}}$ does not give
anything away about $H_{\mbox{\tiny{iv}}}$, therefore (**)
$c(H_{\mbox{\tiny{iv}}}|H_{\mbox{\tiny{iv}}}\equiv{}H_{\mbox{\tiny{v}}})=c(H_{\mbox{\tiny{iv}}})$
by the Principal Principle and in contradiction to (\ref{eq:d3}). The
Principal Principle requires that my knowledge of objective chances is
reflected in my credence, unless there is inadmissible evidence (such
as knowing the outcome of a coin toss, in which case of course I do
not need to have a credence for it corresponding to the bias of the
coin).

Joyce attacks (*), but he cannot do so unless he makes concession
(AC1). For $H_{\mbox{\tiny{iv}}}\equiv{}H_{\mbox{\tiny{v}}}$ is
information that changes the doxastic state without changing the
credence, just as in Example \ref{ex:die}. As such
$H_{\mbox{\tiny{iv}}}\equiv{}H_{\mbox{\tiny{v}}}$ is inadmissible
information, and the argument for (**) fails.\bcut{15} On this point,
I agree with Joyce: given (AC1),
$H_{\mbox{\tiny{iv}}}\equiv{}H_{\mbox{\tiny{v}}}$ is inadmissible and
\textsc{dilation} ceases to be a problem for the Boolean position.
(AC1), however, undermines \textsc{incomplete}, an important argument
which Joyce has used to reject the Laplacean position. If there is a
lot more to a doxastic state than its reflection in a credal state,
both for instates and for sharp credences, then the failure of sharp
credences to report on incompleteness or ambiguity of the evidence is
no longer a major obstacle for Laplaceans.

\subsection{Augustin's Concession (AC2)}
\label{jj2}

(AC2) says that instates do not reflect knowledge claims about
objective chances. White's \emph{Chance Grounding Thesis} is not an
appropriate characterization of the Boolean position.

\begin{quotex}
  \textbf{Chance Grounding Thesis:} Only on the basis of known chances
  can one legitimately have sharp credences. Otherwise one's spread of
  credence should cover the range of possible chance hypotheses left
  open by your evidence. \scite{2}{white10}{174}\bcut{13}
\end{quotex}

Joyce considers (AC2) to be as necessary for a coherent Boolean view
of partial beliefs, blocking \textsc{obtuse}, as (AC1) is, blocking
\textsc{dilation} (see \scite{8}{joyce10}{289f}).\bcut{16} 

\textsc{obtuse} is related to \textsc{vacuity}, another problem for
Booleans:

\begin{itemize}
\item \textsc{vacuity} If one were to be committed to the principle of
  regularity, that all states of the world considered possible have
  positive probability (for a defence see \scite{7}{edwardsetal63}{});
  and to the solution of Henry Kyburg's lottery paradox, that what is
  rationally accepted should have probability 1 (for a defence of this
  principle see \scite{7}{douvenwilliamson06}{}); and the CGT, that
  one's spread of credence should cover the range of possible chance
  hypotheses left open by the evidence (implied by much of Boolean
  literature); then one's instate would always be vacuous.
\end{itemize}

Booleans must deny at least one of the premises to avoid the
conclusion. Joyce denies the CGT, giving us (AC2). 

\section{The Double Task}
\label{TheDoubleTask}

Sharp credences have one task: to represent epistemic uncertainty and
serve as a tool for updating, inference, and decision making. They
cannot fulfill this task without continued reference to the evidence
which operates in the background. To use an analogy, credences are not
sufficient statistics with respect to updating, inference, and
decision making. What is remarkable about Joyce's response to
\textsc{dilation} and \textsc{obtuse} is that Joyce recognizes that
instates are not sufficient statistics either. But this means that
they fail at the double task which has been imposed on them: to
represent both epistemic uncertainty and relevant features of the evidence.

In the following, I will provide a few examples where it becomes clear
that instates have difficulty representing uncertainty because they
are tangled in a double task which they cannot fulfill.

\begin{quotex}
  \beispiel{Aggregating Expert Opinion}\label{ex:aggreg} Bob has no
  information whether it will rain tomorrow ($R$) or not except the
  predictions of two weather forecasters. One of them forecasts 0.3 on
  channel GPY, the other 0.6 on channel QCT. Bob considers the QCT
  forecaster to be significantly more reliable, based on past
  experience.
\end{quotex}

An instate corresponding to this situation may be $[0.3,0.6]$ (see
\scite{8}{walley91}{214}), but it will have a difficult time
representing the difference in reliability of the experts. We could
try $[0.2,0.8]$ (since the greater reliability of QCT suggests that
the chance of rain tomorrow is higher rather than lower) or
$[0.1,0.7]$ (since the greater reliability of QCT suggests that its
estimate is more precise), but it remains obscure what the criteria
might be.

A sharp credence of $P(R)=0.53$, for example, does the right thing.
Such a credence says nothing about any beliefs that the objective
chance is restricted to a subset of the unit interval, but it
accurately reflects the degree of uncertainty that the rational agent
has over the various possibilities. As we will see in the next
example, it is an advantage of sharp credences that they do not
exclude objective chances, even extreme ones, because they are fully
committed to partial belief and do not suggest, as indeterminate
credences do, that there is full belief knowledge that the objective
chance is a member of a proper subset of the possibilities.

\begin{quotex}
  \beispiel{Precise Credences}\label{ex:preccre} Larry's credence for
  rain tomorrow, based on the expert opinion of channel GPY and
  channel QCT (he has no other information) is $0.53$. Is it
  reasonable for Larry, considering how little evidence he has, to
  reject the belief that the chance of rain tomorrow is $0.52$ or
  $0.54$; or to prefer a $52.9$ cent bet on rain to a $47.1$ cent bet
  on no rain?
\end{quotex}

The first question in Example \ref{ex:preccre} is confused, but in
instructive ways. A sharp credence rejects no hypothesis about
objective chances (unlike an instate, unless (AC2) is firmly in
place). It constrains partial beliefs in objective chances by Lewis'
summation formula. No objective chance is excluded by it (principle of
regularity) and any updating will merely change the partial beliefs,
but no full beliefs. Instates, on the other hand, by giving ranges of
acceptable objective chances suggest that there is a full belief that
the objective chance does not lie outside what is indicated by the
instate. A Boolean can avoid this situation by accepting (AC2).

The second question in Example \ref{ex:preccre} is also instructive:
why would we prefer a $52.9$ cent bet on rain to a $47.1$ cent bet on
no rain, given that we do not possess the power of descrimination
between these two bets? The answer to this question ties in with the
issue of incomplete preference structure referred to above as
motiviation (B) for instates.

The development of representation theorems beginning with Frank Ramsey
(followed by increasingly more compelling representation theorems in
\scite{7}{savage54}{}; and \scite{7}{jeffrey65}{}; and numerous other
variants in contemporary literature) bases probability and utility
functions of an agent on her preferences, not the other way around.
Once completeness as an axiom for the preferences of an agent is
jettisoned, indeterminacy follows automatically. Indeterminacy may
thus be a natural consequence of the proper way to think about
credences in terms of the preferences that they represent.

In response, preferences may very well logically and psychologically
precede an agent's probability and utility functions, but that does
not mean that we cannot inform the axioms we use for a rational
agent's preferences by undesirable consequences downstream.
Completeness may sound like an unreasonable imposition at the outset,
but if incompleteness has unwelcome consequences for credences
downstream, it is not illegitimate to revisit the issue. Timothy
Williamson goes through this exercise with vague concepts, showing
that all upstream logical solutions to the problem fail and that it
has to be solved downstream with an epistemic solution (see
\scite{7}{williamson96}{}). Vague concepts, like sharp credences, are
sharply bounded, but not in a way that is luminous to the agent (for
anti-luminosity see chapter 4 in \scite{7}{williamson00}{}).
Anti-luminosity answers the original question: the rational agent
prefers the $52.9$ cent bet on rain to a $47.1$ cent bet on no rain
based on her sharp credence without being in a position to have this
preference necessarily or have it based on physical or psychological
ability (for the analogous claim about knowledge see
\scite{8}{williamson00}{95}).

\begin{quotex}
  \beispiel{Monkey-Filled Urns}\label{ex:monkey} Let urn $A$ contain 4
  balls, two red and two black. A monkey randomly fills urn $B$ from
  urn $A$ with two balls. We draw from urn $B$ (a precursor to this
  example is in \scite{8}{jaynesbretthorst03}{160}).
\end{quotex}

The sharp credence of drawing a red ball is $0.5$, following Lewis'
summation formula for the different combinations of balls in urn $B$.
This solution is more intuitive in terms of further inference,
decision making, and betting behaviour than a credal state of
$\{0,1/2,1\}$ or $[0,1]$ (depending on the convexity requirement),
since this instate would licence an exorbitant bet in favour of one
colour, for example one that costs \$9,999 and pays \$10,000 if red is
drawn and nothing if black is drawn.

To conclude, a Boolean in the light of Joyce's two Augustinian
concessions has three alternatives, of which I favour the third: (a)
to find fault with Joyce's reasoning as he makes those concessions;
(b) to think (as Joyce presumably does) that the concessions are
compatible with the promises of Booleans, such as \textsc{range} and
\textsc{incomplete}, to solve prima facie problems of sharp credences;
or (c) to abandon the Boolean position because (AC1), (AC2), and an
array of examples in which sharp credences are conceptually and
pragmatically more appealing show that the initial promise of the
Boolean position is not fulfilled.

% \nocite{*} 
\bibliographystyle{ChicagoReedweb} 
\bibliography{bib-7293}

\end{document} 
