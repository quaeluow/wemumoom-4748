\documentclass[11pt]{article}

% https://tinyurl.com/lxam8f6

% this makes marginpar more attractive
% see http://www.f.kth.se/~ante/latex.php
\setlength{\marginparwidth}{1.2in}
\let\oldmarginpar\marginpar
\renewcommand\marginpar[1]{\-\oldmarginpar[\raggedleft\footnotesize #1]%
{\raggedright\footnotesize #1}}

%\usepackage{helvet}
\setlength{\parindent}{0in}
\setlength{\parskip}{.3in}

\raggedbottom

%\pagestyle{empty}

% 	PACKAGES
\usepackage{amsfonts}
\usepackage{amssymb}
%\usepackage{german}
%\usepackage{hebtex}
%\usepackage{graphicx}
%\usepackage[german]{babel}
%\usepackage{endnotes}
%\usepackage{rotating}

\begin{document}

Professor Pap,

I am looking for a non-additive, continuous set function from a
simplex $\Pi_{n}$ of finite dimension $n-1$ into $[0,\infty]$. The
motivation is as follows. Shannon defined the entropy of a probability
measure on a finite event space $H(p_{1},\ldots,p_{n})$. He wanted the
entropy to be (i) continuous in the $p_{i}$, (ii) if all the $p_{i}$
are equal ($p_{i}=1/n$) then $H$ should be monotonic increasing in $n$
(more equally probable events mean greater entropy, so for example
$H_{2}(1/2,1/2)<H_{3}(1/3,1/3,1/3)$), and (iii) if choices are
subdivided the entropy should remain constant, such that, for
example, $H_{3}(p_{1},p_{2},p_{3})=
H_{2}(p_{1},q)+qH_{2}(p_{2}/q,p_{3}/q)$ for $q=p_{2}+p_{3}$.

The only function fulfilling these requirements is the Shannon entropy

\begin{equation}
  \label{eq:m1}
  H(p)=-K\sum_{i=1}^{n}p_{i}\log{}p_{i}
\end{equation}

I want to generalize the Shannon entropy from elements of $\Pi_{n}$ to
select subsets of $\Pi_{n}$ (e.g. Borel subsets). Let $\pi$ be a
subset of $\Pi_{n}$. I guess here are some commonsense requirements
for $\eta$, the generalization of $H$. (i) $\eta(\{p\})=H(p)$. (ii)
$\eta$ is continuous in the sense that

\begin{displaymath}
  \mbox{If }\bigcap_{k=1}^{\infty}\pi_{k}=\pi\mbox{ with }\pi_{k}\subseteq\pi_{k-1}\mbox{ then }\eta(\pi)=\lim_{k\rightarrow\infty}\eta(\pi_{k})
\end{displaymath}

and

\begin{displaymath}
  \mbox{If }\bigcup_{k=1}^{\infty}\pi_{k}=\pi\mbox{ with }\pi_{k}\supseteq\pi_{k-1}\mbox{ then }\eta(\pi)=\lim_{k\rightarrow\infty}\eta(\pi_{k})
\end{displaymath}

One requirement that I can't quite articulate yet is that $\eta$ won't
be additive but more of an averageing function, such that
$\eta(\{p,q\})$ is somewhere between $H(p)$ and $H(q)$, certainly not
the sum of them. The intuition is that if a rational agent can
permissibly hold these two probability functions, her information is
roughly the average of the information where she is in the situation
that she can hold only one of them.

My question for you: is it possible to give an explicit formula for
$\eta$ as in (\ref{eq:m1})? I imagine it would look very similar to
(\ref{eq:m1}) and involve some kind of integral over the points of the
simplex.

I would love it if you could help me with this or point me to someone
who could. Thank you,

Stefan

\end{document}
