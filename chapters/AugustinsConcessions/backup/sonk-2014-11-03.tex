% This is the official version. For shorter versions, see directory
% diff.

% http://tinyurl.com/nrhqdxg

% TBD: 

% What exactly is Linda's evidence differential compared to
% Betsy -- the analytic expression may help.

\documentclass[11pt]{article}
\usepackage{october}
% For BJPS
% \hyphenpenalty=10000
% \hbadness=10000

\begin{document}
% For BJPS
% \raggedright
% \doublespacing

\title{Semantics of Not Knowing}
\author{Stefan Lukits}
\date{}
\maketitle
\newcounter{expls}
% \doublespacing

\begin{abstract}
  {\noindent}Many Bayesian epistemologists now accept that it is not
  necessary for a rational agent to hold sharp credences. There are
  various compelling formal theories how such a non-traditional view
  of credences can accommodate decision making and updating. They are
  motivated by a common complaint: that sharp credences can fail to
  represent incomplete evidence. I show that the critical response of
  those who want to maintain the traditional view with only sharp
  credences is insufficient. Their criticism, however, opens the door
  to semantic questions which make indeterminate credences vulnerable
  on a deeper level. This paper uses both conceptual arguments and
  hands-on examples to demonstrate that rational agents should always
  have sharp credences. It is also articulated and substantiated that
  agents with indeterminate credal states make long term gains betting
  against agents with sharp credences. This on the surface
  inconvenient fact can be explained and does not favour indeterminacy
  over sharp credences.
\end{abstract}

\section{Introduction}
\label{Introduction}

Traditionally, Bayesians have maintained that a rational agent, if she
holds a credence, holds a sharp credence. It has recently become
popular to drop the requirement for credence functions to be sharp.
There are now Bayesians who permit a rational agent to hold
indeterminate credal states (from now on instates) based on incomplete
or ambiguous evidence. I will refer to Bayesians who continue to
adhere to the classical theory of sharp credences for rational agents
as \qnull{Laplaceans} (e.g.\ Adam Elga and Roger White). I will refer
to Bayesians who do not believe that a rational agent always holds
sharp credences as \qnull{Booleans} (e.g.\ Peter Walley and James
Joyce).\fcut{1}

There is some terminological confusion around the adjectives
\qnull{imprecise,} \qnull{indeterminate,} and \qnull{mushy} credences.
In the following, I will exclusively refer to indeterminate credences
or credal states (abbreviated \qnull{instates}) and mean by them a
credal state (a set of credence functions, which some Booleans require
to be convex) which it may be rational for an agent to hold within an
otherwise orthodox Bayesian framework. After describing the appeal of
indeterminacy and showing how contemporary Laplacean objections fail,
I will point to more serious failings of indeterminacy in semantic
terms and show how a proper semantics of not knowing, which we could
also call a semantics of partial belief, solves the problems for sharp
credences that Booleans address by introducing instates. There is a
sense in which, by linking knowledge of chances to its reflection in
credences, Booleans seek to reconcile traditional knowledge
epistemology concerned with full belief and formal epistemology
concerned with partial belief. There are other more recent
reconciliation projects (see Spohn, 2012; and Moss, 2013). If my paper
is correct then the Boolean approach will not contribute to this
reconciliation because it mixes full belief and partial belief
metaphors in ways that are semantically problematic.\fcut{2}

Instates are often lauded as doing much better representing
uncertainty together with the evidence that constrains it, but they
can no more give an adequate representation of evidence than sharp
credences. This paper defends a $0.5$ sharp credence in heads in all
three cases: for a coin of whose bias we are completely ignorant; for
a coin whose fairness is supported by a lot of evidence; and even for
a coin about whose bias we know that it is either 1/3 or 2/3 for
heads.

One potential Boolean claim is that agents who use instates do better
than Laplaceans when they bet on the truth of events for which they
have varying degrees of evidence. Walley gives an example where a
Laplacean does much worse at predicting Soccer World Cup games than
Boolean peers who use upper and lower previsions (see
\scite{7}{walley91}{}, appendix I). Upper and lower previsions are
instates for which it is rational to accept or reject bets if they
fall within the margin of indeterminacy.

First I will show on a much more general level how Walley's claims are
justified in practice (bettors using upper and lower previsions do on
average better than bettors using linear previsions, i.e.\ sharp
credences). Then I will explain why this is the case, how a Laplacean
can protect herself against this disadvantage by drawing proper
distinctions between credence and evidence, and how indeterminacy
emerges as the loser when the contest is about clarity in one's
semantics. The problem will be mixed metaphors: the Boolean mixes
semantic levels that ought for good reasons to remain separate.
Indeterminacy imposes a double task on credences (representing both
uncertainty and available evidence) that they cannot coherently
fulfill.

I will present several examples where this double task stretches
instates to the limits of plausibility. Joyce's idea that credences
can represent balance, weight, and specificity of the evidence is
inconsistent with the use of indeterminacy (and Joyce himself, in
response to the dilation problem, gives the argument why this is the
case). The Boolean claim that certain properties of the evidence (its
ambiguity, its completeness, conflicts within it) can be recovered
from instates is inconsistent with an effective Boolean answer to the
dilation problem.

The Laplacean approach of assigning subjective probabilities to
partitions of the event space (e.g.\ objective chances) and then
aggregating them by David Lewis' summation formula into a single
precise credence function is semantically tidy and shares many of the
formal virtues of Boolean theories. If the bad taste about numerical
precision in our fuzzy and nebulous world lingers, I will point to
philosophical projects in other domains where the concepts we use are
sharply bounded, even though our ability to conceive of those sharp
boundaries or know them is limited.

\section{Motivation for Instates}
\label{MotivationForIndeterminateCredalStates}

We want to motivate indeterminacy for the credences of a rational
agent, independent of how they are elicited, as forcefully as possible
so that the reader will see (a) the appeal of such indeterminacy, (b)
the insufficiency of the critical response, and (c) the need for
careful articulation of the Laplacean approach that mandates a
rational agent to hold sharp credences together with an explanation of
how it addresses the concerns which motivate some to resort to
indeterminacy.

\texttt{<x2.1>}When textual criticism compares variant readings of
ancient texts in order to recover the most probable reading of the
original, it follows a well-accepted rule not to recommend the easiest
reading, i.e.\ the one that flows most naturally with the surrounding
text, but the (often more difficult) reading which best explains the
variants in the manuscripts. A similar dynamic is at work here:
motivating indeterminacy will make clear that acceptance of
indeterminacy is the easiest reading and on the surface best responds
to intuitions we have about the credences of a rational agent. We will
address, however, the ways in which indeterminacy itself violates
intuitions, especially those for semantic clarity and simplicity (the
latter viewed in terms of reducing hierarchical nesting of uncertainty
versus tidily separating the evidential and epistemic dimension of
partial beliefs). The Laplacean approach, which does not permit
indeterminacy, explains why this is the case and, despite being the
more difficult reading, ultimately yields a more integrated package
that accords better with our intuitions and requirements for a
coherent formal and semantic theory.\texttt{</x2.1>}

\texttt{<x2.2>}Our conclusion is that a rational agent is best off in
terms of her own goals when she entertains sharp credences with
respect to propositions about events that come her way. Whether this
is advisable for human or machine intelligence is a different kettle
of fish. My topic is the logic of partial beliefs, and I readily admit
that such a logic may be computationally intractable or, given finite
resources, be an irrational way of keeping track of
beliefs.\texttt{</x2.2>}\fcut{3}

Let a \textsc{coin} be a Bernoulli generator that produces successes
and failures with probability $p$ for success, labeled $H$, and $1-p$
for failure, labeled $T$. Physical coins may serve as examples, if we
are willing to set aside that most of them are approximately fair.
Imagine three \textsc{coin}s for which we have evidence that
\textsc{coin}$_{I}$ is fair, \textsc{coin}$_{II}$ has an unknown bias,
and \textsc{coin}$_{III}$ has as bias either $p=1/3$ or $p=2/3$. The
Laplacean approach permits a sharp $0.5$ credence in $H$ for a
rational agent in all three cases. A Boolean approach wants to see the
difference in the evidential situation reflected in a rational agent's
credal state and at least permit, as credence in $H$, $\{x|x=0.5\}$
for \textsc{coin}$_{I}$, $\{x|0\leq{}x\leq{}1\}$ for
\textsc{coin}$_{II}$, and $\{x|1/3\leq{}x\leq{}2/3\}$ or $\{1/3,2/3\}$
for \textsc{coin}$_{III}$.

\texttt{<x2.4>}Here are a few concise statements by Booleans:

\begin{quotex}
  [A] refusal to make a determinate probability judgment does not
  derive from a lack of clarity about one's credal state. To the
  contrary, it may derive from a very clear and cool judgment that on
  the basis of the available evidence, making a numerically
  determinate judgment would be unwarranted and arbitrary.
  \scite{3}{levi85}{395}
\end{quotex}

\begin{quotex}
  If there is little evidence concerning [a claim,] then beliefs about
  [that claim] should be indeterminate, and probability models
  imprecise, to reflect the lack of information. We regard this as the
  most important source of imprecision. \scite{3}{walley91}{212--213}
\end{quotex}

\begin{quotex}
  Imprecise probabilities and related concepts [{\ldots}] provide a
  powerful language which is able to reflect the partial nature of the
  knowledge suitably and to express the amount of ambiguity
  adequately. \scite{3}{augustin03}{34}
\end{quotex}

\begin{quotex}
  As sophisticated Bayesians like Isaac Levi (1980), Richard Jeffrey
  (1983), Mark Kaplan (1996), have long recognized, the proper
  response to symmetrically ambiguous or incomplete evidence is not to
  assign probabilities symmetrically, but to refrain from assigning
  precise probabilities at all. Indefiniteness in the evidence is
  reflected not in the values of any single credence function, but in
  the spread of values across the family of all credence functions
  that the evidence does not exclude. This is why modern Bayesians
  represent credal states using sets of credence functions. It is not
  just that sharp degrees of belief are psychologically unrealistic
  (though they are). Imprecise credences have a clear epistemological
  motivation: they are the proper response to unspecific evidence.
  \scite{3}{joyce05}{170f}
\end{quotex}\fcut{4}

\texttt{</x2.4>}

Consider the following reasons that motivate Booleans to permit
instates for rational agents:

\begin{enumerate}[(A)]
\item The greatest emphasis motivating indeterminacy rests on lack of
  evidence or conflicting evidence and the assumption that single
  probability measures (sharp credences) do not represent such
  evidence as well as credal states composed by sets of probability
  measures (instates).
\item The preference structure of a rational agent may be incomplete
  so that representation theorems do not yield single probability
  measures to represent such incomplete structures.
\item There are more technical and paper-specific reasons, such as
  Thomas Augustin's attempt to mediate between the minimax pessimism
  of objectivists and the Bayesian optimism of subjectivists using
  interval probability (see \scite{8}{augustin03}{35f}); Alan
  H{\'a}jek and Michael Smithson's belief that there may be
  objectively indeterminate chances in the physical world (see
  \scite{8}{hajeksmithson12}{33}); and Jake Chandler's claim that
  \qeins{the sharp model is at odds with a trio of plausible
    propositions regarding agnosticism} \scite{2}{chandler14}{4}.
\end{enumerate}

This paper mostly addresses (A), while taking (B) seriously as well
and pointing towards solutions for it. I am leaving (C) to more
specific responses to the issues presented in the cited articles. I am
adding a reason (D) that is poorly documented in the literature: The
Boolean rational agent may systematically do better accepting bets
than the agent who on principle rejects instates.  Walley conducted an
experiment in which Boolean participants did significantly better than
Laplacean participants, betting on soccer games played in the Soccer
World Cup 1982 in Spain (see \scite{7}{walley91}{}, Appendix I). I
replicated the experiment using two computer players with rudimentary
artificial intelligence and made them specify betting parameters
(previsions) for games played in the Soccer World Cup 2014 in Brazil.
I used the Poisson distribution (which is an excellent predictor for
the outcome of soccer matches) and the FIFA ranking to simulate
millions of counterfactual World Cup results and their associated
bets, using Walley's evaluation method.  The Boolean player had a
slight but systematic advantage. In section \ref{WalleysWorldCupWoes},
I will provide an explanation and show how it undermines any support
the experiment might give to the Boolean position.\fcut{5}

\section{Dilation}
\label{Dilation}\fcut{6}

An objection by Roger White (see \scite{7}{white10}{}) is initially 
semantically minded with an analysis of the Chance Grounding Thesis,
of which I will make use in my criticism of indeterminacy:

\begin{quotex}
  \textbf{(CGT)} Only on the basis of known chances can one
  legitimately have sharp credences. Otherwise one's spread of
  credence should cover the range of possible chance hypotheses left
  open by your evidence. \scite{2}{white10}{174}\tbd{see Chance
  Grounding Thesis in sonk.org -- maybe go into more detail}
\end{quotex}

For the remainder of White's paper, however, the focus moves to
dilation, which according to White saddles the indeterminacy approach
with unacceptable counter-intuitive results. Joyce deals with White's
objection in detail in his 2010 paper (for a concise summary see
\scite{8}{bradleysteele13}{13}). Ironically, White's objection based
on dilation has the implications that are decisive in revealing the
semantic vulnerability of the Boolean approach, whereas the CGT by
itself is not sufficient to undermine the Boolean approach. To
establish this, however, we need Joyce's response to White. Joyce
satisfactorily responds to White's objection and the problem of
dilation, but not without making a semantic concession that undermines
the Boolean approach.\tbd{rephrase}

You have two Bernoulli Generators, \textsc{coin}$_{a}$ and
\textsc{coin}$_{b}$. You have good evidence that \textsc{coin}$_{a}$
is fair and no evidence about the bias of \textsc{coin}$_{b}$.
Furthermore, the two generators are not necessarily independent. Their
results could be 100\% correlated or anticorrelated, they may be
independent, or the correlation could be anywhere between the two
extremes. You toss \textsc{coin}$_{a}$. Without looking at the result,
your credence in $H$ is a sharp $0.5$, even if you are open to
instates, because you have good evidence for the fairness of
\textsc{coin}$_{a}$. Then you toss \textsc{coin}$_{b}$. This time you
look at the result and the moment you learn it, your credence in $H$
for \textsc{coin}$_{a}$ dilates from a sharp $0.5$ to the vacuous
credal state covering the whole interval $[0,1]$ (provided that this
was your credence in $H$ for \textsc{coin}$_{b}$, as stipulated). Even
though you have just received information (the result of
\textsc{coin}$_{b}$'s toss), your credence in $H$ for
\textsc{coin}$_{a}$ dilates. Usually, we would expect more information
to sharpen our credal states (see Walley's \qeins{the more information
  the more precision} principle and his response to this problem in
1991,\fixref{8}{walley91}{207 and 299} 207 and 299).

White did not discover the phenomenon of dilation (see the detailed
study in \scite{7}{seidenfeldwasserman93}{}), but he was able to find
examples where the consequences appear grotesque, especially in his
bonbon case (see \scite{8}{white10}{183}).

\begin{quotex}
  Having noodled about this puzzle on and off for some time, I
  discovered that the general phenomenon of dilation is old news. Some
  statisticians and philosophers have studied how the phenomenon
  arises in other cases and appear to have taken it in their stride.
  This is not a reductio but a result, they might say. I want to
  suggest that the present case brings out particularly forcefully how
  bizarre this phenomenon is, at least in the present case where we
  are assuming evidential symmetry between $p$ and not-$p$.
  \scite{3}{white10}{177}
\end{quotex}

White's claim is also that dilation contradicts Bas van Fraassen's
reflection principle (see van Fraassen,
1984\fixref{7}{vanfraassen84}{}).

In Joyce's response to White, two semantic concessions to White (J1)
and (J2) show why the dilated instates give us the right result. I
agree with Joyce: dilation is what you would expect if (J1) credences
do not adequately represent evidence (the same instate can reflect
different evidential situations); and if (J2) instates do not reflect
knowledge claims about objective chances (Joyce rejects White's Chance
Grounding Thesis CGT, see \scite{8}{joyce10}{289}).

\addtocounter{expls}{1}

\begin{quotex}
  \textbf{Example \arabic{expls}: \addtocounter{expls}{1} Dilating
    Urns.} You draw one ball from an urn with 200 balls (100 red, 100
  black) and receive the information that the urn actually had two
  chambers, one with 99 red balls and 1 black ball, the other with 1
  red ball and 99 black balls.
\end{quotex}

Dilation from a sharp credence of $\{0.5\}$ to an instate of
$\{0.01,0.99\}$ or $[0.01,0.99]$ (depending on whether convexity is
required) is unproblematic, although the example already prefigures
that there is something odd about the Boolean semantic approach. The
example licences a 99:1 bet for one of the colours (if the instate is
interpreted as upper and lower previsions), but this is a problem that
arises out of Boolean semantics without dilation, which we will
address again in Example \ref{monkey}.

Booleans have the resources to extract themselves from the problem of
dilation, but only at the cost of making Joyce's semantic concessions
(J1) and (J2) which we will use against them. These semantic
concessions are inescapable, not only on account of dilation. If one
were to be committed to the principle of regularity, that all states
of the world considered possible have positive probability (for a
defence see \scite{7}{edwardsetal63}{}); and to the solution of Henry
Kyburg's lottery paradox, that what is rationally accepted should have
probability 1 (for a defence of this principle see
\scite{7}{douvenwilliamson06}{}); and the CGT, that one's spread of
credence should cover the range of possible chance hypotheses left
open by the evidence (implied by much of Boolean literature); then
one's instate would always be vacuous. Booleans must deny at least one
of the premises to avoid the conclusion. Joyce denies the CGT, but
then he continues to make implicit use of it when he repeatedly
complains that sharp credences \qeins{ignore a vast number of
  possibilities that are consistent with [the] evidence} (for example
in \scite{8}{joyce05}{170}).

A sharp credence constrains partial beliefs in objective chances by
Lewis' summation formula (which we will provide in the next section).
No objective chance is excluded by it (principle of regularity) and
any updating will merely change the partial beliefs, but no full
beliefs. Instates, on the other hand, by giving ranges of acceptable
objective chances suggest that there is a full belief that the
objective chance does not lie outside what is indicated by the
instate. A Boolean can avoid this situation by accepting Joyce's
concession (J2).

\begin{quotex}
  \textbf{Example \arabic{expls}: \addtocounter{expls}{1} Bavarian
    King.} Matthias Perth, an Austrian civil servant, observes the
  Bavarian king at the Congress of Vienna in 1815 and writes in his
  diary that the king \qeins{appears to be a man between 45 and 47
    years old} (see \texttt{http://www.das-perth-projekt.at}).
\end{quotex}

If Perth then learns that the king was 49 years old, he must revise,
not just update, his earlier judgment. The appropriate formal
instrument is belief revision, not probability update. Whoever wants
to use it leaves the Ample Bosom of Mother Bayes (see
\scite{8}{levi85}{392}), unless there is a substantial reconciliation
project between formal and traditional epistemology operating in the
background. I do not see this project articulated in the Boolean
literature (for an example of such a project see \scite{7}{spohn12}{},
especially chapter 10). Sarah Moss also undertakes it and assumes the
Boolean approach (see \scite{7}{moss13}{}), but I fail to see how the
Boolean approach is essential to her reconciliation or how her
reconciliation gives independent arguments for the Boolean approach.
If Perth had wanted to express a sharp credence, he would have said,
\qeins{my best guess is that the king is 46 years old,} and the
information that the king was 49 would have triggered the appropriate
update, without any revision of full beliefs.

White's objection fails, however, because dilation for indeterminate
credences is in principle not any more surprising than a piece of
information that increases the Shannon entropy of a sharp credence. It
is true for both sharp and indeterminate credences that information
can make us less certain about things. Once Booleans have brought
their house in order to accommodate White's objection using
concessions (J1) and (J2), they open the door to semantic problems. We
finally get to inquire what kind of coherence there is in defending
indeterminacy when it neither fulfills the promise of representing
evidence nor the promise of reconciling traditional full belief
\qnull{knowledge} epistemology and Bayesian partial belief
epistemology as outlined in the CGT, but only adds another
hierarchical layer of uncertainty to a numerical quantity (a sharp
credence) whose job it already is to represent uncertainty, thus
unnecessarily introducing regress problems. We will turn to these
semantic considerations now, show how they display the virtues of
sharp credences in responding to the forceful motivations for instates
while making those instates look semantically otiose. Then we will
show in the last section how sharp credences have an elegant solution
for being outperformed by instates in betting scenarios, and there we
will rest our case.

\section{Semantics of Partial Belief}
\label{SemanticsOfPartialBelief}

Instates are suggestive of a measurement that represents numerically
the mass of an object and then also make claims about its density.
With sharp credences, the semantic roles of evidence, information, and
uncertainty are appropriately differentiated. Rational decision
making, inference, and betting behaviour are based on sharp credences
together with the evidence that is at its foundation. Information
represents evidence, and sharp credences represent uncertainty.
Measurement is in any case a misleading analogy for credences.
Measurements come with imprecision estimates to account for
inaccuracy. Epistemic credences, however, are not measurements,
especially not of objective chances. They represent uncertainty. They
are more like logical truth values than they are like measurements.

\texttt{<x4.1>}It is a slippery affair to determine what evidence is,
which I will leave to others. My claim is that a rational agent is
someone who can distill information from evidence which places
numerically precise constraints on relatively prior probability
distributions, which then can be updated to form posterior probability
distributions and the credences associated with them. Note that
relatively prior probability distributions are not ignorance priors or
non-informative priors, which I would call absolutely prior
probability distributions. I have no answers where absolutely prior
probability distributions come from, how they are justified, or in
what sense they are objective. I am not concerned whether all rational
agents, if they have the same evidence, should arrive at the same
credal states; or even if they should all update a given relatively
prior credal state to the same posterior credal state, if they have
the same evidence. Sometimes there may be different ways to translate
or interpret evidence into information.\texttt{</x4.1>}

It is important not to confuse the claim that it is reasonable to hold
both $X$ and $Y$ with the claim that it is reasonable to hold either
$X$ (without $Y$) or $Y$ (without $X$). It is the reasonableness of
holding $X$ and $Y$ concurrently that is controversial, not the
reasonableness of holding $Y$ (without holding $X$) when it is
reasonable to hold $X$. We will later talk about anti-luminosity, the
fact that a rational agent may not be able to distinguish
psychologically between a $54.9$ cent bet on an event and a $45.1$ bet
on its negation, when her sharp credence is $0.55$. She must reject
one of them not to incur sure loss, so proponents of indeterminacy
suggest that she choose one of them freely without being constrained
by her credal state or reject both of them. I claim that a sharp
credence will make a recommendation between the two so that only one
of the bets is rational given her particular credence, but that does
not mean that another sharp credence which would give a different
recommendation may not also be rational for her to have.

I am sympathetic to the viewpoint that once a rational agent has a
relatively prior credal state and has formalized her evidence in terms
of information, then the probability distributions forming her
posterior credal state should be unique. Joyce, with his
\qnull{committee member} approach, shows how this kind of updating can
be done for instates (see \scite{8}{joyce10}{288}; also
\scite{8}{bradleysteele13}{6}).

On the one hand (the Laplacean approach), you can have partial beliefs
about how a parameter is distributed and then use Lewis' summation
formula (see \scite{8}{lewis81}{266f}) to integrate over them and
condense them to a sharp credence. Walley comments on this
\qeins{reduction} in his section on Bayesian second order
probabilities (see \scite{8}{walley91}{258f}), but he mistakenly
represents the Laplacean approach as a second order approach, as if
the probability distributions that are summarized by Lewis' formula
are of the same kind as the resulting credences. They are not. They
are objective chances or other partitions of the event space and the
subjective probabilities that are associated with them. It is the
Boolean approach which has elements of a second order approach and
thus makes itself vulnerable to regress problems by adding another
dimension of uncertainty to a parameter (the credence) which already
represents uncertainty.

On the other hand (the Boolean approach), you can try to represent
your uncertainty about the distribution of the parameter by an
instate. I want to show that the Laplacean approach can be aligned
with the forceful motivations we listed in the previous section to
introduce instates, as long as we do not require that a sharp credence
represent the evidence as well as the epistemic state of uncertainty
in the agent. We have learned that this requirement can be reduced ad
absurdum even for instates (J1).

One of Joyce's complaints is that a sharp credence of $0.5$ for a
\textsc{coin} contains too much information if there is little or no
evidence that the \textsc{coin} is fair. This complaint, of course, is
only effective if the indeterminacy of the credence is anticorrelated
to the amount of information in the evidence. In (J1), however, Joyce
admits that instates cannot represent the evidence without violating
the reflection principle due to White's dilation problem. He is quite
clear that the same instate can represent different evidential
scenarios (see, for example, \scite{8}{joyce10}{302}). (J1) may not be
sufficient to defend against the information argument, but Walley's
and Joyce's claim that instates are less informative than sharp
credences (see \scite{8}{walley91}{34}; and \scite{8}{joyce10}{311}
for examples, but this attitude is passim) has no foundation in
information theory. To compare instates and sharp credences
informationally, we would need a non-additive set function obeying
Shannon's axioms for information. This is a non-trivial task. I have
not succeeded solving it (nor do I need to carry the Booleans' water),
but I am not convinced that it will result in an information measure
which assigns, for instance, more information to a sharp credence such
as $\{0.5\}$ than to an instate such as $\{x|1/3\leq{}x\leq{}2/3\}$.

Returning to (J1) and the problem of inadequate representation,
Augustin recognizes this long before Joyce, with specific reference to
instates: \qeins{The imprecise posterior does no longer contain all
  the relevant information to produce optimal decisions. Inference and
  decision do not coincide any more} \scite{2}{augustin03}{41} (see
also an example for inadequate representation of evidence by instates
in \scite{8}{bradleysteele13}{16}). At best, instates fare no better
than sharp credences, at worst they unhelpfully mimic saying something
about the evidence that is much better said elsewhere.

Not only can we align sharp credences with the motivations to
introduce instates, we can also show that instates perform worse
semantically because they mix evidential and epistemic metaphors in
deleterious ways. Sharp credences have one task: to represent
epistemic uncertainty and serve as a tool for updating, inference, and
decision making. They cannot fulfill this task without continued
reference to the evidence which operates in the background. To use an
analogy, credences are not sufficient statistics with respect to
updating, inference, and decision making. What is remarkable about
Joyce's response to White's dilation problem is that Joyce recognizes
that instates are not sufficient statistics either. But this means
that they fail at the double task which has been imposed on them: to
represent both epistemic uncertainty and the evidence.

In the following, I will provide a few examples where it becomes clear
that instates have difficulty representing uncertainty because they
are tangled in a double task which they cannot fulfill.

\begin{quotex}
  \textbf{Example \arabic{expls}: \addtocounter{expls}{1} Aggregating
    Expert Opinion.} You have no information whether it will rain
  tomorrow ($R$) or not except the predictions of two weather
  forecasters. One of them forecasts 0.3 on channel GPY, the other 0.6
  on channel QCT. You consider the QCT forecaster to be significantly
  more reliable, based on past experience.
\end{quotex}

An instate corresponding to this situation may be $[0.3,0.6]$ (see
\scite{8}{walley91}{214}), but it will have a difficult time
representing the difference in reliability of the experts. A sharp
credence of $P(R)=0.53$, for example, does the right thing. Such a
credence says nothing about any beliefs that the objective chance is
restricted to a subset of the unit interval, but it accurately
reflects the degree of uncertainty that the rational agent has over
the various possibilities. Beliefs about objective chances make little
sense in many situations where we have credences, since it is doubtful
even in the case of rain tomorrow that there is an urn of nature from
which balls are drawn. What is really at play is a complex interaction
between epistemic states (for example, experts evaluating
meteorological data) and the evidence which influences them.

A sharp credence is often associated with probability distributions
over chances, while an instate puts chances in sets where they all
have an equal voice. This may also be at the bottom of Susanna
Rinard's objection (see \scite{8}{white10}{184}) that Joyce's
committee members are all equally enfranchised and so it is not clear
how extremists among them could not always be replaced by even greater
extremists even after updating on evidence which should serve to
consolidate indeterminacy. Joyce has a satisfactory response to this
objection (see \scite{8}{joyce10}{291}), but I do not see how the
response addresses the problem of aggregating expert opinion without
the kind of summation that Laplaceans find unobjectionable, even
though information is lost and can only be recouped by going back to
the evidence. More generally, the two levels for sharp credences,
representation of uncertainty and distributions over partitions,
tidily differentiate between the epistemic and the evidential
dimension; instates, on the other hand, just add another level of
uncertainty on top of the uncertainty that is already expressed in the
partial belief and thus do not make the appropriate semantic
distinctions.

As we will see in the next example, it is an advantage of sharp
credences that they do not exclude objective chances, even extreme
ones, because they are fully committed to partial belief and do not
suggest, as indeterminate credences do, that there is full belief
knowledge that the objective chance is a member of a proper subset of
the possibilities.

\begin{quotex}
  \textbf{Example \arabic{expls}:
    \addtocounter{expls}{1}\label{monkey} Precise Credences.} Your
  sharp credence for rain tomorrow, based on the expert opinion of
  channel GPY and channel QCT (you have no other information) is
  $0.53$. Is it reasonable, considering how little evidence you have,
  to reject the belief that the chance of rain tomorrow is $0.52$ or
  $0.54$; or to prefer a $52.9$ cent bet on rain to a $47.1$ cent bet
  on no rain?
\end{quotex}

The first question is confused, but in instructive ways (a display of
this confusion is found in \scite{8}{hajeksmithson12}{38f}, and their
doctor and time of the day analogy). A sharp credence rejects no
hypothesis about objective chances (unlike an instate, unless (J2) is
firmly in place). It often has a subjective probability distribution
operating in the background, over which it integrates to yield the
sharp credence (it would do likewise in H{\'a}jek and Smithson's
example for the prognosis of the doctor or the time of the day,
without any problems). \texttt{<x4.2>}This subjective probability
distribution may look like this:

\begin{tabular}{|lcr|}
  \hline
  $P(\pi(R)=0.00)$ & = & $0.0001$ \\ \hline
  $P(\pi(R)=0.01)$ & = & $0.0003$ \\ \hline
  $P(\pi(R)=0.02)$ & = & $0.0007$ \\ \hline
  $\ldots$ & & $\ldots$ \\ \hline
  $P(\pi(R)=0.30)$ & = & $0.0015$ \\ \hline
  $P(\pi(R)=0.31)$ & = & $0.0016$ \\ \hline
  $\ldots$ & & $\ldots$ \\ \hline
  $P(\pi(R)=0.52)$ & = & $0.031$ \\ \hline
  $P(\pi(R)=0.53)$ & = & $0.032$ \\ \hline
  $P(\pi(R)=0.54)$ & = & $0.030$ \\ \hline
  $\ldots$ & & $\ldots$ \\ \hline
\end{tabular}

It is condensed by Lewis' summation formula to a sharp credence,
without being reduced to it:

\begin{equation}
  \label{eq:s2}
  C(R)=\int_{0}^{1}\zeta{}P(\pi(R)=\zeta)\,d\zeta\mbox{ \texttt{</x4.2>}}
\end{equation}

Lewis' 1981 paper \qeins{A Subjectivist's Guide to Objective Chance}
addresses the question what the relationship between $\pi$, $P$, and
$C$ is. The point is that we have properly separated the semantic
dimensions and that the Laplacean approach is not a second order
probability approach. The partial belief epistemology deals with sharp
credences and how they represent uncertainty and serve as a tool in
inference, updating, and decision making; while Lewis' Humean
speculations and his interpretation of the principal principle cover
the relationship between subjective probabilities and objective
chance.

Instates, by contrast, mix these semantic dimensions so that in the
end we get a muddle where a superficial reading of indeterminacy
suddenly follows a converse principal principle of sorts, namely that
objective chances are constrained by the factivity of a rational
agent's credence when this credence is knowledge (Lewis actually talks
about such a converse, but in completely different and
epistemologically more intelligible terms, see
\scite{8}{lewis81}{289}). Sharp credences are more, not less,
permissive with respect to objective chances operating externally
(compared to the internal epistemic state of the agent, which the
credence reflects). By the principle of regularity and in keeping with
statistical practice, all objective chances as possible states of the
world are given positive subjective probabilities, even though they
may be very small. Instates, on the other hand, mix partial belief
epistemology with full belief epistemology and presumably exclude
objective chances which lie outside the credal state from
consideration because they are fully known not to hold (see
\scite{8}{levi81}{540}, \qeins{inference derives credal probability
  from knowledge of the chances of possible outcomes}).

The second question is also instructive: why would we prefer a $52.9$
cent bet on rain to a $47.1$ cent bet on no rain, given that we do not
possess the power of descrimination between these two bets? The answer
to this question ties in with the issue of incomplete preference
structure referred to above as motiviation (B) for instates.

\begin{quotex}
  It hardly seems a requirement of rationality that belief be precise
  (and preferences complete); surely imprecise belief (and
  corresponding incomplete preferences) are at least rationally
  permissible. \scite{3}{bradleysteele13}{2}
\end{quotex}

\texttt{<x4.3>}In personal communication, Yang Liu at Columbia
University posed this problem to me more forcefully: the development
of representation theorems beginning with Frank Ramsey (followed by
increasingly more compelling representation theorems in
\scite{7}{savage54}{}; and \scite{7}{jeffrey65}{}; and numerous other
variants in contemporary literature) puts the horse before the cart
and bases probability and utility functions of an agent on her
preferences, not the other way around. Once completeness as an axiom
for the preferences of an agent is jettisoned, indeterminacy follows
automatically. Indeterminacy may thus be a natural consequence of the
proper way to think about credences in terms of the preferences that
they represent.\texttt{</x4.3>}

In response, preferences may very well logically and psychologically
precede an agent's probability and utility functions, but that does
not mean that we cannot inform the axioms we use for a rational
agent's preferences by undesirable consequences downstream.
Completeness may sound like an unreasonable imposition at the outset,
but if incompleteness has unwelcome semantic consequences for
credences, it is not illegitimate to revisit the issue. Timothy
Williamson goes through this exercise with vague concepts, showing
that all upstream logical solutions to the problem fail and that it
has to be solved downstream with an epistemic solution (see
\scite{7}{williamson96}{}). Vague concepts, like sharp credences, are
sharply bounded, but not in a way that is luminous to the agent (for
anti-luminosity see chapter 4 in \scite{7}{williamson00}{}).
Anti-luminosity answers the original question: the rational agent
prefers the $52.9$ cent bet on rain to a $47.1$ cent bet on no rain
based on her sharp credence without being in a position to have this
preference necessarily or have it based on physical or psychological
ability (for the analogous claim about knowledge see
\scite{8}{williamson00}{95}).

In a way, advocates of indeterminacy have solved this problem for us.
There is strong agreement among most of them that the issue of
determinacy for credences is not an issue of elicitation (sometimes
the term \qnull{indeterminacy} is used instead of \qnull{imprecision}
to underline this difference; see \scite{8}{levi85}{395}). The appeal
of preferences is that we can elicit them more easily than assessments
of probability and utility functions. The indeterminacy issue has been
raised to the probability level (or moved downstream) by indeterminacy
advocates themselves who feel justifiably uncomfortable with an
interpretation of their theory in behaviourist terms. So it shall be
solved there, and this paper makes an appeal to reject indeterminacy
on this level. The solution then has to be carried upstream (or
lowered to the logically more basic level of preferences), where we
recognize that completeness for preferences is after all a desirable
axiom for rationality. Isaac Levi agrees with me on this point: when
he talks about indeterminacy, it proceeds from the level of
probability judgment to preferences, not the other way around (see
\scite{8}{levi81}{533}).

\begin{quotex}
  \textbf{Example \arabic{expls}: \addtocounter{expls}{1}
    Monkey-Filled Urns.} E.T. Jaynes describes an experiment with
  monkeys filling an urn randomly with balls from another urn, for
  which sampling provides no information and so makes updating vacuous
  (see \scite{8}{jaynesbretthorst03}{160}). Here is a variant of this
  experiment for which a sharp credence provides a more compelling
  result than the associated instate: Let urn $A$ contain 4 balls, two
  red and two black. A monkey randomly fills urn $B$ from urn $A$ with
  two balls. We draw from urn $B$.
\end{quotex}

The sharp credence of drawing a red ball is $0.5$, following Lewis'
summation formula for the different combinations of balls in urn $B$.
This solution is more intuitive in terms of further inference,
decision making, and betting behaviour than a credal state of
$\{0,1/2,1\}$ or $[0,1]$ (depending on the convexity requirement),
since this instate would licence an exorbitant bet in favour of one
colour, for example one that costs \$9,999 and pays \$10,000 if red is
drawn and nothing if black is drawn.

\begin{quotex}
  \textbf{Example \arabic{expls}: \addtocounter{expls}{1} Three
    Prisoners.} Prisoner $X_{1}$ knows that two out of three prisoners
  ($X_{1},X_{2},X_{3}$) will be executed and one of them pardoned. He
  asks the warden of the prison to tell him the name of another
  prisoner who will be executed, hoping to gain knowledge about his
  own fate. When the warden tells him that $X_{3}$ will be executed,
  $X_{1}$ erroneously updates his probability of pardon from $1/3$ to
  $1/2$, since either $X_{1}$ or $X_{2}$ will be spared.
\end{quotex}

Walley maintains that for the Monty Hall problem and the Three
Prisoners problem, the probabilities of a rational agent should dilate
rather than settle on the commonly accepted solutions. For the Three
Prisoners problem, there is a compelling case for standard
conditioning and the result that the chances of pardon for prisoner
$X_{1}$ are unchanged after the update (see
\scite{8}{lukits14}{1421f}). Walley's dilated solution would give
prisoner $X_{1}$ hope on the doubtful possibility (and unfounded
assumption) that the warden might prefer to provide $X_{3}$'s (rather
than $X_{2}$'s) name in case prisoner $X_{1}$ was pardoned.

This example brings an interesting issue to the forefront. Sharp
credences often reflect independence of variables where such
independence is unwarranted. Booleans (more specifically, detractors
of the principle of indifference or the principle of maximum entropy,
principles which are used to generate sharp credences for rational
agents) tend to point this out gleefully. They prefer to dilate over
the possible dependence relationships (independence included). White's
dilation problem is an instance of this. The fallacy in the argument
for instates, illustrated by the Three Prisoners problem, is that the
probabilistic independence of sharp credences does not imply
independence of variables (only the converse is correct), but only
that it is unknown whether there is dependence, and if yes, whether it
is correlation or inverse correlation.

In the Three Prisoners problem, there is no evidence about the degree
or the direction of the dependence, and so prisoner $X_{1}$ should
take no comfort in the information that she receives. The prisoner's
probabilities will reflect probabilistic independence, but make no
claims about causal independence. Walley has unkind things to say
about sharp credences and their ability to respond to evidence (for
example that their \qeins{inferences rarely conform to evidence}, see
\scite{8}{walley91}{396}), but in this case it appears to me that they
outperform the Boolean approach.

\begin{quotex}
  \textbf{Example \arabic{expls}: \addtocounter{expls}{1} Wagner's
    Linguist.} A linguist hears the utterance of a native and
  concludes that the native cannot be part of certain population
  groups, depending on what the utterance means. The linguist is
  uncertain between some options about the meaning of the utterance.
  (For full details see \scite{8}{wagner92}{252}; and
  \scite{8}{spohn12}{197}.)
\end{quotex}

The mathematician Carl Wagner proposed a natural generalization of
Jeffrey Conditioning for his Linguist example (see
\scite{7}{wagner92}{}). Since the principle of maximum entropy is
already a generalization of Jeffrey Conditioning, the question
naturally arises whether the two generalizations agree. Wagner makes
the case that they do not agree and deduces that the principle of
maximum entropy is sometimes an inappropriate updating mechanism, in
line with many earlier criticisms of the principle of maximum entropy
(see van Fraassen,\fixref{7}{fraassen81}{} 1981;
\scite{7}{shimony85}{}; \scite{7}{skyrms87updating}{}; and, later on,
\scite{7}{grovehalpern97}{}). What is interesting about this case is
that Wagner uses instates for his deduction, so that even if you agree
with his natural generalization of Jeffrey Conditioning (which I find
plausible), the inconsistency with the principle of maximum entropy
can only be inferred assuming instates. Wagner is unaware of this, and
I am showing in another paper (in process) how on the assumption of
sharp credences Wagner's generalization of Jeffrey conditioning
accords with the principle of maximum entropy.

This will not convince Booleans, since they are already unlikely to
believe in the general applicability of the principle of maximum
entropy (just as Wagner's argument is unlikely to convince a proponent
of the principle of maximum entropy, since they have a tendency to
reject instates). The battle lines are clearly drawn. Wagner's
argument, instead of undermining the principle of maximum entropy,
just shows that instates are as wedded to rejecting the claims of the
principle of maximum entropy as the principle of maximum entropy is
wedded to sharp credences.

Endorsement of instates, however, implies that there are situations of
probability update in which the posterior probability distribution is
more informative than it might be in terms of information theory.
Indeterminate credences violate the relatively natural intuition that
we should not gain information from evidence when a less informative
updated probability will do the job of responding to the evidence.
This is not a strong argument in favour of sharp credences. The
principle of maximum of entropy has received a thorough bashing in the
last forty years. I consider it to be much easier to convince someone
to reject instates on independent (semantic) grounds than to convince
them to give the principle of maximum entropy a second chance. But the
section on semantics comes to an end here, and we want to proceed to
the intriguing issue of who does better in betting situations:
instates or sharp credences.

\section{Evidence Differentials and Cushioning Credences}
\label{WalleysWorldCupWoes}

I have given away the answer already in the introduction: instates do
better. It is surprising that, except for a rudimentary allusion to
this in Walley's book, no Boolean has caught on to this yet. After I
found out that agents with instates do better betting on soccer games,
I let Betsy and Linda play a more basic betting game. An $n$-sided die
is rolled (by the computer). The die is fair, unbeknownst to the
players. Their bets are randomly and uniformly drawn from the simplex
for which the probabilities attributed to the $n$ results add up to 1.
Betsy also surrounds her credences with an imprecision uniformly drawn
from the interval $(0,y)$. I used Walley's pay off scheme (see
\scite{8}{walley91}{632}) to settle the bets.

\texttt{<x5.1>}Here is an example: let $n=2$, so the die is a fair
\textsc{coin}. $X$'s and $Y$'s bets are randomly and uniformly drawn
from the line segment from $(0,1)$ to $(1,0)$ (these are
two-dimensional Cartesian coordinates), the two-dimensional simplex
(for higher $n$, the simplex is a pentatope generalized for $n$
dimensions with side length $2^{1/2}$). The bets may be $(0.21,0.79)$
for Linda and $(0.35\pm{}0.11,0.65\pm{}0.11)$ for Betsy, where the
indeterminacy $\pm{}0.11$ is also randomly and uniformly drawn from
the imprecision interval $(0,y)\subseteq(0,1)$. The first bet is on
$H$, and Betsy is willing to pay $22.5$ cents for it, while $X$ is
willing to pay $77.5$ cents against it. The second bet is on $T$ (if
$n>2$, there will not be the same symmetry as in the \textsc{coin}
case between the two bets), for which Linda is willing to pay $77.5$
cents, and against which Betsy is willing to pay $22.5$ cents. Each
bet pays \$1 if successful. Often, $Y$'s credal state will overlap
with $X$'s sharp credence so that there will not be a
bet.\texttt{</x5.1>}\fcut{7}

The computer simulation clearly shows that Betsy does better. A
defence of sharp credences for rational agents needs to have an
explanation for this. We will call it partial belief cushioning, which
is based on an evidence differential between the bettors.

In many decision-making contexts, we do not have the luxury of calling
off the bet. We have to decide one way or another. This is a problem
for instates, as Booleans have to find a way to decide without
receiving instructions from the credal state. Booleans have addressed
this point extensively (see for example \scite{8}{joyce10}{311ff}; for
an opponent's view of this see \scite{8}{elga10}{6ff}). The problem
for sharp credences arises when bets are noncompulsory, for then the
data above suggest that agents holding instates systematically do
better. Often, decision making happens as betting vis-{\`a}-vis
uninformed nature or opponents which are at least as uninformed as the
rational agent. Sometimes, however, bets are offered by better
informed or potentially better informed bookies. In this case, even an
agent with sharp credences must cushion her credences and is better
off by rejecting bets that look attractive in terms of her partial
beliefs.

If an agent does not cushion her partial beliefs (whether they are
sharp or indeterminate), she will incur a loss in the long run. Since
cushioning is permitted in Walley's experimental setup (the bets are
noncompulsory), Laplacean agents should also have access to it and
then no longer do worse than Boolean agents. One may ask what sharp
credences do if they just end up being cushioned anyway and do not
provide sufficient information to decide on rational bets. The answer
is that sharp credences are sufficient where betting (or decision
making more generally) is compulsory; the cushioning only supplies the
information from the evidence inasmuch as betting is noncompulsory and
so again properly distinguishes semantic categories. This task is much
harder for Booleans, although I do not claim that it is
insurmountable: instates can provide a coherent approach to compulsory
betting. What they cannot do, once cushioning is introduced, is
outperform sharp credences in noncompulsory betting situations.

Here are a few examples: even if I have little evidence on which to
base my opinion, someone may force me to either buy Coca Cola shares
or short them, and so I have to have a share price $p$ in mind that I
consider fair. I will buy Coca Cola shares for less than $p$, and
short them for more than $p$, if forced to do one or the other. This
does not mean that it is now reasonable for me to go (not forced by
anyone) and buy Coca Cola shares for $p$. It may not even be
reasonable to go (not forced by anyone) and buy Coca Cola share for
$p-\delta$ with $\delta{}>0$.

It may in fact be quite unreasonable, since there are many players who
have much better evidence than I do and will exploit my ignorance. I
suspect that most lay investors in the stock market make this mistake:
even though they buy and sell stock at prices that seem reasonable to
them, professional investors are much better and faster at exploiting
arbitrage opportunities and more subtle regularities. If indices rise,
lay investors will make a little less than their professional
counterparts; and when they fall, lay investors lose a lot more. In
sum, unless there is sustained growth and everybody wins, lay
investors lose in the long term.

A case in point is the U.S. Commodity Futures Trading Commission's
crackdown on the online prediction market Intrade. Intrade offered
fair bets for or against events of public significance, such as
election results or other events which had clear yes-or-no outcomes.
Even though the bets were all fair and Intrade only received a small
commission on all bets, and even though Intrade's predictions were
remarkably accurate, the potential for professional arbitrageurs was
too great and the CFTC shut Intrade down (see
\texttt{https://www.intrade.com}).

Cushioning does not stand in the way of holding a sharp credence, even
if the evidence is dim. The evidence determines for a rational agent
the partial beliefs over possible states of the world operating in the
background. The better the evidence, the more pointed the
distributions of these partial beliefs will be and the more willing
the rational agent will be to enter a bet, if betting is
noncompulsory. The mathematical decision rule will be based on the
underlying distribution of the partial beliefs, not only on the sharp
credence. As we have stated before, the sharp credence is not a
sufficient statistic for decision making, inference, or betting
behaviour; and neither is an instate.

The rational agent with a sharp credence has resources at her disposal
to use just as much differentiation with respect to accepting and
rejecting bets as the agent with instates. Often (if she is able to
and especially if the bets are offered to her by a better-informed
agent), she will reject both of two complementary bets, even when they
are fair. On the one hand, any advantage that the agent with an
instate has over her can be counteracted based on her distribution
over partial beliefs that she has with respect to all possibilities.
On the other hand, the agent with instates suffers from a semantic
mixing of metaphors between evidential and epistemic dimensions that
puts her at a real disadvantage in terms of understanding the sources
and consequences of her knowledge and her uncertainties.

\section{References}
\label{References}

% \nocite{*} 
\bibliographystyle{ChicagoReedweb} 
\bibliography{bib-7293}

\end{document} 
