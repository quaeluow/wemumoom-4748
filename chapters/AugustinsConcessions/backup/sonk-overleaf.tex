%\title{IEEE for journals template with bibtex example files included}
%% bare_jrnl.tex
%% V1.4
%% 2012/12/27
%% by Michael Shell
%% see http://www.michaelshell.org/
%% for current contact information.
%%
%% This is a skeleton file demonstrating the use of IEEEtran.cls
%% (requires IEEEtran.cls version 1.8 or later) with an IEEE journal paper.
%%
%% Support sites:
%% http://www.michaelshell.org/tex/ieeetran/
%% http://www.ctan.org/tex-archive/macros/latex/contrib/IEEEtran/
%% and
%% http://www.ieee.org/



% *** Authors should verify (and, if needed, correct) their LaTeX system  ***
% *** with the testflow diagnostic prior to trusting their LaTeX platform ***
% *** with production work. IEEE's font choices can trigger bugs that do  ***
% *** not appear when using other class files.                            ***
% The testflow support page is at:
% http://www.michaelshell.org/tex/testflow/


%%*************************************************************************
%% Legal Notice:
%% This code is offered as-is without any warranty either expressed or
%% implied; without even the implied warranty of MERCHANTABILITY or
%% FITNESS FOR A PARTICULAR PURPOSE! 
%% User assumes all risk.
%% In no event shall IEEE or any contributor to this code be liable for
%% any damages or losses, including, but not limited to, incidental,
%% consequential, or any other damages, resulting from the use or misuse
%% of any information contained here.
%%
%% All comments are the opinions of their respective authors and are not
%% necessarily endorsed by the IEEE.
%%
%% This work is distributed under the LaTeX Project Public License (LPPL)
%% ( http://www.latex-project.org/ ) version 1.3, and may be freely used,
%% distributed and modified. A copy of the LPPL, version 1.3, is included
%% in the base LaTeX documentation of all distributions of LaTeX released
%% 2003/12/01 or later.
%% Retain all contribution notices and credits.
%% ** Modified files should be clearly indicated as such, including  **
%% ** renaming them and changing author support contact information. **
%%
%% File list of work: IEEEtran.cls, IEEEtran_HOWTO.pdf, bare_adv.tex,
%%                    bare_conf.tex, bare_jrnl.tex, bare_jrnl_compsoc.tex,
%%                    bare_jrnl_transmag.tex
%%*************************************************************************

% Note that the a4paper option is mainly intended so that authors in
% countries using A4 can easily print to A4 and see how their papers will
% look in print - the typesetting of the document will not typically be
% affected with changes in paper size (but the bottom and side margins will).
% Use the testflow package mentioned above to verify correct handling of
% both paper sizes by the user's LaTeX system.
%
% Also note that the "draftcls" or "draftclsnofoot", not "draft", option
% should be used if it is desired that the figures are to be displayed in
% draft mode.
%
\documentclass[journal]{IEEEtran}
%
% If IEEEtran.cls has not been installed into the LaTeX system files,
% manually specify the path to it like:
% \documentclass[journal]{../sty/IEEEtran}

% Some very useful LaTeX packages include:
% (uncomment the ones you want to load)


% *** MISC UTILITY PACKAGES ***
%
%\usepackage{ifpdf}
% Heiko Oberdiek's ifpdf.sty is very useful if you need conditional
% compilation based on whether the output is pdf or dvi.
% usage:
% \ifpdf
%   % pdf code
% \else
%   % dvi code
% \fi
% The latest version of ifpdf.sty can be obtained from:
% http://www.ctan.org/tex-archive/macros/latex/contrib/oberdiek/
% Also, note that IEEEtran.cls V1.7 and later provides a builtin
% \ifCLASSINFOpdf conditional that works the same way.
% When switching from latex to pdflatex and vice-versa, the compiler may
% have to be run twice to clear warning/error messages.






% *** CITATION PACKAGES ***
%
\usepackage{cite}
% cite.sty was written by Donald Arseneau
% V1.6 and later of IEEEtran pre-defines the format of the cite.sty package
% \cite{} output to follow that of IEEE. Loading the cite package will
% result in citation numbers being automatically sorted and properly
% "compressed/ranged". e.g., [1], [9], [2], [7], [5], [6] without using
% cite.sty will become [1], [2], [5]--[7], [9] using cite.sty. cite.sty's
% \cite will automatically add leading space, if needed. Use cite.sty's
% noadjust option (cite.sty V3.8 and later) if you want to turn this off
% such as if a citation ever needs to be enclosed in parenthesis.
% cite.sty is already installed on most LaTeX systems. Be sure and use
% version 4.0 (2003-05-27) and later if using hyperref.sty. cite.sty does
% not currently provide for hyperlinked citations.
% The latest version can be obtained at:
% http://www.ctan.org/tex-archive/macros/latex/contrib/cite/
% The documentation is contained in the cite.sty file itself.






% *** GRAPHICS RELATED PACKAGES ***
%
\ifCLASSINFOpdf
  % \usepackage[pdftex]{graphicx}
  % declare the path(s) where your graphic files are
  % \graphicspath{{../pdf/}{../jpeg/}}
  % and their extensions so you won't have to specify these with
  % every instance of \includegraphics
  % \DeclareGraphicsExtensions{.pdf,.jpeg,.png}
\else
  % or other class option (dvipsone, dvipdf, if not using dvips). graphicx
  % will default to the driver specified in the system graphics.cfg if no
  % driver is specified.
  % \usepackage[dvips]{graphicx}
  % declare the path(s) where your graphic files are
  % \graphicspath{{../eps/}}
  % and their extensions so you won't have to specify these with
  % every instance of \includegraphics
  % \DeclareGraphicsExtensions{.eps}
\fi
% graphicx was written by David Carlisle and Sebastian Rahtz. It is
% required if you want graphics, photos, etc. graphicx.sty is already
% installed on most LaTeX systems. The latest version and documentation
% can be obtained at: 
% http://www.ctan.org/tex-archive/macros/latex/required/graphics/
% Another good source of documentation is "Using Imported Graphics in
% LaTeX2e" by Keith Reckdahl which can be found at:
% http://www.ctan.org/tex-archive/info/epslatex/
%
% latex, and pdflatex in dvi mode, support graphics in encapsulated
% postscript (.eps) format. pdflatex in pdf mode supports graphics
% in .pdf, .jpeg, .png and .mps (metapost) formats. Users should ensure
% that all non-photo figures use a vector format (.eps, .pdf, .mps) and
% not a bitmapped formats (.jpeg, .png). IEEE frowns on bitmapped formats
% which can result in "jaggedy"/blurry rendering of lines and letters as
% well as large increases in file sizes.
%
% You can find documentation about the pdfTeX application at:
% http://www.tug.org/applications/pdftex





% *** MATH PACKAGES ***
%
%\usepackage[cmex10]{amsmath}
% A popular package from the American Mathematical Society that provides
% many useful and powerful commands for dealing with mathematics. If using
% it, be sure to load this package with the cmex10 option to ensure that
% only type 1 fonts will utilized at all point sizes. Without this option,
% it is possible that some math symbols, particularly those within
% footnotes, will be rendered in bitmap form which will result in a
% document that can not be IEEE Xplore compliant!
%
% Also, note that the amsmath package sets \interdisplaylinepenalty to 10000
% thus preventing page breaks from occurring within multiline equations. Use:
%\interdisplaylinepenalty=2500
% after loading amsmath to restore such page breaks as IEEEtran.cls normally
% does. amsmath.sty is already installed on most LaTeX systems. The latest
% version and documentation can be obtained at:
% http://www.ctan.org/tex-archive/macros/latex/required/amslatex/math/





% *** SPECIALIZED LIST PACKAGES ***
%
%\usepackage{algorithmic}
% algorithmic.sty was written by Peter Williams and Rogerio Brito.
% This package provides an algorithmic environment fo describing algorithms.
% You can use the algorithmic environment in-text or within a figure
% environment to provide for a floating algorithm. Do NOT use the algorithm
% floating environment provided by algorithm.sty (by the same authors) or
% algorithm2e.sty (by Christophe Fiorio) as IEEE does not use dedicated
% algorithm float types and packages that provide these will not provide
% correct IEEE style captions. The latest version and documentation of
% algorithmic.sty can be obtained at:
% http://www.ctan.org/tex-archive/macros/latex/contrib/algorithms/
% There is also a support site at:
% http://algorithms.berlios.de/index.html
% Also of interest may be the (relatively newer and more customizable)
% algorithmicx.sty package by Szasz Janos:
% http://www.ctan.org/tex-archive/macros/latex/contrib/algorithmicx/




% *** ALIGNMENT PACKAGES ***
%
%\usepackage{array}
% Frank Mittelbach's and David Carlisle's array.sty patches and improves
% the standard LaTeX2e array and tabular environments to provide better
% appearance and additional user controls. As the default LaTeX2e table
% generation code is lacking to the point of almost being broken with
% respect to the quality of the end results, all users are strongly
% advised to use an enhanced (at the very least that provided by array.sty)
% set of table tools. array.sty is already installed on most systems. The
% latest version and documentation can be obtained at:
% http://www.ctan.org/tex-archive/macros/latex/required/tools/


% IEEEtran contains the IEEEeqnarray family of commands that can be used to
% generate multiline equations as well as matrices, tables, etc., of high
% quality.




% *** SUBFIGURE PACKAGES ***
%\ifCLASSOPTIONcompsoc
%  \usepackage[caption=false,font=normalsize,labelfont=sf,textfont=sf]{subfig}
%\else
%  \usepackage[caption=false,font=footnotesize]{subfig}
%\fi
% subfig.sty, written by Steven Douglas Cochran, is the modern replacement
% for subfigure.sty, the latter of which is no longer maintained and is
% incompatible with some LaTeX packages including fixltx2e. However,
% subfig.sty requires and automatically loads Axel Sommerfeldt's caption.sty
% which will override IEEEtran.cls' handling of captions and this will result
% in non-IEEE style figure/table captions. To prevent this problem, be sure
% and invoke subfig.sty's "caption=false" package option (available since
% subfig.sty version 1.3, 2005/06/28) as this is will preserve IEEEtran.cls
% handling of captions.
% Note that the Computer Society format requires a larger sans serif font
% than the serif footnote size font used in traditional IEEE formatting
% and thus the need to invoke different subfig.sty package options depending
% on whether compsoc mode has been enabled.
%
% The latest version and documentation of subfig.sty can be obtained at:
% http://www.ctan.org/tex-archive/macros/latex/contrib/subfig/




% *** FLOAT PACKAGES ***
%
%\usepackage{fixltx2e}
% fixltx2e, the successor to the earlier fix2col.sty, was written by
% Frank Mittelbach and David Carlisle. This package corrects a few problems
% in the LaTeX2e kernel, the most notable of which is that in current
% LaTeX2e releases, the ordering of single and double column floats is not
% guaranteed to be preserved. Thus, an unpatched LaTeX2e can allow a
% single column figure to be placed prior to an earlier double column
% figure. The latest version and documentation can be found at:
% http://www.ctan.org/tex-archive/macros/latex/base/


%\usepackage{stfloats}
% stfloats.sty was written by Sigitas Tolusis. This package gives LaTeX2e
% the ability to do double column floats at the bottom of the page as well
% as the top. (e.g., "\begin{figure*}[!b]" is not normally possible in
% LaTeX2e). It also provides a command:
%\fnbelowfloat
% to enable the placement of footnotes below bottom floats (the standard
% LaTeX2e kernel puts them above bottom floats). This is an invasive package
% which rewrites many portions of the LaTeX2e float routines. It may not work
% with other packages that modify the LaTeX2e float routines. The latest
% version and documentation can be obtained at:
% http://www.ctan.org/tex-archive/macros/latex/contrib/sttools/
% Do not use the stfloats baselinefloat ability as IEEE does not allow
% \baselineskip to stretch. Authors submitting work to the IEEE should note
% that IEEE rarely uses double column equations and that authors should try
% to avoid such use. Do not be tempted to use the cuted.sty or midfloat.sty
% packages (also by Sigitas Tolusis) as IEEE does not format its papers in
% such ways.
% Do not attempt to use stfloats with fixltx2e as they are incompatible.
% Instead, use Morten Hogholm'a dblfloatfix which combines the features
% of both fixltx2e and stfloats:
%
% \usepackage{dblfloatfix}
% The latest version can be found at:
% http://www.ctan.org/tex-archive/macros/latex/contrib/dblfloatfix/




%\ifCLASSOPTIONcaptionsoff
%  \usepackage[nomarkers]{endfloat}
% \let\MYoriglatexcaption\caption
% \renewcommand{\caption}[2][\relax]{\MYoriglatexcaption[#2]{#2}}
%\fi
% endfloat.sty was written by James Darrell McCauley, Jeff Goldberg and 
% Axel Sommerfeldt. This package may be useful when used in conjunction with 
% IEEEtran.cls'  captionsoff option. Some IEEE journals/societies require that
% submissions have lists of figures/tables at the end of the paper and that
% figures/tables without any captions are placed on a page by themselves at
% the end of the document. If needed, the draftcls IEEEtran class option or
% \CLASSINPUTbaselinestretch interface can be used to increase the line
% spacing as well. Be sure and use the nomarkers option of endfloat to
% prevent endfloat from "marking" where the figures would have been placed
% in the text. The two hack lines of code above are a slight modification of
% that suggested by in the endfloat docs (section 8.4.1) to ensure that
% the full captions always appear in the list of figures/tables - even if
% the user used the short optional argument of \caption[]{}.
% IEEE papers do not typically make use of \caption[]'s optional argument,
% so this should not be an issue. A similar trick can be used to disable
% captions of packages such as subfig.sty that lack options to turn off
% the subcaptions:
% For subfig.sty:
% \let\MYorigsubfloat\subfloat
% \renewcommand{\subfloat}[2][\relax]{\MYorigsubfloat[]{#2}}
% However, the above trick will not work if both optional arguments of
% the \subfloat command are used. Furthermore, there needs to be a
% description of each subfigure *somewhere* and endfloat does not add
% subfigure captions to its list of figures. Thus, the best approach is to
% avoid the use of subfigure captions (many IEEE journals avoid them anyway)
% and instead reference/explain all the subfigures within the main caption.
% The latest version of endfloat.sty and its documentation can obtained at:
% http://www.ctan.org/tex-archive/macros/latex/contrib/endfloat/
%
% The IEEEtran \ifCLASSOPTIONcaptionsoff conditional can also be used
% later in the document, say, to conditionally put the References on a 
% page by themselves.




% *** PDF, URL AND HYPERLINK PACKAGES ***
%
%\usepackage{url}
% url.sty was written by Donald Arseneau. It provides better support for
% handling and breaking URLs. url.sty is already installed on most LaTeX
% systems. The latest version and documentation can be obtained at:
% http://www.ctan.org/tex-archive/macros/latex/contrib/url/
% Basically, \url{my_url_here}.




% *** Do not adjust lengths that control margins, column widths, etc. ***
% *** Do not use packages that alter fonts (such as pslatex).         ***
% There should be no need to do such things with IEEEtran.cls V1.6 and later.
% (Unless specifically asked to do so by the journal or conference you plan
% to submit to, of course. )


% correct bad hyphenation here
\hyphenation{op-tical net-works semi-conduc-tor}


%%% BEG OCTOBER
% \setlength{\marginparwidth}{1.2in}
% \let\oldmarginpar\marginpar
% \renewcommand\marginpar[1]{\-\oldmarginpar[\raggedleft\footnotesize #1]%
% {\raggedright\footnotesize #1}}

% \frenchspacing % no extra space at the end of a sentence

% \setlength{\parskip}{.1in}
% \raggedright
% \raggedbottom
% \setlength{\parindent}{.2in}
% \setlength{\parindent}{0in}

% \pagestyle{empty}

% 	PACKAGES
% \usepackage[small,bf]{caption}
% \let\bcode\textbgreek
% \usepackage[bgreek,english]{babel}
\usepackage{setspace}
\usepackage{amsfonts}
\usepackage{amssymb}
\usepackage[fleqn]{amsmath}
\usepackage[table]{xcolor}
\definecolor{lightgray}{gray}{0.9}
% \usepackage{german}
% \usepackage{hebtex}
\usepackage{graphicx}
% \usepackage[german]{babel}
% \usepackage{endnotes}
% \let\footnote=\endnote
% \usepackage{rotating}
\usepackage{enumerate}
% \usepackage{titlesec}
% \titleformat{\section}{\large\bfseries}{\thesection}{1em}{}

\newcommand{\qnull}[1]{`#1'}
\newcommand{\doxnotep}[0]{doxastic}
\newcommand{\qeins}[1]{``#1''}
\newcommand{\qzwei}[1]{`#1'}
\newcommand{\fcut}[1]{}
\newcommand{\bcut}[1]{}
\newcommand{\tbd}[1]{$^{\mbox{\textsc{tbd}}}$}
\newcommand{\nias}{\noindent} % no indent after new section
\newcommand{\nial}{\noindent} % no indent after equation, list, or whatever

\newif\ifNumericalOrYear
\NumericalOrYeartrue
\NumericalOrYearfalse
\ifNumericalOrYear
\usepackage[numbers,colon]{natbib}
\else
\usepackage[round,colon]{natbib}
\fi
\newif\ifPageP
\PagePtrue
\PagePfalse
\ifPageP
\newcommand{\PageP}{p.~}
\else
\newcommand{\PageP}{}
\fi

\newcommand{\fixref}[3]{}
\newcommand{\scite}[3]{\ifnum#1=1\ifNumericalOrYear\citep{#2}\else\citeyearpar{#2}\fi\else
\ifnum#1=2\ifNumericalOrYear\citep[#3]{#2}\else\citep[{\PageP}#3]{#2}\fi\else
% \ifnum#1=3\ifNumericalOrYear(\citet[#3]{#2})\else\citep[{\PageP}#3]{#2}\fi\else
% you may want to use the next line as a template for the others as well <2014-08-29 Fri>
\ifnum#1=3\ifNumericalOrYear(\citet[#3]{#2})\else(\citeauthor{#2}, \citeyear{#2}, {\PageP}#3.)\fi\else
\ifnum#1=4\ifNumericalOrYear\citet{#2}\else\citet{#2}\fi\else
\ifnum#1=5\ifNumericalOrYear(\citet{#2})\else\citep{#2}\fi\else
\ifnum#1=6\ifNumericalOrYear(\citet[#3]{#2})\else\citep[{\PageP}#3]{#2}\fi\else
\ifnum#1=7\ifNumericalOrYear\citep{#2}\else\citealp{#2}\fi\else
\ifnum#1=8\ifNumericalOrYear\citep[#3]{#2}\else\citealp[{\PageP}#3]{#2}\fi\else
\ifnum#1=9\ifNumericalOrYear\citep[#3]{#2}\else{}loc.\ cit., {\PageP}#3\fi\else
\ifnum#1=10\ifNumericalOrYear\citep{#2}\else\citeyear{#2}\fi\else
\textbf{[invalid scite code]}\fi\fi\fi\fi\fi\fi\fi\fi\fi\fi}

\newenvironment{quotex}{\begin{quote}\begin{footnotesize}}{\end{footnotesize}\end{quote}}
%%% END OCTOBER
\newcounter{expls}

\begin{document}
%
% paper title
% can use linebreaks \\ within to get better formatting as desired
% Do not put math or special symbols in the title.
\title{Semantics of Not Knowing}
%
%
% author names and IEEE memberships
% note positions of commas and nonbreaking spaces ( ~ ) LaTeX will not break
% a structure at a ~ so this keeps an author's name from being broken across
% two lines.
% use \thanks{} to gain access to the first footnote area
% a separate \thanks must be used for each paragraph as LaTeX2e's \thanks
% was not built to handle multiple paragraphs
%

\author{Stefan~Lukits}% <-this % stops a space
% \thanks{M. Shell is with the Department
% of Electrical and Computer Engineering, Georgia Institute of Technology, Atlanta,
% GA, 30332 USA e-mail: (see http://www.michaelshell.org/contact.html).}% <-this % stops a space
% \thanks{J. Doe and J. Doe are with Anonymous University.}% <-this % stops a space
% \thanks{Manuscript received April 19, 2005; revised December 27, 2012.}}

% note the % following the last \IEEEmembership and also \thanks - 
% these prevent an unwanted space from occurring between the last author name
% and the end of the author line. i.e., if you had this:
% 
% \author{....lastname \thanks{...} \thanks{...} }
%                     ^------------^------------^----Do not want these spaces!
%
% a space would be appended to the last name and could cause every name on that
% line to be shifted left slightly. This is one of those "LaTeX things". For
% instance, "\textbf{A} \textbf{B}" will typeset as "A B" not "AB". To get
% "AB" then you have to do: "\textbf{A}\textbf{B}"
% \thanks is no different in this regard, so shield the last } of each \thanks
% that ends a line with a % and do not let a space in before the next \thanks.
% Spaces after \IEEEmembership other than the last one are OK (and needed) as
% you are supposed to have spaces between the names. For what it is worth,
% this is a minor point as most people would not even notice if the said evil
% space somehow managed to creep in.



% The paper headers
% \markboth{For Submission}%
% {Shell \MakeLowercase{\textit{et al.}}: Bare Demo of IEEEtran.cls for Journals}
% The only time the second header will appear is for the odd numbered pages
% after the title page when using the twoside option.
% 
% *** Note that you probably will NOT want to include the author's ***
% *** name in the headers of peer review papers.                   ***
% You can use \ifCLASSOPTIONpeerreview for conditional compilation here if
% you desire.




% If you want to put a publisher's ID mark on the page you can do it like
% this:
%\IEEEpubid{0000--0000/00\$00.00~\copyright~2012 IEEE}
% Remember, if you use this you must call \IEEEpubidadjcol in the second
% column for its text to clear the IEEEpubid mark.



% use for special paper notices
%\IEEEspecialpapernotice{(Invited Paper)}




% make the title area
\maketitle

% As a general rule, do not put math, special symbols or citations
% in the abstract or keywords.
\begin{abstract}
  Many Bayesian epistemologists now accept
  that it is not necessary for a rational agent to hold sharp
  credences. There are various compelling formal theories how such a
  non-traditional view of credences can accommodate decision making
  and updating. They are motivated by a common complaint: that sharp
  credences can fail to represent incomplete evidence and exaggerate
  the information contained in it. I show that the critical response
  of those who want to maintain the traditional view with only sharp
  credences is insufficient. Their criticism, however, opens the door
  to semantic questions which make indeterminate credences vulnerable
  on a deeper level. This paper uses both conceptual arguments and
  hands-on examples to demonstrate that rational agents always have
  sharp credences. Moreover, it is articulated and substantiated that
  agents with indeterminate credal states make long term gains betting
  against agents with sharp credences. This on the surface
  inconvenient fact can be explained and does not favour indeterminate
  credences over sharp credences.\tbd{rewrite abstract}
\end{abstract}

% Note that keywords are not normally used for peerreview papers.
% \begin{IEEEkeywords}
% IEEEtran, journal, \LaTeX, paper, template.
% \end{IEEEkeywords}






% For peer review papers, you can put extra information on the cover
% page as needed:
% \ifCLASSOPTIONpeerreview
% \begin{center} \bfseries EDICS Category: 3-BBND \end{center}
% \fi
%
% For peerreview papers, this IEEEtran command inserts a page break and
% creates the second title. It will be ignored for other modes.
% \IEEEpeerreviewmaketitle

\section{Introduction}
\label{Introduction}

Traditionally, Bayesians have maintained that a rational agent, when
holding a credence, holds a sharp credence. It has recently become
popular to drop the requirement for credence functions to be sharp.
There are now Bayesians who permit a rational agent to hold
indeterminate credal states (from now on instates) based on incomplete
or ambiguous evidence. I will refer to Bayesians who continue to
adhere to the classical theory of sharp credences for rational agents
as \qnull{Laplaceans} (e.g.\ Adam Elga and Roger White). I will refer
to Bayesians who do not believe that a rational agent's credences are
sharp as \qnull{Booleans} (e.g.\ Peter Walley and James
Joyce).\fcut{1}

There is some terminological confusion around the adjectives
\qnull{imprecise,} \qnull{indeterminate,} and \qnull{mushy} credences.
In the following, I will exclusively refer to indeterminate credences
or credal states (abbreviated \qnull{instates}) and mean by them a set
of sharp credence functions (which some Booleans require to be convex)
which it may be rational for an agent to hold within an otherwise
orthodox Bayesian framework.\bcut{1}

There is a sense in which, by linking knowledge of chances to its
reflection in credences, Booleans seek to reconcile traditional
knowledge epistemology concerned with full belief and formal
epistemology concerned with partial belief. There are other more
recent reconciliation projects (see Spohn, 2012; and Moss, 2013). If
my paper is correct then the Boolean approach will not contribute to
this reconciliation because it mixes full belief and partial belief
metaphors in ways that are problematic.\tbd{rephrase}\fcut{2}\bcut{2}

The main complaint of this paper is that instates represent in one
credence both degree of belief and properties of the evidence.
Instates thus incorporate both {\doxnotep} and evidential features of
the credal state. Representing multiple features of a state is not per
se a bad thing when we fix the terms of a theory, in this case the
terms of our theory about beliefs. In colour theory, the term
\qnull{red} represents both a phenomenological quality and a
neighbourhood on the visible light spectrum. When we say one thing is
more red than another thing, we effectively describe a relation based
on both phenomenological and physical properties.

My examples and conceptual arguments show that instates, in spite of
how plausible they initially sound in representing credal states, fail
to give us a coherent terminology in our theory about beliefs. It
turns out that we want to separate the {\doxnotep} and evidential
components, just as the scientific mineralogist wants to separate
jadeite and nephrite instead of lumping them together under the single
concept of \qnull{jade.} 

When we first hear of the advantages of instates, two of them sound
particularly compelling.

\begin{itemize}
\item [\textsc{range}] Instates represent the possibility range for objective
  chances.
\item [\textsc{incomp}] Instates represent incompleteness or ambiguity of the
  evidence.\bcut{3}
\end{itemize}

Let a \textit{coin} be a Bernoulli generator that produces successes
and failures with probability $p$ for success, labeled $H$, and $1-p$
for failure, labeled $T$. Physical coins may serve as examples, if we
are willing to set aside that most of them are approximately fair.

\addtocounter{expls}{1}

\begin{quotex}
  \textbf{Example \arabic{expls}: \addtocounter{expls}{1} Range.} You
  have two Bernoulli Generators, \textit{coin}$_{a}$ and
  \textit{coin}$_{b}$. You have a database of \textit{coin}$_{a}$
  results and conclude on excellent evidence that \textit{coin}$_{a}$
  is fair. You have no evidence about the bias of \textit{coin}$_{b}$.
  As a Boolean, you assume a sharp credence of $\{0.5\}$ for success
  on the next toss of \textit{coin}$_{a}$ and an indeterminate credal
  state of $[0,1]$ for success on the next toss of
  \textit{coin}$_{b}$. You feel bad for your Laplacean colleague who
  cannot distinguish between the two cases and who must assign a sharp
  credence of $\{0.5\}$ for success on the next toss of both
  \textit{coin}$_{a}$ and \textit{coin}$_{b}$.
\end{quotex}

\begin{quotex}
  \textbf{Example \arabic{expls}: \addtocounter{expls}{1} Incomp.} You
  have a Bernoulli Generator, \textit{coin}$_{c}$. A graduate student
  has submitted \textit{coin}$_{c}$ to countless experiments and
  emailed you the resulting bias, but failed to include whether the
  bias of $2/3$ was in favour of $H$ or in favour of $T$. As a
  Boolean, you assume an indeterminate credal state of $[1/3,2/3]$ (or
  $\{1/3,2/3\}$) for success on the next toss of \textit{coin}$_{c}$.
  You feel bad for your Laplacean colleague who must assign a sharp
  credence of $\{0.5\}$ for success on the next toss of
  \textit{coin}$_{c}$ and who concurrently knows that her credence
  gets the bias wrong.
\end{quotex}

Here are a few concise statements by Booleans:

\begin{quotex}
  [A] refusal to make a determinate probability judgment does not
  derive from a lack of clarity about one's credal state. To the
  contrary, it may derive from a very clear and cool judgment that on
  the basis of the available evidence, making a numerically
  determinate judgment would be unwarranted and arbitrary.
  \scite{3}{levi85}{395}
\end{quotex}

\begin{quotex}
  If there is little evidence concerning [a claim,] then beliefs about
  [that claim] should be indeterminate, and probability models
  imprecise, to reflect the lack of information. We regard this as the
  most important source of imprecision. \scite{3}{walley91}{212--213}
\end{quotex}

\begin{quotex}
  Imprecise probabilities and related concepts [{\ldots}] provide a
  powerful language which is able to reflect the partial nature of the
  knowledge suitably and to express the amount of ambiguity
  adequately. \scite{3}{augustin03}{34}
\end{quotex}

\begin{quotex}
  As sophisticated Bayesians like Isaac Levi (1980), Richard Jeffrey
  (1983), Mark Kaplan (1996), have long recognized, the proper
  response to symmetrically ambiguous or incomplete evidence is not to
  assign probabilities symmetrically, but to refrain from assigning
  precise probabilities at all. Indefiniteness in the evidence is
  reflected not in the values of any single credence function, but in
  the spread of values across the family of all credence functions
  that the evidence does not exclude. This is why modern Bayesians
  represent credal states using sets of credence functions. It is not
  just that sharp degrees of belief are psychologically unrealistic
  (though they are). Imprecise credences have a clear epistemological
  motivation: they are the proper response to unspecific evidence.
  \scite{3}{joyce05}{170f}
\end{quotex}\fcut{4}

Consider the following reasons that motivate Booleans to permit
instates for rational agents:

\begin{enumerate}[(A)]
\item The greatest emphasis motivating indeterminacy rests on
  \textsc{range} and \textsc{incomp}.
\item The preference structure of a rational agent may be incomplete
  so that representation theorems do not yield single probability
  measures to represent such incomplete structures.
\item There are more technical and paper-specific reasons, such as
  Thomas Augustin's attempt to mediate between the minimax pessimism
  of objectivists and the Bayesian optimism of subjectivists using
  interval probability (see \scite{8}{augustin03}{35f}); Alan
  H{\'a}jek and Michael Smithson's belief that there may be
  objectively indeterminate chances in the physical world (see
  \scite{8}{hajeksmithson12}{33}); and Jake Chandler's claim that
  \qeins{the sharp model is at odds with a trio of plausible
    propositions regarding agnosticism} \scite{2}{chandler14}{4}.
\end{enumerate}

This paper mostly addresses (A), while taking (B) seriously as well
and pointing towards solutions for it. I am leaving (C) to more
specific responses to the issues presented in the cited articles. I am
adding a reason (D) that is poorly documented in the literature: The
Boolean rational agent may systematically do better accepting bets
than the agent who on principle rejects instates.  Walley conducted an
experiment in which Boolean participants did significantly better than
Laplacean participants, betting on soccer games played in the Soccer
World Cup 1982 in Spain (see \scite{7}{walley91}{}, Appendix I). I
replicated the experiment using two computer players with rudimentary
artificial intelligence and made them specify betting parameters
(previsions) for games played in the Soccer World Cup 2014 in Brazil.
I used the Poisson distribution (which is an excellent predictor for
the outcome of soccer matches) and the FIFA ranking to simulate
millions of counterfactual World Cup results and their associated
bets, using Walley's evaluation method.  The Boolean player had a
slight but systematic advantage. In section \ref{WalleysWorldCupWoes},
I will provide an explanation and show how it undermines any support
the experiment might give to the Boolean position.\fcut{5}

There is also a problem for Booleans:

\begin{itemize}
\item [\textsc{dilation}] Instates are vulnerable to dilation.
\end{itemize}

\begin{quotex}
  \textbf{Example \arabic{expls}: \addtocounter{expls}{1} Dilation.}
  You have two Bernoulli Generators, \textit{coin}$_{d}$ and
  \textit{coin}$_{e}$. You have excellent evidence that
  \textit{coin}$_{d}$ is fair and no evidence about the bias of
  \textit{coin}$_{e}$. Furthermore, the two generators are not
  necessarily independent. Their results could be 100\% correlated or
  anticorrelated, or the correlation could be anywhere between the two
  extremes, including independence. You toss \textit{coin}$_{d}$.
  Without looking at the result, your credence in $H_{d}$
  (\textit{coin}$_{d}$ coming up heads) is a sharp $0.5$, even if you
  are a Boolean, because you have excellent evidence for the fairness
  of \textit{coin}$_{d}$. Then you toss \textit{coin}$_{e}$. This time
  you look at the result and the moment you learn it, your credence in
  $H_{d}$ dilates from a sharp $0.5$ to the vacuous credal state
  covering the whole interval $[0,1]$ (provided that this was your
  credence in $H_{e}$, as stipulated). Even though you have just
  received information (the result of \textit{coin}$_{e}$'s toss),
  your credence in $H_{d}$ dilates.
\end{quotex}

At first blush, this may not sound like a knock-down argument against
Booleans (it was investigated in detail in
\scite{7}{seidenfeldwasserman93}{}), but Roger White was able to use
it to derive more absurd results from instates.

\begin{quotex}
  \textbf{Example \arabic{expls}: \addtocounter{expls}{1} White
    Chocolates.} Four out of five chocolates in the box have cherry
  fillings, while the rest have caramel. Picking one at random, what
  should my credence be that it is cherry-filled? Everyone, including
  the staunchest [Booleans], seems to agree on the answer $4/5$. Now
  of course the chocolate I've chosen has many other features, for
  example this one is circular with a swirl on top. Noticing such
  features could hardly make a difference to my reasonable credence
  that it is cherry filled (unless of course I have some information
  regarding the relation between chocolate shapes and fillings). Often
  chocolate fillings do correlate with their shapes, but I haven't the
  faintest clue how they do in this case or any reason to suppose they
  correlate one way rather than another {\ldots} the further result is
  that while my credence that the chosen chocolate is cherry-filled
  should be $4/5$ prior to viewing it, once I see its shape (whatever
  shape it happens to be) my credence that it is cherry-filled should
  dilate to become [indeterminate]. But this is just not the way we
  think about such matters. \scite{3}{white10}{183}
\end{quotex}

One of the most authoritative Boolean voices is James Joyce. He has
defended instates against \textsc{dilation}, making two important
concessions, (J1) and (J2).

\begin{description}
\item[{\bf (J1)}] Credences do not adequately represent evidence (the
  same instate can reflect different evidential states).
\item[{\bf (J2)}] Instates do not reflect knowledge claims about
  objective chances (White's \emph{Chance Grounding Thesis} is not an
  appropriate characterization of the Boolean position).
\end{description}

I agree with Joyce that (J1) and (J2) are both necessary and
sufficient to resolve \textsc{dilation} for instates. I will go into
more detail as we move along. I disagree with Joyce what this means
for an overall recommendation to accept the Boolean rather than the
Laplacean position. I will show that (J1) and (J2) neutralize both
\textsc{range} and \textsc{incomp}, the two major motivators to reject
the Laplacean position.

The problem will be mixed metaphors: the Boolean mixes features that
ought for reasons to remain separate. Indeterminacy imposes a double
task on credences (representing both uncertainty and available
evidence) that they cannot coherently fulfill. I will present several
examples where this double task stretches instates to the limits of
plausibility. Joyce's idea that credences can represent balance,
weight, and specificity of the evidence (in \scite{7}{joyce05}{}) is
inconsistent with the use of indeterminacy (and Joyce himself, in
response to \textsc{dilation}, gives the argument why this is the
case, see \scite{8}{joyce10}{13ff}). 

The Laplacean approach of assigning subjective probabilities to
partitions of the event space (e.g.\ objective chances) and then
aggregating them by David Lewis' summation formula into a single
precise credence function is conceptually tidy and shares many of the
formal virtues of Boolean theories. If the bad taste about numerical
precision in a fuzzy and nebulous world lingers, I will point to
philosophical projects in other domains where the concepts we use are
sharply bounded, even though our ability to conceive of those sharp
boundaries or know them is limited (in particular Timothy Williamson's
account of vagueness and knowledge).

To put it provocatively, this paper defends a $0.5$ sharp credence in
heads in all three cases: for a coin of whose bias we are completely
ignorant; for a coin whose fairness is supported by a lot of evidence;
and even for a coin about whose bias we know that it is either 1/3 or
2/3 for heads. The credence represents the {\doxnotep} features of the
credal state and filters out some evidential features, which need to
be independently represented. If instates could successfully integrate
the two sets of features, they would be more successful in
representing the credal state. \textsc{dilation} combined with Joyce's
two concessions (J1) and (J2) show that they cannot.\bcut{4}\bcut{5}

xxyxx

There\fcut{10} are in the literature two kinds of objections to the
Boolean position. One is semantic, the other derives unacceptable
consequences from instates and then urges to give them up in favour of
sharp credences. A paradigm example for the latter kind is Adam Elga's
objection, which leads him to the following conclusion:
\qeins{Perfectly rational agents always have perfectly sharp
  probabilities} \scite{2}{elga10}{1}.

Elga tries to show that instates are not coherent in the sense that
they allow bets which will lead to a sure loss (for example, if an
agent has $0.4$ as a lower prevision and $0.6$ as an upper prevision,
then her credal state permits a $45$ cent bet on the proposition in
question, with \$1 being the prize money, and a $56$ cent bet against
the proposition in question). Joyce addresses Elga's point (see
\scite{8}{joyce10}{314}) and successfully defends the Boolean
position.

Roger White, in the meantime, articulates his objections on both
levels: in terms of semantics and in terms of undesirable consequences
of instates. Let us abbreviate the former objection as (Anti-CGT) and
the latter objection as (Anti-DIL). CGT stands for Chance Grounding
Thesis. It is a semantic claim about instates, and according to White
it is semantically unacceptable---some of White's objections are
echoed and re-articulated throughout this paper.

\begin{quotex}
  \textbf{Chance Grounding Thesis:} Only on the basis of known chances
  can one legitimately have sharp credences. Otherwise one's spread of
  credence should cover the range of possible chance hypotheses left
  open by your evidence. \scite{2}{white10}{174}\tbd{see Chance
    Grounding Thesis in sonk.org -- maybe go into more detail}
\end{quotex}

It is not sufficient for the Laplacean, however, to criticize the
Boolean position on the basis of the CGT, so White's (Anti-CGT)
argument will not do. This problem is intimately linked to White's
(Anti-DIL), to which Laplaceans can respond by giving up on the claims
of the CGT. DIL stands for dilation, which provides the grounds for
White's objection of the latter kind: not semantic, but drawing out
undesirable consequences from the Boolean approach. I will describe
dilation in a moment, but for the argument of this paper it is
important to understand that Booleans have a line of defence against
(Anti-DIL), and (Anti-CGT) to boot. They can, as Joyce does in his
response to White, give up on the CGT. More precisely, they can make
two concessions, which in honour of Joyce, who makes these
concessions, I call (J1) and (J2). 

Just as Joyce successfully defends the Boolean position against Elga's
objection, he successfully defends the Boolean position against
(Anti-CGT) and (Anti-DIL) by making concessions (J1) and (J2). The
rest of this paper shows how (J1) and (J2) are necessary to defend the
Boolean position against White's objections and how they leave 
Booleans in a situation so semantically vulnerable that their position
must be rejected anyways, despite their successful defences against
objections made on the grounds of undesirable consequences. We will
look at a few hands-on examples where Booleans, given (J1) and (J2),
under weak assumptions give the wrong answers.

Now we must make good on our promise to explain dilation (using
\scite{8}{white10}{175ff}) and why Joyce needs (J1) and (J2) in order
to save the Boolean position from its intuitive appeal.

Usually, we would expect more information to sharpen our credal states
(see Walley's \qeins{the more information the more precision}
principle and his response to this problem in
1991,\fixref{8}{walley91}{207 and 299} 207 and 299). White did not
discover the phenomenon of dilation (see the detailed study in
\scite{7}{seidenfeldwasserman93}{}), but he was able to find examples
where the consequences appear grotesque, especially in his chocolates
case (see \scite{8}{white10}{183}).\fcut{9}

White's claim is also that dilation contradicts Bas van Fraassen's
reflection principle (see van Fraassen,
1984\fixref{7}{vanfraassen84}{}). If you know that soon you will take
a dilated doxastic attitude towards a proposition without loss of
information and no surprising information coming in, you can just as
well assume the dilated doxastic attitude now, which is clearly
counter-intuitive (see \scite{8}{white10}{178}).

Joyce does an admirable job showing both that (J1) and (J2) address
White's dilation problem and that they are necessary in order to avoid
White's counter-intuitive results (see \scite{8}{joyce10}{13ff}). Joyce
rejects the CGT on the grounds that it would make learning impossible
(see \scite{8}{joyce10}{7f}); and he rejects the notion that identical
instates encode identical beliefs by giving a simple example:

\begin{quotex}
  \textbf{Example \arabic{expls}: \addtocounter{expls}{1} Three-Sided
    Die.} Suppose $\mathcal{C}$ and $\mathcal{C}^{*}$ are defined on a
  partition $\{X,Y,Z\}$ corresponding to the result of a roll of a
  three sided-die. Let $\mathcal{C}$ contain all credence functions
  defined on $\{X,Y,Z\}$ such that $\mathbf{c}(Z)\geq{}1/2$, and let
  $\mathcal{C}^{*}$ be the subset of $\mathcal{C}$ whose members also
  satisfy $\mathbf{c}(X)=\mathbf{c}(Y)$. \scite{3}{joyce10}{12}
\end{quotex}

$\mathcal{C}$ and $\mathcal{C}^{*}$ generate the same instates, but
they surely differ in the beliefs that they encode, as
$\mathcal{C}^{*}$ regards $X$ and $Y$ as equiprobable, whereas
$\mathcal{C}$ does not. To say that dilation is a problem for the
Boolean position presupposes that instates encode beliefs, since if
they do not it becomes clear why the correlation between the two coin
tosses is evidentially relevant to $H_{a}$ in Example 1. If instates
encoded the evidential basis for a belief, however, there could be no
weights attached to the various credence functions represented by the
instate based on the evidence. All \qnull{committee members,} as Joyce
quite helpfully calls them in illustrating how instates work by
committee rather than one single credence function, would be equally
enfranchised, which would inhibit learning and either introduce
regress problems on the non-trivial (neither $0$ nor $1$) margins of
the indeterminate intervals or render all instates vacuous.

Therefore, the Boolean position cannot be upheld without concessions
(J1) and (J2). Joyce, for example, clearly rejects the CGT and
underlines how \qnull{committee members} can be discriminately
enfranchised by a \qeins{directionality of the spread} (see
\scite{8}{joyce10}{318}), which clarifies that instates, as little as
sharp credences, adequately represent the evidence supporting the
partial belief. Joyce gives a nice formal description of this, but I
want to show by example how dilation is unproblematic (and therefore
agree with Joyce that instates cannot encode beliefs).\fcut{11}

\begin{quotex}
  \textbf{Example \arabic{expls}: \addtocounter{expls}{1} Dilating
    Urns.} You draw one ball from an urn with 200 balls (100 red, 100
  black) and receive the information that the urn actually had two
  chambers, one with 99 red balls and 1 black ball, the other with 1
  red ball and 99 black balls.
\end{quotex}

Dilation from a sharp credence of $\{0.5\}$ to an instate of
$\{0.01,0.99\}$ or $[0.01,0.99]$ (depending on whether convexity is
required) is unproblematic, although the example already prefigures
that there is something odd about the Boolean semantic approach. The
example licences a 99:1 bet for one of the colours (if the instate is
interpreted as upper and lower previsions), but this is a problem that
arises out of Boolean semantics without dilation, which we will
address again in Example 8.

Booleans have the resources to extract themselves from the problem of
dilation, but only at the cost of making Joyce's semantic concessions
(J1) and (J2) which we will use against them. These semantic
concessions are inescapable, not only on account of dilation. If one
were to be committed to the principle of regularity, that all states
of the world considered possible have positive probability (for a
defence see \scite{7}{edwardsetal63}{}); and to the solution of Henry
Kyburg's lottery paradox, that what is rationally accepted should have
probability 1 (for a defence of this principle see
\scite{7}{douvenwilliamson06}{}); and the CGT, that one's spread of
credence should cover the range of possible chance hypotheses left
open by the evidence (implied by much of Boolean literature); then
one's instate would always be vacuous. Booleans must deny at least one
of the premises to avoid the conclusion (Joyce denies the CGT).

A sharp credence constrains partial beliefs in objective chances by
Lewis' summation formula (which we will provide in the next section).
No objective chance is excluded by it (principle of regularity) and
any updating will merely change the partial beliefs, but no full
beliefs. Instates, on the other hand, by giving ranges of acceptable
objective chances suggest that there is a full belief that the
objective chance does not lie outside what is indicated by the
instate. A Boolean can avoid this situation by accepting Joyce's
concession (J2).

Here is a brief example to illustrate the difference between Laplacean
semantics of partial beliefs based on the principle of regularity and
Boolean semantics which introduce an obscure grey zone between partial
beliefs and full belief, update and revision, traditional epistemology
and formal epistemology.

\begin{quotex}
  \textbf{Example \arabic{expls}: \addtocounter{expls}{1} Bavarian
    King.} Matthias Perth, an Austrian civil servant, observes the
  Bavarian king at the Congress of Vienna in 1815 and writes in his
  diary that the king \qeins{appears to be a man between 45 and 47
    years old} (see \texttt{http://www.das-perth-projekt.at}).
\end{quotex}

If Perth then learns that the king was 49 years old, he must revise,
not just update, his earlier judgment. The appropriate formal
instrument is belief revision, not probability update, requiring a
substantial reconciliation project between formal and traditional
epistemology operating in the background. I do not see this project
articulated in the Boolean literature (for an example of such a
project see \scite{7}{spohn12}{}, especially chapter 10). Sarah Moss
also undertakes it and assumes the Boolean approach (see
\scite{7}{moss13}{}), but I fail to see how the Boolean approach is
essential to her reconciliation or how her reconciliation gives
independent arguments for the Boolean approach. If Perth had wanted to
express a sharp credence, he would have said, \qeins{my best guess is
  that the king is 46 years old,} and the information that the king
was 49 would have triggered the appropriate update, without any
revision of full beliefs.

In summary, I return to White's objections (Anti-CGT) and (Anti-DIL)
which, as they stand, fail because the CGT is not a necessary
ingredient of the Boolean position; and dilation for indeterminate
credences is in principle not any more surprising than a piece of
information that increases the Shannon entropy of a sharp credence
(see Example 4). It is true for both sharp and indeterminate credences
that information can make us less certain about things, and it is true
for both sharp and indeterminate credences that they do not encode the
evidence. 

Once Booleans have brought their house in order to accommodate White's
objection using concessions (J1) and (J2), they open the door to
semantic problems. We finally get to inquire what kind of coherence
there is in defending indeterminacy when it neither fulfills the
promise of adequately representing evidence nor the promise of
reconciling traditional full belief \qnull{knowledge} epistemology and
Bayesian partial belief epistemology as outlined in the CGT, but only
adds another hierarchical layer of uncertainty to a numerical quantity
(a sharp credence) whose job it already is to represent uncertainty,
thus unnecessarily introducing regress problems. We will turn to these
semantic considerations now, show how they display the virtues of
sharp credences in responding to the forceful motivations for instates
while making those instates look semantically otiose. Then we will
show in the last section how sharp credences have an elegant solution
for being outperformed by instates in betting scenarios, and there we
will rest our case.

\section{Semantics of Partial Belief}
\label{SemanticsOfPartialBelief}

Instates are suggestive of a measurement that represents numerically
the mass of an object and then also make claims about its density.
With sharp credences, the semantic roles of evidence, information, and
uncertainty are appropriately differentiated. Rational decision
making, inference, and betting behaviour are based on sharp credences
together with the evidence that is at its foundation. Information
represents evidence, and sharp credences represent uncertainty.
Measurement is in any case a misleading analogy for credences.
Measurements come with imprecision estimates to account for
inaccuracy. Credences, however, are not measurements, especially not
of objective chances. They represent uncertainty. They are more like
logical truth values than they are like measurements.

It is a slippery affair to determine what evidence is,
which I will leave to others. My claim is that a rational agent is
someone who can distill information from evidence which places
numerically precise constraints on relatively prior probability
distributions, which then can be updated to form posterior probability
distributions and the credences associated with them. Note that
relatively prior probability distributions are not ignorance priors or
non-informative priors, which I would call absolutely prior
probability distributions. I have no answers where absolutely prior
probability distributions come from, how they are justified, or in
what sense they are objective. I am not concerned whether all rational
agents, if they have the same evidence, should arrive at the same
credal states; or even if they should all update a given relatively
prior credal state to the same posterior credal state, if they have
the same evidence. Sometimes there may be different ways to translate
or interpret evidence into information.

It is important not to confuse the claim that it is reasonable to hold
both $X$ and $Y$ with the claim that it is reasonable to hold either
$X$ (without $Y$) or $Y$ (without $X$). It is the reasonableness of
holding $X$ and $Y$ concurrently that is controversial, not the
reasonableness of holding $Y$ (without holding $X$) when it is
reasonable to hold $X$. We will later talk about anti-luminosity, the
fact that a rational agent may not be able to distinguish
psychologically between a $54.9$ cent bet on an event and a $45.1$ bet
on its negation, when her sharp credence is $0.55$. She must reject
one of them not to incur sure loss, so proponents of indeterminacy
suggest that she choose one of them freely without being constrained
by her credal state or reject both of them. I claim that a sharp
credence will make a recommendation between the two so that only one
of the bets is rational given her particular credence, but that does
not mean that another sharp credence which would give a different
recommendation may not also be rational for her to have.

I am sympathetic to the viewpoint that once a rational agent has a
relatively prior credal state and has formalized her evidence in terms
of information, then the probability distributions forming her
posterior credal state should be unique. Joyce, with his
\qnull{committee member} approach, shows how this kind of updating can
be done for instates (see \scite{8}{joyce10}{288}; also
\scite{8}{bradleysteele13}{6}).

On the one hand (the Laplacean approach), you can have partial beliefs
about how a parameter is distributed and then use Lewis' summation
formula (see \scite{8}{lewis81}{266f}) to integrate over them and
condense them to a sharp credence. Walley comments on this
\qeins{reduction} in his section on Bayesian second order
probabilities (see \scite{8}{walley91}{258f}), but he mistakenly
represents the Laplacean approach as a second order approach, as if
the probability distributions that are summarized by Lewis' formula
are of the same kind as the resulting credences. They are not. They
refer to partitions of the event space, sometimes corresponding to
objective chances, and represent the subjective probabilities that are
associated with them. The credence, by contrast, is a quantity
representing partial belief and assisting in the making of decisions
and inferences. It is the Boolean approach which has elements of a
second order approach and thus makes itself vulnerable to regress
problems by adding another dimension of uncertainty to a parameter
(the credence) which already represents uncertainty.

On the other hand (the Boolean approach), you can try to represent
your uncertainty about the distribution of the parameter by an
instate. I want to show that the Laplacean approach can be aligned
with the forceful motivations we listed in the previous section to
introduce instates, as long as we do not require that a sharp credence
represent the evidence as well as the agent's state of uncertainty. We
have learned that this requirement can be reduced ad absurdum even for
instates, see (J1) and (J2).

One of Joyce's complaints is that a sharp credence of $0.5$ for a
\textit{coin} contains too much information if there is little or no
evidence that the \textit{coin} is fair. This complaint, of course, is
only effective if the indeterminacy of the credence is anticorrelated
to the amount of information in the evidence. In (J1), however, Joyce
admits that instates cannot represent the evidence without violating
the reflection principle due to White's dilation problem. He is quite
clear that the same instate can represent different evidential
scenarios (see, for example, \scite{8}{joyce10}{302}). (J1) may not be
sufficient to defend against the information argument, but Walley's
and Joyce's claim that instates are less informative than sharp
credences (see \scite{8}{walley91}{34}; and \scite{8}{joyce10}{311}
for examples, but this attitude is passim) has no foundation in
information theory. To compare instates and sharp credences
informationally, we would need a non-additive set function obeying
Shannon's axioms for information. This is a non-trivial task. I have
not succeeded solving it (nor do I need to carry the Booleans' water),
but I am not convinced that it will result in an information measure
which assigns, for instance, more information to a sharp credence such
as $\{0.5\}$ than to an instate such as $\{x|1/3\leq{}x\leq{}2/3\}$.

Returning to (J1) and the problem of inadequate representation,
Augustin recognizes this long before Joyce, with specific reference to
instates: \qeins{The imprecise posterior does no longer contain all
  the relevant information to produce optimal decisions. Inference and
  decision do not coincide any more} \scite{2}{augustin03}{41} (see
also an example for inadequate representation of evidence by instates
in \scite{8}{bradleysteele13}{16}). At best, instates fare no better
than sharp credences, at worst they unhelpfully mimic saying something
about the evidence that is much better said elsewhere.

Not only can we align sharp credences with the motivations to
introduce instates, we can also show that instates perform worse
semantically because they mix evidential and {\doxnotep} metaphors in
deleterious ways. Sharp credences have one task: to represent
epistemic uncertainty and serve as a tool for updating, inference, and
decision making. They cannot fulfill this task without continued
reference to the evidence which operates in the background. To use an
analogy, credences are not sufficient statistics with respect to
updating, inference, and decision making. What is remarkable about
Joyce's response to White's dilation problem is that Joyce recognizes
that instates are not sufficient statistics either. But this means
that they fail at the double task which has been imposed on them: to
represent both epistemic uncertainty and the evidence.

In the following, I will provide a few examples where it becomes clear
that instates have difficulty representing uncertainty because they
are tangled in a double task which they cannot fulfill.

\begin{quotex}
  \textbf{Example \arabic{expls}: \addtocounter{expls}{1} Aggregating
    Expert Opinion.} You have no information whether it will rain
  tomorrow ($R$) or not except the predictions of two weather
  forecasters. One of them forecasts 0.3 on channel GPY, the other 0.6
  on channel QCT. You consider the QCT forecaster to be significantly
  more reliable, based on past experience.
\end{quotex}

An instate corresponding to this situation may be $[0.3,0.6]$ (see
\scite{8}{walley91}{214}), but it will have a difficult time
representing the difference in reliability of the experts. We could
try $[0.2,0.8]$ (since the greater reliability of QCT suggests that
the chance of rain tomorrow is higher rather than lower) or
$[0.1,0.7]$ (since the greater reliability of QCT suggests that its
estimate is more precise), but it remains obscure what the criteria
might be.

A sharp credence of $P(R)=0.53$, for example, does the right thing.
Such a credence says nothing about any beliefs that the objective
chance is restricted to a subset of the unit interval, but it
accurately reflects the degree of uncertainty that the rational agent
has over the various possibilities. Beliefs about objective chances
make little sense in many situations where we have credences, since it
is doubtful even in the case of rain tomorrow that there is an urn of
nature from which balls are drawn. What is really at play is a complex
interaction between epistemic states (for example, experts evaluating
meteorological data) and the evidence which influences them.

A sharp credence is often associated with probability distributions
over chances, while an instate puts chances in sets where they all
have an equal voice. This may also be at the bottom of Susanna
Rinard's objection (see \scite{8}{white10}{184}) that Joyce's
committee members are all equally enfranchised and so it is not clear
how extremists among them could not always be replaced by even greater
extremists even after updating on evidence which should serve to
consolidate indeterminacy. Joyce has a satisfactory response to this
objection (see \scite{8}{joyce10}{291}), but I do not see how the
response addresses the problem of aggregating expert opinion without
the kind of summation that Laplaceans find unobjectionable, even
though information is lost and can only be recouped by going back to
the evidence. More generally, the two levels for sharp credences,
representation of uncertainty and distributions over partitions,
tidily differentiate between the {\doxnotep} and the evidential
dimension; instates, on the other hand, just add another level of
uncertainty on top of the uncertainty that is already expressed in the
partial belief and thus do not make the appropriate semantic
distinctions.

As we will see in the next example, it is an advantage of sharp
credences that they do not exclude objective chances, even extreme
ones, because they are fully committed to partial belief and do not
suggest, as indeterminate credences do, that there is full belief
knowledge that the objective chance is a member of a proper subset of
the possibilities.

\begin{quotex}
  \textbf{Example \arabic{expls}: \addtocounter{expls}{1} Precise
    Credences.} Your sharp credence for rain tomorrow, based on the
  expert opinion of channel GPY and channel QCT (you have no other
  information) is $0.53$. Is it reasonable, considering how little
  evidence you have, to reject the belief that the chance of rain
  tomorrow is $0.52$ or $0.54$; or to prefer a $52.9$ cent bet on rain
  to a $47.1$ cent bet on no rain?
\end{quotex}

The first question is confused, but in instructive ways (a display of
this confusion is found in \scite{8}{hajeksmithson12}{38f}, and their
doctor and time of the day analogy). A sharp credence rejects no
hypothesis about objective chances (unlike an instate, unless (J2) is
firmly in place). It often has a subjective probability distribution
operating in the background, over which it integrates to yield the
sharp credence (it would do likewise in H{\'a}jek and Smithson's
example for the prognosis of the doctor or the time of the day,
without any problems). This subjective probability
distribution may look like this:

\begin{tabular}{|lcr|}
  \hline
  $P(\pi(R)=0.00)$ & = & $0.0001$ \\ \hline
  $P(\pi(R)=0.01)$ & = & $0.0003$ \\ \hline
  $P(\pi(R)=0.02)$ & = & $0.0007$ \\ \hline
  $\ldots$ & & $\ldots$ \\ \hline
  $P(\pi(R)=0.30)$ & = & $0.0015$ \\ \hline
  $P(\pi(R)=0.31)$ & = & $0.0016$ \\ \hline
  $\ldots$ & & $\ldots$ \\ \hline
  $P(\pi(R)=0.52)$ & = & $0.031$ \\ \hline
  $P(\pi(R)=0.53)$ & = & $0.032$ \\ \hline
  $P(\pi(R)=0.54)$ & = & $0.030$ \\ \hline
  $\ldots$ & & $\ldots$ \\ \hline
\end{tabular}

It is condensed by Lewis' summation formula to a sharp credence,
without being reduced to it:

\begin{equation}
  \label{eq:s2}
  C(R)=\int_{0}^{1}\zeta{}P(\pi(R)=\zeta)\,d\zeta\mbox{ }
\end{equation}

Lewis' 1981 paper \qeins{A Subjectivist's Guide to Objective Chance}
addresses the question what the relationship between $\pi$, $P$, and
$C$ is. The point is that we have properly separated the semantic
dimensions and that the Laplacean approach is not a second order
probability approach. The partial belief epistemology deals with sharp
credences and how they represent uncertainty and serve as a tool in
inference, updating, and decision making; while Lewis' Humean
speculations and his interpretation of the principal principle cover
the relationship between subjective probabilities and objective
chance.

Instates, by contrast, mix these semantic dimensions so that in the
end we get a muddle where a superficial reading of indeterminacy
suddenly follows a converse principal principle of sorts, namely that
objective chances are constrained by the factivity of a rational
agent's credence when this credence is knowledge (Lewis actually talks
about such a converse, but in completely different and
epistemologically more intelligible terms, see
\scite{8}{lewis81}{289}). Sharp credences are more, not less,
permissive with respect to objective chances operating externally
(compared to the internal belief state of the agent, which the
credence reflects). By the principle of regularity and in keeping with
statistical practice, all objective chances as possible states of the
world are given positive subjective probabilities, even though they
may be very small. Instates, on the other hand, mix partial belief
epistemology with full belief epistemology and presumably exclude
objective chances which lie outside the credal state from
consideration because they are fully known not to hold (see
\scite{8}{levi81}{540}, \qeins{inference derives credal probability
  from knowledge of the chances of possible outcomes}).

The second question is also instructive: why would we prefer a $52.9$
cent bet on rain to a $47.1$ cent bet on no rain, given that we do not
possess the power of descrimination between these two bets? The answer
to this question ties in with the issue of incomplete preference
structure referred to above as motiviation (B) for instates.

\begin{quotex}
  It hardly seems a requirement of rationality that belief be precise
  (and preferences complete); surely imprecise belief (and
  corresponding incomplete preferences) are at least rationally
  permissible. \scite{3}{bradleysteele13}{2}
\end{quotex}

In personal communication, Yang Liu at Columbia University posed this
problem to me more forcefully: the development of representation
theorems beginning with Frank Ramsey (followed by increasingly more
compelling representation theorems in \scite{7}{savage54}{}; and
\scite{7}{jeffrey65}{}; and numerous other variants in contemporary
literature) puts the horse before the cart and bases probability and
utility functions of an agent on her preferences, not the other way
around. Once completeness as an axiom for the preferences of an agent
is jettisoned, indeterminacy follows automatically. Indeterminacy may
thus be a natural consequence of the proper way to think about
credences in terms of the preferences that they represent.

In response, preferences may very well logically and psychologically
precede an agent's probability and utility functions, but that does
not mean that we cannot inform the axioms we use for a rational
agent's preferences by undesirable consequences downstream.
Completeness may sound like an unreasonable imposition at the outset,
but if incompleteness has unwelcome semantic consequences for
credences, it is not illegitimate to revisit the issue. Timothy
Williamson goes through this exercise with vague concepts, showing
that all upstream logical solutions to the problem fail and that it
has to be solved downstream with an epistemic solution (see
\scite{7}{williamson96}{}). Vague concepts, like sharp credences, are
sharply bounded, but not in a way that is luminous to the agent (for
anti-luminosity see chapter 4 in \scite{7}{williamson00}{}).
Anti-luminosity answers the original question: the rational agent
prefers the $52.9$ cent bet on rain to a $47.1$ cent bet on no rain
based on her sharp credence without being in a position to have this
preference necessarily or have it based on physical or psychological
ability (for the analogous claim about knowledge see
\scite{8}{williamson00}{95}).

In a way, advocates of indeterminacy have solved this problem for us.
There is strong agreement among most of them that the issue of
determinacy for credences is not an issue of elicitation (sometimes
the term \qnull{indeterminacy} is used instead of \qnull{imprecision}
to underline this difference; see \scite{8}{levi85}{395}). The appeal
of preferences is that we can elicit them more easily than assessments
of probability and utility functions. The indeterminacy issue has been
raised to the probability level (or moved downstream) by indeterminacy
advocates themselves who feel justifiably uncomfortable with an
interpretation of their theory in behaviourist terms. So it shall be
solved there, and this paper makes an appeal to reject indeterminacy
on this level. The solution then has to be carried upstream (or
lowered to the logically more basic level of preferences), where we
recognize that completeness for preferences is after all a desirable
axiom for rationality. Isaac Levi agrees with me on this point: when
he talks about indeterminacy, it proceeds from the level of
probability judgment to preferences, not the other way around (see
\scite{8}{levi81}{533}).

\begin{quotex}
  \textbf{Example \arabic{expls}: \addtocounter{expls}{1}
    Monkey-Filled Urns.} E.T. Jaynes describes an experiment with
  monkeys filling an urn randomly with balls from another urn, for
  which sampling provides no information and so makes updating vacuous
  (see \scite{8}{jaynesbretthorst03}{160}). Here is a variant of this
  experiment for which a sharp credence provides a more compelling
  result than the associated instate: Let urn $A$ contain 4 balls, two
  red and two black. A monkey randomly fills urn $B$ from urn $A$ with
  two balls. We draw from urn $B$.
\end{quotex}

The sharp credence of drawing a red ball is $0.5$, following Lewis'
summation formula for the different combinations of balls in urn $B$.
This solution is more intuitive in terms of further inference,
decision making, and betting behaviour than a credal state of
$\{0,1/2,1\}$ or $[0,1]$ (depending on the convexity requirement),
since this instate would licence an exorbitant bet in favour of one
colour, for example one that costs \$9,999 and pays \$10,000 if red is
drawn and nothing if black is drawn.

To make this example more vivid consider a Hand Urn, where you draw by
hand from an urn with 100 balls, 50 red balls and 50 black balls. When
your hand retreats from the urn, does it not contain either a red ball
or a black ball and so serve itself as an urn, from which in a sense
you draw a ball? Your hand contains one ball, either red or black, and
the indeterminate credal state that it is one or the other should be
$[0,1]$. This contradicts our intuition that our credence should be a
sharp $0.5$. As is the case with the White Chocolate example, instates
appear to be highly contingent on a problem's mode of representation,
more so than intuition allows.

\begin{quotex}
  \textbf{Example \arabic{expls}: \addtocounter{expls}{1} Three
    Prisoners.} Prisoner $X_{1}$ knows that two out of three prisoners
  ($X_{1},X_{2},X_{3}$) will be executed and one of them pardoned. He
  asks the warden of the prison to tell him the name of another
  prisoner who will be executed, hoping to gain knowledge about his
  own fate. When the warden tells him that $X_{3}$ will be executed,
  $X_{1}$ erroneously updates his probability of pardon from $1/3$ to
  $1/2$, since either $X_{1}$ or $X_{2}$ will be spared.
\end{quotex}

Walley maintains that for the Monty Hall problem and the Three
Prisoners problem, the probabilities of a rational agent should dilate
rather than settle on the commonly accepted solutions. For the Three
Prisoners problem, there is a compelling case for standard
conditioning and the result that the credence for prisoner $X_{1}$ to
have been pardoned ought to be unchanged after the update (see
\scite{8}{lukits14}{1421f}). Walley's dilated solution would give
prisoner $X_{1}$ hope on the doubtful possibility (and unfounded
assumption) that the warden might prefer to provide $X_{3}$'s (rather
than $X_{2}$'s) name in case prisoner $X_{1}$ was pardoned.

This example brings an interesting issue to the forefront. Sharp
credences often reflect independence of variables where such
independence is unwarranted. Booleans (more specifically, detractors
of the principle of indifference or the principle of maximum entropy,
principles which are used to generate sharp credences for rational
agents) tend to point this out gleefully. They prefer to dilate over
the possible dependence relationships (independence included). White's
dilation problem is an instance of this. The fallacy in the argument
for instates, illustrated by the Three Prisoners problem, is that the
probabilistic independence of sharp credences does not imply
independence of variables (only the converse is correct), but only
that it is unknown whether there is dependence, and if yes, whether it
is correlation or inverse correlation.

In the Three Prisoners problem, there is no evidence about the degree
or the direction of the dependence, and so prisoner $X_{1}$ should
take no comfort in the information that she receives. The prisoner's
probabilities will reflect probabilistic independence, but make no
claims about causal independence. Walley has unkind things to say
about sharp credences and their ability to respond to evidence (for
example that their \qeins{inferences rarely conform to evidence}, see
\scite{8}{walley91}{396}), but in this case it appears to me that they
outperform the Boolean approach.

\begin{quotex}
  \textbf{Example \arabic{expls}: \addtocounter{expls}{1} Wagner's
    Linguist.} A linguist hears the utterance of a native and
  concludes that the native cannot be part of certain population
  groups, depending on what the utterance means. The linguist is
  uncertain between some options about the meaning of the utterance.
  (For full details see \scite{8}{wagner92}{252}; and
  \scite{8}{spohn12}{197}.)
\end{quotex}

The mathematician Carl Wagner proposed a natural generalization of
Jeffrey Conditioning for his Linguist example (see
\scite{7}{wagner92}{}). Since the principle of maximum entropy is
already a generalization of Jeffrey Conditioning, the question
naturally arises whether the two generalizations agree. Wagner makes
the case that they do not agree and deduces that the principle of
maximum entropy is sometimes an inappropriate updating mechanism, in
line with many earlier criticisms of the principle of maximum entropy
(see van Fraassen,\fixref{7}{fraassen81}{} 1981;
\scite{7}{shimony85}{}; \scite{7}{skyrms87updating}{}; and, later on,
\scite{7}{grovehalpern97}{}). What is interesting about this case is
that Wagner uses instates for his deduction, so that even if you agree
with his natural generalization of Jeffrey Conditioning (which I find
plausible), the inconsistency with the principle of maximum entropy
can only be inferred assuming instates. Wagner is unaware of this, and
I am showing in another paper (in process) how on the assumption of
sharp credences Wagner's generalization of Jeffrey conditioning
accords with the principle of maximum entropy.

This will not convince Booleans, since they are already unlikely to
believe in the general applicability of the principle of maximum
entropy (just as Wagner's argument is unlikely to convince a proponent
of the principle of maximum entropy, since they have a tendency to
reject instates). The battle lines are clearly drawn. Wagner's
argument, instead of undermining the principle of maximum entropy,
just shows that instates are as wedded to rejecting the claims of the
principle of maximum entropy as the principle of maximum entropy is
wedded to sharp credences (these marriages are only unilaterally
monogamous, however, as it is perfectly coherent to reject both the
principle of maximum entropy and the Boolean position; or to reject
both the Laplacean position and instates). 

Endorsement of instates, however, implies that there are situations of
probability update in which the posterior probability distribution is
more informative than it might be in terms of information theory.
Indeterminate credences violate the relatively natural intuition that
we should not gain information from evidence when a less informative
updated probability will do the job of responding to the evidence.
This is not a strong argument in favour of sharp credences. The
principle of maximum of entropy has received a thorough bashing in the
last forty years. I consider it to be much easier to convince someone
to reject instates on independent (semantic) grounds than to convince
them to give the principle of maximum entropy a second chance. But the
section on semantics comes to an end here, and we want to proceed to
the intriguing issue of who does better in betting situations:
instates or sharp credences.

\section{Evidence Differentials and Cushioning Credences}
\label{WalleysWorldCupWoes}

I have given away the answer already in the introduction: instates do
better. It is surprising that, except for a rudimentary allusion to
this in Walley's book, no Boolean has caught on to this yet. After I
found out that agents with instates do better betting on soccer games,
I let Betsy and Linda play a more basic betting game. An $n$-sided die
is rolled (by the computer). The die is fair, unbeknownst to the
players. Their bets are randomly and uniformly drawn from the simplex
for which the probabilities attributed to the $n$ results add up to 1.
Betsy also surrounds her credences with an imprecision uniformly drawn
from the interval $(0,y)$. I used Walley's pay off scheme (see
\scite{8}{walley91}{632}) to settle the bets.

Here is an example: let $n=2$, so the die is a fair \textit{coin}.
Betsy's and Linda's bets are randomly and uniformly drawn from the
line segment from $(0,1)$ to $(1,0)$ (these are two-dimensional
Cartesian coordinates), the two-dimensional simplex (for higher $n$,
the simplex is a pentatope generalized for $n$ dimensions with side
length $2^{1/2}$). The previsions (limits at which bets are accepted)
may be $(0.21,0.79)$ for Linda and $(0.35\pm{}0.11,0.65\pm{}0.11)$ for
Betsy, where the indeterminacy $\pm{}0.11$ is also randomly and
uniformly drawn from the imprecision interval $(0,y)\subseteq(0,1)$.
The first bet is on $H$, and Linda is willing to pay $22.5$ cents for
it, while Betsy is willing to pay $77.5$ cents against it. The second
bet is on $T$ (if $n>2$, there will not be the same symmetry as in the
\textit{coin} case between the two bets), for which Betsy is willing
to pay $77.5$ cents, and against which Linda is willing to pay $22.5$
cents. Each bet pays \$1 if successful. Often, Linda's credal state
will overlap with Betsy's sharp credence so that there will not be a
bet.\fcut{7}

The computer simulation clearly shows that Linda does better than
Betsy in the long run. A defence of sharp credences for rational
agents needs to have an explanation for this. We will call it partial
belief cushioning, which is based on an evidence differential between
the bettors.

In many decision-making contexts, we do not have the luxury of calling
off the bet. We have to decide one way or another. This is a problem
for instates, as Booleans have to find a way to decide without
receiving instructions from the credal state. Booleans have addressed
this point extensively (see for example \scite{8}{joyce10}{311ff}; for
an opponent's view of this see \scite{8}{elga10}{6ff}). The problem
for sharp credences arises when bets are noncompulsory, for then the
data above suggest that agents holding instates systematically do
better. Often, decision making happens as betting vis-{\`a}-vis
uninformed nature or opponents which are at least as uninformed as the
rational agent. Sometimes, however, bets are offered by better
informed or potentially better informed bookies. In this case, even an
agent with sharp credences must cushion her credences and is better
off by rejecting bets that look attractive in terms of her partial
beliefs.

If an agent does not cushion her partial beliefs (whether they are
sharp or indeterminate), she will incur a loss in the long run. Since
cushioning is permitted in Walley's experimental setup (the bets are
noncompulsory), Laplacean agents should also have access to it and
then no longer do worse than Boolean agents. One may ask what sharp
credences do if they just end up being cushioned anyway and do not
provide sufficient information to decide on rational bets. The answer
is that sharp credences are sufficient where betting (or decision
making more generally) is compulsory; the cushioning only supplies the
information from the evidence inasmuch as betting is noncompulsory and
so again properly distinguishes semantic categories. This task is much
harder for Booleans, although I do not claim that it is
insurmountable: instates can provide a coherent approach to compulsory
betting. What they cannot do, once cushioning is introduced, is
outperform sharp credences in noncompulsory betting situations.

Here are a few examples: even if I have little evidence on which to
base my opinion, someone may force me to either buy Coca Cola shares
or short them, and so I have to have a share price $p$ in mind that I
consider fair. I will buy Coca Cola shares for less than $p$, and
short them for more than $p$, if forced to do one or the other. This
does not mean that it is now reasonable for me to go (not forced by
anyone) and buy Coca Cola shares for $p$. It may not even be
reasonable to go (not forced by anyone) and buy Coca Cola share for
$p-\delta$ with $\delta{}>0$.

It may in fact be quite unreasonable, since there are many players who
have much better evidence than I do and will exploit my ignorance. I
suspect that most lay investors in the stock market make this mistake:
even though they buy and sell stock at prices that seem reasonable to
them, professional investors are much better and faster at exploiting
arbitrage opportunities and more subtle regularities. If indices rise,
lay investors will make a little less than their professional
counterparts; and when they fall, lay investors lose a lot more. In
sum, unless there is sustained growth and everybody wins, lay
investors lose in the long term.

A case in point is the U.S. Commodity Futures Trading Commission's
crackdown on the online prediction market Intrade. Intrade offered
fair bets for or against events of public significance, such as
election results or other events which had clear yes-or-no outcomes.
Even though the bets were all fair and Intrade only received a small
commission on all bets, and even though Intrade's predictions were
remarkably accurate, the potential for professional arbitrageurs was
too great and the CFTC shut Intrade down (see
\texttt{https://www.intrade.com}).

Cushioning does not stand in the way of holding a sharp credence, even
if the evidence is dim. The evidence determines for a rational agent
the partial beliefs over possible states of the world operating in the
background. The better the evidence, the more pointed the
distributions of these partial beliefs will be and the more willing
the rational agent will be to enter a bet, if betting is
noncompulsory. The mathematical decision rule will be based on the
underlying distribution of the partial beliefs, not only on the sharp
credence. As we have stated before, the sharp credence is not a
sufficient statistic for decision making, inference, or betting
behaviour; and neither is an instate.

The rational agent with a sharp credence has resources at her disposal
to use just as much differentiation with respect to accepting and
rejecting bets as the agent with instates. Often (if she is able to
and especially if the bets are offered to her by a better-informed
agent), she will reject both of two complementary bets, even when they
are fair. On the one hand, any advantage that the agent with an
instate has over her can be counteracted based on her distribution
over partial beliefs that she has with respect to all possibilities.
On the other hand, the agent with instates suffers from a semantic
mixing of metaphors between evidential and {\doxnotep} dimensions that
puts her at a real disadvantage in terms of understanding the sources
and consequences of her knowledge and her uncertainties.

\section{Conclusion}
The conclusion goes here.

% if have a single appendix:
%\appendix[Proof of the Zonklar Equations]
% or
%\appendix  % for no appendix heading
% do not use \section anymore after \appendix, only \section*
% is possibly needed

% use appendices with more than one appendix
% then use \section to start each appendix
% you must declare a \section before using any
% \subsection or using \label (\appendices by itself
% starts a section numbered zero.)
%

% \appendices
% \section{Proof of the First Zonklar Equation}
% Appendix one text goes here.

% you can choose not to have a title for an appendix
% if you want by leaving the argument blank
% \section{}
% Appendix two text goes here.


% use section* for acknowledgement
% \section*{Acknowledgment}


% The authors would like to thank...
% \cite{IEEEhowto:IEEEtranpage}

% Can use something like this to put references on a page
% by themselves when using endfloat and the captionsoff option.
% \ifCLASSOPTIONcaptionsoff
%   \newpage
% \fi



% trigger a \newpage just before the given reference
% number - used to balance the columns on the last page
% adjust value as needed - may need to be readjusted if
% the document is modified later
%\IEEEtriggeratref{8}
% The "triggered" command can be changed if desired:
%\IEEEtriggercmd{\enlargethispage{-5in}}

% references section

% can use a bibliography generated by BibTeX as a .bbl file
% BibTeX documentation can be easily obtained at:
% http://www.ctan.org/tex-archive/biblio/bibtex/contrib/doc/
% The IEEEtran BibTeX style support page is at:
% http://www.michaelshell.org/tex/ieeetran/bibtex/
\bibliography{bibtex/bib/IEEEabrv.bib,bibtex/bib/bib-7293.bib}{}
\bibliographystyle{IEEEtran}
% argument is your BibTeX string definitions and bibliography database(s)
%\bibliography{}
%
% <OR> manually copy in the resultant .bbl file
% set second argument of \begin to the number of references
% (used to reserve space for the reference number labels box)
%\begin{thebibliography}{1}

%\bibitem{IEEEhowto:kopka}
%H.~Kopka and P.~W. Daly, \emph{A Guide to \LaTeX}, 3rd~ed.\hskip 1em plus
%  0.5em minus 0.4em\relax Harlow, England: Addison-Wesley, 1999.
%
%\end{thebibliography}

% biography section
% 
% If you have an EPS/PDF photo (graphicx package needed) extra braces are
% needed around the contents of the optional argument to biography to prevent
% the LaTeX parser from getting confused when it sees the complicated
% \includegraphics command within an optional argument. (You could create
% your own custom macro containing the \includegraphics command to make things
% simpler here.)
%\begin{IEEEbiography}[{\includegraphics[width=1in,height=1.25in,clip,keepaspectratio]{mshell}}]{Michael Shell}
% or if you just want to reserve a space for a photo:

\begin{IEEEbiography}{Michael Shell}
Biography text here.
\end{IEEEbiography}

% insert where needed to balance the two columns on the last page with
% biographies
%\newpage

% You can push biographies down or up by placing
% a \vfill before or after them. The appropriate
% use of \vfill depends on what kind of text is
% on the last page and whether or not the columns
% are being equalized.

%\vfill

% Can be used to pull up biographies so that the bottom of the last one
% is flush with the other column.
%\enlargethispage{-5in}



% that's all folks
\end{document}