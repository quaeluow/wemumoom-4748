\section{Introduction}
\label{Introduction}

See 750 word synopsis.

\section{Motivation for Indeterminate Credal States}
\label{MotivationForIndeterminateCredal States}

Booleans claim that it is rational for an agent to hold
an indeterminate credal state towards an event. The
indeterminate credal state represents uncertainty about
the truth value of the event and informs
decision-making, inference, and betting behaviour.
Booleans are Bayesians in all other respects and defend
Bayesian epistemology, proposing that it is better off
without the requirement for sharp credences.
Laplaceans, by contrast, require sharp credences for
rational agents. We want to motivate indeterminacy,
independent of how they are elicited from the agent, as
forcefully as possible so that the reader will see (a)
the appeal of such indeterminacy, (b) the insufficiency
of the critical response, and (c) the need for careful
articulation of the Laplacean approach that mandates a
rational agent to hold sharp credences together with an
explanation of how Laplaceans can address Finally, (d)
the undesirable semantics of the Boolean approach,
documented both conceptually and in practice, will lead
to the conclusion that the Laplacean approach is
desirable.

Let a \textsc{coin} be a Bernoulli generator that
produces successes and failures with probability $p$
for success, labeled $H$, and $1-p$ for failure,
labeled $T$. Physical coins may serve as examples if we
are willing to set aside that most of them are
approximately fair. Imagine three \textsc{coin}s for
which we have evidence that \textsc{coin}$_{I}$ is
fair, \textsc{coin}$_{II}$ has an unknown bias, and
\textsc{coin}$_{III}$ has as bias either $p=1/3$ or
$p=2/3$. The Laplacean approach, on the one hand,
permits a sharp $0.5$ credence in $H$ for a rational
agent in all three cases. A Boolean approach, on the
other hand, sees the difference in the evidential
situation reflected in a rational agent's credal state
and for example at least permits, as credence in $H$,
$\{x|x=0.5\}$ for \textsc{coin}$_{I}$,
$\{x|0\leq{}x\leq{}1\}$ for \textsc{coin}$_{II}$, and
$\{x|1/3\leq{}x\leq{}2/3\}$ or $\{1/3,2/3\}$ for
\textsc{coin}$_{III}$.

Here is a list of reasons why credal states for ideally
rational agents do not need to be representable by
single probability measures.

\begin{enumerate}[(A)]
\item By far the greatest emphasis in motivation for
  indeterminacy rests on lack of evidence or
  conflicting evidence and the assumption that single
  probability measures (sharp credences) do not
  represent such evidence as well as credal states
  composed by sets of probability measures
  (indeterminacy credal states).
\item The preference structure of a rational agent may
  be incomplete so that representation theorems do not
  yield single probability measures to represent such
  incomplete structures.
\item There are more technical and paper-specific
  reasons, such as Thomas Augustin's attempt to mediate
  between the minimax pessimism of objectivists and the
  Bayesian optimism of subjectivists using interval
  probability (see \scite{8}{augustin03}{35f}); Alan
  H{\'a}jek and Michael Smithson's belief that there
  may be objectively indeterminate chances in the
  physical world (see \scite{8}{hajeksmithson12}{33});
  and Jake Chandler's claim that \qeins{the sharp model
    is at odds with a trio of plausible propositions
    regarding agnosticism} \scite{2}{chandler14}{4}.
\end{enumerate}

This paper mostly addresses (A), while taking (B)
seriously as well and pointing towards solutions for
it. I am leaving (C) to more specific responses to the
issues presented in the cited articles, and for the
remainder of this section I am adding a reason (D) that
is poorly documented in the literature.

One motivation for permitting a rational agent to have
a range of probability measures rather than mandating
her to hold a single one as her credence is that she
would do better making decisions and accepting
advantageous bets. Upper and lower previsions,
components of a plausible version of indeterminacy (see
\scite{7}{walley91}{}), have a behavioural
interpretation as maximum buying prices and minimum
selling prices for gambles where two bets can both be
rejected if they offer a reward of \$1 for a price of
$x$ in case of $H$ and a reward of \$1 for a price of
$1-x$ in case of $T$ as long as $x$ falls between the
upper and lower prevision.

Walley conducted an experiment with 17 participants,
who had various levels of understanding Bayesian
theory, asking them for upper and lower probabilities
entering into bets about the games played in the Soccer
World Cup 1982 in Spain. One of the participants, a
\qeins{dogmatic Bayesian lecturer} (see
\scite{8}{walley91}{633}), only used single sharp
subjective probabilities, while the others used
intervals. The dogmatic Bayesian lecturer finished a
distant last when the bets were evaluated.

I was curious what a more rigorous treatment of the
comparison between linear previsions (when upper and
lower previsions coincide) and interval previsions
would yield. I equipped two computer players with some
rudimentary artificial intelligence and made them
specify previsions for games played in the Soccer World
Cup 2014 in Brazil, ordering player $X$ to specify
linear previsions and player $Y$ to specify upper and
lower previsions. I used the Poisson distribution
(which is an excellent predictor for the outcome of
soccer matches) and the FIFA ranking to simulate
millions of counterfactual World Cup results and their
associated bets, using Walley's evaluation method.
Player $Y$ had a slight but systematic advantage. In
section \ref{WalleysWorldCupWoes}, I will provide an
explanation and show how it can support rejecting
indeterminate credal states for rational agents,
counter-intuitive as that may sound.

\section{Dilation}
\label{Dilation}

Roger White introduces a problem for indeterminate
credal states, which I will call White's dilation
problem, claiming that indeterminate credal states lead
to unacceptable doxastic scenarios involving dilation.
In dilation, there is a widening of the indeterminate
credal state upon updating instead of a narrowing
(especially Walley reiterates that more information
generally means more precision, see
\scite{8}{walley91}{34}, although he also addresses
dilation, see \scite{8}{walley91}{299}). Sometimes this
widening is extreme, from a maximally precise credal
state to a vacuous credal state, based on very little
information (see \scite{7}{white10}{} for examples).
White's objection, even though I think it fails as it
stands, triggers semantic concessions on the part of
Booleans defending their position which will be
important to my semantic criticism of indeterminacy.

Joyce's response to White, for example, uses two lines
of defence: the results are not counter-intuitive upon
closer inspection and Lewis' indadmissibility criterion
as well as two significant semantic concessions to
White show why the indeterminate credal states give us
the right result. This response articulates just the
right response to White: yes, dilation is what you
would expect (1) if credences do not represent evidence
(the same indeterminate credal state can reflect
different evidential situations); and (2) if the CGT is
not a necessary consequence of the indeterminacy
approach.

Based on Joyce's amendments to the Boolean approach
(and his own version of it in \scite{7}{joyce05}{}), I
agree with Joyce in White's own words that White's
dilation problem \qeins{is not a reductio but a result}
\scite{3}{white10}{177}. The possible dependence
between \textsc{coin}$_{a}$ and \textsc{coin}$_{b}$ and
the unknown bias of \textsc{coin}$_{b}$ dilate our
credences in the result of tossing \textsc{coin}$_{a}$.
The same would happen if we were drawing from an urn
with 200 balls (100 red, 100 black) and received the
information that the urn actually had two chambers, one
with 99 red balls and 1 black ball, the other with 1
red ball and 99 black balls. Our credence would dilate,
given this additional piece of information, from a
sharp $0.5$ to an indeterminate $[0.01,0.99]$ without
much mystery (the indeterminate credence would licence
an unreasonable 99:1 bet, but that is part of the
semantic criticism below).

A sharp credence, on the one hand, constrains partial
beliefs in objective chances by Lewis' summation
formula (which we will provide in the next section). No
objective chance is excluded by it and any updating
will merely change the partial beliefs, but no full
beliefs. Indeterminate credal states, on the other
hand, by giving ranges of acceptable objective chances
suggest that there is a full belief that the objective
chance does not lie outside what is indicated by the
indeterminate credal state.

If one were to be committed to the principle of
regularity, that all states of the world considered
possible have positive probability (for a defence see
\scite{7}{edwardsetal63}{}); and to the solution of
Henry Kyburg's lottery paradox, that what is rationally
accepted should have probability 1 (for a defence of
this principle see \scite{7}{douvenwilliamson06}{});
and to White's CGT, that one's spread of credence
should cover the range of possible chance hypotheses
left open by evidence; then one's indeterminate credal
state must always be vacuous. Booleans must deny at
least one of the premises to avoid the conclusion.
Joyce, for example, denies White's CGT (see
\scite{8}{joyce10}{289}), but then he continues to make
implicit use of it when he repeatedly complains that
sharp credences \qeins{ignore a vast number of
  possibilities that are consistent with [the]
  evidence} (for example in \scite{2}{joyce05}{170}).

When updating dilates the credal state, it appears that
the prior credal state was in some sense incorrect and
did not properly reflect the state of the world, even
though it properly reflected the epistemic state of the
agent. Such a divergence between the proper reflections
of epistemic state and state of the world undermines
the subjective interpretation of probabilities at the
heart of Bayesian epistemology. Credences are not
knowledge claims about the world, but represent
uncertainty in the agent's epistemic state. Yet an
indeterminate credal state blurs the line between the
two (for two rigorous attempts to reconcile traditional
full belief epistemology and formal partial belief
epistemology see \scite{7}{moss13}{}; and
\scite{7}{spohn12}{}, especially chapter 10).

White's objection fails because dilation for
indeterminate credences is in principle not any more
surprising than a piece of information that increases
the Shannon entropy of a sharp credence. It is true for
both sharp and indeterminate credences that information
can make us less certain about things. Once Booleans
have brought their house in order to accommodate
White's objection, however, they open wide the door on
semantic problems. 

We finally get to inquire what kind of coherence there
is in defending indeterminacy when it neither fulfills
the promise of representing evidence nor the promise of
reconciling traditional full belief \qnull{knowledge}
epistemology and Bayesian partial belief epistemology
as outlined in the CGT, but only adds another
hierarchical layer of uncertainty to a numerical
quantity (a sharp credence) whose job it already is to
represent uncertainty. We will turn to these semantic
considerations now, show how they display the virtues
of sharp credences in responding to the forceful
motivations for indeterminate credal states while
making those indeterminate credal states look
semantically otiose. Then we will show in the last
section how sharp credences have an elegant solution
for being outperformed by indeterminate credal states
in betting scenarios, and there we will rest our case.

\section{Semantics of Partial Belief}
\label{SemanticsOfPartialBelief}

At the heart of my project is a proper semantic
distinction between evidence, information, and partial
belief. Both sharp credences and indeterminate credal
states, within a Bayesian framework, try to represent
the uncertainty of an agent with respect to the truth
of a proposition. Indeterminate credal states have a
greater ambition: they also claim to represent
properties of the evidence, such as its weight,
conflict between its constituents, or its ambiguity.

Indeterminate credal states are suggestive of a
measurement that wants to represent numerically the
mass of an object and then also make claims about its
density. With sharp credences, the semantic roles of
evidence, information, and uncertainty are
appropriately differentiated. Rational decision-making,
inference, and betting behaviour are based on sharp
credences together with the evidence that is at its
foundation. Information represents evidence, and sharp
credences represent uncertainty. Measurement is in any
case a misleading analogy for credences. Measurements
are always imprecise. Epistemic credences, however, are
not measurements, especially not of objective chances.
They represent uncertainty. They are more like logical
truth values than they are like measurements.
\scite{8}{levi85}{407}, makes this category mistake,
whereas \scite{8}{walley91}{249} is clear on the
difference.

It is important not to confuse the claim that it is
reasonable to hold both $X$ and $Y$ with the claim that
it is reasonable to hold either $X$ (without $Y$) or
$Y$ (without $X$). It is the reasonableness of holding
$X$ and $Y$ concurrently that is controversial, not the
reasonableness of holding $Y$ (without holding $X$)
when it is reasonable to hold $X$. We will later talk
about anti-luminosity, the fact that a rational agent
may not be able to distinguish psychologically between
a 54.9 cent bet on an event and a 45.1 bet on its
negation. She must reject one of them not to incur sure
loss, so proponents of indeterminacy suggest that she
choose one of them freely without being constrained by
her credal state or reject both of them. I claim that a
sharp credence will make a recommendation between the
two so that only one of the bets is rational given her
particular credence, but that does not mean that
another sharp credence which would give a different
recommendation may not also be rational for her to
have.

There are two approaches you can take towards the
indeterminacy question, both of them involving two
levels. On the one hand, you can have partial beliefs
about how a parameter is distributed and then use
Lewis' summation formula (see \scite{8}{lewis81}{266f})
to integrate over them and condense them to a sharp
credence. Walley comments on this \qeins{reduction} in
his section on Bayesian second order probabilities (see
\scite{8}{walley91}{258f}), but he mistakenly
represents the Laplacean approach as a second order
approach, as if the probability distributions that are
summarized by Lewis' formula are of the same kind as
the resulting credences. They are not. They are
objective chances or other partitions of the event
space and the subjective probabilities that are
associated with them. It is the Boolean approach which
has elements of a second order approach and thus makes
itself vulnerable to regress problems by adding another
dimension of uncertainty to a parameter (the credence)
which already represents uncertainty.

On the other hand, you can try to represent your
uncertainty about the distribution of the parameter by
an indeterminate credal state. I want to show that the
first approach can be aligned with the forceful
motivations we listed in the previous section to
introduce indeterminate credal states, as long as we do
not require that a sharp credence represent the
evidence as well as the epistemic state of uncertainty
in the agent. We have learned that this requirement can
be reduced ad absurdum even for indeterminate credal
states.

One of Joyce's complaints is that a sharp credence of
$0.5$ for a \textsc{coin} contains too much information
if there is little or no evidence that the
\textsc{coin} is fair. This complaint, of course, is
only effective if we make a credence say something
about the evidence. Joyce himself, however, admits that
indeterminate credal states cannot represent the
evidence without violating the reflection principle due
to White's dilation problem. He is quite clear that the
same indeterminate credal state can represent different
evidential scenarios (see, for example,
\scite{8}{joyce10}{302}).

In any case, Walley's and Joyce's claim that
indeterminate credal states are less informative than
sharp credences has no foundation in information theory
(see \scite{8}{walley91}{34}; and
\scite{8}{joyce10}{311} for examples, but this attitude
is passim). To compare indeterminate credal states and
sharp credences informationally, we would need a
non-additive set function obeying Shannon's axioms for
information. This is a non-trivial task. I have not
succeeded solving it, but I am not at all convinced
that it will result in an information measure which
assigns more information to a sharp credence such as
$\{0.5\}$ than to an indeterminate credal state such as
$\{x|1/3\leq{}x\leq{}2/3\}$ for a binomial random
variable.

Augustin recognizes the problem of inadequate
representation long before Joyce, with specific
reference to indeterminate credal states: \qeins{The
  imprecise posterior does no longer contain all the
  relevant information to produce optimal decisions.
  Inference and decision do not coincide any more}
\scite{2}{augustin03}{41} (see also an example for
inadequate representation of evidence by indeterminate
credal states in \scite{8}{bradleysteele13}{16}).
Indeterminate credal states fare no better than sharp
credences, except perhaps for the problem that they
unhelpfully mimic saying something about the evidence
that is much better said elsewhere.

Not only can we align sharp credences with the
motivations to introduce indeterminate credal states,
we can also show that indeterminate credal states
perform worse semantically because they mix evidential
and epistemic metaphors in deleterious ways. Sharp
credences have one task: to represent epistemic
uncertainty and serve as a tool for updating,
inference, and decision-making. They cannot fulfill
this task without continued reference to the evidence
which operates in the background. To use an analogy,
credences are not sufficient statistics with respect to
updating, inference, and decision-making. What is
remarkable about Joyce's response to White's dilation
problem is that Joyce recognizes that indeterminate
credal states are not sufficient statistics either. But
this means that they fail at the double task which has
been imposed on them: to represent both epistemic
uncertainty and the evidence.

In the following, I will provide a few examples where
it becomes clear that indeterminate credal states have
difficulty representing uncertainty because they are
tangled in a double task which they cannot fulfill.

\begin{quotex}
  \textbf{Example 1: Aggregating Expert Opinion} You
  have no information whether it will rain tomorrow
  ($R$) or not except the predictions of two weather
  forecasters. One of them forecasts 0.3 on channel
  GPY, the other 0.6 on channel QCT. You consider the
  QCT forecaster to be significantly more reliable,
  based on past experience.
\end{quotex}

An indeterminate credal state corresponding to this
situation may be $[0.3,0.6]$ (see
\scite{8}{walley91}{214}), but it will have a difficult
time representing the difference in reliability of the
experts. A sharp credence of $P(R)=0.55$, for example,
does the right thing. Such a credence says nothing
about any beliefs that the objective chance of $R$ is
$x\in{}I$ or restricted to $X\subseteq{}I$ (where $I$
is the unit interval), but it accurately reflects the
degree of uncertainty that the rational agent has over
the various possibilities. Beliefs about objective
chances make little sense in many situations where we
have credences, since it is doubtful even in the case
of rain tomorrow that there is an urn of nature from
which balls are drawn. What is really at play is a
complex interaction between epistemic states (for
example, experts evaluating meteorological data) and
the evidence which influences them.

A sharp credence is often associated with distributions
over chances or other partitions of the event space,
while an indeterminate credal state puts chances in
sets where they all have an equal voice. This may also
be at the bottom of Susanna Rinard's objection (see
\scite{8}{white10}{184}) that Joyce's committee members
are all equally enfranchised and so it is not clear how
extremists among them could not always be replaced by
even greater extremists even after updating on evidence
which should serve to consolidate indeterminacy. Joyce
has a satisfactory response to this objection (see
\scite{8}{joyce10}{291}), but I do not see how the
response addresses the problem of aggregating expert
opinion without the kind of summation that Laplaceans
find unobjectionable, even though information is lost
and can only be recouped by going back to the evidence.
More generally, the two levels for sharp credences,
representation of uncertainty and distributions over
partitions, tidily differentiate between the epistemic
and the evidential dimension; indeterminate credal
states, on the other hand, just add another level of
uncertainty on top of the uncertainty that is already
expressed in the partial belief and thus do not make
the appropriate semantic distinctions.

\begin{quotex}
  \textbf{Example 2: How Precise Can a Rational Agent
    Reasonably Be} Your sharp credence for rain
  tomorrow, based on the expert opinion of channel GPY
  and channel QCT (you have no other information) is
  $0.55$. Is it reasonable, considering how little
  evidence you have, to reject the belief that the
  chance of rain tomorrow is $0.54$ or $0.56$; or to
  prefer a 54.9 cent bet on rain to a 45.1 cent bet on
  no rain?
\end{quotex}

The first question, of course, is confused, but in
instructive ways (a display of this confusion is found
in \scite{8}{hajeksmithson12}{38f}, and their doctor
and time of the day analogy). A sharp credence rejects
no hypothesis about objective chances (unlike an
indeterminate credal state). It often has a subjective
probability distribution operating in the background,
over which it integrates to yield the sharp credence
(it would do likewise in H{\'a}jek and Smithson's
example for the prognosis of the doctor or the time of
the day, without any problems). This subjective
probability distribution may look like this:

\begin{tabular}{|lcr|} 
  \hline
  $P(\pi(R)=0.00)$ & = & $0.0001$ \\ \hline
  $P(\pi(R)=0.01)$ & = & $0.0003$ \\ \hline
  $P(\pi(R)=0.02)$ & = & $0.0007$ \\ \hline
  $\ldots$ & & $\ldots$ \\ \hline
  $P(\pi(R)=0.30)$ & = & $0.0015$ \\ \hline
  $P(\pi(R)=0.31)$ & = & $0.0016$ \\ \hline
  $\ldots$ & & $\ldots$ \\ \hline
  $P(\pi(R)=0.54)$ & = & $0.031$ \\ \hline
  $P(\pi(R)=0.55)$ & = & $0.032$ \\ \hline
  $P(\pi(R)=0.56)$ & = & $0.054$ \\ \hline
  $\ldots$ & & $\ldots$ \\ \hline
\end{tabular}

It is condensed by Lewis' summation formula to a sharp
credence, without being reduced to it:

\begin{equation}
  \label{eq:s2}
  P(R)=\int_{0}^{1}\zeta{}P(\pi(R)=\zeta)\,d\zeta
\end{equation}

There are more semantic questions here: what is
$\pi(R)$, the objective chance that it rains tomorrow,
and how do we get to use $P(\pi(R)=\zeta)$ in our
calculation of $P(R)$ without begging the question.
Lewis' 1981 paper \qeins{A Subjectivist's Guide to
  Objective Chance} remains the gold standard in
addressing these questions, and I will no longer pursue
them here. The point is that we have properly separated
the semantic dimensions and that the Laplacean approach
is not a second order probability approach. The partial
belief epistemology deals with sharp credences and how
they represent uncertainty and serve as a tool in
inference, updating, and decision making; while Lewis'
Humean speculations and his interpretation of the
principal principle cover the relationship between
subjective probabilities and objective chance.

Indeterminate credal states, by contrast, mix these
semantic dimensions so that in the end we get a muddle
where a superficial reading of indeterminacy suddenly
follows a converse principal principle of sorts, namely
that objective chances are constrained by the factivity
of a rational agent's credence when this credence is
knowledge (Lewis actually talks about such a converse,
but in completely different and epistemologically more
intelligible terms, see \scite{8}{lewis81}{289}; for
credences being based on knowledge see
\scite{8}{levi81}{540}). Sharp credences are more, not
less, permissive with respect to objective chances
operating externally (compared to the internal
epistemic state of the agent, which the credence
reflects).

The second question is also instructive: why would we
prefer a 54.9 cent bet on rain to a 45.1 cent bet on no
rain, given that we do not possess the power of
descrimination between these two bets? The answer to
this question ties in with the issue of incomplete
preference structure referenced above as motiviation
(B) for indeterminate credal states.

The development of representation theorems beginning
with Frank Ramsey (followed by increasingly more
compelling representation theorems in
\scite{7}{savage54}{}; and \scite{7}{jeffrey65}{}; and
numerous other variants in contemporary literature)
puts the horse before the cart and bases probability
and utility functions of an agent on her preferences,
not the other way around. Once completeness as an axiom
for the preferences of an agent is jettisoned,
indeterminacy follows automatically. Indeterminacy may
thus be a natural consequence of the proper way to
think about credences in terms of the preferences that
underlie them.

In response, preferences may very well logically and
psychologically precede an agent's probability and
utility functions, but that does not mean that we
cannot inform the axioms we use for a rational agent's
preferences by undesirable consequences downstream.
Completeness may sound like an unreasonable imposition
at the outset, but if incompleteness has unwelcome
semantic consequences for credences, it is not
illegitimate to revisit the issue. Timothy Williamson
goes through this exercise with vague concepts, showing
that all upstream logical solutions to the problem fail
and that it has to be solved downstream with an
epistemic solution (see \scite{7}{williamson96}{}).
Vague concepts, like sharp credences, are sharply
bounded, but not in a way that is luminous to the agent
(for anti-luminosity see chapter 4 in
\scite{7}{williamson00}{}). Anti-luminosity answers the
original question: the rational agent prefers the 54.9
cent bet on rain to a 45.1 cent bet on no rain based on
her sharp credence without being in a position to have
this preference necessarily or have it based on
physical or psychological ability (for the analogous
claim about knowledge see \scite{8}{williamson00}{95}).

In a way, advocates of indeterminacy have solved this
problem for us. There is strong agreement among most of
them that the issue of determinacy for credences is not
an issue of elicitation. The appeal of preferences is
that we can elicit them more easily than assessments of
probability and utility functions. The indeterminacy
issue has been raised to the probability level (or
moved downstream) by indeterminacy advocates themselves
who feel justifiably uncomfortable with an
interpretation of their theory in behaviourist terms.
So it shall be solved there, and this paper makes an
appeal to reject indeterminacy on this level.
Ironically, Isaac Levi seems to agree with me on this
point: when he talks about indeterminacy, it proceeds
from the level of probability judgment to preferences,
not the other way around (see \scite{8}{levi81}{533}).

\begin{quotex}
  \textbf{Example 3: Jaynes' Monkeys} E.T. Jaynes
  describes an experiment with monkeys filling an urn
  randomly with balls from another urn, for which
  sampling provides no information and so makes
  updating vacuous (see
  \scite{8}{jaynesbretthorst03}{160}). Here is a
  variant of this experiment for which a sharp credence
  provides a more compelling result than the associated
  indeterminate credal state: Let urn $A$ contain 4
  balls, two red and two black. A monkey randomly fills
  urn $B$ from urn $A$ with two balls. We draw from urn
  $B$.
\end{quotex}

The sharp credence of drawing a red ball is $0.5$,
following Lewis' summation formula for the different
combinations of balls in urn $B$. I find this solution
more intuitive in terms of further inference, decision
making, and betting behaviour than a credal state of
$\{0,1/2,1\}$, since this indeterminate credal state
would licence an exorbitant bet in favour of one
colour, for example one that costs \$9,999 and pays
\$10,000 if red is drawn and nothing if black is drawn.

\begin{quotex}
  \textbf{Example 4: Three Prisoners} Prisoner $X_{1}$
  knows that two out of three prisoners
  ($X_{1},X_{2},X_{3}$) will be executed and one of
  them pardoned. He asks the warden of the prison to
  tell him the name of another prisoner who will be
  executed, hoping to gain knowledge about his own
  fate. When the warden tells him that $X_{3}$ will be
  executed, $X_{1}$ erroneously updates his probability
  of pardon from $1/3$ to $1/2$, since either $X_{1}$
  or $X_{2}$ will be spared.
\end{quotex}

Peter Walley maintains that for the Monty Hall problem
and the Three Prisoners problem, the probabilities of a
rational agent should dilate rather than settle on the
commonly accepted solutions. Consider the three
prisoners problem. There is a compelling case for
standard conditioning and the result that the chances
of pardon for prisoner $X_{1}$ are unchanged after the
update (see \scite{8}{lukits14}{1421f}). Walley's
dilated solution would give prisoner $X_{1}$ hope on
the doubtful possibility (and unfounded assumption)
that the warden might prefer to provide $X_{3}$'s name
in case prisoner $X_{1}$ was pardoned.

This example brings an interesting issue to the
forefront. Sharp credences often reflect independence
of variables where such independence is unwarranted.
Booleans (more specifically, detractors of the
principle of indifference or the principle of maximum
entropy, principles which are used to generate sharp
credences for rational agents) tend to point this out
gleefully. They prefer to dilate over the possible
dependence relationships (independence included).
White's dilation problem is an instance of this. The
fallacy in the argument for indeterminate credal
states, illustrated by the three prisoners problem, is
that the probabilistic independence of sharp credences
does not imply independence of variables (the converse
is correct), but only that it is unknown whether there
is dependence, and if yes, whether it is correlation or
inverse correlation.

\begin{quotex}
  \textbf{Example 5: Wagner's Linguist} A linguist
  hears the utterance of a native and concludes that
  the native cannot be part of certain population
  groups, depending on what the utterance means. The
  linguist is uncertain between some options about the
  meaning of the utterance. (For full details see
  \scite{8}{wagner92}{252}; and
  \scite{8}{spohn12}{197}.)
\end{quotex}

The mathematician Carl Wagner proposed a natural
generalization of Jeffrey Conditioning for his Linguist
example (see \scite{7}{wagner92}{}). Since the
principle of maximum entropy is already a
generalization of Jeffrey Conditioning, the question
naturally arises whether the two generalizations agree.
Wagner makes the case that they do not agree and
deduces that the principle of maximum entropy is
sometimes an inappropriate updating mechanism, in line
with many earlier criticisms of the principle of
maximum entropy (see \scite{7}{fraassen81}{};
\scite{7}{shimony85}{}; \scite{7}{skyrms87updating}{};
and, later on, \scite{7}{grovehalpern97}{}). What is
interesting about this case is that Wagner uses
indeterminate credal states for his deduction, so that
even if you agree with his natural generalization of
Jeffrey Conditioning (which I find plausible), the
inconsistency with the principle of maximum entropy can
only be inferred assuming indeterminate credal states.
Wagner is unaware of this, and I am showing in another
paper (in process) how on the assumption of sharp
credences Wagner's generalization of Jeffrey
conditioning perfectly accords with the principle of
maximum entropy.

This will not convince a proponent of indeterminate
credences, since they are already unlikely to believe
in the general applicability of the principle of
maximum entropy (just as Wagner's argument is unlikely
to convince a proponent of the principle of maximum
entropy, since they are more likely to reject
indeterminate credal states). The battle lines are
clearly drawn. Wagner's argument, instead of
undermining the principle of maximum entropy, just
shows that indeterminate credal states are as wedded to
rejecting the claims of the principle of maximum
entropy as the principle of maximum entropy is wedded
to sharp credences.

Endorsement of indeterminate credal states, however,
implies that there are situations of probability update
in which the posterior probability distribution is more
informative than it might be in terms of informative
theory. Indeterminate credences violate the relatively
natural intuition that we should not gain information
from evidence when a less informative updated
probability will do the job of responding to the
evidence. This is not a strong argument in favour of
sharp credences. The principle of maximum of entropy
has received a thorough bashing in the last thirty
years. I consider it to be much easier to convince
someone to reject indeterminate credal states on
independent (semantic) grounds than to convince them to
give the principle of maximum entropy a second chance.
But the section on semantics comes to an end here, and
we want to proceed to the intriguing issue of who does
better in betting situations: indeterminate credal
states or sharp credences.

\section{Evidence Differentials and Cushioning
  Credences}
\label{WalleysWorldCupWoes}

I have given away the answer already in the
introduction: indeterminate credal states do better.
After I found out that agents with indeterminate credal
states do better betting on soccer games, I let player
$X$ (who uses sharp credences) and player $Y$ (who uses
indeterminate credal states) play a more basic betting
game. An $n$-sided die is rolled (by the computer). The
die is fair, unbeknownst to the players. Their bets are
randomly and uniformly drawn from the simplex for which
the probabilities attributed to the $n$ results add up
to 1. Player $Y$ also surrounds her credences with an
imprecision uniformly drawn from the interval $(0,y)$.
I used Walley's pay off scheme (see
\scite{8}{walley91}{632}) to settle the bets.

The simulation results show that player $Y$ does better
for $n>2$ while player $X$ does better for $n=2$. We
should get similar results if we do this analytically
instead of using computer simulation. I will pursue
this further for the final version of the paper. The
math is not complicated, but unwieldy. The following
expression yields the expected gain for player $X$:

\begin{eqnarray}
  \label{eq:s3}
  EX =
  \frac{p_{x}}{n}\left(\sum_{j=0}^{n-1}\int_{0}^{1}\int_{0}^{x}\int_{0}^{x-y}\sum_{k=0}^{n-1}g(x,y,s,k,j)\,ds\,d\upsilon(y)\,d\xi(x)\right)+
  \notag \\
  \frac{p_{y}}{n}\left(\sum_{j=0}^{n-1}\int_{0}^{1}\int_{0}^{1}\int_{0}^{y-s}\sum_{k=0}^{n-1}g(x,y,s,k,j)\,d\xi(x)\,d\upsilon(y)\,ds\right)
\end{eqnarray}

where $p_{x}$ and $p_{y}$ are the respective
probabilities that $X$ and $Y$ win a bet and $g$ is
$X$'s gain given $X$'s credence $x$ and $Y$'s credal
state $y\pm{}s$ on roll $j$, given result $k$. $\xi(x)$
and $\upsilon(y)$ are the distributions of the
credences given our method of simplex point picking
(for these distributions, one must use the
Cayley-Menger Determinant to find out the volume of
generalized pentatopes involved). A defence of sharp
credences for rational agents needs to have an
explanation why for $n>2$, player $Y$ does better. We
will call it partial belief cushioning, which is based
on an evidence differential between the bettors.

In many decision-making context, we do not have the
luxury of calling off the bet. We have to decide one
way or another. This is a problem for indeterminate
credal states, as Booleans have to find a way to decide
without receiving clear instructions from the credal
state. Booleans have addressed this point extensively
(see for example \scite{8}{joyce10}{311ff}; for an
opponent's view of this see \scite{8}{elga10}{6ff}).
The problem for sharp credences arises when bets are
noncompulsory, for then the data above suggest that
agents holding indeterminate credal states sometimes do
better. Often, decision-making happens as betting
vis-{\`a}-vis uninformed nature or opponents which are
at least as uninformed as the rational agent.
Sometimes, however, bets are offered by better informed
or potentially better informed bookies. In this case,
even an agent with sharp credences must cushion her
credences and is better off by rejecting bets that look
attractive in terms of her partial beliefs. If an agent
does not cushion her partial beliefs (whether they are
sharp or indeterminate), she will incur a loss in the
long run. Since cushioning is permitted in Walley's
experimental setup, Laplacean agents should also have
access to it and then no longer do worse than Boolean
agents.

Cushioning does not stand in the way of holding a sharp
credence, even if the evidence is dim. The evidence
determines for a rational agent the partial beliefs
over possible states of the world operating in the
background. The better the evidence, the more pointed
the distributions of these partial beliefs will be and
the more willing the rational agent will be to enter a
bet. The mathematical decision rule will be based on
the underlying distribution of the partial beliefs, not
only on the sharp credence. As we have stated before,
the sharp credence is not a sufficient statistic for
decision-making, inference, or betting behaviour; and
neither is an indeterminate credal state. If a rational
agent perceives an evidence differential and lends some
belief to the proposition that the bet is offered by
someone who has more evidence about the outcome of an
event than she does, then it is likely that the
rational agent will update her sharp credence, as she
would do if she were informed of another source of
expert opinion. She will certainly not be willing to
enter a bet based on her outdated sharp credence.

The rational agent with a sharp credence has resources
at her disposal to use just as much differentiation
with respect to accepting and rejecting bets as the
agent with indeterminate credal states. Often (if she
is able to and especially if the bets are offered to
her by a better-informed agent), she will reject both
of two complementary bets, even when they are fair. On
the one hand, any advantage that the agent with an
indeterminate credal state has over her can be
counteracted based on her distribution over partial
beliefs that she has with respect to all possibilities.
On the other hand, the agent with indeterminate credal
states suffers from a semantic mixing of metaphors
between evidential and epistemic dimensions that puts
her at a real disadvantage in terms of understanding
the sources and consequences of her knowledge and her
uncertainties.
