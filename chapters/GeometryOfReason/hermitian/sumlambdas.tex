%\documentclass[10pt]{article}
\documentclass[landscape,12pt]{article}

\setlength{\parindent}{0in}
\setlength{\parskip}{.3in}

\raggedbottom

\pagestyle{empty}

% 	PACKAGES
\usepackage{amsfonts}
\usepackage{amssymb}
\usepackage[fleqn]{amsmath}
%\usepackage{german}
%\usepackage{hebtex}
%\usepackage{pdfpages} % to include pdf pages
%\usepackage{graphics}
% \usepackage[german]{babel}
%\usepackage{helvet}

\newcommand{\R}{\mathbb{R}}
\newcommand{\C}{\mathbb{C}}
\newcommand{\N}{\mathbb{N}}
\newcommand{\Q}{\mathbb{Q}}
\newcommand{\Z}{\mathbb{Z}}
\newcommand{\PP}{\mathbb{P}}

\begin{document}

Naohito,

Thank you very much for responding to my inquiries so far. I have been
working through your 1993 paper, Geometrical Structures of Some
Non-Distance Models for Asymmetric MDS. Here is something I am
wondering about. You define the seminorm (page 38)

\begin{equation}
  \label{eq:n1}
  \|\zeta\|=\sqrt{\varphi(\zeta,\zeta)}
\end{equation}

and then say, ``in particular, if $\varphi$ is positive for any
$\zeta\neq{}0,\|\zeta\|$ defines a \emph{norm}.'' Let me defend the
claim that $\sqrt{\varphi(\zeta,\zeta)}$ \textbf{never} defines a norm
because the diagonal matrix $\Lambda$ is always indefinite (i.e.\ it
always contains at least one negative and one positive real number).
Let's call this claim \textsc{always-indef}.

Recall that $\varphi$ was defined as follows:

\begin{equation}
  \label{eq:n2}
  \varphi(\zeta,\tau)=\zeta\Lambda\tau^{*}
\end{equation}

and 

\begin{equation}
  \label{eq:n3}
  \Lambda=\mbox{diag}(\lambda_{1},\ldots,\lambda_{n})
\end{equation}

where the $\lambda_{j}$ are the eigenvalues (repeated according to
multiplicity) of $H$ as defined on page 36. $H$ is a Hermitian matrix
with $\mbox{tr}(H)=0$, and according to a well-known theorem in linear
algebra (see link in email)

\begin{equation}
  \label{eq:n4}
  \sum_{j=1}^{n}\lambda_{j}=\mbox{tr}(H)=0.
\end{equation}

In other words, the trace of $\Lambda$ is $0$. \textsc{always-indef}
follows immediately unless $\Lambda=0$, which is a trivial case.

This appears to be sad news for both you and me. You were trying to
model asymmetries. The Hermitian Form Model looks initially very
promising and elegant, but now it turns out that the seminorm defined
on the Hilbert Space is always indefinite. I was hoping that I could
classify asymmetry as follows: (null) no asymmetry, for example
Euclidean distance; (well-behaved) asymmetry obeying the triangle
inequality and transitivity, for example Tobler's wind model;
(ill-behaved) asymmetry violating the triangle inequality and
transitivity, for example the Kullback-Leibler divergence. Your
distinction between only seminorm-inducing asymmetries and
norm-inducing asymmetries may have delivered such a classification, as
you expressed in your email:

\begin{quote}
  Very interesting! I suppose that your finding may be explained by
  examining the definiteness of the Hermitian matrix
  $H=(S+S')/2+i*(S-S')/2$ constructed from the proximity matrix $S$
  which you specified above. Here, $S'$ denotes the transposed matrix
  of $S$, and $i$ denotes the pure imaginary number. This inspection
  comes from the theory of the Hermitian Form Model (HFM) proposed by
  Chino and Shiraiwa (1993), Behaviormetrika. Our theory states that
  members (or objects) are embedded in a finite-dimensional Hilbert
  space if and only if the matrix $H$ is positive-semi definite.
\end{quote}

Unofortunately, as it turns out, all cats are grey at night and there
are no norm-inducing asymmetries. Even if $H$ is symmetrical and
$S_{sk}=0$ the seminorm on the Hilbert Space is still indefinite.

% Consider

% \begin{equation}
%   \label{eq:n5}
%   \det(H-\lambda{}I)=\det
%       \left[
%       \begin{array}{cccc}
%         -\lambda & h_{12} & \cdots & h_{1n} \\
%         h_{21} & -\lambda & \cdots & h_{2n} \\
%         \vdots & \vdots & \ddots & \vdots \\
%         h_{n1} & \cdots & \cdots & -\lambda 
%       \end{array}
% \right].
% \end{equation}

\end{document}