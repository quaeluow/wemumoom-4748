% TEX PROPERTIES

\documentclass[11pt]{article}

\setlength{\parindent}{0in}
\pagestyle{empty}

\setlength{\parskip}{.1in}

\raggedbottom

% \pagestyle{myheadings}

% PACKAGES

\usepackage{amsfonts}

\usepackage{amssymb}

% \usepackage{graphixs}

% NEW COMMANDS

\newcommand{\R}{\mathbb{R}}

\newcommand{\C}{\mathbb{C}}

\newcommand{\N}{\mathbb{N}}

\newcommand{\Q}{\mathbb{Q}}

\newcommand{\Z}{\mathbb{Z}}

\newcommand{\PP}{\mathbb{P}}

% DOCUMENT BODY

\begin{document}

Thomas,

Beweise es oder gib mir ein Gegenbeispiel.

Wenn

\begin{equation}
\label{eq:eins}
  \log\left(\frac{p}{q}\right)>\log\left(\frac{\hat{q}}{\hat{p}}\right)
\end{equation}

dann

\begin{equation}
\label{eq:zwei}
  (p+q)\log\left(\frac{p}{q}\right)>(\hat{p}+\hat{q})\log\left(\frac{\hat{q}}{\hat{p}}\right).
\end{equation}

Hiebei ist zu beachten dass

\begin{equation}
\label{eq:drei}
  \hat{p}=1-p\mbox{ und }\hat{q}=1-q
\end{equation}

und sowohl $0<p<1$ als auch $0<q<1$.

M{\"o}glichkeit (hat bei mir aber nicht funktioniert): (\ref{eq:eins})
hei{\ss}t schlichtweg, dass die Wahrscheinlichkeiten $p$ und $\hat{p}$
mehr in der Mitte liegen als $q$ und $\hat{q}$, also
$0<q<p\leq{}\hat{p}<\hat{q}<1$. Man sollte das ausnutzen und
(\ref{eq:eins}) als folgende Bedingung umformulieren:

\begin{equation}
  \label{eq:vier}
  p=\frac{1-\pi}{2}\mbox{ und }q=\frac{1-\kappa}{2}
\end{equation}

also $\pi=\hat{p}-p$ und $\kappa=\hat{q}-q$. Bedingung (\ref{eq:eins})
ist dann 

\begin{equation}
  \label{eq:fuenf}
  \frac{\pi-\kappa}{2}<0
\end{equation}

und das zu Beweisende (\ref{eq:zwei}) ist $A<\frac{\pi-\kappa}{2}B$, wobei:

\begin{equation}
  \label{eq:sechs}
  A=(1+\pi)\log\left(\frac{1+\kappa}{1+\pi}\right)-(1-\pi)\log\left(\frac{1-\pi}{1-\kappa}\right)
\end{equation}

\begin{equation}
  \label{eq:sieben}
  B=\log\left(\frac{1+\kappa}{1+\pi}\right)+\log\left(\frac{1-\pi}{1-\kappa}\right)
\end{equation}

\end{document}


