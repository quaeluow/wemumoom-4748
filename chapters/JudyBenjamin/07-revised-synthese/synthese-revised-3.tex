%%%%%%%%%%%%%%%%%%%%%% file template.tex %%%%%%%%%%%%%%%%%%%%%%%%%
%
% This is a general template file for the LaTeX package SVJour3
% for Springer journals.          Springer Heidelberg 2010/09/16
%
% Copy it to a new file with a new name and use it as the basis
% for your article. Delete % signs as needed.
%
% This template includes a few options for different layouts and
% content for various journals. Please consult a previous issue of
% your journal as needed.
%
%%%%%%%%%%%%%%%%%%%%%%%%%%%%%%%%%%%%%%%%%%%%%%%%%%%%%%%%%%%%%%%%%%%

\RequirePackage{fix-cm}

%\documentclass{svjour3}                     % onecolumn (standard format)
%\documentclass[smallcondensed]{svjour3}     % onecolumn (ditto)
\documentclass[smallextended]{svjour3}       % onecolumn (second format)
%\documentclass[twocolumn]{svjour3}          % twocolumn

\smartqed  % flush right qed marks, e.g. at end of proof

% \usepackage[small,bf]{caption}
% \usepackage{titlesec}
% \titleformat{\section}{\large\bfseries}{\thesection}{1em}{}
\usepackage{graphicx}
\usepackage{amsfonts}
\usepackage{amssymb}
\usepackage{amsmath}
\usepackage[table]{xcolor}
\definecolor{lightgray}{gray}{0.9}
% \usepackage{german}
% \usepackage{hebtex}
% \usepackage[german]{babel}
% \usepackage{endnotes}
% \let\footnote=\endnote
% \usepackage{rotating}
\usepackage{enumerate}

% please place your own definitions here and don't use \def but
% \newcommand{}{}
%
\newcommand{\nias}{\noindent} % no indent after new section
\newcommand{\nial}{\noindent} % no indent after equation, list, or whatever
\newcommand{\nonsc}[1]{}
\newcommand{\qnull}[1]{`#1'}
\newcommand{\qeins}[1]{``#1''}
\newcommand{\qzwei}[1]{`#1'}
\newcommand{\erf}[0]{\mbox{erf}}
\newcommand{\anum}[0]{a'}
\newcommand{\bnum}[0]{b'}
\newcommand{\cnum}[0]{c'}
\newcommand{\hnum}[0]{h'}
\newcommand{\knum}[0]{k'}
\newcommand{\wnum}[0]{w'}
\newcommand{\aden}[0]{a''}
\newcommand{\bden}[0]{b''}
\newcommand{\cden}[0]{c''}
\newcommand{\hden}[0]{h''}
\newcommand{\kden}[0]{k''}
\newcommand{\wden}[0]{w''}
\newcommand{\lwv}[0]{0.6}
\newcommand{\qvu}[0]{\vartheta}
% \newif\ifNumericalOrYear
% \NumericalOrYeartrue
% \NumericalOrYearfalse
% \ifNumericalOrYear
% \usepackage[numbers,colon]{natbib}
% \else
% \usepackage[round,colon]{natbib}
% \usepackage{natbib}
% \fi
\newif\ifPageP
\PagePtrue
\PagePfalse
\ifPageP
\newcommand{\PageP}{p.~}
\else
\newcommand{\PageP}{}
\fi
% \newcommand{\scite}[3]{\ifnum#1=1\ifNumericalOrYear\citep{#2}\else\citeyearpar{#2}\fi\else
% \ifnum#1=2\ifNumericalOrYear\citep[#3]{#2}\else\citep[{\PageP}#3]{#2}\fi\else
% \ifnum#1=3\ifNumericalOrYear(\citet[#3]{#2})\else\citep[{\PageP}#3]{#2}\fi\else
% \ifnum#1=4\ifNumericalOrYear\citet{#2}\else\citet{#2}\fi\else
% \ifnum#1=5\ifNumericalOrYear(\citet{#2})\else\citep{#2}\fi\else
% \ifnum#1=6\ifNumericalOrYear(\citet[#3]{#2})\else\citep[{\PageP}#3]{#2}\fi\else
% \ifnum#1=7\ifNumericalOrYear\citep{#2}\else\citealp{#2}\fi\else
% \ifnum#1=8\ifNumericalOrYear\citep[#3]{#2}\else\citealp[{\PageP}#3]{#2}\fi\else
% \ifnum#1=9\ifNumericalOrYear\citep[#3]{#2}\else{}loc.\ cit., {\PageP}#3\fi\else
% \textbf{[invalid scite code]}\fi\fi\fi\fi\fi\fi\fi\fi\fi}
\newcommand{\scite}[3]{\ifnum#1=1\cite{#2}\else
\ifnum#1=2\cite[{\PageP}~#3]{#2}\else
\ifnum#1=3\cite[{\PageP}~#3]{#2}\else
\ifnum#1=4\cite{#2}\else
\ifnum#1=5\cite{#2}\else
\ifnum#1=6\cite[{\PageP}~#3]{#2}\else
\ifnum#1=7\cite{#2}\else
\ifnum#1=8\cite[{\PageP}~#3]{#2}\else
\ifnum#1=9\cite[{\PageP}~#3]{#2}\else
\textbf{[invalid scite code]}\fi\fi\fi\fi\fi\fi\fi\fi\fi}
% \newcommand{\scite}[3]{#2}
% \newcommand{\scite}[3]{\cite{#2}}

\newenvironment{quotex}{\begin{quote}\begin{footnotesize}}{\end{footnotesize}\end{quote}}

% Insert the name of "your journal" with
\journalname{Synthese}

\begin{document}

\title{The Principle of Maximum Entropy and a Problem in Probability Kinematics}
% \subtitle{Do you have a subtitle?\\ If so, write it here}

\author{Stefan Lukits}

\institute{Stefan Lukits \at
University of British Columbia \\
Department of Philosophy \\
1866 Main Mall E370 \\
Vancouver BC Canada V6T 1Z1 \\
\email{saiserit@streetgreek.com}
}
% \institute{for blind review}

\date{Received: date / Accepted: date}
% The correct dates will be entered by the editor

\maketitle

\begin{abstract}
  \noindent Given a more general type of evidence than Bayes' formula
  will accommodate, the principle of maximum entropy (\textsc{maxent})
  provides a unique solution for the posterior probability
  distribution based on the intuition that the information gain
  consistent with assumptions and evidence should be minimal.
  \keywords{Judy Benjamin \and Principle of Maximum Entropy \and
    Coarsening at Random \and Full Employment Theorem \and Probability
    Kinematics \and Epistemic Entrenchment}
\end{abstract}

\section{Introduction}
\label{Introduction}

Probability kinematics is the field of inquiry asking how we should
update a probability distribution in the light of evidence. 
% BibTeX users please use one of
% \bibliographystyle{stefan-2010-08-28} 
\bibliographystyle{spbasic}      % basic style, author-year citations
%\bibliographystyle{spmpsci}      % mathematics and physical sciences
%\bibliographystyle{spphys}       % APS-like style for physics
%\bibliography{}   % name your BibTeX data base
\bibliography{bib-3306}

% % Non-BibTeX users please use
% \begin{thebibliography}{}
% %
% % and use \bibitem to create references. Consult the Instructions
% % for authors for reference list style.
% %
% \bibitem{RefJ}
% % Format for Journal Reference
% Author, Article title, Journal, Volume, page numbers (year)
% % Format for books
% \bibitem{RefB}
% Author, Book title, page numbers. Publisher, place (year)
% % etc
% \end{thebibliography}

\end{document}
% end of file template.tex
