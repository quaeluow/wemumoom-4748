\documentclass[11pt]{article}

\setlength{\marginparwidth}{1.2in}
\let\oldmarginpar\marginpar
\renewcommand\marginpar[1]{\-\oldmarginpar[\raggedleft\footnotesize #1]%
{\raggedright\footnotesize #1}}

\setlength{\parindent}{0in}
\setlength{\parskip}{.3in}

\raggedbottom

%\pagestyle{empty}

% 	PACKAGES
\usepackage[colon,round]{natbib}
\usepackage{amsfonts}
\usepackage{amssymb}
%\usepackage{german}
%\usepackage{hebtex}
%\usepackage{graphicx}
%\usepackage[german]{babel}
%\usepackage{endnotes}
%\usepackage{rotating}

% \newcommand{\kapt}[1]{\bigskip\noindent\textbf{{\thechap}. #1}\addtocounter{chap}{1}\medskip}
\newcommand{\kapt}[1]{\textbf{{\thechap}. #1}\addtocounter{chap}{1}}

% \input{./zz-ReferencesAtBeg-InformationEpistemology.tex}

\newcommand{\tbd}[1]{}
\newcommand{\qnull}[1]{`#1'}
\newcommand{\qeins}[1]{``#1''}
\newcommand{\qzwei}[1]{`#1'}

\newenvironment{quotex}{\begin{quote}\begin{footnotesize}}{\end{footnotesize}\end{quote}}

\begin{document}

The idea was originally to show that it would be more intuitive to
rest one's epistemology, especially experimental design in science, on
information theory rather than probability theory. It turns out that
information theory and probability theory may just be two sides of the
same coin, however. Information (if in bits, then use base 2 for the
logarithm) and probability can be related to each other by a simple
formula \citep[37]{guiasu77}:
\begin{equation}
  P(X)={2^{-I(X)}}
\label{eq:probdef}
\end{equation}

It may then be vacuous to decide which of the two is the primary
epistemological notion, although it is worth pointing out that
that the use of probabilities among epistemologists is ubiquitous,
whereas few epistemologists seem to concern themselves with
information theory.

I thought it would be interesting to show how far we can get by basing
our probabilistic notions on the concepts of information theory rather
than the reverse. I found out that there are precedents in the
literature: \citep{ingardenurbanik62} and \citep{kampe67} developed
their own ways of doing this. 

My strategy looks basically like this:
\begin{enumerate}
\item Show that ({\ref{eq:probdef}}) is a consequence of the intuition
  in counterfactuals that probability is the number of possible worlds
  where $X$ is true divided by the number of possible worlds in toto.
  Information, on the other hand, is related to the reciprocal of this
  quantity because information `constrains' the real world by
  shrinking the number of possible worlds (the car is now blue rather
  than any colour).
\item Both the concept of evidence and the concept of knowledge could
  be interpreted to mean that they no longer yield up information: $X$
  is evidence, $X$ is knowledge iff $I(X)=0$. This is a rather bold
  claim, but there is also precedent for it in \citep{williamson00}.
\item Then defend the thesis that when choosing the hypothesis which
  is optimally related to the evidence (or `epistemically justified')
  one should either take into account or definitely choose the one
  which is minimally informative based on the evidence.
\end{enumerate}

If the probabilities are numerically defined, Bayesian
conditionalization will give you the hypothesis that is epistemically
justified. If there are $n$ hypotheses exhausting the event
space then their posterior probability is:
\begin{equation}
  \label{eq:bayes}
Q(H_{i})=P(H_{i}|E)=\frac{P(E|H_{i})P(H_{i})}{\sum_{k=1}^{n}P(E|H_{k})}  
\end{equation}

Now, here is my question: can I show that the posterior probability
given by Bayes' formula is also the one that is informatively minimal?
If I don't start out with probabilities I need to start out with a
mathematical notion of information. Shannon's Entropy comes to mind:
\begin{equation}
    H(X)=-\sum_{i=1}^{n}p(x_{i})\log p(x_{i})
\label{eq:shannon}
  \end{equation}

Shannon's Entropy, however, is ill-defined for continuous random
variables. There appear to be several ways of doing this. I found the
Kullback-Leibler Divergence to be the most promising candidate. It
measures the additional information needed if a probability
distribution $Q$ differs from the `true' probability distribution $P$:
\begin{equation}
  \label{eq:kulei}
  D_{\mbox{\tiny{KL}}}(P\|Q)=\int_{\Omega}p\log\frac{p}{q}d\mu
\end{equation}
($\mu$ is any measure on the event space $\Omega$ for which
$p$ and $q$ are the Radon-Nikodym derivatives $\frac{dP}{d\mu}$ and
$\frac{dQ}{d\mu}$ respectively.) I am now entering the realm of
measure theory where my footing is very loose and where I need your
help. It would be nice to show conclusively that (a) $Q$ (the
posterior probability distribution provided by Bayes' formula) is
informatively minimal, given the evidence; and (b) that any other
probability distribution $Q'$ which is also informatively minimal,
given the evidence, is equal to $Q$ wherever it matters (for all
events except those whose measure is $0$, for example).

Here is my attempt at (a), although I can only imagine that it's full
of holes:

For any $F\subset E$,
\begin{displaymath}
  \frac{P(F)}{Q(F)}=\frac{P(F)}{\frac{P(F|E)P(F)}{P(E)}}=\frac{P(E)}{P(F|E)}=P(E)
\end{displaymath}
Now take any countable set $\mathcal{F}=\{F_{i}\}$ of disjoint subsets
of $E$. Then
\begin{displaymath}
  \int_{\cup F_{i}}p\log\frac{p}{q}d\mu=\sum_{i}P(F_{i}|\cup F_{i})\log\frac{P(F_{i}|\cup F_{i})}{Q(F_{i}|\cup F_{i})}=
\end{displaymath}
\begin{displaymath}
  \sum_{i}P(F_{i}|\cup F_{i})\log P(E|\cup F_{i})=0
\end{displaymath}
No probability distribution $Q'$ can beat this, although I haven't
been able to show (b), that $Q'$ must basically be the same as $Q$.
The probability distribution $Q$ given by Bayes' formula delivers a
probability distribution which is informatively minimal. More
strongly, Bayesian conditionalization adds no information to the prior
probability distribution and the evidence.

My question is: does this proof hold water, where does it need
improvement, and how do I show that basically only $Q$, the Bayesian
posterior probability distribution, is informatively minimal and not
others?

\nocite{*}
\bibliographystyle{plainnat}
\bibliography{email-MartinBarlow-2010-08-23}


\end{document}
