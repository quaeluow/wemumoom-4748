\documentclass[11pt]{article}

\setlength{\marginparwidth}{1.2in}
\let\oldmarginpar\marginpar
\renewcommand\marginpar[1]{\-\oldmarginpar[\raggedleft\footnotesize #1]%
{\raggedright\footnotesize #1}}

\frenchspacing % no extra space at the end of a sentence

\setlength{\parindent}{0in}
\setlength{\parskip}{.1in}

\raggedbottom

%\pagestyle{empty}

% 	PACKAGES
% \usepackage[small,bf]{caption}
% \let\bcode\textbgreek
% \usepackage[bgreek,english]{babel}
% \usepackage{setspace}
\usepackage{amsfonts}
\usepackage{amssymb}
\usepackage{amsmath}
% \usepackage{german}
% \usepackage{hebtex}
% \usepackage{graphicx}
% \usepackage[german]{babel}
% \usepackage{endnotes}
% \let\footnote=\endnote
% \usepackage{rotating}
% \usepackage{enumerate}

\newcommand{\kapt}[1]{\textbf{{\thechap}. #1}\addtocounter{chap}{1}}
\newcommand{\nootag}{}

\newcommand{\tbd}[1]{}
\newcommand{\qnull}[1]{`#1'}
\newcommand{\qeins}[1]{``#1''}
\newcommand{\qzwei}[1]{`#1'}

\newif\ifNumericalOrYear
\NumericalOrYeartrue
% \NumericalOrYearfalse
\ifNumericalOrYear
\usepackage[numbers,colon]{natbib}
\else
\usepackage[round,colon]{natbib}
\fi
\newif\ifPageP
\PagePtrue
% \PagePfalse
\ifPageP
\newcommand{\PageP}{p.~}
\else
\newcommand{\PageP}{}
\fi

\newcommand{\scite}[3]{\ifnum#1=1\ifNumericalOrYear\citep{#2}\else\citeyearpar{#2}\fi\else
\ifnum#1=2\ifNumericalOrYear\citep[#3]{#2}\else\citep[{\PageP}#3]{#2}\fi\else
\ifnum#1=3\ifNumericalOrYear(\citet[#3]{#2})\else\citep[{\PageP}#3]{#2}\fi\else
\ifnum#1=4\ifNumericalOrYear\citet{#2}\else\citet{#2}\fi\else
\ifnum#1=5\ifNumericalOrYear(\citet{#2})\else\citep{#2}\fi\else
\ifnum#1=6\ifNumericalOrYear(\citet[#3]{#2})\else\citep[{\PageP}#3]{#2}\fi\else
\ifnum#1=7\ifNumericalOrYear\citep{#2}\else\citealp{#2}\fi\else
\ifnum#1=8\ifNumericalOrYear\citep[#3]{#2}\else\citealp[{\PageP}#3]{#2}\fi\else
\textbf{[invalid scite code]}\fi\fi\fi\fi\fi\fi\fi\fi}

\newenvironment{quotex}{\begin{quote}\begin{footnotesize}}{\end{footnotesize}\end{quote}}
% \newenvironment{quotex}{\begin{quote}\begin{footnotesize}\begin{singlespace}}{\end{singlespace}\end{footnotesize}\end{quote}}

\begin{document}

\title{A Note on the Judy Benjamin Problem}

\author{Stefan Lukits}

\maketitle

\newcounter{chap}

\setcounter{chap}{1}

Grove and Halpern \scite{1}{grovehalpern97}{} make a claim that, given
the Judy Benjamin problem introduced in \scite{4}{fraassen81}{}, it is
not necessary and produces unintuitive consequences to employ the
principle of maximum entropy (from now on \textsc{maxent}) in order to
achieve a reasonable posterior probability distribution. The authors
maintain that despite the appearance of partial information (which may
be interpreted to call for the use of maximum entropy rather than
Bayesian conditionalization), they can handle the problem using Bayes'
formula and produce results that are both intuitive and contingent on
information not provided in the problem, as they should be because
Judy Benjamin's information is so vague.

If we use the constraint rule and \textsc{maxent}, Judy Benjamin's
posterior probability of being in blue territory is greater than its
prior probability. The normalized odds vectors of the prior and
posterior probability distribution are, respectively:
\begin{align}
&v_{0}=(.25,.25,.5)\nootag \\
&v_{1}=(.12,.35,.53)\nootag
\end{align}

Grove and Halpern consider this development unintuitive, because they
think that Judy Benjamin's information given to her by her
headquarters is independent of the probability of being in blue
territory. I want to show that Grove and Halpern's assumption of
independence is unwarranted, because it imposes information on Judy
Benjamin's situation that she does not have. I will provide scenarios
where independence is given and other scenarios where it is not. The
scenarios are not far-fetched, and so we conclude that independence
must not be one of our assumptions. 

Grove and Halpern claim that it would be natural for Judy Benjamin to
assume independence and uniformity in the probability assignments of
her headquarters (they explicitly state, however, that these
assumptions go far beyond the information provided in the problem). In
contrast to Grove and Halpern, we will look at a scenario where Judy
Benjamin makes no assumptions about independence and much weaker
assumptions than uniformity by considering measurable partitions of
the event space whose grain goes to infinity. 

As her partitions become more fine-grained, her posterior probability
approaches the posterior probability suggested by \textsc{maxent}.
Thus, we say, it is true that there is a lot of information Judy
Benjamin does not have. She still needs to make a reasonable posterior
probability assessment, and we maintain that it is \textsc{maxent}
which provides it, partly because it is so good at incorporating
ignorance. The results will be close to the results suggested by
independence, but because independence is not one of our assumptions
we have to pay attention to the scenarios in which it is not true and
accordingly increase the probability of the event we know less about.
The fine-grained partition scenario confirms that this procedure is in
full accordance with intuitions.

Here are three scenarios, in which Judy may have received her
information:

\begin{enumerate}
\item[\textbf{S1}] Judy was dropped off by a pilot who flipped two coins. If
  the first coin landed H, then Judy was dropped off in Blue
  territory, otherwise in Red territory. If the second coin landed H,
  she was dropped off on Headquarters ground, otherwise on Second
  Company ground. Judy's headquarters find out that the second coin
  was biased $q:1-q$ toward H with $q=.75$. The normalized odds vector is
  $v=(.125,.375,.5)$ and agrees with (T1), because the choice of Blue
  or Red is completely independent from the choice of Headquarters or
  Second Company.
\item[\textbf{S2}] The pilot randomly lands in any of the four quadrants and
  rolls a die. If she rolls an even number, she drops off Judy. If
  not, she takes her to another (or the same) randomly selected
  quadrant to repeat the procedure. Headquarters find out, however,
  that for $A_{1}$, the pilot requires a six to drop off Judy, not
  just an even number. Thus they relay (HQ) to Judy, which she
  correctly interprets with the normalized odds vector $v=(.1,.3,.6)$.
\item[\textbf{S3}] Judy's Headquarters has divided the map into $24$ congruent
  rectangles, $A_{3}$ into twelve, and $A_{1}$ and $A_{2}$ into six
  rectangles each. They have information that the only subsets of the
  $24$ rectangles in which Judy Benjamin may be located are such that
  they contain three times as many $A_{2}$ rectangles than $A_{1}$
  rectangles. Thus they relay (HQ) to Judy, which she correctly
  interprets with the normalized odds vector $v=(.108,.324,.568)$
  (evaluating the $16777216$ subsets).
\end{enumerate}

Scenarios S1 and S2 are good examples of where independence is true
and where it is not, respectively. In the following section, we will
focus on S3 and consider what happens when the grain of the partition
becomes finer. To do this, let
\begin{equation}
  A_{1}\cup{}A_{2}\cup{}A_{3}=A=\bigcup_{i\in{}I}B_{i}\mbox{ and }  \mathcal{A}=2^{\{B_{i}\}_{i\in{}I}}\nootag
\end{equation}
 where $I$ is a finite set of indices and the $B_{i}$ are a pairwise
 disjoint covering of $A$ with $B_{i}\cap{}B_{j}=\emptyset$ for
 $i\neq{}j$ as well as $\mu(B_{i})=\mu(B_{j})$ for all $i,j\in{}I$ and
 an appropriate measure $\mu$.

 Now let $\mathcal{B}\subset{}\mathcal{A}$ be the set that contains
 only those collections of $B_{i}$ for which there are $t=q/(1-q)$ (in
 Judy Benjamin's case, $t=3$) times as many $B_{i}$ in $A_{2}$ as
 there are in $A_{1}$. In other words, $B\in\mathcal{B}$ iff
\begin{equation}
  \mbox{ and }q\mu\left(\bigcup_{i\in{}J\subset{}I}B_{i}\cap{}A_{1}\right)=(1-q)\mu\left(\bigcup_{i\in{}J\subset{}I}B_{i}\cap{}A_{2}\right)\nootag
\end{equation}

For simplicity's sake, we assume that $\#J=4n=4ts$ (where $t$ depends
on $q$ as above and $s$ indicates the grain of the partition). (Show a
graph with the example $t=3$ and $s=2$.) Brute combinatorics tells us
that $H_{3}(s,t)$, the average ratio of $\mu(B\cap{}A_{3})/\mu(B)$ for
all $B\in\mathcal{B}$ is
\begin{equation}
  H_{3}(s,t)=\frac{1}{N}\sum_{j=0}^{2ts}\sum_{i=0}^{s}\binom{2ts}{j}\binom{ts}{i}\binom{ts}{ti}\varphi_{ij}\nootag
\end{equation}
where
\begin{equation}
  N=2^{2ts}\sum_{i=0}^{s}\binom{ts}{i}\binom{ts}{ti}\nootag
\end{equation}
and
\begin{equation}
  \varphi_{ij}=j(j+i(1+t))^{-1}\nootag
\end{equation}

This would be the proper posterior $Q(A_{3})$ for scenario S3 with
$t=3$ and $s=2$. We are interested in what happens to $H_{3}(s,t)$ as
$s\rightarrow\infty$ (the partition becomes more fine-grained; (T1)
suggests that $H_{3}(s,t)$ will remain at approximately $1/2$, while
(T2) suggests that $H_{3}(s,t)$ is consistently larger than $1/2$ and
more so with greater $t$) and as $t\rightarrow\infty$ (in this case,
(T2) suggests that $H_{3}(s,t)$ should approach $2/3$, while (T1)
suggest that it should approach $1/2$).

% (fset 'stefan-insert-sum "\\sum_{}^{}\C-u5\C-h\C-l")

Here is a table with the results for $t=1,\ldots{},10$ and $s=1,\ldots{},25$:

\begin{tabular}{|c|c|c|c|c|c|c|c|c|c|c|}
\hline 
 &1 &2 &3 &4 &5 &6 &7 &8 &9 &10 \\
 \hline 
1 &0.2917 &0.3732 &0.4094 &0.4297 &0.4426 &0.4515 &0.4580 &0.4629 &0.4669 &0.4700 \\
 \hline 
2 &0.4327 &0.5187 &0.5682 &0.5976 &0.6143 &0.6241 &0.6305 &0.6351 &0.6386 &0.6413 \\
 \hline 
3 &0.4812 &0.5271 &0.5512 &0.5635 &0.5705 &0.5751 &0.5784 &0.5809 &0.5829 &0.5845 \\
 \hline 
4 &0.4958 &0.5274 &0.5539 &0.5734 &0.5893 &0.6028 &0.6141 &0.6233 &0.6306 &0.6363 \\
 \hline 
5 &0.4997 &0.5269 &0.5531 &0.5711 &0.5834 &0.5917 &0.5973 &0.6012 &0.6040 &0.6062 \\
 \hline 
6 &0.5006 &0.5265 &0.5526 &0.5705 &0.5832 &0.5930 &0.6013 &0.6085 &0.6151 &0.6212 \\
 \hline 
7 &0.5006 &0.5263 &0.5523 &0.5705 &0.5837 &0.5937 &0.6013 &0.6070 &0.6114 &0.6147 \\
 \hline 
8 &0.5005 &0.5261 &0.5521 &0.5702 &0.5831 &0.5925 &0.5998 &0.6056 &0.6106 &0.6150 \\
 \hline 
9 &0.5004 &0.5259 &0.5520 &0.5701 &0.5831 &0.5930 &0.6008 &0.6071 &0.6123 &0.6166 \\
 \hline 
10 &0.5004 &0.5258 &0.5519 &0.5700 &0.5830 &0.5927 &0.6001 &0.6059 &0.6105 &0.6144 \\
 \hline 
11 &0.5003 &0.5257 &0.5518 &0.5699 &0.5829 &0.5926 &0.6002 &0.6064 &0.6116 &0.6161 \\
 \hline 
12 &0.5002 &0.5256 &0.5517 &0.5698 &0.5828 &0.5926 &0.6001 &0.6061 &0.6110 &0.6149 \\
 \hline 
13 &0.5002 &0.5256 &0.5516 &0.5698 &0.5828 &0.5925 &0.6001 &0.6061 &0.6112 &0.6155 \\
 \hline 
14 &0.5002 &0.5255 &0.5516 &0.5697 &0.5827 &0.5925 &0.6000 &0.6061 &0.6111 &0.6152 \\
 \hline 
15 &0.5002 &0.5255 &0.5515 &0.5697 &0.5827 &0.5924 &0.6000 &0.6060 &0.6110 &0.6152 \\
 \hline 
16 &0.5001 &0.5254 &0.5515 &0.5696 &0.5826 &0.5924 &0.6000 &0.6060 &0.6110 &0.6152 \\
 \hline 
17 &0.5001 &0.5254 &0.5514 &0.5696 &0.5826 &0.5923 &0.5999 &0.6060 &0.6110 &0.6151 \\
 \hline 
18 &0.5001 &0.5254 &0.5514 &0.5695 &0.5826 &0.5923 &0.5999 &0.6060 &0.6110 &0.6152 \\
 \hline 
19 &0.5001 &0.5254 &0.5514 &0.5695 &0.5825 &0.5923 &0.5999 &0.6059 &0.6109 &0.6151 \\
 \hline 
20 &0.5001 &0.5253 &0.5513 &0.5695 &0.5825 &0.5923 &0.5998 &0.6059 &0.6109 &0.6151 \\
 \hline 
21 &0.5001 &0.5253 &0.5513 &0.5695 &0.5825 &0.5922 &0.5998 &0.6059 &0.6109 &0.6151 \\
 \hline 
22 &0.5001 &0.5253 &0.5513 &0.5694 &0.5825 &0.5922 &0.5998 &0.6059 &0.6109 &0.6151 \\
 \hline 
23 &0.5001 &0.5253 &0.5513 &0.5694 &0.5824 &0.5922 &0.5998 &0.6059 &0.6109 &0.6150 \\
 \hline 
24 &0.5001 &0.5253 &0.5512 &0.5694 &0.5824 &0.5922 &0.5998 &0.6059 &0.6109 &0.6150 \\
 \hline 
25 &0.5001 &0.5252 &0.5512 &0.5694 &0.5824 &0.5922 &0.5998 &0.6059 &0.6108 &0.6150 \\
 \hline 
\end{tabular}

It appears that (T2) is in this case the better intuition to follow.
The appearance is confirmed by the following consideration. Let $X$ be
the random variable
\begin{equation}
  X=\frac{\mu(B\cap{}A_{3})}{\mu(B)}\nootag
\end{equation}
for a randomly chosen $B\in\mathcal{B}$. Let $X_{i}=\mu(B\cap{}A_{i})$
also be random variables for $i=1,2,3$. Then the expectation of $X$ is
\begin{equation}
  \label{eq:ex}
  EX=\frac{EX_{3}}{\sum_{i=1}^{3}EX_{i}}
\end{equation}

We know that according to the Moivre-Laplace Theorem and continuity
correction (basically approximating the binomial distribution, which
is difficult to calculate for large integers, by the normal
distribution)
\begin{equation}
  \sum_{k=0}^{m}\binom{n}{k}\approx{}2^{n}\int_{-\infty}^{m+\frac{1}{2}}N\left(\frac{n}{2},\frac{n}{4}\right)(x)dx\nootag
\end{equation}
and (I don't have proof or reference for the following approximation,
but have confirmed it numerically in test-intxN.pl)
\begin{equation}
  \sum_{k=0}^{m}k\binom{n}{k}\approx{}2^{n}\int_{-\infty}^{m+\frac{1}{2}}xN\left(\frac{n}{2},\frac{n}{4}\right)(x)dx\nootag
\end{equation}
with
\begin{equation}
  N(\mu,\sigma^{2})=\frac{1}{\sqrt{2\pi\sigma^{2}}}e^{-\frac{(x-\mu)^{2}}{2\sigma^{2}}}
\end{equation}

As above, combinatorics show that
\begin{align}
  &EX_{1}=\frac{\sum_{i=0}^{s}i\binom{ts}{i}}{\sum_{i=0}^{s}\binom{ts}{i}}\nootag \\
  &EX_{2}=\frac{\sum_{i=0}^{ts}i\binom{ts}{i}}{\sum_{i=0}^{ts}\binom{ts}{i}}\nootag \\
  &EX_{3}=\frac{\sum_{i=0}^{2ts}i\binom{2ts}{i}}{\sum_{i=0}^{2ts}\binom{2ts}{i}}\nootag
\end{align}

Consequently,
\begin{equation}
\label{eq:ex1}
  EX_{1}\approx\frac{\int_{-\infty}^{s+\frac{1}{2}}xN\left(\frac{n}{2},\frac{n}{4}\right)(x)dx}{\int_{-\infty}^{s+\frac{1}{2}}N\left(\frac{n}{2},\frac{n}{4}\right)(x)dx}\nootag
\end{equation}

Using the well-known integrals (erf is the Gauss error function)
\begin{equation}
\int_{a}^{b}\omega{}e^{-\omega{}^{2}}d\omega=\frac{1}{2}e^{-a^{2}}-\frac{1}{2}e^{-b^{2}}
\end{equation}
and
\begin{equation}
\int_{a}^{b}e^{-\omega^{2}}d\omega=\frac{\sqrt{\pi}}{2}\left(\mbox{erf}(b)-\mbox{erf}(a)\right)
\end{equation}
as well as the substitution
\begin{equation}
  y=\sqrt{\frac{2}{n}}\left(x-\frac{n}{2}\right)
\end{equation}
({\ref{eq:ex1}}) simplifies to
\begin{equation}
  % EX_{1}\approx\frac{n\left(\frac{1}{4}+\frac{1}{4}\mbox{erf}(w_{1})\right)-\sqrt{\frac{n}{8\pi}}e^{-w_{1}^{2}}}{\frac{1}{2}\left(\mbox{erf}(w_{1})+1\right)}\nootag
EX_{1}=\frac{n}{2}\left(1-\sqrt{\frac{2}{\pi{}n}}\cdot\frac{e^{-w_{1}^{2}}}{1+\mbox{erf}(w_{1})}\right)
\end{equation}
with (remember that $n=ts$)
\begin{equation}
  w_{1}=\sqrt{\frac{2}{n}}\left(s+\frac{1}{2}\left(1-n\right)\right)\nootag
\end{equation}

Analogously, $EX_{2}$ and $EX_{3}$ approximately simplify to
(respectively)
\begin{equation}
EX_{2}=\frac{n}{2}\left(1-\sqrt{\frac{2}{\pi{}n}}\cdot\frac{e^{-w_{2}^{2}}}{1+\mbox{erf}(w_{2})}\right)
\end{equation}
\begin{equation}
% EX_{3}\approx\frac{n\left(\frac{1}{2}+\frac{1}{2}\mbox{erf}(w_{3})\right)-\sqrt{\frac{n}{4\pi}}e^{-w_{3}^{2}}}{\frac{1}{2}\left(\mbox{erf}(w_{3})+1\right)}\nootag
EX_{3}=n\left(1-\sqrt{\frac{1}{\pi{}n}}\cdot\frac{e^{-w_{3}^{2}}}{1+\mbox{erf}(w_{3})}\right)
\end{equation}
with
\begin{equation}
  w_{2}=\frac{n+1}{n\sqrt{2}}\hspace{.5in}w_{3}=\frac{2n+1}{2\sqrt{n}}
\end{equation}

Evaluating these expressions using L'H{\^o}pital's rule it becomes
apparent that they show the following behaviour as
$s\rightarrow\infty$ and $t$ remains fixed:
\begin{align}
  &EX_{1}\rightarrow\frac{s}{2}\left(t-\sqrt{2}\cdot\frac{t^{2}-4}{t}\right) \\
  &EX_{2}\rightarrow{}\frac{n}{2} \\
  &EX_{3}\rightarrow{}n
\end{align}

$X_{2}$ and $X_{3}$ behave precisely as we would expect them to
behave: sloppily speaking, taking a random subset from $A_{2}$ or
$A_{3}$, we would expect it to be half the size of $A_{2}$ or $A_{3}$
respectively. (What if $t$ is $1$: wouldn't you expect $EX_{1}=n/2$ as
well?) The constraint that the randomly chosen subset $B$ fulfills
(again, expressed sloppily) $B\cap{}A_{2}=t\cdot{}B\cap{}A_{1}$,
however, introduces an intriguing new factor for $EX_{1}$ that is
dependent on the two factors of $n$, the grain $s$ and the constraint
$t$.

We now have several different ways of calculating $q_{t}$, the
posterior probability of the event that Judy Benjamin is in $A_{1}$.
If we want to use a particular partition in the spirit of S3, we need
to consult the table provided above. Using either \textsc{minxent} or
the constraint rule and differentiating the Kullback-Leibler
Divergence to achieve the result according to \textsc{maxent}, $q_{t}$
will be value solely dependent on $t$:
\begin{equation}
  \label{eq:fromq1toq}
  G(q)=q_{1}=\frac{C}{1+Ct+C}
\end{equation}
where
\begin{equation*}
  t=\frac{q}{1-q}
\end{equation*}
and
\begin{equation*}
  C=2^{-\frac{t\log{}t+t+1}{1+t}}
\end{equation*}

Grove and Halpern suggest that simply
\begin{equation}
  q_{t}=\frac{1}{2}\left(1-\frac{1}{1+t}\right)
\end{equation}

And now our fine grain partition method suggests that $q_{t}$ should
approximately follow equation ({\ref{eq:ex}}) using the explicit
expressions given for $EX_{i}$ above.

% \noindent\textbf{Endnotes}

% \theendnotes

% \medskip

% \noindent\textbf{References}

% \nocite{*} 
\bibliographystyle{stefan-2010-08-28}
\bibliography{bib-3306}

\end{document}

