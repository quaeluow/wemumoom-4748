When we come to know a conditional, we cannot straightforwardly apply
Jeffrey conditioning to gain an updated probability distribution. Carl
Wagner has proposed a natural generalization of Jeffrey conditioning
to accommodate this case (Wagner conditioning). The generalization
rests on an ad hoc but plausible intuition (W). Wagner shows how the
principle of maximum (M) entropy disagrees with intuition (W). This
article presents a natural generalization of Wagner conditioning which
is derived from (M) and implied by it. (M) does therefore not only not
disagree with (W), it seamlessly and elegantly generalizes it (just as
it generalizes standard conditioning and Jeffrey conditioning).
Wagner's inconsistency result for (M) and (W) rests on Wagner's
rejection of (L), the Laplacean Principle. The article explains (L)
and why all advocates of (M) accept it. Therefore, Wagner only shows
that one cannot hold (W) and (M) while rejecting (L). Since all
advocates of (M) accept (L), (W) and (M) are consistent and,
furthermore, (M) provides a much less ad hoc and much more integrated
generalization of Jeffrey conditioning than Wagner conditioning.

When we come to know a conditional, we cannot straightforwardly apply
Jeffrey conditioning to gain an updated probability distribution. Carl
Wagner has proposed a natural generalization of Jeffrey conditioning
to accommodate this case (Wagner conditioning). The generalization
rests on an ad hoc but plausible intuition (W). Wagner shows how the
principle of maximum entropy (M) disagrees with intuition (W) and
therefore considers (M) to be undermined. This article presents a
natural generalization of Wagner conditioning which is derived from
(M) and implied by it. (M) is therefore not only consistent with (W),
it seamlessly and elegantly generalizes it (just as it generalizes
standard conditioning and Jeffrey conditioning). Wagner's
inconsistency result for (W) and (M) is instructive. It rests on the
assumption (I) that the credences of a rational agent may be
indeterminate. While many Bayesians now hold (I) it is difficult to
articulate (M) on its basis because to date there is no proposal how
to measure indeterminate probability distributions in terms of
information theory. Most, if not all, advocates of (M) resist (I). If
they did not they would be vulnerable to Wagner's inconsistency
result. Wagner has therefore not, as he believes, undermined (M) but
only demonstrated that advocates of (M) must accept that rational
agents ought to have sharp credences.
