It is well-known that with known marginal probabilities a_{i} and
b_{j} the joint probability distribution maximizing the entropy

[tex]H(P)=-\sum_{i=1}^{m}\sum_{j=1}^{n}p_{ij}\log{}p_{ij}[/tex]

is [tex]p_{ij}=a_{i}b_{j}[/tex] 

For m=3 and n=3, a=(0.2,0.3,0.5), b=(0.1,0.6,0.3), for example,

[tex]\begin{equation}
  \label{eq:r1}
P=\left(
  \begin{array}{rrr}
    0.02 & 0.12 & 0.06 \\
    0.03 & 0.18 & 0.09 \\
    0.05 & 0.30 & 0.15
  \end{array}
\right)
\end{equation}[/tex]

jNow, I have a problem where the joint probability distribution is
constrained (much like a random walk where the path from one node to
another is blocked). For example, let m, n, a, and b be as above with
the constraint that

[tex]\begin{equation}
  \label{eq:r2}
P=\left(
  \begin{array}{rrr}
    p_{11} & 0 & p_{13} \\
    p_{21} & p_{22} & p_{23} \\
    p_{31} & p_{32} & p_{33} \\
  \end{array}
\right)
\end{equation}[/tex]

Because a and b are known, it suffices to find out any 2x2 matrix
contained in the 3x3 matrix, for example (x,y,z the variables by which
w_{1}, w_{2}, v_{1}, v_{2}, and sigma are expressible, given a and b)

[tex]\begin{equation}
  \label{eq:r3}
P=\left(
  \begin{array}{rrr}
    x & 0 & w_{1} \\
    y & z & w_{2} \\
    v_{1} & v_{2} & \sigma \\
  \end{array}
\right)
\end{equation}[/tex]

I use this to write out the entropy and differentiate with respect to
x, y, and z to find out that the maximum will be where

[tex]\begin{equation}
  \label{eq:r4}
  \det\left(
    \begin{array}{ll}
      x & w_{1} \\
      v_{1} & \sigma
    \end{array}
\right)=0
\end{equation}[/tex]

[tex]\begin{equation}
  \label{eq:r5}
  \det\left(
    \begin{array}{ll}
      y & w_{2} \\
      v_{1} & \sigma
    \end{array}
\right)=0
\end{equation}[/tex]

[tex]\begin{equation}
  \label{eq:r6}
  \det\left(
    \begin{array}{ll}
      z & w_{2} \\
      v_{2} & \sigma
    \end{array}
\right)=0
\end{equation}[/tex]

This is a system of 3 non-linear equations which are awkward to solve
algebraically. In the end, I am interested to know which property a
and b need to have to ensure that P is a proper probability
distribution (i.e. no negative elements). For now, however, I'd be
thrilled if anybody could give me a hint how I could find the
solutions for x, y, and z algebraically without these non-linear
equations.

The numeric solution for this is

[tex]\begin{equation}
  \label{eq:r1}
P=\left(
  \begin{array}{rrr}
    0.093 & 0.000 & 0.107 \\
    0.001 & 0.113 & 0.186 \\
    0.006 & 0.487 & 0.007
  \end{array}
\right)
\end{equation}[/tex]

But I have definitely seen systems where all solutions were complex
and some of the probabilities ended up <0, especially when m and n are
larger than 3 and there are more than one zero constraint in the joint
probability matrix.