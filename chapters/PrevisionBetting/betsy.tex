% What exactly is Linda's evidence differential compared to
% Betsy -- the analytic expression may help.

\documentclass[11pt]{article}
\usepackage{october}
% For BJPS
% \hyphenpenalty=10000
% \hbadness=10000

\begin{document}
% For BJPS
% \raggedright
% \doublespacing

\title{Sharp Credences and Prevision Betting}
\author{Stefan Lukits}
\date{}
\maketitle

\begin{abstract} 
  {\noindent}TBA.
\end{abstract}

\section{Introduction}
\label{Introduction}

I am adding a reason (D) that is poorly documented in the literature:
The Boolean rational agent may systematically do better accepting bets
than the agent who on principle rejects instates. Walley conducted an
experiment in which Boolean participants did significantly better than
Laplacean participants, betting on soccer games played in the Soccer
World Cup 1982 in Spain (see \scite{7}{walley91}{}, Appendix I). I
replicated the experiment using two computer players with rudimentary
artificial intelligence and made them specify betting parameters
(previsions) for games played in the Soccer World Cup 2014 in Brazil.
I used the Poisson distribution (which is an excellent predictor for
the outcome of soccer matches) and the FIFA ranking to simulate
millions of counterfactual World Cup results and their associated
bets, using Walley's evaluation method. The Boolean player had a
slight but systematic advantage. In section TBA, I will provide an
explanation and show how it undermines any support the experiment
might give to the Boolean position.

\section{Evidence Differentials and Cushioning Credences}
\label{WalleysWorldCupWoes}

I want to proceed to the intriguing issue of who does better in
betting situations: instates or sharp credences. I have given away the
answer already in the introduction: instates do better. It is
surprising that, except for a rudimentary allusion to this in Walley's
book, no Boolean has caught on to this yet. After I found out that
agents with instates do better betting on soccer games, I let Betsy
and Linda play a more basic betting game. An $n$-sided die is rolled
(by the computer). The die is fair, unbeknownst to the players. Their
bets are randomly and uniformly drawn from the simplex for which the
probabilities attributed to the $n$ results add up to 1. Betsy also
surrounds her credences with an imprecision uniformly drawn from the
interval $(0,y)$. I used Walley's pay off scheme (see
\scite{8}{walley91}{632}) to settle the bets.

Here is an example: let $n=2$, so the die is a fair \textit{coin}.
Betsy's and Linda's bets are randomly and uniformly drawn from the
line segment from $(0,1)$ to $(1,0)$ (these are two-dimensional
Cartesian coordinates), the two-dimensional simplex (for higher $n$,
the simplex is a pentatope generalized for $n$ dimensions with side
length $2^{1/2}$). The previsions (limits at which bets are accepted)
may be $(0.21,0.79)$ for Linda and $(0.35\pm{}0.11,0.65\pm{}0.11)$ for
Betsy, where the indeterminacy $\pm{}0.11$ is also randomly and
uniformly drawn from the imprecision interval $(0,y)\subseteq(0,1)$.
The first bet is on $H$, and Linda is willing to pay $22.5$ cents for
it, while Betsy is willing to pay $77.5$ cents against it. The second
bet is on $T$ (if $n>2$, there will not be the same symmetry as in the
\textit{coin} case between the two bets), for which Betsy is willing
to pay $77.5$ cents, and against which Linda is willing to pay $22.5$
cents. Each bet pays \$1 if successful. Often, Linda's credal state
will overlap with Betsy's sharp credence so that there will not be a
bet.\fcut{7}

The computer simulation clearly shows that Linda does better than
Betsy in the long run. A defence of sharp credences for rational
agents needs to have an explanation for this. We will call it partial
belief cushioning, which is based on an evidence differential between
the bettors.

In many decision-making contexts, we do not have the luxury of calling
off the bet. We have to decide one way or another. This is a problem
for instates, as Booleans have to find a way to decide without
receiving instructions from the credal state. Booleans have addressed
this point extensively (see for example \scite{8}{joyce10}{311ff}; for
an opponent's view of this see \scite{8}{elga10}{6ff}). The problem
for sharp credences arises when bets are noncompulsory, for then the
data above suggest that agents holding instates systematically do
better. Often, decision making happens as betting vis-{\`a}-vis
uninformed nature or opponents which are at least as uninformed as the
rational agent. Sometimes, however, bets are offered by better
informed or potentially better informed bookies. In this case, even an
agent with sharp credences must cushion her credences and is better
off by rejecting bets that look attractive in terms of her partial
beliefs.

If an agent does not cushion her partial beliefs (whether they are
sharp or indeterminate), she will incur a loss in the long run. Since
cushioning is permitted in Walley's experimental setup (the bets are
noncompulsory), Laplacean agents should also have access to it and
then no longer do worse than Boolean agents. One may ask what sharp
credences do if they just end up being cushioned anyway and do not
provide sufficient information to decide on rational bets. The answer
is that sharp credences are sufficient where betting (or decision
making more generally) is compulsory; the cushioning only supplies the
information from the evidence inasmuch as betting is noncompulsory and
so again properly distinguishes semantic categories. This task is much
harder for Booleans, although I do not claim that it is
insurmountable: instates can provide a coherent approach to compulsory
betting. What they cannot do, once cushioning is introduced, is
outperform sharp credences in noncompulsory betting situations.

Here are a few examples: even if I have little evidence on which to
base my opinion, someone may force me to either buy Coca Cola shares
or short them, and so I have to have a share price $p$ in mind that I
consider fair. I will buy Coca Cola shares for less than $p$, and
short them for more than $p$, if forced to do one or the other. This
does not mean that it is now reasonable for me to go (not forced by
anyone) and buy Coca Cola shares for $p$. It may not even be
reasonable to go (not forced by anyone) and buy Coca Cola share for
$p-\delta$ with $\delta{}>0$.

It may in fact be quite unreasonable, since there are many players who
have much better evidence than I do and will exploit my ignorance. I
suspect that most lay investors in the stock market make this mistake:
even though they buy and sell stock at prices that seem reasonable to
them, professional investors are much better and faster at exploiting
arbitrage opportunities and more subtle regularities. If indices rise,
lay investors will make a little less than their professional
counterparts; and when they fall, lay investors lose a lot more. In
sum, unless there is sustained growth and everybody wins, lay
investors lose in the long term.

A case in point is the U.S. Commodity Futures Trading Commission's
crackdown on the online prediction market Intrade. Intrade offered
fair bets for or against events of public significance, such as
election results or other events which had clear yes-or-no outcomes.
Even though the bets were all fair and Intrade only received a small
commission on all bets, and even though Intrade's predictions were
remarkably accurate, the potential for professional arbitrageurs was
too great and the CFTC shut Intrade down (see
\texttt{https://www.intrade.com}).

Cushioning does not stand in the way of holding a sharp credence, even
if the evidence is dim. The evidence determines for a rational agent
the partial beliefs over possible states of the world operating in the
background. The better the evidence, the more pointed the
distributions of these partial beliefs will be and the more willing
the rational agent will be to enter a bet, if betting is
noncompulsory. The mathematical decision rule will be based on the
underlying distribution of the partial beliefs, not only on the sharp
credence. As we have stated before, a sharp credence is not a
sufficient statistic for decision making, inference, or betting
behaviour; and neither is an instate.

The rational agent with a sharp credence has resources at her disposal
to use just as much differentiation with respect to accepting and
rejecting bets as the agent with instates. Often (if she is able to
and especially if the bets are offered to her by a better-informed
agent), she will reject both of two complementary bets, even when they
are fair. On the one hand, any advantage that the agent with an
instate has over her can be counteracted based on her distribution
over partial beliefs that she has with respect to all possibilities.
On the other hand, the agent with instates suffers under both
conceptual and practical problems that put her at a real disadvantage
in terms of understanding the sources and consequences of her
knowledge and her uncertainties.

\section{Other Text}
\label{OtherText}

I replicated the experiment using two computer players, Betsy and
Linda, with rudimentary artificial intelligence and made them specify
betting parameters (previsions) for games played in the Soccer World
Cup 2014 in Brazil. I used the Poisson distribution (which is an
excellent predictor for the outcome of soccer matches) and the FIFA
ranking to simulate millions of counterfactual World Cup results and
their associated bets, using Walley's evaluation method. Betsy, who
used upper and lower previsions, had a slight but systematic advantage
over Linda, who used linear previsions. Betsy's expected gain for a
whole tournament was approximately \$2 (very little compared to the
stakes, as one can see by considering the standard deviation of
approximately \$46 for this expected gain, i.e. 67\% of Betsy's
overall gain from a tournament was between -\$44 and +\$48). The
surprising fact that she did consistently better than Linda remains in
need of explanation. In section \ref{WalleysWorldCupWoes}, I will
provide an explanation and show how it can support rejecting instates
for rational agents, counter-intuitive as that may sound.

^^^

Here is a table of the results. Each result is based on
100,000 die rolls. The second column shows the mean gain for Linda,
the third column shows the standard deviation, the fourth column shows
the percentage of bets which are called off. The table is for $y=1$.

\begin{tabular}{|l|r|r|r|}
  \hline
  $n=2$ & 0.14 & 18.8 & 25.2 \\ \hline
  $n=3$ & -1.67 & 13.7 & 39.1 \\ \hline
  $n=4$ & -1.73 & 10.4 & 48.1 \\ \hline
  $n=5$ & -1.39 & 8.2 & 54.5 \\ \hline
  $n=6$ & -1.22 & 6.6 & 59.1 \\ \hline
  $n=7$ & -1.01 & 5.5 & 62.8 \\ \hline
\end{tabular}

Here are the results if Betsy is not as generous with her
indeterminacy, $y=0.3$ (note that fewer bets are called off). The
second column still shows Linda's mean gain.

\begin{tabular}{|l|r|r|r|}
  \hline
  $n=2$ & 5.82 & 28.0 & 0.0 \\ \hline
  $n=3$ & -0.55 & 21.8 & 0.0 \\ \hline
  $n=4$ & -2.32 & 17.0 & 0.0 \\ \hline
  $n=5$ & -2.89 & 13.7 & 0.4 \\ \hline
  $n=6$ & -2.74 & 11.3 & 1.4 \\ \hline
  $n=7$ & -2.52 & 9.6 & 3.2 \\ \hline
\end{tabular}

We should get similar results if we do this analytically instead of
using computer simulation. I will pursue this further for the final
version of the paper. The math is not complicated, but unwieldy. The
following expression yields the expected gain for Linda:

\begin{eqnarray}
  \label{eq:s3}
  EX =
  \frac{p_{x}}{n}\left(\sum_{j=0}^{n-1}\int_{0}^{1}\int_{0}^{x}\int_{0}^{x-y}\sum_{k=0}^{n-1}g(x,y,s,k,j)\,ds\,d\upsilon(y)\,d\xi(x)\right)+
  \notag \\
  \frac{p_{y}}{n}\left(\sum_{j=0}^{n-1}\int_{0}^{1}\int_{0}^{1}\int_{0}^{y-s}\sum_{k=0}^{n-1}g(x,y,s,k,j)\,d\xi(x)\,d\upsilon(y)\,ds\right)
\end{eqnarray}

where $p_{x}$ and $p_{y}$ are the respective probabilities that $X$
and $Y$ win a bet and $g$ is $X$'s gain given $X$'s credence $x$ and
$Y$'s credal state $y\pm{}s$ on roll $j$, given result $k$. $\xi(x)$
and $\upsilon(y)$ are the distributions of the credences given our
method of simplex point picking (for these distributions, one must use
the Cayley-Menger Determinant to find out the volume of generalized
pentatopes involved).

^^^

One potential Boolean claim is that agents who use instates do better
than Laplaceans when they bet on the truth of events for which they
have varying degrees of evidence. Walley gives an example where a
Laplacean does much worse at predicting Soccer World Cup games than
Boolean peers who use upper and lower previsions (see
\scite{7}{walley91}{}, appendix I). Upper and lower previsions are
instates for which it is rational to accept or reject bets if they
fall within the margin of indeterminacy.

First I will show on a much more general level how Walley's claims are
justified in practice (bettors using upper and lower previsions do on
average better than bettors using linear previsions, i.e.\ sharp
credences). Then I will explain why this is the case, how a Laplacean
can protect herself against this disadvantage by drawing proper
distinctions between credence and evidence, and how indeterminacy
emerges as the loser when the contest is about clarity in one's
semantics.

For our purposes, we accept Walley's theory of upper and lower
previsions, which is motivated by coherence on the one hand and
avoidance of sure loss on the other hand. Upper and lower previsions
have a behavioural interpretation as maximum buying prices and minimum
selling prices for gambles. If the agent's lower prevision is $1/3$
and the upper prevision $2/3$, she would accept a bet paying \$1 if
$H$ for thirty cents and reject such a bet for seventy cents, but if
the price of the bet is between one and two thirds of a dollar it
would be rational for her to either accept or reject the bet. Walley
conducted an experiment with 17 participants, who had various levels
of understanding Bayesian theory, asking them for upper and lower
probabilities entering into bets about the games played in the Soccer
World Cup 1982 in Spain. One of the participants, a \qeins{dogmatic
Bayesian lecturer} (see \scite{8}{walley91}{633}), only used single
sharp subjective probabilities, while the others used intervals. The
dogmatic Bayesian lecturer finished a distant last when the bets were
evaluated. Walley admits that this result may have been due to the
lecturer's eccentric assessments (for the game between the Soviet
Union and Brazil, for example, his distribution between a Win, a Draw,
and a Loss was 10-80-10).

I replicated the experiment using two computer players, Betsy and
Linda, with rudimentary artificial intelligence and made them specify
betting parameters (previsions) for games played in the Soccer World
Cup 2014 in Brazil. I used the Poisson distribution (which is an
excellent predictor for the outcome of soccer matches) and the FIFA
ranking to simulate millions of counterfactual World Cup results and
their associated bets, using Walley's evaluation method. Betsy, who
used upper and lower previsions, had a slight but systematic advantage
over Linda, who used linear previsions. Betsy's expected gain for a
whole tournament was approximately \$2 (very little compared to the
stakes, as one can see by considering the standard deviation of
approximately \$46 for this expected gain, i.e. 67\% of Betsy's
overall gain from a tournament was between -\$44 and +\$48). The
surprising fact that she did consistently better than Linda remains in
need of explanation. In section \ref{WalleysWorldCupWoes}, I will
provide an explanation and show how it can support rejecting instates
for rational agents, counter-intuitive as that may sound.

Here is a table of the results. Each result is based on
100,000 die rolls. The second column shows the mean gain for Linda,
the third column shows the standard deviation, the fourth column shows
the percentage of bets which are called off. The table is for $y=1$.

\begin{tabular}{|l|r|r|r|}
  \hline
  $n=2$ & 0.14 & 18.8 & 25.2 \\ \hline
  $n=3$ & -1.67 & 13.7 & 39.1 \\ \hline
  $n=4$ & -1.73 & 10.4 & 48.1 \\ \hline
  $n=5$ & -1.39 & 8.2 & 54.5 \\ \hline
  $n=6$ & -1.22 & 6.6 & 59.1 \\ \hline
  $n=7$ & -1.01 & 5.5 & 62.8 \\ \hline
\end{tabular}

Here are the results if Betsy is not as generous with her
indeterminacy, $y=0.3$ (note that fewer bets are called off). The
second column still shows Linda's mean gain.

\begin{tabular}{|l|r|r|r|}
  \hline
  $n=2$ & 5.82 & 28.0 & 0.0 \\ \hline
  $n=3$ & -0.55 & 21.8 & 0.0 \\ \hline
  $n=4$ & -2.32 & 17.0 & 0.0 \\ \hline
  $n=5$ & -2.89 & 13.7 & 0.4 \\ \hline
  $n=6$ & -2.74 & 11.3 & 1.4 \\ \hline
  $n=7$ & -2.52 & 9.6 & 3.2 \\ \hline
\end{tabular}

We should get similar results if we do this analytically instead of
using computer simulation. I will pursue this further for the final
version of the paper. The math is not complicated, but unwieldy. The
following expression yields the expected gain for Linda:

\begin{eqnarray}
  \label{eq:s3}
  EX =
  \frac{p_{x}}{n}\left(\sum_{j=0}^{n-1}\int_{0}^{1}\int_{0}^{x}\int_{0}^{x-y}\sum_{k=0}^{n-1}g(x,y,s,k,j)\,ds\,d\upsilon(y)\,d\xi(x)\right)+
  \notag \\
  \frac{p_{y}}{n}\left(\sum_{j=0}^{n-1}\int_{0}^{1}\int_{0}^{1}\int_{0}^{y-s}\sum_{k=0}^{n-1}g(x,y,s,k,j)\,d\xi(x)\,d\upsilon(y)\,ds\right)
\end{eqnarray}

where $p_{x}$ and $p_{y}$ are the respective probabilities that $X$
and $Y$ win a bet and $g$ is $X$'s gain given $X$'s credence $x$ and
$Y$'s credal state $y\pm{}s$ on roll $j$, given result $k$. $\xi(x)$
and $\upsilon(y)$ are the distributions of the credences given our
method of simplex point picking (for these distributions, one must use
the Cayley-Menger Determinant to find out the volume of generalized
pentatopes involved).

% \nocite{*} 
\bibliographystyle{ChicagoReedweb} 
\bibliography{bib-6858}

\end{document} 
