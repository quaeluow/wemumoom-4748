I just read a paper by Abner Shimony (The Status of the Principle of
Maximum Entropy), and he is making a claim about a Lebesgue integral
in appendix A that I don't fully understand yet.

Let $f(x)$ be a function which has strictly one extremum at $x=0$
(it's a maximum, but according to Shimony that doesn't matter). We
also know that 

$$
f(0)=1/4
$$

and 

$$
\int_{\mathbb{R}}f(x)d\mu=1/4
$$

where $\mu$ is a probability measure. Shimony says that it follows
that $\mu$ puts all of its weight on the point $0$, i.e.
$\mu(\{0\})=1$. Why, and how can you show this to be true?