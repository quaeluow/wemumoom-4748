% FOR POST PROSPECTUS IDEAS AND FEEDBACK SEE DISSERTATION.ORG

\documentclass[11pt]{article}

\usepackage{october}
\hyphenation{Brett-horst}

% http://tinyurl.com/kjr8lj6 (pdf) 
% http://tinyurl.com/qfkjcgc (docx for Paul)

\begin{document}
% \onehalfspacing
% \doublespacing

\title{Information Theory and Partial Belief Reasoning}
\subtitle{Dissertation Proposal Accepted by Committee April 10, 2014}
% Instructions: The prospectus should be no longer than 10,000 words,
% everything included. The thesis prospectus is defended in a
% presentation attended by the student’s supervisor and supervisory
% committee, and any other graduate students and members of the
% Department who may wish to attend.

\author{Stefan Lukits}

\date{}

\maketitle

\newpage

\tableofcontents

\newpage

% \begin{verbatim}
% 8455    whole - prospectus.tex                                 
%         800     Section 1 - Abstract                           
%         2695    Section 2 - Literature Review                  
%                 1210    Subsection 1 - Inductive Logic and Entr
%                 494     Subsection 2 - Information Theory and t
%                 175     Subsection 3 - Early Criticism         
%                 393     Subsection 4 - Late Criticism          
%                 420     Subsection 5 - Acceptance versus Probab
%         2024    Section 3 - Proposal                           
%         2926    Section 4 - Chapter Outline                    
%                 692     Subsection 1 - Introduction            
%                 1365    Subsection 2 - The Principle of Maximum
%                         101     Subsubsection 2 - Judy Benjamin
%                         202     Subsubsection 2 - The Shimony O
%                         200     Subsubsection 2 - The Seidenfel
%                         302     Subsubsection 2 - The Wagner Ob
%                         164     Subsubsection 2 - Coarsening at
%                 636     Subsection 3 - Conceptual Problems     
%                 230     Subsection 4 - Epistemological Implicat
%         4       Section 5 - Bibliography                       
% \end{verbatim}

\section{Abstract}
\label{Abstract}
% Instructions: Ranging in length anywhere from a single paragraph to a
% single page, the abstract provides a short summary of your thesis
% proposal.

If an agent's belief state is representable by a probability
distribution (or density), then there may be guidelines or norms for
changing it in the light of new evidence. Richard Jeffrey calls the
investigation of these guidelines or norms \qnull{probability
  kinematics} (see \scite{7}{jeffrey65}{}). Standard conditioning is
the best-known (though not uncontested) updating procedure: the $\mid$
operator (probabilistic conditioning) provides a posterior probability
distribution (or density) which obeys both basic probability axioms
and intuitions about desired properties of updated probabilities.

Although almost all the claims, examples, and formal methods presented
in the following pertain to discrete probability distributions, there
are usually equivalent things to be said for continuous probability
densities, sometimes (as is the case for the Shannon entropy)
requiring some mathematical manoeuvering, sometimes (more commonly)
requiring no more than substituting an integral sign for a sum sign.
I will exclusively refer to probability distributions, but many of the
claims can be extended to probability densities.

Bayesian probability kinematics (1) requires a prior probability
distribution (or, in softened versions, a more imprecise credal state)
to precede any meaningful evaluation of evidence and (2) considers
standard conditioning to be mandatory for a rational agent when she
forms her posterior probabilities, just in case her evidence is
expressible in a form which makes standard conditioning an option.
There are various situations, such as the \emph{Judy Benjamin} problem
or the \emph{Brandeis Dice} problem, in which standard conditioning
does not appear to be an option (although some make the case that it
is, so this point needs to be established independently), and
therefore the question arises whether there is room for a more general
updating method, whose justification will entail a justification of
standard conditioning, but not be entailed by it.

E.T. Jaynes has suggested a unified updating procedure which
generalizes standard conditioning called the principle of maximum
entropy, or \textsc{pme} for short. The \textsc{pme} additionally to
Bayesian probability kinematics (3) uses information theory to develop
updating methods which keep the entropy of probability distributions
high (in the synchronic case) and their cross-entropy low (in the
diachronic case). The \textsc{pme} is based on a subjective
interpretation of probabilities as representing the degree of
uncertainty, or the lack of information, of the agent holding these
probabilities. In this interpretation, probabilities do not represent
frequencies or objective probabilities, although there are models of
how probabilities relate to them.

Most Bayesians reject the notion that the \textsc{pme} enjoys the same
level of justification as standard conditioning and maintain that
there are cases in which it delivers results which a rational agent
should not, or is not required to, accept. Note that we distinguish
between separate problems within the Bayesian camp: on the one hand,
there may be objective methods of determining probabilities prior to
any evidence or observation (call these \qnull{absolutely} prior
probabilities), for example from some type of principle of
indifference; on the other hand, there may be objective methods of
determining posterior probabilities from given prior probabilities
(which themselves could be posterior probabilities in a previous
instance of updating, call these \qnull{relatively} prior
probabilities) in case standard conditioning does not apply. My work
is concerned with the latter problem, although the \textsc{pme} can
also be used to defend objectivism about the former problem.
Absolutely prior probabilities, however, have little relation to
relatively prior probabilities.

% use Zellner's antedata postdata nomenclature

Formal epistemologists widely concur that the \textsc{pme} is beset
with too many conceptual problems and counterexamples to yield a
generally valid objective updating procedure. Against the tide of this
agreement, my work establishes that considering the tests that have
been applied to it the \textsc{pme} as the only candidate for a
generally valid, objective updating procedure is also a successful
candidate. Both the conceptual problems and the counterexamples are
surmountable, as we will show in great detail.

Many of the portrayals of the \textsc{pme}'s failings are flawed and
motivated by a desire to demonstrate that the labour of the
epistemologist in interpreting probability kinematics on a
case-by-case basis is indispensable. This \qnull{full employment
  theorem} of probability kinematics (which has a formally proven
equivalent in computer science) is widely promulgated by textbooks and
passed on to students as expert consensus. By contrast, the
\textsc{pme} combines a powerful and simple idea (update your
probabilities in accordance with constraints revealed by the evidence
without gaining more information than necessary) with a sophisticated
formal theory which confirms that the powerful and simple idea
consistently works.

Although the role of the \textsc{pme} in probability kinematics as a
whole will be the scope of my work, I will pay particular attention to
a problem which has stymied its acceptance by epistemologists:
updating or conditioning on conditionals. Two counterexamples, Bas van
Fraasen's \emph{Judy Benjamin} and Carl Wagner's \emph{Linguist}
problem, are specifically based on updating given an observation
expressed as a conditional and on the \textsc{pme}'s alleged failure
to update on such observations in keeping with strong intuitions.

The \textsc{pme} also faces much of its conceptual criticism on the
same issue with respect to \qnull{epistemic entrenchment.} Epistemic
entrenchment updates on conditionals by assuming a second tier of
commitment to propositions beneath the primary tier of quantitative
degrees of uncertainty of belief such as probabilities (or ranks). For
example, if I am confident that a coin is fair my epistemic
entrenchment that the probability of heads on the next toss is $1/2$
is much more pronounced (and resilient to countervailing evidence)
than the epistemic entrenchment in the same belief if I have no
information or confidence about the bias of the coin. The \textsc{pme}
conceptualizes this second tier differently and is consequently at
odds with the voluminous recent literature on epistemic entrenchment.
A large part of my task is to address and defend the \textsc{pme}'s
performance with respect to conditionals, both conceptually and with a
view to threatening counterexamples.

\section{Literature Review}
\label{LiteratureReview}
% Instructions: The prospectus should begin with a review of the
% relevant literature, 10 to 20 pages long, outlining the nature and
% indicating the importance of the topic, issue, or problem that will be
% discussed in the dissertation.

\subsection{Inductive Logic}
\label{InductiveLogicAndEntropy}

David Hume posed one of the fundamental questions for the philosophy
of science, the problem of induction. There is no deductive
justification that induction works, as the observations which serve as
a basis for inductive inference are not sufficient to make an argument
for the inductive conclusion deductively valid. An inductive
justification would beg the question. The late 19th and the 20th
century brought us two responses to the problem of induction relevant
to our project: (i) Bayesian epistemology and the subjective
interpretation of probability focused attention on uncertainty and
beliefs of agents, rather than measuring frequencies or hypothesizing
about objective probabilities in the world, and on decision problems
(John Maynard Keynes, Harold Jeffreys, Bruno de Finetti, Frank Ramsey;
against, for example, R.A. Fisher and Karl Popper). (ii) Philosophers
of science argued that some difficult-to-nail-down principle
(indifference, simplicity, laziness, symmetry, entropy) justified
entertaining certain hypotheses more seriously than others, even
though more than one of them may be compatible with experience (Ernst
Mach, Rudolf Carnap).

Two pioneers of Bayesian epistemology and subjectivism were Harold
Jeffreys (see \scite{7}{jeffreys31}{}, and \scite{7}{jeffreys39}{})
and Bruno de Finetti (see de Finetti, \scite{10}{definetti31}{} and
\scite{10}{definetti37}{}). They personified a divide in the camp of
subjectivists about probabilities. While de Finetti insisted that any
probability distribution could be rational for an agent to hold as
long as it obeyed the axioms of probability theory, Jeffreys
considered probability theory to be an inductive logic with rules,
resembling the rules of deductive logic, about the choice of prior and
posterior probabilities. While both agreed on subjectivism in the
sense that probabilities reflect an agent's uncertainty (or, in
Jeffreys' case, more properly a lack of information), they disagreed
on the subjectivist versus objectivist interpretation of how these
probabilities are chosen by a rational agent (or, in Jeffreys' case,
more properly by the rules of an inductive logic---as even maximally
rational agents may not be able to implement them). The logical
interpretation of probabilities began with John Maynard Keynes (see
\scite{7}{keynes21}{}), but soon turned into a fringe position with
Harold Jeffreys (for example \scite{7}{jeffreys31}{}) and E.T. Jaynes
(for example \scite{7}{jaynesbretthorst03}{}) as advocates who were
standardly invoked for refutation.

The problem in part was that the logical interpretation could not get
off the ground with plausible rules about how to choose absolutely prior
probabilities. No one was able to overcome the problem of
transformation invariance for the principle of indifference (consider
Bertrand's paradox, see \scite{8}{paris06}{71f}), not even E.T. Jaynes
(for his attempts see \scite{7}{jaynes73}{}; for a critical response
see \scite{8}{howsonurbach06}{285} and \scite{8}{gillies00}{48}).

One especially intractable problem for the principle of indifference
was Ludwig von Mises' water/wine paradox. According to van Fraassen,
it showed why we should \qeins{regard it as clearly settled now that
  probability is not uniquely assignable on the basis of a principle
  of indifference} (van Fraassen, 1989,
292\citpro{\scite{9}{fraassen89}{292}}). Van Fraassen went on to claim
that the paradox signals the ultimate defeat of the principle of
indifference, nullifying the Pyrrhic victory won by Poincar{\'e} and
Jaynes in solving other Bertrand paradoxes (see
\scite{8}{mikkelson04}{137}). Donald Gillies called von Mises' paradox
a \qeins{severe, perhaps in itself fatal, blow} to Keynes' logical
theory of probability (see \scite{8}{gillies00}{43}). De Finetti's
subjectivism was an elegant solution to this problem and marginalized
the logical theory.

While Jaynes threw up his hands over von Mises' paradox, despite the
success he had landed addressing Bertrand's paradox (see
\scite{7}{jaynes73}{}), Jeffrey Mikkelson recently suggested a
promising solution to von Mises' paradox (see
\scite{7}{mikkelson04}{}). There may still be hope for an objectivist
approach to absolutely prior probabilities. Nevertheless, my thesis
remains agnostic about this problem. The domain of my project is
probability kinematics. Relatively prior probabilities are assumed,
and their priority only refers to the fact that they are prior to the
posterior probabilities (one may call these distributions or densities
antedata and postdata rather than prior and posterior, to avoid
confusion) and not necessarily prior to earlier evidence.

This raises a conceptual problem: why would anybody be interested in a
defence of objectivism in probability kinematics when the sense is
that objectivism has failed about absolutely prior probabilities? My
intuition is in line with Keynes, who maintains that all probabilities
are conditional: \qeins{No proposition is in itself either probable or
  improbable, just as no place can be intrinsically distant; and the
  probability of the same statement varies with the evidence
  presented, which is, as it were, its origin of reference} (see
\scite{7}{keynes09}{}, chapter 1).

The problem of absolutely prior probabilities is therefore moot, and
it becomes clear that \qnull{objectivism} is not really what we are
advocating. Jaynes, who was initially interested in objectivism about
absolutely prior probabilities as well, seemed to have come around to
this position when in his last work \emph{Probability Theory: The
  Logic of Science} he formally introduced probabilities as
conditional probabilities (and later asserted that \qeins{one man's
  prior probability is another man's posterior probability,} see
\scite{8}{jaynesbretthorst03}{89}). 

A logic of induction only provides rules for how to proceed from one
probability distribution to another, given certain evidence. It does
not claim that everybody should arrive at the same distribution
(inasmuch as they claim to be rational), because there is no initial
point at which two rational agents must agree. Just as in deductive
logic, we may come to a tentative and voluntary agreement on a set of
rules and presuppositions and then go part of the way together (for a
more systematic analogy between deductive inference and probabilistic
inference see \scite{8}{walley91}{485}). [Added 2015-05-21: see
kaplan10:43 for an interesting take on this.]

Another paradigm case for this kind of objectivity is Carnap's
conventionalism in geometry. A subjectivist interpretation of
probability, which both the more strictly subjectivist probability
theorists (such as de Finetti) endorse as well as those who advocate
the logical interpretation of probability---call both of these schools
together the Bayesian school---sideline the frequentist's question
about which probability/frequency corresponds to the real world as the
conventionalist sidelines the metaphysician's question about which
geometry corresponds to the real world. The logical interpretation
goes further along with the conventionalist in attending to what is
reasonable to believe given certain formal rules that we accept even
if we have no Archimedean leverage point to start.

Probability kinematics rests on the idea that there are not only
static norms about the probabilities of a rational agent, but also
dynamic norms. The rational agent is not only constrained by the
probability axioms, but also by standard conditioning as she adapts
her probabilities to incoming evidence. Paul Teller gave a diachronic
Dutch-book argument for standard conditioning (see
\scite{7}{teller73}{}; \scite{7}{teller76}{}), akin to de Finetti's
more widely accepted synchronic Dutch-book argument (for detractors of
the synchronic Dutch-book argument see \scite{7}{seidenfeldetal90}{};
\scite{8}{foley93}{{\S}4.4}; and more recently
\scite{7}{rowbottom07}{}; for a defence see \scite{7}{skyrms87a}{}).
Brad Armendt expanded Teller's argument for Jeffrey conditioning (see
\scite{7}{armendt80}{}). In contrast to the synchronic argument,
however, there was considerable opposition to the diachronic
Dutch-book argument (see \scite{7}{hacking67}{}; \scite{7}{levi87}{};
and \scite{7}{maher92}{}). Colin Howson and Peter Urbach even made the
argument that standard conditioning itself as a diachronic norm of
updating was inconsistent (see \scite{8}{howsonurbach06}{81f}).

An alternate route to justify subjective degrees of belief and their
probabilistic nature is to use Ramsey's approach of providing a
representation theorem. Representation theorems make rationality
assumptions for preferences such as transitivity (the standard
reference work is still \scite{7}{sen71}{}) and derive from them a
probability and a utility function which are unique up to acceptable
transformations. Ramsey only furnished a sketch of how this could be
done. The first fully formed representation theorem was given by
Leonard Savage (see \scite{7}{savage54}{}); but soon Jeffrey noted
that its assumptions were too strong. Based on mathematical work by
Ethan Bolker (see \scite{7}{bolker66}{}; and a summary for
philosophers in \scite{7}{bolker67}{}), Jeffrey provided a
representation theorem with more acceptable assumptions (in
\scite{7}{jeffrey78}{}). Since then, representation theorems have
proliferated (there is, for example, a representation theorem for an
acceptance-based belief function in \scite{7}{maher93}{}, and one for
decision theory in \scite{7}{joyce99}{}). They are formally more
complex than Dutch-book arguments, but well worth the effort because
they make less controversial assumptions.

In summary, despite the success among epistemologists of de Finetti's
more strictly subjectivist viewpoint, which is suspicious towards the
claims of objectivity on part of the logical interpretation (with
which, confusingly, de Finetti shares an overall subjectivist
interpretation of probability as a measure of an agent's uncertainty,
lack of information, or partial belief), the logical interpretation
still commands intuitive appeal, internal consistency, and formal
substance.

% Paul's Comment: ``Something missing here: you need to bring the
% discussion back to the logical interpretation. Are you suggesting that
% we might offer either a DB argument or a representation theorem
% argument for PME? This might also answer my question about why there
% is more hope for PME kinematics than for PME as fixing ultimate
% priors.''

\subsection{Information Theory and the Principle of Maximum Entropy}
\label{InformationTheoryAndThePrincipleOfMaximumEntropy}

When Jaynes introduced the \textsc{pme} (see \scite{7}{jaynes57a}{};
\scite{10}{jaynes57b}{}), he was less indebted to the philosophy of
science project of giving an account of semantic information (as in
\scite{7}{carnapbarhillel52}{}; \scite{10}{carnapbarhillel53}{}) than
to Claude Shannon's mathematical theory of information and
communication. Shannon identified information entropy with a numerical
measure of a probability distribution fulfilling certain requirements
(for example, that the measure is additive over independent sources of
uncertainty). The focus is not on what information is but how we can
formalize an axiomatized measure. Entropy stands for the uncertainty
that is still contained in information (certainty is characterized by
zero entropy).

Shannon introduced information entropy in 1948 (see
\scite{7}{shannon01}{}), based on work done by Norbert Wiener
connecting probability theory to information theory (see
\scite{7}{wiener39}{}). Jaynes also traced his work back to Ludwig
Boltzmann and Josiah Gibbs, who built the mathematical foundation of
information entropy by investigating entropy in statistical mechanics
(see \scite{7}{boltzmann77}{}; \scite{7}{gibbs02}{}).

For the further development of the \textsc{pme} in probability
kinematics it is important to refer to the work of Richard Jeffrey,
who established the discipline (see \scite{7}{jeffrey65}{}), and
Solomon Kullback, who provided the mathematical foundations of minimum
cross-entropy (see \scite{7}{kullback59}{}). In probability
kinematics, contrasted with standard conditioning, evidence is
uncertain (for example, the ball drawn from an urn may have been
observed only briefly and under poor lighting conditions).

Jeffrey addressed many of the conceptual problems attending
probability kinematics by providing a much improved representation
theorem, thereby creating a tight connection between preference theory
and its relatively plausible axiomatic foundation and a probabilistic
view of \qnull{beliefs.} Jeffrey and Isaac Levi's (see
\scite{7}{levi67}{}) debates on partial belief and acceptance, which
Jeffrey considered to be as opposed to each other as Dracula and
Wolfman (see \scite{7}{jeffrey70}{}), set the stage for two
\qnull{epistemological dimensions} (Henry Kyburg's term, see
\scite{8}{kyburg95}{343}), which will occupy us in detail and towards
which I will take a more conciliatory approach, as far as their
opposition or mutual exclusion is concerned.

Kullback's divergence relationship between probability distributions
made possible a smooth transition from synchronic arguments about
absolutely prior probabilities to diachronic argument about
probability kinematics (this transition was much more troublesome from
the synchronic Dutch-book argument to the diachronic Dutch-book
argument; for the information-theoretic virtues of the
Kullback-Leibler divergence see \scite{7}{kullbackleibler51}{};
\scite{8}{seidenfeld86}{262ff}; \scite{8}{guiasu77}{308ff}).

Jaynes' project of probability as a logic of science was orginally
conceived to provide objective absolutely prior probabilities by using
the \textsc{pme}, rather than to provide objective posterior
probabilities, given relatively prior probabilities. It was, however,
easy to turn the \textsc{pme} into a method of probability kinematics
using the Kullback-Leibler divergence. Jaynes presented this method in
1978 at an MIT conference under the title \qeins{Where Do We Stand on
  Maximum Entropy?} (see \scite{7}{jaynes78}{}), where he explained
the \emph{Brandeis} problem and demonstrated the use of Lagrange
multipliers in probability kinematics.

Ariel Caticha and Adom Giffin recently demonstrated, using Lagrange
multipliers, that the \textsc{pme} seamlessly generalizes standard
conditioning (see \scite{7}{catichagiffin06}{}). Many others, however,
thought that in one way or another the \textsc{pme} was inconsistent
with standard conditioning, to the detriment of the \textsc{pme} (see
\scite{8}{seidenfeld79}{432f}; \scite{7}{shimony85}{}; van Fraassen,
1993, 288ff\citpro{\scite{8}{fraassen93}{288ff}};
\scite{8}{uffink95}{14}; and \scite{8}{howsonurbach06}{278}); Jon
Williamson believed so, too, but to the detriment of standard
conditioning (see \scite{7}{williamson11}{}).

Arnold Zellner, however, proved that standard conditioning as a
diachronic updating rule (Bayes' theorem) is the \qeins{optimal
  information processing rule} \scite{2}{zellner88}{278}, also using
Lagrange multipliers. Standard conditioning is neither inefficient
(using a suitable information metric), diminishing the output
information compared to the input information, nor does it add
extraneous information. This is just the simple conceptual idea behind
the \textsc{pme}, although the \textsc{pme} only requires optimality,
not full efficiency. Full efficiency implies optimality, therefore
standard conditioning fulfills the \textsc{pme}.

Once the \textsc{pme} was formally well-defined and its scope
established (for the latter, Imre Csisz{\'a}r's work on affine
constraints was important, see \scite{7}{csiszar67}{}), its virtues
came to the foreground. While Richard Cox (see \scite{7}{cox46}{}) and
E.T. Jaynes defended the idea of probability as a formal system of
logic, John Shore and Rodney Johnson provided the necessary detail to
establish the uniqueness of the \textsc{pme} in meeting intuitively
compelling axioms (see \scite{7}{shorejohnson80}{}).

\subsection{Early Criticism}
\label{EarlyCriticism}

By then, however, an avalanche of criticism against the \textsc{pme}
as an objective updating method had been launched. Papers by Abner
Shimony (see \scite{7}{friedmanshimony71}{};
\scite{7}{diasshimony81}{}; \scite{7}{shimony93}{}) convinced Brian
Skyrms that the \textsc{pme} and its objectivism were not tenable (see
\scite{7}{skyrms85}{}, \scite{10}{skyrms86}{}, and
\scite{10}{skyrms87}{}). Bas van Fraassen's \emph{Judy Benjamin}
problem (see van Fraassen, \scite{10}{fraassen81}{}) dealt another
blow to the \textsc{pme} in the literature, motivating Joseph Halpern
(who already had reservations against Cox's theorem, see
\scite{7}{halpern99}{}) to reject it in his textbook on uncertainty
(see \scite{7}{halpern03}{}).

Teddy Seidenfeld ran his own campaign against objective updating
methods in articles such as \qeins{Why I Am Not an Objective Bayesian}
(see \scite{7}{seidenfeld79}{}; \scite{10}{seidenfeld86}{}), while Jos
Uffink took issue with Shore and Johnson, casting doubt on the
uniqueness claims of the \textsc{pme} (see \scite{7}{uffink95}{};
\scite{10}{uffink96}{}). Carl Wagner introduced a counterexample to
the \textsc{pme} (see \scite{7}{wagner92}{}), again, as in the \emph{Judy
Benjamin} counterexample (but in much greater generality), involving
conditioning on conditionals.

\subsection{Late Criticism}
\label{LateCriticism}

In 2003, Halpern renewed his attack against the \textsc{pme} with the
help of Peter Gr{\"u}nwald and the concept of \qnull{coarsening at
  random,} which according to the authors demonstrated that the
\textsc{pme} \qeins{essentially never gives the right results} (see
\scite{8}{gruenwaldhalpern03}{243}).

In 2009, Igor Douven and Jan-Willem Romeijn wrote an article on the
\emph{Judy Benjamin} problem (see \scite{7}{douvenromeijn09}{}) in
which they asked probing questions about the compatibility of
objective updating methods with epistemic entrenchment.

Malcolm Forster and Elliott Sober's attack on Bayesian epistemology
using Akaike's Information Criterion was articulated in the 1990s (see
\scite{7}{forstersober94}{}) but reverberated well into the next
decade (Howson and Urbach call the authors the \qnull{scourges of
  Bayesianism}). Because the attack concerns Bayesian methodology as a
whole, it is not within our purview to defend the \textsc{pme} against
it (for a defence of Bayesianism see
\scite{8}{howsonurbach06}{292ff}), but it deserves mention for its
direct reference to information as a criterion for inference and
provides an interesting point of comparison for maximum entropy.

Another criticism which affected both the weaker Bayesian claim for
standard conditioning and the stronger \textsc{pme} was its purported
excessive apriorism, i.e.\ the concern that the agent can never really
move away from beliefs once formed---and that those beliefs always
need to be fully formed all the time! It can be found as early as 1945
in Carl Hempel (see \scite{8}{hempel45}{107}) and is vigorously raised
again as late as 2005 by James Joyce (see \scite{8}{joyce05}{170f}).
In \emph{Bayes or Bust?,} excessive apriorism (among other things) led
John Earman to his famous position of being a Bayesian only on
Mondays, Wednesdays, and Fridays (see \scite{8}{earman92}{1}; for the
detailed criticism \scite{8}{earman92}{139f}).

Gillies had similar reservations (see \scite{8}{gillies00}{81; 84} and
also for a pertinent quote by de Finetti see
\scite{8}{gillies00}{57}), while Seidenfeld militated against
objective Bayesianism in 1979 using excessive apriorism (we owe the
term to him, see \scite{8}{seidenfeld79}{414}). Again, because the
charge was directed at Bayesians more generally, we do not need to
address it, but mention it because the \textsc{pme} may have resources
at its disposal that the more general Bayesian position lacks (for
this position, see \scite{7}{williamson11}{}).

% PB wants a summary of charges against the \textsc{pme} here.}

\subsection{Acceptance versus Probabilistic Belief}
\label{ThePrincipleOfMaximumEntropyVices}

Epistemic entrenchment figures prominently in the AGM literature on
belief revision (for one of its founding documents see
\scite{7}{agm85}{}) and is based on two levels of uncertainty about a
proposition: its static inclusion in belief sets on the one hand, and
its dynamic behaviour under belief revision on the other hand. It is
one thing, for example, to think that the probability of a coin
landing heads is 1/2 and consider it fair because you have observed
one hundred tosses of it, or to think that the probability of a coin
landing heads is 1/2 because you know nothing about it. In the former
scenario, your belief that $P(X=H)=0.5$ is more entrenched.

Wolfgang Spohn provided an excellent overview of the interplay between
Bayesian probability theory, AGM belief revision, and ranking
functions (see \scite{7}{spohn12}{}). The extent to which the
\textsc{pme} is compatible with epistemic entrenchment and a
distinction between the static and the dynamic level will be a major
topic of my investigation. At first glance, the \textsc{pme} and
epistemic entrenchment are at odds, because the \textsc{pme} operates
without recourse to a second epistemic layer behind probabilities
expressing uncertainty. Our conclusion will be that the content of
this layer is expressible in terms of evidence and is not epistemic.

For a long time, there was unease between defenders of partial belief
(such as Richard Jeffrey) and defenders of full (and usually
defeasible or fallible) belief (such as Isaac Levi). This issue was
viewed more pragmatically beginning in the 1990s with Patrick Maher's
\emph{Betting on Theories} (see \scite{7}{maher93}{}) and Wolfgang
Spohn's work in several articles (later summarized in
\scite{7}{spohn12}{}). Both authors sought to downplay the
contradictory nature of these two approaches and emphasized how both
were necessary and able to inform each other.

Maher argued that representation theorems were superior to Dutch-book
arguments in justifying Bayesian methodology, but then distinguished
between practical utility and cognitive utility. Whereas probabilism
is appropriate in the arena of acting, based on practical utility,
acceptance is appropriate in the arena of asserting, based on
cognitive utility. Maher then provided his own representation theorem
with respect to cognitive utility, underlining the resemblance in
structure between the probabilistic and the acceptance-based approach.

In a similar vein, Spohn demonstrated the structural similarities
between the two approaches using ranking theory for the
acceptance-based approach. Together with the formal methods of the AGM
paradigm, ranking theory delivered results that were astonishingly
analogous to the already well-formulated results of Bayesian
epistemology. Maher and Spohn put us on the right track of
reconciliation between the two epistemological dimensions, and I hope
to contribute to it by showing that the \textsc{pme} can be coherently
coordinated with this reconciliation. This will only be possible if we
clarify the relation that the \textsc{pme} has to epistemic
entrenchment and how it conditions on conditionals, because on a
surface level the two are difficult to accommodate to each other.

\section{Proposal}
\label{Proposal}
% Instructions: Typically ranging in length from five to ten pages, the
% proposal describes how the topic, issue, or problem that is the
% subject of the dissertation will be dealt with. While recognizing that
% research sometimes leads one to conclusions or into areas that one did
% not have in mind at the outset of the research, the prospectus should
% nevertheless indicate what hypothesis will be defended or what goal
% will be reached.

My thesis is that the principle of maximum entropy (\textsc{pme}) is
defensible against all counterexamples and conceptual issues raised so
far as a generally valid objective updating method in probability
kinematics. Subjectivists need to work harder to undermine the
validity of the \textsc{pme}, and a fortiori the validity of
objectivism.

The \textsc{pme} operates on the basis of an astonishingly simple
principle: when updating your probabilities, waste no useful
information and do not gain information unless the evidence compels
you to gain it (see \scite{8}{jaynes88}{280},
\scite{8}{fraassenetal86}{376}, and \scite{8}{zellner88}{278}). The
astonishingly simple principle comes with its own formal apparatus
(not unlike probability theory itself): Shannon's information entropy,
the Kullback-Leibler divergence, the use of Lagrange multipliers, and
the sometimes intricate, sometimes straightforward relationship
between information and probability.

My interpretation of the \textsc{pme} is an intermediate position
between what I would call Jaynes' Laplacean idealism, where evidence
logically prescribes unique and determinate probability distributions
to be held by rational agents; and a softened version of Bayesianism
exemplified by, for example, Richard Jeffrey and James Joyce (for the
latter see \scite{7}{joyce05}{}). 

I side with Jaynes in so far as I am committed to determinate prior
probabilities, whether they are absolute or relative. Once a rational
agent considers a well-defined event space, the agent is able to
assign fixed numerical probabilities to it (this ability is logical,
not practical---in practice, the assignment may not be computationally
feasible). Because I consider this process to be contingent on
previous commitments (there is no objectivity in choosing absolutely
prior probabilities, if there is such a thing as an absolutely prior
probability) and the interpretation of evidence, both of which
introduce elements of subjectivity, I part ways with Jaynes about
objectivity.

I part ways with the \qnull{humanly faced} Bayesians because of my
commitment to determinate probabilities. \qnull{Humanly faced}
Bayesians claim that rational agents typically lack determinate
subjective probabilities and that their opinions are characterized by
imprecise credal states in response to unspecific and equivocal
evidence. There is a difference, however, between (1) appreciating the
imprecision in interpreting observations and, in the context of
probability updating, casting them into appropriate mathematical
constraints for updated probability distributions, and (2) bringing to
bear formal methods to probability updating which require numerically
precise priors. The same is true for using calculus with imprecise
measurements. The inevitable imprecision in our measurements does not
attenuate the logic of using real analysis to come to conclusions
about the volume of a barrel or the area of a circular flower bed. My
project will seek to articulate these distinctions more explicitly.

% While I appreciate the equivocality of evidence, I would separate
% the disambiguation of the evidence in articulating constraints from
% bringing to bear formal methods to probability updating which
% requires numerically precise priors. When we apply mathematics to
% daily life, we do this by measuring imprecisely and then processing
% the disambiguated measurements using calculus.

One particularly strong advocate of imprecise credal states is James
Joyce (see \scite{8}{joyce05}{156f}), with the unfortunate consequence
that the updating strategies that Joyce proposes for these credal
states are impotent. No amount of evidence can modify the imprecise
credal state, because each member of the set of credal states that an
agent accepts has a successor with respect to updating that is also a
member of these credal states and that is consistent with its
predecessor and the evidence. Although the feeling is that the
imprecise credal state is narrowed by evidence towards more precision,
set theory indicates that the credal state remains static, no
matter what the evidence is, unless we introduce a higher-level
distribution over these sets---but then the same problems arise on the
higher level.

My project therefore promotes what I would call Laplacean realism,
contrasting it with Jaynes' Laplacean idealism (Sandy Zabell uses the
less complimentary term \qeins{right-wing totalitarianism} for Jaynes'
position, see \scite{8}{zabell05}{28}), but also distinguishing it
from contemporary softened versions of Bayesianism such as Joyce's or
Jeffrey's (Zabell's corresponding term is \qeins{left-wing dadaists,}
although he does not apply it to Bayesians). What is distinctive about
my approach to Bayesianism is the high value I assign to the role that
information theory plays within it. My contention is that information
theory, much like Jaynes intended it, provides a logic for belief
revision. Almost all epistemologists, who are Bayesians, currently
have severe doubts that information theory can deliver on this
promise, not to mention their doubts about the logical nature of
belief revision (see for example Zabell's repeated charge that
advocates of logical probability have never successfully addressed
Ramsey's \qeins{simple criticism} about how to apply observations to
the logical relations of probabilities, see for example
\scite{8}{zabell05}{25}).

One way in which these doubts can be addressed is by referring them to
the more general debate about the relationship between mathematics and
the world. The relationship between probabilities and the events to
which they are assigned is not unlike the relationship between the
real numbers we assign to the things we measure and calculate and
their properties in the physical world. As unsatisfying as our
understanding of the relationship between formal apparatus and
physical reality may be, the power, elegance, and internal consistency
of the formal methods is rarely in dispute. Information theory is one
such apparatus, probability theory is another. In contemporary
epistemology, their relationship is held to be at best informative of
each other. Whenever there are conceptual problems or counterintuitive
examples, the two come apart. I consider the relationship to be more
substantial than currently assumed.

There have been promising and mathematically sophisticated attempts to
define probability theory in terms of information theory (see for
example \scite{7}{ingardenurbanik62}{}; \scite{7}{kolmogorov68}{};
\scite{7}{kampe67}{}---for a detractor who calls information theory a
\qeins{chapter of the general theory of probability} see
\scite{7}{khinchin57}{}). 
% For the straightforward relationship consider that information is
% added where probabilities are multiplied, as in the following formula,
% $P(X)=\exp{-I(X)}$.
While interesting, however, making information theory or probability
theory derivative of the other is not my project. What is at the core
of my project is the idea that information theory delivers the unique
and across the board successful candidate for an objective updating
mechanism in probability kinematics. This idea is unpopular in the
literature, but as evidenced in the chapter outline the arguments on
which the literature relies are not robust, neither in quantity nor in
quality.

Carnap advises pragmatic flexibility with respect to inductive
methods, although he presents only a one-dimensional parameter system
of inductive methods which curbs the flexibility. On the one hand,
Dias and Shimony report that Carnap's $\lambda$-continuum of inductive
methods is consistent with the \textsc{pme} only if $\lambda=\infty$
(see \scite{7}{diasshimony81}{}). This choice of inductive method is
unacceptable even to Carnap, albeit allowed by the parameter system,
because it gives no weight to experience (see
\scite{8}{carnap52}{37ff}). On the other hand, Jaynes makes the case
that the \textsc{pme} entails Laplace's Rule of Succession
($\lambda=2$) and thus occupies a comfortable middle position between
giving all weight to experience ($\lambda=0$, for the problems of this
position see \scite{8}{carnap52}{40ff}) or none at all
($\lambda=\infty$). While Carnap's parameter system of inductive
methods rests on problematic assumptions, we will show why Dias and
Shimony's assignment of $\lambda$, given the \textsc{pme}, is
erroneous, and why Jaynes' assignment is better justified.

Jeffrey, van Fraassen, Halpern, Skyrms, Persi Diaconis and Sandy
Zabell (see \scite{7}{diaconiszabell82}{}), Colin Howson and Allan
Franklin (see \scite{7}{howsonfranklin94}{}), Douven and Romeijn,
however, all follow Carnap in giving weight not only to a logical and
an empirical component in induction, but also to pragmatic
considerations. The pragmatic considerations often turn into a
requirement that an expert epistemologist study each situation calling
for belief revision on its own and address it with recourse to a
toolkit of updating procedures, each being context-appropriate under
different circumstances.

In order to defend my position, I need to address three important
counterexamples which at first glance discredit the \textsc{pme}:
Shimony's Lagrange multiplier problem, van Fraassen's \emph{Judy
  Benjamin} case, and Wagner's \emph{Linguist}. For the latter two, I
am confident that we can make a persuasive case for the \textsc{pme},
based on formal features of these problems which favour the
\textsc{pme} upon closer examination. For the former (Shimony), Jaynes
has written a spirited rebuttal: \qeins{This brings us, obviously, to
  the matter of Shimony. I am not a participant, but, like other
  readers, only a bewildered onlooker in the spectacle of his epic
  struggles with himself} \scite{3}{jaynes85}{134}. According to
Jaynes, errors in Shimony's argument have been pointed out five times
(see \scite{7}{hobson72}{}; \scite{7}{tribusmotroni72}{};
\scite{7}{gagehestenes73}{}; \scite{7}{jaynes78}{};
\scite{7}{cyranski79}{}). This does not, however, keep Brian Skyrms,
Jos Uffink, and Teddy Seidenfeld from referring again to Shimony's
argument in rejecting the \textsc{pme} in the 1980s, so the matter
must be sorted out, and we promise to do so.

Once the counterexamples are out of the way, the more serious
conceptual issues loom. There is good news and bad news for advocates
of the \textsc{pme}. On the one hand, there are powerful conceptual
arguments affirming the special status of the \textsc{pme}. Shore and
Johnston, who use the axiomatic strategy of Cox's theorem in
probability kinematics, show that relatively intuitive axioms only
leave us with the \textsc{pme} to the exclusion of all other objective
updating methods. Van Fraassen, R.I.G. Hughes, and Gilbert Harman's
MUD method, for example, or their maximum transition probability
method from quantum mechanics both fulfill their five requirements
(see van Fraassen et al., \scite{10}{fraassenetal86}{}), but do not
fulfill Shore and Johnston's axioms. Neither does Uffink's more
general class of inference rules, which maximize the so-called
R{\'e}nyi entropies, but Uffink argues that Shore and Johnston's
axioms rest on unreasonably strong assumptions (see
\scite{7}{uffink95}{}). Caticha and Giffin counter that Skilling's
method of induction (see \scite{7}{skilling88}{}) and Jaynes'
empirical results in statistical mechanics and thermodynamics imply
the uniqueness of Shannon's information entropy over rival entropies.

To continue with the good news, the \textsc{pme} seamlessly
generalizes standard conditioning and Jeffrey's rule where they are
applicable (see \scite{7}{catichagiffin06}{}). It underlies the
entropy concentration phenomenon described in Jaynes' standard work
\emph{Probability Theory: the Logic of Science}, which also contains a
sustained conceptual defence of the \textsc{pme} and its underlying
logical interpretation of probabilities. Entropy concentration refers
to the unique property of the \textsc{pme} solution to have other
distributions which obey the affine constraint cluster around it. When
used to make predictions whose quality is measured by a logarithmic
score function, posterior probabilities provided by the \textsc{pme}
result in minimax optimal decisions (see \scite{7}{topsoe79}{};
\scite{7}{walley91}{}; \scite{7}{grunwald00a}{}) so that by a
logarithmic scoring rule these posterior probabilities are in some
sense optimal.

Jeff Paris has investigated different belief functions (probabilities,
Dempster-Shafer, and truth-functional, see \scite{7}{paris06}{}) from
a mathematical perspective and come to the conclusion that given
certain assumptions about the constraints that experience normally
imposes (we will have to examine their relationship to the affine
constraints assumed by the \textsc{pme}), if a belief function is a
probability function, only minimum cross entropy belief revision
satisfies a host of desiderata (continuity, equivalence, irrelevant
information, open-mindedness, renaming, obstinacy, relativization, and
independence) while competitors fail on multiple counts (see
\scite{7}{parisvencovska90}{}).

On the other hand, now turning to the bad news, the belief revision
literature has in the last twenty years mostly turned its attention to
the AGM paradigm (named after Carlos Alchourr{\'o}n, Peter
G{\"a}rdenfors, and David Makinson), which operates on the basis of
fallible beliefs and their logical relationships. These are really at
this point two different epistemic dimensions (to use Henry Kyburg's
expression): the one where doxastic states are cashed out in terms of
fallible beliefs which move in and out of belief sets; the other the
dimension of probabilities where \qnull{beliefs} are vague labels for a
more deeply rooted, graded notion of uncertainty. 

Jeffrey with his radical probabilism pursues a project of
epistemological monism (see \scite{7}{jeffrey65}{}) which would reduce
beliefs to probabilities, while Spohn and Maher seek reconciliation
between the two dimensions, showing how fallible full beliefs are
epistemologically necessary and how the formal structure of the two
dimensions reveals many shared features so that in the end they have
more in common than what separates them (see \scite{8}{spohn12}{201}
and \scite{7}{maher93}{}).

In the end, our project is not about the semantics of doxastic states.
We do not argue the eliminativism of beliefs in favour of
probabilities; on the contrary, the belief revision literature has
opened an important door for inquiry in the Bayesian dimension with
its concept of epistemic entrenchment. This is a good example for the
kind of cross-fertilization between the two different dimensions that
Spohn had in mind, mostly in terms of formal analogies and with little
worry about semantics, important as they may be. Maher has given
similar parallels between the two dimensions, also with an emphasis on
formal relationships, in terms of representation theorems. Pioneering
papers in probabilistic versions of epistemic entrenchment are recent
(see \scite{7}{bradley05}{}; \scite{7}{douvenromeijn09}{}).

The guiding idea behind epistemic entrenchment is that once an agent
is apprised of a conditional (indicative or material), she has a
choice of either adjusting her credence in the antecedent or the
consequent (or both). Often, the credence in the antecedent remains
constant and only the credence in the consequent is adjusted (Bradley
calls this \qnull{Adams conditioning}). Douven and Romeijn give an
example where the opposite is plausible and only the credence in the
consequent is left constant (see \scite{8}{douvenromeijn09}{12}).
Douven and Romeijn speculate that an agent could theoretically take
any position in between, and they use Hellinger's distance to
represent these intermediary positions formally (see
\scite{8}{douvenromeijn09}{14}).

Even though Bradley, Douven, and Romeijn are in the dimension of
probabilities, they are using a notion frequently used and introduced
by the AGM literature to capture formally analogous structures in
probability theory. The question is how compatible the use of
epistemic entrenchment in probabilistic belief revision (probability
kinematics) is with the \textsc{pme}. The \textsc{pme} appears to
assign probability distributions to events without any heed to
epistemic entrenchment. The \emph{Judy Benjamin} problem is a case in
point. \textsc{pme}'s posterior probabilities are somewhere in between
the possible epistemic entrenchments, as though mediating between
them, but they affix themselves to a determinate position (which in
some quarters raises worries analogous to excessive apriorism).

My claim is that the \textsc{pme} does not accept two levels of
epistemic commitment: the static and the dynamic. AGM belief revision
theory, Spohn's ranking functions, and epistemic entrenchments
according to Bradley, Douven, and Romeijn suppose that beneath our
credences (static probabilities), believers entertain a second set of
dynamic probabilities which are determinative of the kinematics once
doxastic states are subject to change. This view is inconsistent with
the \textsc{pme}, and so I hold against it that the \textsc{pme} can
only understand this second dynamic set of graded commitments as
information. In other words, entrenchments are evidential, not
epistemic. To refer to the earlier example, the fact that a coin has
been tossed a hundred times, so that now we consider it to be a fair
coin (rather than assigning a 50:50 probability to heads and tails
because we do not know any better), is information and part of our
evidence; it is not part of the epistemic state of a rational agent,
such as a belief or a probability function would be.

Despite the potholes in the historical development of the
\textsc{pme}, on account of its unifying features, its simple and
intuitive foundations, and its formal success it deserves more
attention in the field of belief revision and probability kinematics,
definitely more attention than the many competing ad hoc methods (such
as Carl Wagner's) which patch one problem while raising many more
somewhere else. The \textsc{pme} is the single principle which can
hold things together over vast stretches of epistemological terrain
(intuitions, formal consistency, axiomatization, case management) and
calls into question the scholarly consensus that such a principle is
not needed.

Epistemologists are generally agreed that such a principle is not
needed because they expect pragmatic latitude in addressing questions
of belief revision. Similar to how scientists subjectively choose
absolutely prior probabilities, the full employment theorem gives
epistemologists access to a wide array of updating methods. As
there is already widespread consensus that objectivism has died on the
operating table of absolutely prior probabilities, nobody sees any
reason to prolong the dead patient's life on the sickbed of
probability kinematics. The problem with this position is that the
wide array of updating methods systematically leads to solutions which
contradict the principle that one should use relevant information and
not gain unwarranted information. 

Often, independence assumptions sneak in through the back door where
they are indefensible. Well-posed problems are dismissed as ambiguous
(e.g.\ the \emph{Judy Benjamin} problem), while problems that are
over-determined may be treated as open to a whole host of solutions
because they are deemed under-determined (e.g.\ von Mises' water and
wine paradox). Ad hoc updating methods proliferate which can often be
subsumed under the \textsc{pme} with little mathematical effort (e.g.\
Wagner's \emph{Linguist} problem). 

Worst of all, epistemologists suffer from the mistaken perception that
they will always be indispensable to the scientist's and the quotidien
reasoner's quest for proper belief revision in the face of new
evidence. I call this perception \qnull{full employment,} so named
after a theorem in computer science, where it can be formally proven
that no computer program can write itself all necessary computer
programs without the assistance of a computer scientist. Formal
epistemologists would like to be in the same position with respect to
probability updating. I maintain that information theory provides all
the necessary tools to update probabilities effectively, universally,
and consistently.

\section{Chapter Outline}
\label{ChapterOutline}
% Instructions: Typically short and to the point, the chapter outline
% should still contain enough information to give the reader an
% indication of how the dissertation will be structured.

\subsection{Introduction}
\label{Introduction}

The introduction provides a first framework for the problem of
probability kinematics, underlines the philosophical relevance, and
embeds the discussion in a wider epistemological context. It also
introduces the problem of how we can identify evidence with affine
constraints and presents the principle of maximum entropy both in its
synchronic form (\textsc{maxent}) and in its diachronic form
(\emph{Infomin}). There is a view which holds \textsc{maxent} and
\emph{Infomin} to be inconsistent with each other, which needs to be
addressed (see \scite{7}{wagner02}{}). 

Information itself is notoriously difficult to define and comes in
different varieties (Shannon information, Solomonoff complexity,
quantum information, semantic information) that have not been
successfully reduced to a single concept. We are primarily interested
in Shannon information, i.e.\ the information associated with
probability distributions or densities. It is an open question what
the relationship between information theory and probability theory is,
whether one is derivative of the other or whether they are independent
accounts of uncertainty informing each other in interesting ways.

An affine constraint restricts evidence to subsets which are closed
and convex in a suitable information topology of probability
distributions or densities. For example, when standard conditioning is
applicable (\qnull{a die was rolled, and the result is an even
  number}) only those probability distributions which assign $1$ to
the observed event are eligible as posterior probabilities. I have not
been able to find a systematic justification for the restrictions of
affine constraints (the subset must be closed and convex), except that
once they are in place they provide just the right formal assumptions
to warrant the existence and uniqueness of a maximum entropy solution
(Paris is mathematically most explicit, but also most unabashed about
citing mathematical convenience as uppermost in accepting these
restrictions, see \scite{7}{paris06}{}. \scite{7}{csiszar67}{} is
another place to consider and often referred to for moral support by
advocates of the \textsc{pme}. It would constitute great progress if
someone could bring more clarity to this question, although due to the
mathematical challenges this may not be the task of a philosophy
student (both Paris and Csisz{\'a}r are mathematicians). For this, see
also \scite{8}{debbahmueller05}{1673}.

To my knowledge there are no counterexamples in which evidence is not
an affine constraint, although it is probably possible to construct
one. Paris, for example, refers to non-linear constraints of belief
functions as too complicated to handle (undermining certain uniqueness
claims, see \scite{8}{paris06}{70}) and in any case unlikely to be
encountered in real-world problems (see \scite{8}{paris06}{7}). Affine
constraints are equivalent to expectations provided by evidence (I
have so far not been able to locate a formal proof for this result,
but see \scite{7}{hobson71}{}), and those in turn cover all the
classic examples of evidence: standard conditioning, Jeffrey
conditioning, and partial information.

Affine constraints come in three forms: observation with certainty of
an event (as above, \qnull{a die was rolled, and the result is an
  even number}); complete re-partitioning of the event space (\qnull{a
  die was rolled, and the probability that the result is an even
  number rather than an odd number is 60:40}); and reassessment of
expectation (\qnull{a die was rolled, and the expected value of the
  result is 4.5,} the \emph{Brandeis} problem). This chapter shows how
these forms are related. The third form is a generalization of the
second form, and the second form is a generalization of the first
form. Consequently, a method for solving the problem of probability
kinematics for the third form automatically solves this problem for
the other forms.

This chapter introduces standard conditioning and reviews the
arguments why standard conditioning is widely accepted as a solution
for affine constraints of the first form. This chapter introduces
Jeffrey conditioning and reviews the arguments why Jeffrey
conditioning is less widely accepted as a solution for affine
constraints of the second form. The thesis of this work is conditional
on the acceptance of Jeffrey conditioning as a solution to the problem
of probability kinematics for the second form, although I will provide
a brief overview of counterarguments. Then this chapter introduces
affine constraints which are not covered by Jeffrey conditioning. Let
us call these strictly affine constraints.

Strictly affine constraints exist, and they complete the list that we
need to consider for the problem of probability kinematics. There is a
formal basis for these two claims. My work provides counterarguments
to the claim that Jeffrey conditioning and the \textsc{pme} can be
subsumed under standard conditioning (for example in
\scite{7}{domotor85}{}; \scite{7}{skyrms85}{}; or retrospective
conditioning, summary and problems in
\scite{8}{diaconiszabell82}{822}; applied to the \emph{Judy Benjamin}
problem see \scite{7}{grovehalpern97}{}). Especially (but not
exclusively) in statistical physics there are applications where
standard conditioning is not an effective method, while Jeffrey
conditioning and the \textsc{pme} are. As mentioned above, the claim
that the three forms exhaust the list is formally challenging, and I
am uncertain to what extent I can address the question in my project.

\subsection{The Principle of Maximum Entropy: Virtues and Vices}
\label{ThePrincipleOfMaximumEntropyVirtuesAndVices}

This chapter introduces the \textsc{pme} and highlights its strengths
and vulnerabilities. In terms of strengths, the \textsc{pme} provides
a unique solution for all affine constraints and is as well-immunized
against transformation variance as one can reasonably expect. Jaynes
has identified the most important transformation groups (for example
scale, translation, rotation, and location; for language invariance
see \scite{8}{paris06}{76}), with respect to which the \textsc{pme} is
transformation invariant (see \scite{7}{jaynes73}{}; and
\scite{8}{jaynesbretthorst03}{378}). Howson and Urbach mention
transformation groups (for example, arbitrary variations in space-time
curvature and arbitrary coordinate transformations), with respect to
which the \textsc{pme} is not transformation invariant (see
\scite{8}{howsonurbach06}{285}).

The \textsc{pme} solution uniquely fulfills several desiderata, and
thus we are able to provide an axiomatic basis for its use (see
\scite{7}{shorejohnson80}{}; \scite{7}{tikochinskyetal84}{};
\scite{7}{skilling88}{} for details; \scite{7}{uffink96}{} for
criticism). The \textsc{pme} accords with standard conditioning if the
affine constraint is of the first form and with Jeffrey conditioning
if the affine constraint is of the second form. While it is highly
contested as an objective updating method for strictly affine
constraints, it exhibits important virtues as such a method. It
underlies the entropy concentration phenomenon. 

A large majority of philosophers of science and epistemologists,
however, rejects the idea that the \textsc{pme} is normative as an
objective updating procedure given strictly affine constraints.
Instead, they adhere to a type of \qnull{full employment theorem,}
according to which affine constraints must be submitted to the
individualized attention of an expert before the problem can be
considered properly investigated. I will call this camp
\qnull{opponents,} not because I necessarily believe that their claims
are false, but because I believe that the arguments on which their
claims rest are either faulty or incomplete. 

They have not made a convincing case that the \textsc{pme} lacks
generality. Their case rests on the vulnerabilities of the
\textsc{pme}, which we will try to address in a comprehensive manner.
This chapter portrays these vulnerabilities in their strongest form:
the \emph{Judy Benjamin} problem; the Shimony objection; the
Seidenfeld objection; the Wagner objection; and Gr{\"u}nwald and
Halpern's \qnull{Coarsening at Random.} All of these objections derive
counterintuitive results from the \textsc{pme} and claim that
advocates of the \textsc{pme} find themselves in the awkward position
of having to assent to something to which they would intuitively, and
even after reflection, withhold assent.

\subsubsection{Judy Benjamin}
\label{JudyBenjamin}

Opponents prominently cite van Fraassen's \emph{Judy Benjamin} case to
undermine the generality of the \textsc{pme}. This chapter shows that
an intuitive approach to the \emph{Judy Benjamin} case supports the
\textsc{pme}. This is surprising because based on independence
assumptions the anticipated result is that it would support the
opponents. The chapter also demonstrates that opponents improperly
apply independence assumptions to the problem. Not dissimilar to the
(1/2)er camp and the (1/3)er camp in the \emph{Sleeping Beauty} case,
there is a (1/2)er camp and a (1/3)er camp in the \emph{Judy Benjamin}
case. This chapter gives rigorous arguments for the (1/3)er camp in
the \emph{Judy Benjamin} case (the analogy with Sleeping Beauty goes
deeper than the labels, see \scite{7}{bovens10}{}).

\subsubsection{The Shimony Objection}
\label{TheShimonyObjection}

In a series of papers, Abner Shimony has highlighted cases in which
the \textsc{pme} results in the counterintuitive claim that being
informed of a Lagrange multiplier requires that one has expected it
with certainty. This chapter introduces Shimony's objection and mounts
a defence of \textsc{pme} against it. Attempts at a defence are
available (see \scite{7}{hobson72}{}; \scite{7}{tribusmotroni72}{};
\scite{7}{gagehestenes73}{}; \scite{7}{cyranski79}{};
\scite{7}{jaynes85}{}), but there is also strong support for Shimony's
objection, coupled with various attempts at strengthening his results
(see \scite{7}{seidenfeld86}{}; \scite{7}{skyrms87}{}). I am not yet
in a position to articulate a solution or a route towards one. It will
probably require careful distinctions of what it looks like to gain
information legitimately, especially with reference to using
mathematical tools. 

(I do not know if this is a helpful analogy, but it is alleged that
the number $\pi$ is normal, i.e.\ there are no patterns in its digits,
independent of the base. A formal proof that $pi$ is normal is
elusive. Consequently, its Shannon entropy is high, whereas its
Kolmogorov complexity is low, as $pi$ is expressible by a relatively
simple formula. Therefore, a certain amount of care is needed in
describing what it means to be informed that $x=\pi$. To be informed
of the value of a Lagrange multiplier may reveal similar
complications.)

One particular problem for the \textsc{pme} is that Shimony's argument
identifies it with $\lambda=\infty$ in Carnap's parameter system of
inductive logic. This is the case in which observation has no weight
for induction, a state of affairs universally held to be undesirable.
The game of deprecating an inductive method by showing that it implies
$\lambda=\infty$ is played already by Carnap himself, his targets
being C.S. Peirce, Ludwig Wittgenstein, and John Maynard Keynes (see
\scite{8}{carnap52}{40}). Jaynes seeks to fend off the attack by
showing that the \textsc{pme} entails $\lambda=2$ (Laplace's Rule of
Succession) and that Shimony's Lagrange multiplier argument is an
instance of hysteron proteron.

\subsubsection{The Seidenfeld Objection}
\label{TheSeidenfeldObjection}

Teddy Seidenfeld claims that the \textsc{pme} as a rule to update on
partial information (an affine constraint of the third form) is
unacceptable. It leads to precise probabilities that are excessively
aprioristic, containing more information than the evidence generating
them allows. This being a common objection to \textsc{pme} among
philosophers, we will show how it cannot be coherently raised against
the \textsc{pme} without being raised against Bayesian epistemology as
a whole. This section also provides a brief overview of reasons why within
appropriate contexts it is advisable to accept the normative claims of
Bayesian epistemology. Therefore, the normative claims of the
\textsc{pme} are immune to Seidenfeld's objection to the degree to
which we have already accepted the normative claims of Bayesian
epistemology.

Besides this more general objection, Teddy Seidenfeld develops several
formal lines of argument against the \textsc{pme}, for example that
under certain circumstances involving noise factors the \textsc{pme}
will inappropriately provide more information based on less evidence.
Another example involves an alleged incompatibility between Bayes'
Theorem and the \textsc{pme}. Seidenfeld also explains and expands the
Shimony objection in a new light. This section will look in detail at
available lines of defence for the \textsc{pme} against Seidenfeld's
objections.

\subsubsection{The Wagner Objection}
\label{TheWagnerObjection}

Carl Wagner is a typical representative of the full employment line of
argument, who in his own words values the hard work of judging by
warring against the temptations of mechanical updating (see
\scite{8}{wagner92}{255f}). Other representatives of full employment
with similar sentiments are Peter Gr{\"u}nwald, Joseph Halpern,
Richard Bradley, Persi Diaconis, and Sandy Zabell. E.T. Jaynes, the
most prominent proponent of the \textsc{pme}, sometimes speaks
derisively about the relationship of the statistician (or formal
epistemologist) and the client as one between a doctor and his
patient, which I have characterized as full employment in allusion to
the full employment theorem in computer science.

Wagner's formal attack against the \textsc{pme} concerns constraints
that are a generalization of Jeffrey conditioning, for which Wagner
suggests an intuitively plausible revision procedure. Wagner's
procedure is based on what I call Jeffrey's principle: Leave the ratio
of probabilities alone if they are not affected by your observation or
your evidence. Let Mr.\ A, Mr.\ B, Ms.\ X, and Ms.\ Y be the exclusive
group of people suspected of a crime and the ratio of probabilities
which the detective assigns to them (as having committed the crime) be
2:3:1:4 (out of ten). Jeffrey's principle states that if the detective
finds out that the culprit is male, the probabilities update to
4:6:0:0, all else being equal.

Wagner constructs a case, the \emph{Linguist} problem, where Jeffrey
conditioning is not applicable. He generalizes Jeffrey conditioning,
using Jeffrey's principle, and finds that his generalization violates
the \textsc{pme}. There are now two plausible intuitions inconsistent
with each other: Jeffrey's principle and the principle of maximum
entropy. Wagner concludes that, given all the other conceptual
problems of the \textsc{pme} and the counterexamples, we ought to
accept Jeffrey's principle and feel that the case against the
\textsc{pme} has been corroborated once again.

Wagner's application of the \textsc{pme} which violates Jeffrey's
principle, however, is incorrect. A correct application agrees with
both with his procedure and with Jeffrey's principle. I present a
formal proof that the \textsc{pme} seamlessly and elegantly
generalizes Wagner's generalization of Jeffrey conditioning, with the
added benefit that it is not merely an ad hoc rule such as Wagner's,
but integrated in a unified approach to probability kinematics. This
section conducts a careful analysis of Wagner's procedure, which is
based on Dempster's Rule, and argues that the \textsc{pme} offers an
elegant generalization of everything that Wagner wants except his
rejection of determinate prior probabilities. We will make a case for
them within a Bayesian framework.

It is generally not difficult to find alternatives to the \textsc{pme}
(consider for example van Fraassen's Maximum Transition Probability
and his less delicately named MUD method, both of which beat the
\textsc{pme} on select performance criteria). My claim is that based
on the virtues of the \textsc{pme} and based on the fact that all such
alternatives violate the principle that belief revision should not
result in unwarranted information gain, we should reject such
alternatives, no matter how appealing some of their qualities are. For
van Fraassen's MUD and MTP, for example, on the first of two
performance criteria MUD places first, the \textsc{pme} second, and
the MTP third. On van Fraassen's other performance criterion, the
ranking is just the reverse. I count it as a virtue of the
\textsc{pme} to do relatively well on both criteria and would not be
surprised if the \textsc{pme} in some sense optimizes its performance
across criteria rather than doing particularly well on smaller
subsets.

\subsubsection{Coarsening at Random}
\label{CoarseningAtRandom}

Coarsening at random (\textsc{CAR}) involves using more naive (or
coarse) event spaces in order to arrive at solutions to probability
updating problems. Gr{\"u}nwald and Halpern show that updating on the
naive space rather than the sophisticated space is legitimate for
event type observations when the set of observations is pairwise
disjoint or when the \textsc{CAR} condition (as defined by them)
holds. For Jeffrey type observations, there is a generalized
\textsc{CAR} condition which applies likewise. For strictly affine
constraints, the \textsc{pme} essentially never gives the right
results, according to the authors.

Gr{\"u}nwald and Halpern mention the \emph{Judy Benjamin} problem as
their prime example for a case in which the \textsc{pme} delivers the
wrong result. It does so in analogy to the evidently wrong result in
the \emph{Monty Hall} or the \emph{Three Prisoners} problem by naive
conditioning. This chapter shows how the analogy is misguided and the
authors' claim misleading that the \textsc{pme} essentially always
gives the wrong results.

\subsection{Conceptual Problems}
\label{ConceptualProblems}

This chapter addresses conceptual problems that have been raised with
respect to the \textsc{pme} rather than specific counterexamples to
the procedure. First, Jos Uffink targets especially Shore and
Johnson's assumptions when they identify the \textsc{pme} as the
unique method for determining updated probability distributions, given
certain types of rationality constraints. Uffink shows how a more
reasonable restatement of Shore and Johnson's assumptions results in a
whole class of updating procedures, the so-called R{\'e}nyi entropies.

Van Fraassen, with collaborators R.I.G. Hughes and Gilbert Harman,
also concludes that there is a family of candidates which fulfills
five requirements or principles that he establishes with the
\emph{Judy Benjamin} problem in view. If the five requirements are
complemented by two performance criteria, the \textsc{pme} is not the
best updating method with respect to any of the performance criteria.
The objection that instead of a unique rational updating method there
is a family of such methods is again common among philosophers,
especially because it accords with Rudolf Carnap's continuum of
inductive methods in a different context. This chapter presents and
expands on Ariel Caticha and Adom Giffin's arguments against families
of updating methods.

Second, there is a large branch of belief revision literature which
assumes that a person can only have meaningful updating methods, for
ranks, probabilities, or other measures of uncertainty, if behind a
first (static) layer there is a second (dynamic) layer which
determines how the first layer behaves when changes occur. The
principle of maximum entropy, by contrast, only operates on one level
and must incorporate the second level as information. Degrees of
belief about degrees of belief may be coherent and useful (see
\scite{7}{jeffrey74}{}; \scite{7}{skyrms80}{}; Domotor
\scite{10}{domotor80}{}; \scite{10}{domotor81}{}). Our contention is
that they can be subsumed under evidence-handling and are thus of a
completely different source and nature than first-order degrees of
belief.

Third, and closely related to the previous point, upon learning the
truth of a conditional, degree of belief in the antecedent and the
consequent may be re-evaluated. AGM belief revision theory has at its
heart the idea of epistemic entrenchment, the degree to which an agent
wants to maintain belief in a proposition when she is informed of its
antecedent role in conditionals. Igor Douven and Jan-Willem Romeijn
have applied epistemic entrenchment to the \emph{Judy Benjamin}
problem and evaluate it in terms of \qnull{Adams conditioning,} the
kind of conditioning where the degree of belief in the antecedent
remains unaltered. This solution, which receives support in a number
of other papers (for example, by Joseph Halpern), in contrast to the
\textsc{pme} contradicts van Fraassen, Hughes, and Harman's five
requirements.

Epistemic entrenchment and the \textsc{pme} are incompatible at first glance.
Wolfgang Spohn's work in ranking theory illustrates a strong
commitment to the idea that a rational agent needs both a quantitative
assessment (either in terms of rankings or in terms of probability) as
well as parameters of epistemic entrenchment in order to make sense
out of updating in light of new observation or evidence. The \textsc{pme},
however, updates by operating on probability distributions alone, only
in conjunction with an objective principle of minimum information
gain. This chapter will seek reconciliation between the strong
intuitive appeal of both the \textsc{pme} and epistemic entrenchment.

Fourth, in the spirit of this reconciliation we will try to establish
that the epistemic dimensions of probabilistic belief and acceptance
are complementary, i.e.\ irreducible with respect to each other on the
one hand, while on the other hand rich in cross-fertilization. They do
not exist in parallel universes with unrelated sets of formal
relations and strictly separate domains. On the contrary, they inform
each other and are best kept in view simultaneously, both formally and
in contexts of application. It will be the task of this chapter to
clarify the nature of the relationship.

\subsection{Epistemological Implications and Conclusion}
\label{EpistemologicalImplications}

In the concluding chapter, I want to look at the wider epistemological
implications of a more affirmative stance towards the \textsc{pme} as
an objective updating method. Importantly, information becomes a
basic notion to which philosophers will increasingly pay attention,
possibly at the expense of probability. There may be a parallel to the
salience that preference gained in epistemology following Ramsey's
work and the numerous representation theorems in its wake. It is
plausible that information will fill a similar need for explicating
the notion of probability, this time, however, not in the more
subjective terms of preference, but in the more objective terms of
information.

Probability theory is easily expressible by information theory,
whereas the reverse is not true. For example, there is no analogue to
Chaitin's incompleteness theorem in probability theory---in general,
information theory is easily linked to algorithmic compressibility
(Shannon information to Solomonoff/\-Kolmogorov complexity), which may
very well have substantial epistemological implications once belief
revision is viewed in information-theoretic rather than
probability-theoretic terms. 

Questions in epistemology, in the theory of causation, in the
philosophy of science, and perhaps also in the philosophy of cognition
may be rearticulated and answered in different ways if information is
more widely used as a common currency between these fields. In
physics, information already plays a vital role, whereas the
philosophy of information in many respects is in its infancy.

\section{Bibliography}
\label{Bibliography}
% Instructions: A list of writings especially relevant to the
% dissertation, including all works discussed in the Literature Review,
% in the form of a standard bibliography.

% \nocite{*} 
\bibliographystyle{ChicagoReedweb} 
\bibliography{bib-3415}

\end{document} 
