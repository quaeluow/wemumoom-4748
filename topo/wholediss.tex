\documentclass[phd,12pt,oneside,paper=letterpaper]{ubcthesis}
\usepackage[pdftex]{geometry}
\usepackage[round]{natbib}
\usepackage{setspace}
\onehalfspacing
\institution{The University Of British Columbia}
\faculty{The Faculty of Graduate Studies}
\institutionaddress{Vancouver}
% \previousdegree{BFA, The School of Visual Arts, 1999}
% \previousdegree{BA/MA, CUNY Queens College, 2005}
\previousdegree{Magister rerum naturalium (M.\ Sc.), University of Graz, 1995}
\previousdegree{Master of Divinity (M.\ Div.), Regent College, 1998}
\previousdegree{Master of Arts (M.\ A.), University of British Columbia, 2010}
\degreetitle{Doctor of Philosophy}
\title{Information Theory and Partial Belief Reasoning}
\author{Stefan Hermann Lukits}
\copyrightyear{2016}
\submitdate{February\ 2016}
\program{Philosophy}
\begin{document}
\frontmatter
\maketitle
\begin{abstract}
A rational agent has partial beliefs that are exclusively in the form
of sharp credences and updates them using methods derived from
information theory.
\end{abstract}
\tableofcontents
% \chapter{Acknowledgments}

% \chapter{Dedication}

\mainmatter
\chapter{Are We Responsible?}
Responsibility is intimately linked with our personhood. The core claim of this dissertation is that moral responsibility and personhood are interdependent. Only persons can be morally responsible for things, and that it is through moral responsibility, we come to be persons. This interdependence is complex, and a misunderstanding of this interdependence has generated a number of skeptical worries about whether or not human beings really are morally responsible. Many of these skeptical worries about responsibility are grounded in challenges to the claim, implicit or explicit, that persons are different from the other things of the universe in a way that sustains holding them morally responsible. 
\section{Responsibility, Free Will, and Their Discontents}
The core operating assumption of moral responsibility is that there is a meaningful difference between an event ``just happening'' and an action that someone is responsible for.  When we say a person is responsible for some event, we are picking them out as having a special relationship to an action. Whatever the relationship is, it must be such that it sustains the social and individual practices and feelings connected to moral responsibility; actions like punishing or rewarding, scolding or praising, and feelings such as pride, shame, resentment, gratitude, etc. Most of these can be subsumed under the categories of blame and praise, and I will follow Bennett \citeyearpar{bennett1980} in calling accountability. So responsibility is whatever it is about people in certain situations that makes it permissible or even obligatory for us to blame or praise them for particular actions (or even failures to act), and which is absent in cases where we think it is impermissible or inappropriate to blame or praise people.

The most familiar way of expressing that ``special something'' is that people have free will, and people are responsible for things that they have done ``of their own free will.'' Skeptical worries about moral responsibility have generally proceeded by attacking the supposition that we have free will. This problem has been expressed in many ways, and indeed as somewhat different kinds of problems throughout history. Generally, the core of the worry about moral responsibility has been based on something like the following bit of reasoning, which I will call the core skeptical argument. 

\begin{description}
\item[(1)] To hold persons morally responsible, they must have free will.
\item[(2)] x is incompatible with anyone having free will.
\item[(3)] x is true
\item[(c)] therefore we cannot hold persons morally responsible.
\end{description}

Throughout the history of the debate about moral responsibility, the x that threatens free will has shifted, as has the reasoning behind it. In the Western tradition, early worries that we might not really have free will were grounded in a particular theological idea - that God has perfect foreknowledge of all that will happen. God's perfect foreknowledge of what we would do seemed to threaten the sense that the things that people choose to do are genuinely choices, and it was worried that this means that people do not really have free will. Some, however, argued that free will was still possible in the light of theological fatalism, and proposed a number of different approaches to the problem. Some, such as Boethius \citeyearpar{boethius2008} proposed that God is outside of time, so God's omniscience is not really foreknowledge at all. The Augustinian response to the problem affirms God's foreknowledge in time alongside freedom of the will, arguing that God's foreknowledge is of human wills as causes \citep{augustine2010}. While debates about problems of divine foreknowledge are still ongoing, particularly in theological circles, the problem of moral responsibility has since taken other forms. 
Determinism presents the second major iteration of the problem. The thesis of determinism is articulated alongside the rise of Newtonian science, which sees physical processes being governed by fixed laws of nature. Determinism claims that any event is the inevitable outcome of prior conditions and the laws of nature, following a chain of causation that extends from the beginning of time. If determinism is true, then anything a person seems to choose to do was determined in the initial stages of the universe, and everything in the universe is simply playing out the results of those initial conditions according to the laws of nature. Much like the problem of divine foreknowledge, the problem of causal determinism threatens free will by rendering moments of choice illusory in some way, since the outcome of a choice is the inevitable result of the antecedent determining factors. 

There are two primary issues in the free will debate as related to determinism. The first issue is whether determinism is itself true - challenging or defending premise (3) in the above argument. The second issue is whether free will or moral responsibility are possible if determinism is true, which challenge (1) or (2). The interplay between these two issues stakes out a number of possible positions in the debate. There are two varieties of incompatibilism which both hold that if determinism is true, then moral responsibility is impossible/free will does not exist. The libertarian believes that determinism rules out free will/moral responsibility, but that determinism is false, and there are two main variations among libertarians. Event causal libertarians claim that there are nondeterministic events which leave room for human beings to exercise their will, and thus really be morally responsible for their decisions \citep{kane1999, balaguer2004}. Agent causal libertarians believe that human beings are special kinds of causes - a solution that interestingly parallels Augustine's solution to the problem of divine fatalism \citep{oconnor2005, clarke1993}. Hard incompatibilists agree with libertarians that determinism and free will are not compatible, and believe that determinism is true. \citep{strawsong2010, pereboom2001}. Compatibilists, on the other hand, need not take any position on whether determinism is true, because they argue that determinism poses no threat to moral responsibility or free will, see especially: \citep{strawsonp1974, watson1986, bok1998, smart1961, frankfurt1969, frankfurt1971, fischerravizza1998}.

If determinism poses a threat to moral responsibility, it seems to do so by rendering impossible a ``garden of forking paths'' view of freedom. This amounts to the demand that a free choice involves ``metaphysically open possibilities'' - such that if I am choosing between a or b, it is metaphysically possible that the universe may be such that I choose a, or that I choose b. If determinism is true, then it is simply not the case that either of those outcomes may happen; one of those outcomes is inevitable, given initial conditions and the laws of nature. Whatever I choose was inevitably going to happen, though this does not deny the fact that my choice was a part of the outcome. Many compatibilists have focused on this point, denying the necessity of metaphysically alternate possibilities, with a particular influential series of challenges starting with Harry Frankfurt \citeyearpar{frankfurt1969, frankfurt1971}. This strategy attempts to show that alternate possibilities are not actually required for moral responsibility. Frankfurt introduced a case in which we still seem to be willing to make moral judgments of someone even with knowledge that they could not have done other than they did. Many papers have generated similar cases attempting to challenge and refine the situation to show whether we really are or are not abandoning the principle of alternate possibilities \citep{pereboom2001, kane1998, mele1995, fischer1986, inwagen1986}

This debate has helped to develop a problem that is further reaching than determinism, however. Criticisms of event causal libertarianism \citep{oconnor2005, clarke1993}, as well as further pressing of the issue by others \citep{pereboom2001, strawsong2010} have shown that the problem may be deeper than determinism. Some such as Galen Strawson, have argued that the falsity of determinism isn't capable of rescuing moral responsibility. The alternative to determinism seems to be indeterminism, which claims that natural processes also contain randomness - such that the state of the universe is partially decided by random but probabilistic events. Thus it is not the case that the outcome of the universe was entirely decided in its very beginning. However, it is also unclear how the introduction of randomness is supposed to help protect freedom of the will and moral responsibility.

This further development of the thesis of determinism has been called the ``Causal Thesis'' by T. M. Scanlon, who articulates it as follows:
\begin{quote}
[...] the events which are human actions, thoughts, and decisions are linked to antecedent events by causal laws as deterministic as those governing other goings-on in the universe. According to this thesis, given antecedent conditions and the laws of nature, the occurrence of an act of a specific kind follows, either with certainty or with a certain degree of probability, the indeterminacy being due to chance factors of the sort involved in other natural processes. \citep[p.152]{scanlon1988} 
\end{quote}

This formulation problematizes both the thesis of determinism, as well as probabilistic randomness, which covers a wide array of ways that science might end up telling us the universe is really like. 

This extends past the problem of the ``garden of forking paths'' since a universe with some randomly deterministic events would indeed involve multiple possible outcomes, yet if the Causal Thesis is incompatible with moral responsibility, then something more is needed than alternate possibilities. One feature taken to be essential for moral responsibility that the Causal Thesis undermines is the possibility that any person can be understood as the source of their actions. The Causal Thesis traces the causes of a person's actions beyond themselves, and this is sometimes taken to be the deeper problem for moral responsibility. For arguments forwarding this position, see especially \citep{strawsong2010, pereboom2001, oconnor2005}.

When considering the Causal Thesis the shape of the debate shifts slightly, because the event causal libertarian becomes a kind of compatibilist - claiming that moral responsibility is consistent with the Causal Thesis, and attempting to show how indeterministic events represent a choice that can be attributed to the agent \citep{kane1999, balaguer2004}. Agent Causal libertarianism is the only position available that maintains that free will exists, and it maintains this position by denying the Causal Thesis. Compatibilists who are primarily concerned with the Causal Thesis must shift their strategy to establishing how a person can be the source of their actions. The principle of alternate possibilities may or may not remain as a pressing concern. Some argue that if a person could be shown to be the source of their actions, then there is no need for alternate possibilities at all \citep{pereboom2001}. Others maintain that source worries and alternate possibility worries both need to be addressed. 

The debate at present does not have especially clear or sharp lines, given that there are often slightly different worries really being considered. Some philosophers have shifted focus to the Causal Thesis \citep{strawsong2010, wallace1994}. There are also subtle but important differences in the arguments, depending on whether free will, moral responsibility, or our practices of moral responsibility are taken to be the target of the challenge. It is possible to believe that free will is possible even if the causal thesis is true - that the libertarian definition of free will is mistaken or false. It is possible to believe that free will is impossible, but moral responsibility is fine. Such a position accepts the libertarian definition of free will, while denying that this is necessary for moral responsibility. It is also possible to believe that moral responsibility is impossible, but that our practices of moral responsibility are merely conventional to begin with, and thus do not depend on whether or not anyone is actually morally responsible. The lines between positions, and thus the success conditions for an argument can shift from debate to debate, and the debate can just as easily be about where the lines are drawn as what the status of moral responsibility actually is.

There are a few important things to notice about the overall shape of the debate about free will and moral responsibility. Perhaps the most important is the fact that there is, and has always been disagreement about whether the complicating factor (God's omniscience, determinism, or the causal thesis) actually threatens moral responsibility (or free will) at all. Part of the reason for this is that the nature of what these complicating factors are threatening is not clearly defined or even widely agreed upon. It is not as if free will is or has been well defined and clearly understood, and then some new insight or perspective into the running of the universe has come to be accepted and we have then understood that our previous understanding of free will cannot exist alongside this new insight. We seem to feel the threat before we understand what it is that is supposed to be threatened, and it sometimes seems as if the accounts of free will we come up with are attempts to explain or explain away the threats we feel. If this suspicion is correct, then theories of free will are ways of explaining why we feel that some complicating factor threatens moral responsibility.

Whether or not this suspicion is true, the absence of any widely agreed upon understanding of free will has opened up another front in the debate about moral responsibility - whose ``version'' of free will is the right one. Attempts to settle this question have varied from appeals to intuition and thought experiments, pragmatic approaches that settle the question by whichever version of free will works best, and even attempts at an empirical foundation in more recent X-phi work \citep{nahmias2004}.

These approaches all proceed as if there is a tacit concept of ``free will'' operating in the background of our thinking that needs to somehow be made explicit, and from there, we can figure out whether the complicating factor is really in conflict with it. This assumes some fairly sophisticated metaphysical thinking operating in the subconscious of not only philosophers, but those untrained in philosophy. This approach also starts with an assumption about what the answer is going to look like - that it will involve finding some fundamental intuitive understanding of freedom or a free will, and from this model, we judge whether or not particular cases qualify as free and morally responsible. Essentially, this means that we think people are morally responsible because we believe they fit some model of a morally responsible thing, and the complicating factors threaten moral responsibility because they conflict with some feature or capacity that the tacit model of a ``morally responsible thing'' is supposed to have. I believe that this approach is wrong, and the problems with this approach are responsible for the intractable nature of the debate around moral responsibility. 

The model of moral responsibility I will be arguing for is grounded in a Strawsonian reactive attitudes approach,which starts from our feelings toward one another and ourselves. However, the moral structures, guidelines, and expectations around responsibility do not generate judgments of responsibility. Instead, these judgments curtail many attributions of responsibility we might otherwise make, and significantly transform the ones that remain, making them into judgments of moral responsibility. If this is correct, then the question is not really ``why do we sometimes hold people morally responsible?'' but rather ``why don't we hold people morally responsible sometimes?'' Ultimately, I will argue, the answer to this is intimately connected to our being persons, a status which makes possible the moral categories of responsibilities and rights. 

However, if we remove the assumption that we refer to some tacit understanding of free will in order to decide whether or not we believe a person is morally responsible, then we will first have to propose an alternate idea of what people are doing when they make these intuitive moral judgments. As was suggested above, this will involve getting a clearer idea of what relationship between persons and their actions must obtain if they are to be held morally responsible, and then understanding how we come to judge when that relationship obtains, or when it is interrupted. But before doing any of that, I will first clarify what I think makes something count as an acceptable answer to these questions in the first place. 

\section{Methodology: The Role of Intuitions}
Debates in moral responsibility can very easily become mired in conflicting intuitions - particularly concerning what is supposed to count as moral responsibility. The challenge in giving an account of moral responsibility is that the account given is usually supposed to be about moral responsibility as it exists and is practiced in the everyday world. However, because judgments of moral responsibility often appear through behaviors of holding responsible or not rather than as explicitly stated principles of responsibility, this becomes difficult. The explanations for moral responsibility that are usually generated are generally tested against typical or problem cases of the kinds of moral judgments we typically make. 

There are a number of problems with this approach, however. People can and often do have conflicting judgments about whether a person is (or should be held) responsible, especially in problem cases that rely on the ambiguity of the line between brainwashing or upbringing, for example. See especially \citep[p.110-7]{pereboom2001}. It is also not clear how to integrate the possibility that someone is judging a case of moral responsibility incorrectly - if a person judges differently than the theory in question, is that a problem with the judgment or is it a case that shows a problem with the theory? Some \citep{nahmias2004} have attempted to settle this by appealing to surveys that track what folk intuition about moral responsibility actually is, though it is not immediately clear why a majority, even an overwhelming majority in a survey of folk intuitions about moral responsibility should decide anything at all about what is right about moral responsibility.  

Alternately, some attempt to argue for a particular understanding of moral responsibility because the model in question works. On a pragmatic approach like this, the question we ought to be asking is whether we can hold people morally responsible without running into major inconsistencies or incoherences. Pragmatic approaches are often accused of being revisionist about moral responsibility - the claim is that they have departed from any attempt to understand moral responsibility as it currently exists and operates in human life, and instead are prescribing an alternate system of how we ought to hold people morally responsible. Some \citep{vargas2009, fischerravizza1998} openly embrace that approach, believing that the moral responsibility that people are operating with on a day-to-day basis does not really work, but that something very much like it would work. 

The approach I will take lands between the intuitive and pragmatic approaches, and might be called semi-revisionist. On this approach, the key deciding factor in favor or against a model of moral responsibility will be whether it ``works''  - it is able to make the sorts of distinctions in a consistent and coherent manner that it is supposed to be able to make. However, intuitions will still have a role in deciding what kind of work moral responsibility is supposed to be able to do. On this approach, there is a principled distinction between acceptable and unacceptable revisionism.

As an example, J.C.C. Smart's \citep{smart1961} account of moral responsibility takes moral responsibility as a kind of social control whose deciding principles are the utility of its related behaviors (punishment and reward) in building and sustaining a well ordered society. Smart's model conflicts with a number of intuitions and some of these intuitions are connected to the idea that what moral responsibility is supposed to do is drastically revised under Smart's description. Importantly, the emotional component of praise and blame are removed, and some have found those elements to be essential to what moral responsibility does, making Smart's account revisionary in the negative sense. However, others might say that Smart's account runs contrary to intuitions that moral responsibility requires a metaphysically free will. But this intuition is not about what moral responsibility is supposed to do, but rather what is required for a correct judgment of moral responsibility, and as such is an intuition about how moral responsibility works. Moral intuitions are not playing the role of inborn ``theoretical'' knowledge of various facts about moral responsibility. They are knowledge of the practical aims of morality, and moral responsibility. These sorts of claims are foundational simply because they are about what we are trying to accomplish, which is something we (in some sense) just decide, rather than discover.

Once we decide (or figure out) what we are trying to do, there is a further question of whether the task is possible, and whether the way we have chosen to go about doing it is actually successful. In this stage of the investigation, appeals to intuition will have a much more limited scope, particular in disputing how some practical aim of moral responsibility is actually accomplished. That we are metaphysically free agents may be a way of accomplishing everything that moral responsibility is supposed to do, but there may be other ways of accomplishing everything that moral responsibility is supposed to do, and that one or the other doesn't seem right, intuitively, will not, on its own be offered or accepted as a meaningful objection to an account. However, it is also important not to simply ignore these claims about friction with existing intuitions - I suspect many ``conflicting intuitions'' about whether compatibilist accounts are acceptable as accounts of moral responsibility are actually related to expectations about what moral responsibility is supposed to be able to do that aren't always clearly articulated. Overall, this methodology takes what Daniel Dennett \citeyearpar{dennett1987} has called the ``design stance'' and applies it toward moral responsibility.

In the next section, I will take a first pass at articulating what expectations concerning practical aims have attached to moral responsibility. As mentioned, I suspect that much of the resistance that pragmatic accounts of moral responsibility have encountered has been grounded in the fact that these accounts leave out things that moral responsibility is supposed to do.  These practical ends are seldom articulated by either side. Part of the challenge of the ongoing debate about moral responsibility is that these concerns often operate in the background - driving intuitions without themselves being exposed. I will try to make as clear as possible the things that I believe moral responsibility is actually supposed to be accomplishing.

\section{What Is Moral Responsibility Trying to Accomplish?}
The essential relationship between persons and actions for moral responsibility is that the action the person is held responsible for is theirs. Intuition pumping arguments that undermine moral responsibility all undermine the judgment that a person owns their actions, and intuition pumping arguments that restore belief in moral responsibility do so by restoring a sense that a person owns their actions. I think that this is the common target behind a number of the main debates in the literature on moral responsibility. 

The consequence argument is perhaps one of the most important incompatibilist criticisms of moral responsibility, and can help us to more clearly articulate what expectations are connected to moral responsibility. This argument created great difficulty for the classical compatibilist strategy, a long standing compatibilist defense of moral responsibility which said that a person was free, and morally responsible in any case that they were able to do what they willed - only interference and coercion mitigated moral responsibility, because the truth of determinism did not interfere with a person's actions in that morally relevant sort of way. The consequence argument has the following basic form (adapted from Van Inwagen \citeyearpar{inwagen1978}):

\begin{quote}
The state of the world at time t plus the laws of nature entail any state of the world after t. 

Peter raises his hand at t1

Peter raising his hand at t1 is entailed by t+ the laws of nature

For Peter to refrain from raising his hand at t1, he would have to either change the state of the world at t, or change the laws of nature

No one (including Peter) can change the state of the world in the past, nor can anyone change the laws of nature

Therefore, Peter could not refrain from raising his hand at t
\end{quote}

This argument shows that if determinism is true, then a person doing anything other than what they actually do or did is impossible. This served to break down the distinction made by classical compatibilists between coerced/forced acts, and what were regarded to be free acts, by showing that people were just as ``forced'' into these so-called free acts as they were to unfree acts. The consequence argument is meant to show that for anything we do that we might be held responsible for, we couldn't have done other than we did, an expectation that has become known as the principle of alternate possibilities (PAP). Compatibilists initially responded by offering a differing analysis of what Peter ``can'' do called the Conditional Analysis, which stated that Peter is free to refrain from raising his hand such that if Peter's motivations or reasoning gone differently, he could have refrained from raising his hand. This strategy, however, confronted some challenges - one of the most critical being the demonstration where the conditional analysis holds, yet it would be false to say that the agent could do otherwise.

However, the PAP has also been challenged, initially by Harry Frankfurt \citeyearpar{frankfurt1969}. His challenge was followed by a new wing of literature on moral responsibility responding to, and offering variations of, his challenge. Frankfurt attempted to show that people would still make moral judgments in cases where it was made clear to them that the person doing wrong actually wouldn't have been able to do otherwise, such as in the ``Sam the assassin'' case given by Fischer and Ravizza \citep{fischerravizza1998}. These Frankfurt-style cases have in common the idea that a person ``wants to be moved by the desire that actually moves them'', and these cases have seen substantial development. In Fischer \& Ravizza's case we are told about Sam, who plans to assassinate the Mayor. His friend, Jack, also wishes to see the Mayor assassinated, and Jack has implanted a control mechanism in Sam that will force Sam to go through with the assassination plan if he shows any sign of not doing so. However, Sam shows no such sign and actually shoots the Mayor - there is no cause for intervention from Jack's device. In cases like these, most people seem to think that Sam is morally responsible for killing the mayor, even though he could not have done otherwise (thanks to the presence of the unactivated counterfactual intervener). 

There has been a good deal of debate over these cases, attempting to show that even in these cases, there is a brief instant of potential freedom where Sam might not choose to assassinate the mayor, triggering the intervener, and that moment of potential freedom shows that there really is, in some sense, an alternate possibility available. We might think of this debate as a debate over whether the attitudes around moral responsibility really do depend on the claim that person could have done other than they did at all, and again it is telling that the cases really generate intuitions about whether a person is still responsible, and offers varying explanations for those intuitions. 

The Consequence Argument and the Frankfurt-style counterexamples are clearly both concerned with the nature of the relationship between a person and the actions we hold them responsible for, but each proposes a different characterization of the responsibility-binding relationship. In the consequence argument, the responsible agent must have been able to do other than they did, whereas in the Frankfurt-style counterexamples, the proposed principle is simply that a person wants to be moved by the desire that actually moves them. However, both arguments seem to work by calling our attention to whether or not the person held responsible ``owns'' their actions. In the consequence argument, the sense of ownership is undermined because we come to see that the person's relationship to their action was due to factors that have nothing to do with that person (namely, the laws of nature and the state of the world at t). The Frankfurt-style arguments offer a different sense in which a person might be said to ``own'' their action - because they wanted to be moved by the desire that actually caused them to act. Setting aside which, if either of these accounts gives the understanding of ownership that is actually supposed to sustain moral responsibility (if indeed, any ``account of ownership'' plays that role at all) we can see that both of these arguments are impacting a person's sense of the ownership of an action. 

Importantly, the question of a person's ownership of their action comes up in debates outside the question of the principle of alternate possibilities. Derk Pereboom \citeyearpar{pereboom2001} calls into question the importance of the PAP, and Tim O'Connor sees alternate possibilities as a ``merely contingent'' feature of freedom, and that freedom itself is not to be identified with the existence of the alternatives. \citep{oconnor2005}. Both Pereboom and O'Connor suggest that what matters for moral responsibility has something to do with the connection between a person and the act they are being held responsible for. 

We see this connection explained by both agent causal libertarians and hard incompatibilists. Randolph Clarke's agent causal account of free will attempts to address two problems typically brought against agent causal accounts. The first is the objection that it is not clear how agent causation can be consistent with an agent's actions being caused by her reasons for action, while the second is the unintelligibility of agents as causes. Clarke's strategy is to argue that causation is basic and irreducible in the case of event causation (so there is no reductive account that explains why a causes b), and equally so in the case of agent causation - agents are just special relata in the causation relation. This agent causes action based on different alternative sets of reasons (an agent chooses to act on reasons A\&B, rather than reasons C\&D, etc). \citep{clarke1993} But what motivates this particular structure of agency? Clarke describes the necessary condition for genuinely free will as follows:

\begin{quote}
When an agent acts with free will, her action is causally brought about by something that (a)  is not itself causally brought about by anything over which she has no control, and that (b) is related to her in such a way that, in virtue of its causing her action, she determines which action she performs \citep{clarke1993}.
\end{quote}

Tim O'Connor argues for a similar priority, claiming that the crucial aspect of free will is the fact that ``I am a self-determining agent'' \citep[p.210]{oconnor2005}. Derk Pereboom and Galen Strawson, both hard incompatibilists seem to agree that what would be required for moral responsibility tracks closely onto what Clarke and O'Connor insist on. Pereboom gives the following account:

\begin{quote}
If an agent is morally responsible for her deciding to perform an action, then the production of this decision must be something over which the agent has control, and an agent is not morally responsible for the decision if it is produced by a source over which she has no control. \citep{pereboom2001}.
\end{quote}

To support this account, Pereboom gives his four-stage argument, where we are given four scenarios where a person was ``manipulated'' in various ways to commit a murder. In the first, there are neuroscientists who have created a person, Plum, whose thoughts, beliefs, and intentions can be monitored and manipulated by them from moment to moment. The kinds of thoughts are fairly complex, and include second order desires, rationally egoistic reasoning about those desires. He is reasons responsive, and sometimes acts in less rationally egoistic ways. Plum seems to fulfill the qualifications for numerous compatibilist versions of free will, and yet, given that all of these mental states are manipulated by neuroscientists, it does not seem that Plum is responsible. If we argue that the constant intervention is why Plum is not responsible, the second case is modified so that these dispositions are all set up by neuroscientists from the beginning, so that through initial programming, the same outcomes (a murder) result. The third case switches the neuroscientists with rigorous cultural training to program him to be as he is and thus commit the murder, and in the final case the rigorous cultural programming is replaced by an ordinary person who commits the murder as an outcome of determinism. \citep[p.112-6]{pereboom2001}. 

The argument attempts to show that the distinctions that the compatibilist uses to maintain that moral responsibility does not apply in the first case, but still does apply in the fourth are insufficient - according to Pereboom, the reasons why Plum cannot be morally responsible do not change from the first to the fourth case, and the four cases just make this apparent that the same reasons have always applied. Plum is not in control of his actions, in the sense that the decision to perform the action, in all cases, is an ``alien-deterministic event.'' Pereboom's expectation for a case of genuine moral responsibility is that an agent can meaningfully say of an action that it ``belongs'' to them. Pereboom's four stage argument is designed to show that the distinctions that compatibilists try to make between cases of coercion and control - where an agent is alienated from their actions, and ordinary actions, where an agent is thought to have an important relationship to their actions - is illusory. 

Galen Strawson sees a similar problem. In his basic argument (for the impossibility of moral responsibility), he says that what we do is a result of who we are, and that true moral responsibility for what we do entails true moral responsibility for who we are. Strawson gives us a particularly vivid account of ``true moral responsibility'' such that, if we have it, eternal punishment in hell and bliss in heaven make sense (even if one doesn't believe in either). His point is that this indicates a kind of ultimate and absolute accountability for our actions. \citep{strawsong2010}.

In all four of these cases, we have an insistence on the condition of ``ultimacy'' or ``sourcehood'' all framed in slightly different language, and including similar rationales. Each of these descriptions attempts to pick out a way that the action needs to ``belong'' to the agent - by the agent's being an originating cause, and not determined to do what they did by something alien to themselves. Much like the debate concerning the PAP, we can see that the core concern about ultimacy is related to whether or not a person can be said to own their actions. In the PAP, the presence of alternate possibilities signal where the agent was an independent cause, and thus, what actions really did or did not belong to the agent, where as the ultimacy condition picks a person out as owning their actions based on the idea that they alone caused them. 

If the debate over moral responsibility operates by shifting intuitions about whether or not a person owns their actions, then the possibility of moral responsibility will really depend on whether or not a person can rightly be said to own any of their actions at all. One of the challenges that is immediately clear is that intuitions we have seen thus far have all depended on a different ``mechanism'' behind the ownership relationship. In order to understand whether a person really owns their actions, we could try appealing to intuitions about ownership. However, those intuitions, I propose, are actually based on ideas about what the ownership relationship is supposed to do. If this is correct, then understanding moral responsibility will require understanding how an attribution of ownership between a person and an action is supposed to work - or, in another way of putting it, we need to understand what moral responsibility is supposed to be doing. 

\section{What Responsibility is Supposed to Do}
In order to get a starting idea it will be helpful to take a closer look at an account of moral responsibility that has a starkly pragmatic orientation and see what, if anything, is missing from that account and build from there. J.J.C. Smart \citeyearpar{smart1961} offers a compatibilist account of moral responsibility that to some extent explicitly examines what an account of moral responsibility is trying to do, and then sets aside some of the aims as empty and impossible, while focusing on the fact that other aims are still possible. 

Smart offers an argument refuting the possibility of ``metaphysically'' free will and embraces a particularly vivid picture of determinism. In his discussion of moral responsibility in light of this picture, Smart argues that free will as described by libertarians is utterly irrelevant. For Smart, what is important is whether holding someone morally responsible is or is not the sort of thing that can weigh on behavior in the proper way. In his example of a schoolboy who has not done his homework, Smart distinguishes between a case where stupidity prevents the boy from doing his homework, and laziness prevents the boy from doing his homework. Smart says it would be wrong to hold the boy responsible (and punish him) for not doing his homework to correct ``stupidity.'' Punishment doesn't change that feature of a person at all. However, if the boy did not do his homework out of laziness, then punishing him may change his behavior by introducing an element into the environment that would figure into the decision making process - Tommy might do his homework, despite his laziness, if the possibility of punishment also weighed on his deliberations. Punishment has a clear practical justification.

Praise and blame, on the other hand, are ``more difficult'' as the features that we might praise and blame end up being, on balance, no different than things like physical beauty, height, or hair color. However, these moral properties are typically treated differently - they are praised and blamed, while non-moral properties are subject to praise or dispraise; the latter has more in common with grading, while ordinary moral evaluation is, according to Smart ``grading plus an ascription of responsibility.'' This responsibility must be understood along the same lines as the lazy student's form of responsibility, and not robust metaphysical responsibility. That is to say that when we blame a student for not doing their homework from laziness, we are simply applying dispraise to the student, coupled with the expectation that they are to do something about it (perhaps on pain of punishment). 

According to Smart, this ``clear and dispassionate'' stance toward praise and blame is clearly different than how most people usually consider it. Smart writes:

\begin{quote}
Nevertheless, we can see that a rather Pharisaical attitude to sinners and an almost equally unhealthy attitude to saints is bound up with this metaphysics in the thinking of the ordinary man... The upshot of this discussion is that we should be quite ready to grade a person for his moral qualities as for his nonmoral qualities, but we should stop judging him.'' \citep[p.00]{smart1961}. 
\end{quote}

Smart seems to agree in some measure with the philosophers he is arguing against that the falsity of the metaphysical version of free will entails a particular change of attitude toward moral responsibility. Smart's justification for moral responsibility is fully grounded on the practical value of the practices around moral responsibility: holding responsible, praise, and blame. Thus, Smart is explicitly endorsing one of the ends of moral responsibility, responsibility as way of bringing about and reinforcing behavior that serves society well. Smart takes it as fairly obvious that the truth of determinism has no bearing on our ability to reinforce or punish certain behaviors. Furthermore, we are still able to make a clear distinction between cases where we think people are responsible as opposed to those where they are not on Smart's account. This is decided based on whether holding a person responsible will tend to make them behave better or not. 

If this was all that moral responsibility is supposed to do, then we would have no real reason to be dissatisfied with this account. However, Smart's account is clearly missing a number of important elements that become clear when we investigate some of the elements of moral responsibility that Smart sets aside. We will see that there are important elements that we usually expect moral responsibility to be responsive to which Smart's account clearly is not,  and correspondingly we see that the way in which a person is said to ``own'' their actions is fairly thin. 

One prominent feature of moral responsibility that Smart sets aside is the notion of desert. Desert is the idea that certain states are ``appropriate'' for people given considerations of who they are or what they did. Desert is typically considered important in cases of punishment, because punishment involves the deliberate imposition of unpleasantness upon a person. If a person deserves punishment, then it is judged to be appropriate that they are being punished, despite the unpleasantness involved. Smart's defense shows that punishment is not only permissible but obviously necessary from a social perspective, as this practice changes the social landscape in which we make decisions for the better. He does not seem to be troubled by the fact that if a person isn't really responsible in the metaphysical sense. While Smart's account does consider the appropriateness of punishment, this consideration rests on the social efficacy of punishment, rather than considerations of who that person is and what they have done. For Smart, there is no sense in which a person could deserve something, since this entails the kind of judgment of a person that Smart's position requires that we cease engaging in. Furthermore, the boy who doesn't do his homework ``owns'' his actions only in the sense that punishing or rewarding him seems to be the best way of getting that behavior to change. On Smart's account, we all end up as beings merely ``situated near the right (or wrong) levers'' of a large moral machine. \citep{Williams1973}. 

If this is correct, then one of the important missing elements in Smart's account is desert, and for someone to really deserve something, it is not sufficient that they own their actions in virtue of being situated near particular causal levers. We can see concern with desert expressed by Galen Strawson in his ``heaven and hell'' standard for moral responsibility, which will help to clarify what is involved:

\begin{quote}
As I understand it, true moral responsibility is responsibility of such a kind that, if we have it, then it makes sense, at least, to suppose that it could be just to punish some of us with (eternal) torment and hell and reward others with (eternal) bliss in heaven'' \citep[p.216]{strawsong2010}
\end{quote}

This standard is important for a number of reasons, including the fact that it is firmly embedded in the history of the debate around moral responsibility. There are also reasons to question this standard - not the least of which is that there may be no conditions, even the most metaphysically extravagant version of agent causation - in which such a punishment/reward system can be understood as just. However, Strawson's claim here suggests that one essential element of desert is that both attributing an action to, and subsequently praising or blaming a person for some action must be just or fair. If, on Smart's account, we punish or praise the boy because doing so is efficacious we show a lack of concern about whether it is fair to punish or praise the boy, What is it that makes it right for the buck to stop with him, and for him to bear the burden of punishment for the greater good? 

There is also a second issue with Smart's account, related to the first but more difficult to describe. It can be best captured by the notion that our attitude toward moral responsibility is itself a part of moral responsibility - what Smart describes as the ``pharisiacal attitude toward sinners and [...] almost equally unhealthy attitude toward saints'' is somehow a part of the aims of moral responsibility itself. Smart's recommendation that we jettison these attitudes has seemed counterintuitive to a number of philosophers. Indeed, P. F. Strawson \citep{strawsonp1974} argued that emotions and attitudes, the very things that Smart would have us discard, are the core of moral responsibility. According to Strawson, moral responsibility is grounded in emotion, particularly emotional responses around expectations for interpersonal regard, which Strawson calls Reactive Attitudes. For Strawson, the judgmental response that Smart sees as extraneous to moral responsibility is actually its core.  Strawson attempts to deflate the worries of ``pessimists'' who think that moral responsibility is not possible if determinism is true. According to Strawson, we have a grounded commitment to the reactive attitudes which result in our behaviors around moral responsibility. The distinctions we make are coherent, and inevitable, according to Strawson.

If Strawson is right, then the emotional core of moral responsibility would, on the approach I am taking here, have to be related to something that moral responsibility is supposed to do. For Strawson, we have to feel a certain way about moral responsibility, and I will argue later that the importance of feeling is actually related to two distinct things that moral responsibility is supposed to do. In the first place, moral responsibility needs to be responsive to our lived experience of deliberation and choice - it must in some sense be a response to the feeling that we choose what we do, a criteria that I will later describe as phenomenological adequacy. The feeling that we choose what we do also maintains the sense that we own our actions.

Moral responsibility must also be able to sustain the value that we place on our lives as ethical beings (or persons), a criteria I will later explain as the existential weight of moral responsibility. Thomas Nagel \citeyearpar{nagel1979b} has rightly pointed out that moral judgments are supposed to take a certain kind of object - a person - which is supposed to be in some way distinct from a mere thing, and human actions are supposed to be in some way distinct from events. The expectation that moral responsibility requires something like agent causation reflects that moral judgment requires a special status for human beings that not only provides a distinction between the judgment-apt and not, but a meaningful one. The distinction becomes meaningful because in making the distinction, it affirms the central location of the person as a person in moral life. Smart's instrumental treatment of reward and punishment are not the sort of thing that can sustain the emotional gravity that is usually attributed to moral responsibility. Holding persons morally responsible is an affirmation of a special status for human beings. We can start to see that part of why theological fatalism, determinism, and the Causal Thesis have been a threat is because they each challenge something about the special status of human beings.

On this analysis, an account of moral responsibility that ``works'' will have to be fair, phenomenologically adequate, and able to bear existential weight. Many existing compatibilist accounts tend to focus on only the first of these criteria, with typically only indirect attention to the second. Very few address the existential weight criteria, and I suspect that a good bit of the skeptical resistance to compatibilist accounts can be attributed to the fact that most fail to address this existential criteria. The picture I eventually hope to explain and defend is that moral responsibility helps give us a form of teleological unity as persons, a unity whose prominent role in human life extends far back into human history, and forms of which have been explained and defended by philosophers as foundational as Plato and Kant, and which has found contemporary expression in the work of Taylor \citeyearpar{taylor1989}, Korsgaard \citeyearpar{korsgaard1996, korsgaard2009} and Frankfurt \citeyearpar{frankfurt1971}. I hope to extend and modify these accounts to make clearer the role that moral responsibility has in providing that unity - part of the function of moral responsibility is to express and enact a commitment to being a person with this kind of unity.

\chapter{The Naturalistic Strategy}
In the previous chapter, I argued that a proper account of moral responsibility must be able to sustain the objectives that we have for it. Pragmatic defenses of moral responsibility, such as that given by J.J.C. Smart \citeyearpar{smart1961}, have focused primarily on whether the account is able to manage our attributions of blame, and thus justify the practices associated with that (punishment and reward). Incompatibilists find such defenses to be inadequate, and I have suggested that these objections of inadequacy must be grounded in claims about things that the account of moral responsibility fails to do. I will argue that pragmatic approaches like Smart's have been problematic, not because of their pragmatism, but because a full account of moral responsibility has been expected to do much more than simply justify things like praise and blame. The missing element, I will argue, has been the connection to the way moral responsibility gives us a form of identity, as well as supports the meaningfulness that many people expect from the moral life. Thus, a defense of moral responsibility against skepticism must show that moral responsibility can sustain both meaning and identity despite the challenges that come from the Causal Thesis.
 
In order to make this defense, I will first develop an account of moral responsibility that I believe will pass this standard, and then show how it does so. In this chapter and next, I will develop a picture of moral responsibility as the application of abstract principles to moderate and control fundamental emotional responses. The guiding concept behind the abstract principles is the idea of the human being as a full moral agent, which is both the source of the connection between moral responsibility and identity, as well as the deep expectation for meaning that motivates some of the objections to pragmatic accounts. Moral responsibility has at its very foundation an account of human beings which the Causal Thesis, Determinism, and divine foreknowledge have all been taken to threaten. In order to determine whether this threat is legitimate, we will need to understand what expectations of our humanity are really built into moral responsibility in the first place.

In this chapter, I will begin to develop a model of moral responsibility where holding someone morally responsible is an affective response subject to a regulative ideal, which both regulates when and where a response is appropriate, and alters the nature of the response itself, turning it into what Strawson has called a reactive attitude. However, I will also argue that a reactive attitudes-based account does not and cannot provide a full defense for moral responsibility, because it lacks the second crucial ``regulative'' element, and that it is through concerns from the ``regulative'' side of moral responsibility that a general skepticism about moral responsibility can take hold. I will take P. F. Strawson's reactive attitudes account as my starting point, and argue that skepticism about moral responsibility must really be distinguished into two distinct concerns - concerns about why we hold people morally responsible, and concerns about why we cannot. Strawson's naturalistic strategy is an effective defense against the first concern, which I will call motivational skepticism. Difficulties with the naturalistic strategy have emerged because it is entirely inadequate to deal with challenges of the second type, which I will call defeasibility skepticism. As a consequence, the incomplete defense developed by the naturalistic strategy does not defuse skeptical worries about moral responsibility. However, it does put important constraints on the shape of the debate which will be helpful in the account I am developing.
 
Locating our practices of moral responsibility in our natural emotional lives - particularly as that intersects with our social nature has been an influential approach for defending moral responsibility since P. F. Strawson's seminal paper ``Freedom and Resentment.'' \citep{strawsonp1974} Strawson attempts to show that moral responsibility is grounded in reactive attitudes, a vicarious extension of the emotional engagement we have concerning basic expectations we have concerning our treatment and regard by others. According to Strawson, we place a great deal of weight on what we take to be the attitudes and intentions that others have toward us, and when these are found to be amiss our attitudes and feelings toward that person are deeply impacted. \citep[p.5]{strawsonp1974} For example, we will feel resentful if a person shoves us into a puddle because doing so indicates a lack of due regard for us. On the other hand, we would more likely be merely displeased with the situation should we find out that the person merely tripped and accidentally pushed us into the puddle. According to our Strawson, moral responsibility is grounded in vicarious analogs of these reactive attitudes (where we witness acts which reveal the high regard or disregard of one person for another).

These reactive attitudes can be suspended, either temporarily or permanently towards certain people if we find that they are not appropriate targets. For example, if we find that the person who shoved us into the puddle was hypnotized, drugged against their will, or mentally ill, we tend to adopt what Strawson calls the objective stance toward that person.\citep[p.9]{strawsonp1974} This stance excludes the offending person from participation in the moral community because we decide that we can or should no longer expect ``appropriate regard'' for them because we see them as having been made incapable of holding us in that regard. For the same reason, we no longer see it as appropriate to offer moral reasons to them, or to try to come to a common moral understanding with them. We cease to think of the person as a moral agent, and view them instead as a candidate for ``understanding, management, treatment and control.'' \citep[p.9]{strawsonp1974} As such, we change the way we interact with and think of the person, adopting an emotional distance and a more pragmatic and instrumental approach to dealing with them. 
 
With this understanding of moral responsibility, Strawson's defense against deterministic worries about moral responsibility employs two distinct strategies as identified by Paul Russell \citeyearpar{russell1992}. The first line of defense Russell identifies is called the naturalistic strategy, which claims that there is an error inherent in asking for a justification for the reactive attitudes, because they are simply a natural feature of human beings, and do not stand in need of justification any more than any other feature of human nature (that we have a heart, for example) stands in need of justification \citep[p.293]{russell1992}. Worries about a lack of justification for moral responsibility if determinism is true wrongly take moral responsibility to be justified only if determinism is false, when in reality, the question of justification cannot arise in either case because our reactive attitudes, as features of what we are as human beings do not stand in need of justification in the first place.

Russell calls the second line of defense the rationalistic strategy, which claims that there is something rationally incoherent about a demand to universally adopt the objective attitude based on excusing conditions, because it is contradictory to suppose that the abnormality by which someone is excused from moral responsibility is a universal condition. \citep[p.298]{russell1992}If determinism were true, then being determined just is the normal condition, and it no longer makes any sense to treat it as an excusing condition for being morally responsible, since, according to this argument, something gets to be an excusing condition because it is an exception from the norm. 

Russell argues that there are deep difficulties with both strategies, however. According to Russell. the naturalistic strategy contains an ambiguity between two types of naturalism. On the first approach, we might read Strawson as providing a plausible but very weak form of naturalism. In this form, Strawson deflates the demand for justification for the fact that human beings are the kinds of creatures that can respond with resentment, or any of the other reactive attitudes. The reactive attitudes are just part of our phenotype, and this is a fact that cannot stand in need of justification any more than the fact that we have two lungs stands in need of justification. This line of defense seems fairly obviously true.  However, it is important to note that this response is really only an objection to the argument that moral responsibility is wrong because we ought not be able to feel things like resentment at all, a form of doubt about moral responsibility that Russell calls ``type pessimism''. However, according to Russell, it is implausible that the pessimist is really saying that we ought not be able to feel the reactive attitudes. \citep[p.294]{russell1992} 

Much more plausible as a concern is what Russell calls ``token pessimism''. The token pessimist can accept that attitudes like resentment are part of our natural capacities, and yet worry that there is a skeptical argument that shows that it is never appropriate to entertain those reactive attitudes, and whenever we do, we are making some kind of mistake. This form of pessimist is not looking for a justification for our capacity for holding reactive attitudes. Instead, the token pessimist is worried that if determinism is true, then it will never be appropriate to resent, blame, or praise anyone, because we are all systematically morally incapacitated.  Strawson's type naturalist response does not defuse these worries because the issue is not about our emotional repertoire, which is really all that type naturalism can defend. Whether it is okay to engage those responses from our repertoire is an entirely separate issue.

Against this, we might read Strawson as arguing for a ``token naturalist'' response. A token naturalist version of the argument claims that because the reactive attitudes are natural features of human beings, we need not justify our general tendency to entertain tokens of those attitudes. Criticism of particular reactive attitudes is still possible under token naturalism, but it is limited to appeals to the consistent and correct application of the norms and conventions of the reactive attitudes themselves. An example of this sort of criticism would be the claim ``It is inappropriate to be angry at Bill because his knocking you into the puddle was an accident.'' This criticism is internal to the reactive attitudes because the principle it appeals to is part and parcel of the reactive attitudes themselves. We might imagine this principle being expressed as ``moral indignation is only appropriate at people who intended harm, or were reckless.'' Token naturalism would not make trouble for this sort of criticism, because the criticism in essence, only checks for consistency within the norms of the reactive attitudes themselves. 

Contrast this with another possible objection: ``It is inappropriate to be angry at Bill because he, like everyone else, is entirely morally incapacitated given the truth of the Causal Thesis.'' This kind of objection is blocked by token naturalism, because it is not questioning whether the rules of the reactive attitudes are being applied, but whether it is ever appropriate to apply the reactive attitudes at all, and questioning this is, for the token naturalist, something we cannot legitimately do because our entertaining tokens of the reactive attitudes is just part of who we are. In essence, the system of reactive attitudes are immune from ``external'' criticisms. \footnote{I adopt ``internal'' and ``external'' as ways of distinguishing between reasons given within the system of reactive attitudes themselves. Reasons that appeal to the rules of the correct application of the reactive attitudes given their acceptability are internal reasons - ``he didn't mean it'' is an appeal to withhold resentment from within the context of the reactive attitudes themselves, not calling into question the legitimacy of resentment itself. External criticisms of the reactive attitudes do not appeal to the structure of the reactive attitudes themselves, nor our capacity for reactive attitudes, but call into question our ever being moved by those structures. R. J. Wallace (1994) makes a similar distinction, though internal and external are indexed to moral norms.  See especially p. 105-117.} We can be criticized for not ``playing by the rules'' of the reactive attitudes (internal criticism), but no reactive response can be criticized by questioning the legitimacy of the substructure upon which they are founded. According to Russell, token naturalism is far too strong and ``calls on us to accept or reconcile ourselves to the reactive attitudes (and their associated retributive practices) even in circumstances where we have reason to repudiate them.'' \citep[p.297]{russell1992}

The rationalistic strategy fares no better. Against this element of Strawson's overall strategy, Russell claims that Strawson misconstrues our excusing people from moral responsibility as excusing them because they are abnormal. People are excused from the reactive attitudes because they are thought to be morally incapacitated, and there is simply nothing contradictory about everyone being morally incapacitated. The worry that Strawson needs to respond to is the worry that if determinism is true, everyone is, in fact morally incapacitated \citep[p.299]{russell1992}. An adequate response to this worry cannot be merely deflationary - it seems as if some positive account of moral capacity will be needed to show that people are not, in fact, morally incapacitated if determinism is true.

According to Russell, Strawson's project accomplishes less than he believes, though it still accomplishes something. Type naturalism may provide a framework from which a rationalistic defense of moral responsibility may be mounted, though the rationalistic strategy that Strawson himself employs is not adequate to the task. \citep[p.299]{russell1992}

I agree with Russell here, though I think that the problem with the naturalistic strategy is not with the defense itself, but rather with the scope of defense that Strawson thinks it is capable of. Although the naturalistic strategy does not provide us with a complete defense of moral responsibility, it nevertheless does something incredibly important. Properly situated, the naturalistic strategy undermines arguments that attempt to show that the Causal Thesis (or determinism, or divine foreknowledge) is a problem for moral responsibility by showing that the skeptic is not entitled to demand an explanation for why we initially hold people responsible - because the overall practice of holding people responsible is not itself motivated by reasons. However, we will see, this will not insulate moral responsibility from skeptical challenges. If this analysis is correct, then while determinism can still present a challenge for moral responsibility, it will be unable to do so along some of the more familiar paths of criticism. 

\section{The Naturalistic Strategy Reimagined}
What we have seen thus far shows the naturalistic strategy as both appealing and unappealing. Oddly enough, the appeal and the concerns connect to the same feature of the approach. It is appealing because it seems to provide a very firm foundation for moral responsibility. Demands for justification must come to an end when they are met by basic facts about human nature - it makes no sense to demand a justification for some aspect of our humanity that did not come about ``for a reason'' in the first place. This very same feature, however, also makes the strategy unappealing. There is something about moral responsibility which requires ongoing justification, whether it is a natural feature or not, in a way that other natural features simply do not require. If it turned out that our practice of attributing moral responsibility to agents was not susceptible to reason as a whole, that would be disturbing because it makes that practice seem arbitrary. If everyone were morally incapacitated, we really should stop responding to what people do with resentment or admiration, and Strawson's defense as it stands is not sensitive to this. This version of a reactive attitudes-based account of moral responsibility fails to do one of the things a successful account of moral responsibility needs to do. It must be responsive to reasons down to its foundations, since it is only by being responsive to reasons that it can respond properly to a discovery that, for example, we are all systematically morally incapacitated. 

Token naturalism is disturbing because it dismisses all external reasons in virtue of the fact that they are external to the reactive attitudes themselves. The only rational criticism of our reactive responses must take the system of reactive attitudes itself for granted. This doesn't seem to be an adequate response to a worry like ``all people are systematically morally incapacitated by the Causal Thesis.'' If we did find that people were all morally incapacitated, that is exactly the sort of fact we would want to have an impact on us and change our practices. We can see that Strawson's account makes some of the same mistake here that Smart's account makes. It does not take seriously the fact that moral responsibility is more than a kind of behavior management strategy, or a mere outburst of emotion. Moral responsibility is supposed to be rational and meaningful. If an account of moral responsibility is numb to an objection that we are systematically incapacitated for moral life, then it can no longer be said to take who we are seriously. 

It seems that the demands we have for the naturalistic strategy run in opposite directions. For it to do what it is supposed to do, it must make moral responsibility immune from demands for justification because it is naturally ``given'' while at the same time keeping it responsive to demands for justification. This apparent contradiction is suggestive, however - perhaps there are different kinds of demands for justification, some of which we want to deflate, and some of which are owed a response. Looking more closely at ways we give moral reasons can suggest a helpful distinction, and it may be clearest to start by considering epistemic reason-giving. 

When we attempt to justify ourselves epistemically, we give the reasons why we believe or act as we do. One way that we can do this is by giving our grounds - a set of positive reasons for believing as we do. These sorts of reasons provide a ground our belief. For example, I might believe that Tom is in North America because I just saw him, and I am in North America. I take seeing tom in North America to give me sufficient reason for believing that Tom is in North America. A reason in this ``grounding'' sense might be thought of as answering the question of why I believe anything at all about where Tom is, as opposed to having no particular belief about Tom's whereabouts.

However, sometimes our beliefs run into defeaters - reasons against believing what we believed before - suppose that we find out that Tom has an identical twin brother Ted, and Tom is actually supposed to be in Paris. This new information can be said to defeat my previous certainty that Tom was in North America by providing an alternate explanation for why we saw what we saw, and showing that the new explanation is more plausible. In order to maintain that Tom was in North America, we come to need to seek out additional reasons that can overcome the defeater (finding out that the trip was canceled at the last minute, and in fact the reason for Tom's trip was to visit his twin brother Ted in Paris). In contrast to grounding reasons which justify our taking a position at all, when we reason against defeaters, we are attempting to maintain a position despite what appear to be defeating reasons by generating reasons to dismiss those defeaters. 

These two ways of thinking about reasons map very well onto the types of reasons external to the system of reactive attitudes we might be asked for when justifying our holding someone morally responsible. If someone demands an external justification for the practice of holding morally responsible, they may be asking what the grounding reasons are for holding someone morally responsible - essentially asking for the reasons why one has a reactive response at all. On the other hand, they may also demand justification for your reactive response given some defeating external reason against having reactive responses at all. The naturalistic strategy I will defend here deflates demands for grounding reasons, while not denying the need to take defeating reasons seriously, fulfilling both the protection of reactive attitudes against (certain) demands for justification, while leaving it open to (certain) rational criticisms.  In order to do that, however, we will first need a clearer account of the subject of this naturalistic defense - the reactive attitudes themselves.

\section{Reactive Attitudes}
There have been extensive debates about what exactly the reactive attitudes are supposed to be. In Freedom and Resentment, Strawson himself limits the moral reactive attitudes to our feelings when persons other than us are wronged \citep[p.14]{strawsonp1974}. Since then, there have other accounts which attempt to limit the reactive attitudes further \citep{wallace1994}, or else give a clearer articulation of them \citep{bennett1980}. The account I will give here attempts to pick out reactive attitudes as a distinct kind of response that have particularly important links to moral responsibility and to human interpersonal relationships. Furthermore, the account should cover the various sorts of things that are taken to be reactive attitudes. 

It will be helpful to begin by looking at some of the features of our practices of holding morally responsible itself. In Chapter 1's discussion of Smart's account of moral responsibility, we saw that the only thing really being defended was a kind of ``moral grading'' and the practices of punishment and reward (in certain contexts) \citep{smart1961}. Thomas Nagel argues that moral judgment is distinct from the mere evaluation that something is good or bad because `` …when we blame someone for his actions we are not merely saying that it is bad that they happened, or bad that he exists. We are judging him, saying he is bad, which is different from his being a bad thing. This judgment only takes a certain kind of object'' \citep[67]{nagel1979b}. Ultimately, we will see that for Nagel, moral judgments require the object be a person. For Smart, judgment must be disentangled from moral responsibility which should simply be a case of praise and dispraise, while for Nagel, judgment of the person, and not simply an approval or disapproval of what has happened is essential to moral responsibility. This difference carries through to the details - moral emotions for Smart are akin to the sorts of things we feel about physical beauty or natural talent - evaluative but not judgmental. The sorts of emotions that come into play in reactive attitudes are judgmental, and for precisely the same reason - there is a fundamental difference in the proper object of those attitudes.

But what is the proper object? Nagel's later discussion on free will offers us a solution to this question. Moral responses like praise and blame are themselves deflated when there is a change in our understanding of the object of those responses - one which starts to understand the person and their acts as being absorbed into the class of things and events \citep[67]{nagel1979b} This suggests that the proper target of those responses is a person. This translates well onto the reactive attitudes: One can feel angry that it is raining on your picnic day, or that someone insulted you. There is something of a mistake, however, in feeling resentful that it is raining - it immediately raises the question ``resentful of whom?'' \footnote{For the time being, we will need to set aside questions about what persons are, and whether they are something that is relevantly distinct from the class of objects and events, an issue I will be concerned with in later chapters. Nagel does suggest that the concept of person and action that are being used in this distinction come from the internal sense of one's agency, a suggestion I will pursue in the following chapter. Until then, we can take this sense of agency as the source of the distinction that we want to make in the reactive attitudes without claiming that it is a legitimate distinction.}

It is an important point that persons are the proper target of these attitudes. We are still capable of feeling resentment at the rainy day itself, but when we do so, we would fairly be accused of being irrational. So, too, might we be accused of being irrational or unfair if we continued to regard someone with resentment if it turns out they had a good excuse (they accidentally tripped and knocked us into the puddle), or presented us with some other kind of excluding condition (they were hypnotized). These constraints of rationality and fairness are normative constraints applied to our own emotional responses, which points to a second essential feature of the reactive responses - that they are structured by norms.

This feature will require more explanation. Affective responses in general might fairly be said to be norm governed - there are situations in which it is inappropriate to express anger, and there are situations in which it is inappropriate to be angry, yet anger alone is something that we might want to distinguish from the reactive attitudes such as resentment. 

Alan Gibbard's account of reactive attitudes makes no such sharp distinction. For Gibbard, moral convictions just are norms for anger and guilt, and resentment is just a kind of anger \citep[p.126]{Gibbard1990}. Reactive attitudes are simply anger at the violation of moral norms that we accept - the anger being an expression of our acceptance of those norms. R. J. Wallace offers a different account, arguing that Gibbard's account fails to account for important differences in the content of resentment as opposed to anger - specifically that they are a response to demands on conduct that we hold others to \citep[p.49]{wallace1994}. On Wallace's account, reactive attitudes are emotional responses with a particular kind of cognitive content, what Wallace calls a quasi-evaluative state. For Wallace, we hold ourselves and others to certain expectations (consciously or not), and are subsequently prone to certain emotional responses, or else prone to thinking a certain emotional response would be appropriate. A reactive response is an emotion with this particular quasi-evaluative content \citep[p.49]{wallace1994}.

While I agree with Wallace's claim that Gibbard's account ignores important differences between resentment and anger, Wallace's account has its own costs. For Wallace, the reactive attitudes have, as part of their content, expectations about our treatment by others which captures what is missing from Gibbard's account. However, in making the reactive attitudes include these expectations, Wallace is forced to narrow the scope of the reactive attitudes. According to Strawson, reactive attitudes are not only linked to resentment, but also things like love, sympathy, and shame \citep[p.9]{strawsonp1974}. Wallace just accepts this result, and argues that excluding love, sympathy, and shame actually helps address the challenge of figuring out just what makes an emotion ``reactive'' in the first place \citep[p.30]{wallace1994}. However, as I will argue much more extensively later, the connection between emotions like resentment, and those like love, sympathy, and shame is actually an important one - all of these attitudes must take a person as their target, and this is what makes these attitudes ``reactive.'' 

Furthermore, we sometimes have a reactive response - such as resentment - in cases where we reject the expectations that seem to be connected to them. Despite attempts to address this problem, Wallace's approach makes irrational reactive attitudes difficult to comprehend. We can still feel guilt when we consciously reject the reasons we ought to feel guilty, as in the case of a lapsed Catholic feeling guilty for violating Catholic sexual norms which he has rejected. Wallace attempts to address this concern by proposing that the expectations need not be consciously accepted. In such cases, Wallace supposes that there is a ``rift in the self'' between the emotional state and one's sincere evaluative belief \citep[p.46]{wallace1994}. This rift requires Wallace to carve out a fairly complicated psychological space that makes an ad hoc differentiation between ``accepting'' and ``internalizing'' a reason, and, I will argue, there is a much simpler way to account for irrational reactive attitudes. 

Furthermore, it is also possible to fail to feel the appropriate response even when the norms prescribe the response in question - one might not feel angry when lied to by a charming friend, even though one feels that anger would be appropriate. Wallace attempts to carve out a space between consciously accepting a demand, and internalizing a demand to make room for both of these kinds of cases, but this third case - ``holding to a demand'' seems resistant to any sort of positive characterization. Wallace characterizes it primarily by what it is not.

According to Gibbard, we accept certain moral norms, and our acceptance of those norms is expressed through emotional responses when those norms are violated, met, or exceeded. For Gibbard, if you stomp on my toe, I might respond with anger because you violated my expectations about respect for my personal space, and not causing needless pain.  For Wallace, the norms are bundled into the emotional response as expectations - the reactive attitude I would feel in the above case is ``resentment-at-having-my-toe-stomped-when-it-shouldn't-have-been''. The expectation around not having my toe stomped is part of the quasi-evaluative content in the reactive attitude. For Wallace, it is this pairing of emotion with evaluation that makes a reactive attitude what it is, rather than simply anger. 

The model I am proposing lands between Gibbard and Wallace. On my account, reactive attitudes start out like Gibbard suggests - I feel angry because someone stepped on my toe, and that anger is not a special kind of reactive emotion yet. It is motivated by the fact that an expectation I hold and am emotionally invested in (especially when it comes to my toes) has been violated. However, at this level, it is just anger, not resentment. It only becomes resentment after the situation and my own response undergo a further degree of evaluation, particularly concerning the appropriateness of my own affective response in the situation, an evaluation into which many factors can be brought to bear. It is only when I come to think that I ought to be angry that the anger turns into resentment. In some sense, I follow Wallace in claiming that evaluation is an important part of the ``reactivity'' of the reactive attitude and changes the character of the feeling. However, on my account, it is not that the emotion must be ``bundled'' with quasi-evaluative content, but rather that the emotion is itself subject to evaluation. Our attitudes become reactive attitudes because of a process by which we reflect upon them. 

The reflective element is the crucial element of this account, and can be most easily understood by contrast with Wallace's quasi-evaluative content. For Wallace, a reactive attitude is an emotional state that is essentially about an evaluation of someone or something against a set of expectations. Unlike Gibbard, for whom resentment would be just another way of saying anger but in a specific context (supplied by the fact that it is anger at a moral norm being violated), Wallace rightly insists that there is an difference in the character of the emotion itself - resentment is different than anger \citep[p.49]{wallace1994}. For Wallace, however, that difference is accounted for by the fact that the feeling is intrinsically linked to this evaluative content -in a sense, my anger has a different character because it is about the violation of a moral expectation, and this is changed by bundling the evaluation into the emotion by making the reactive emotions about expectations. This link is proposed as an explanation of what makes reactive attitudes reactive \citep[p.24]{wallace1994}. 

However, this also creates the complications that were mentioned earlier. Some of what Strawson identifies as reactive attitudes - such as love, sympathy, and shame - cannot be easily understood as connected to evaluative content. Wallace says that this is so much the worse for those attitudes, but this also undermines what I take to be a very crucial point that Strawson makes - that the reactive attitudes are really closely and essentially linked to what we understand as particularly human kinds of interaction. The reactive attitudes, for Strawson, are what make the difference between interacting with a person as a kind of complex object, rather than interacting with them as a person \citep[p.9]{strawsonp1974}. I think that Strawson was right about this, and it would certainly be more elegant if the particularly ``human'' forms of interaction - blame, praise, love, sympathy, etc, all had the same mechanism behind it. I think that this is right, but will argue this further in chapters 5-7. For now, I will just propose that the elegance of a single mechanism behind all of what Strawson suggests are the ``human'' ways of interacting would be a more simple and elegant solution, so we ought not jettison those other reactive emotions if we don't need to, and on the reflective version of the reactive attitudes, we do not have to. 

On the reflective account, instead of the emotion being intrinsically linked to the ``quasi evaluative content'' and tied explicitly with some set of expectations, a reactive attitude becomes what it is as a result of reflectively evaluating the emotion and the situation it arises in itself. The core difference in the feeling of anger and resentment, I will argue, is that resentment is not simply anger, but it is anger colored by the reflective evaluation that I ought to be angry. One can feel angry without thinking about the appropriateness of anger at all - one might feel a flash of anger when suddenly stubbing one's toe on a rock. Furthermore, one can even believe that anger is inappropriate or irrational and yet continue to feel angry - anger is at least compatible with the evaluation that anger is not an appropriate response. This is not true of resentment. \footnote{Resentment here understood as a kind of ``morally righteous'' anger, and not to be confused with the sense of resentment which has the same sense as ``envy'' -  I do not take the latter to be a reactive attitude.} One cannot believe that one should not feel resentment and still feel resentment, because the judgment that one should not feel resentment would negate the judgment that the anger felt is anger one ought to feel. To feel resentment and judge that one ought not to feel as one does would turn the resentment into mere anger.

This is not merely a case of stipulating that resentment is anger-plus judgment. Reactive attitudes, as Strawson has argued, are what back the idea of responsibility. One can be angry that x happened, but that anger will be a mere expression of disapproval or dislike. Resentment, with the corresponding judgment of the appropriateness of the anger helps to sustain the emotionally charged version of ``x is bad'' - not merely I, but all should feel this way. Such a judgment could certainly come through the violation of expectations as Wallace says it must, but on this account, the judgment about the emotion need not be based on or about specific expectations. Moral or interpersonal norms are one of many candidates to form that judgment - a simplified analysis of love as a reactive attitude might be ``Lynda makes me happy'' alongside the judgment ``Lynda is wonderful'', where Lynda's wonderfulness is then taken to validate the happiness, for example. 

Importantly, our evaluation of our emotional responses need not always come out in favor of the emotion we are feeling. If I feel angry when I stub my toe on a rock, I may realize that anger is not the appropriate response and that judgment may be enough to diffuse the anger or turn it into something else. However, it may not be enough, and I may become irrationally angry in a way that reason cannot overcome. Alternately, I may try and find another way to justify my anger (``Who left this rock here in the middle of the road where people walk?!''). However, what I'm supposed to do is dissipate or control the emotional response such that it is appropriate to the situation. 

So, when our emotional responses are subject to reflection, those responses can be defeated by reasons, or they can be endorsed and transformed by those reasons into reactive attitudes. On this account, then the essential feature of the reactive attitudes is that they are a result of this reflective process where the emotions themselves are subject to some sort of evaluation. This account of the reactive attitudes is able to absorb a great deal of Wallace's account because what Wallace calls quasi-evaluation is just one possible reflection we can have on our emotions. One clear benefit of this account over Wallace's however, is that by detaching the content from the feeling and locating it in the understanding we have of our own emotional response, there is much less restriction on what that content needs to be. It can be relative to moral expectations, but it can also be informed by other things, generating reactive responses like reverence, love, and gratitude. This allows us to bring back Strawson's original point that the reactive attitudes are deeply implicated in our ordinary human interpersonal interaction, rather than restricting the reactive attitudes to cases where quasi-evaluative content might plausibly be said to attach. 

Furthermore, it expands the scope of the moral reactive attitudes to include interpersonal, self, and other regarding attitudes - the attitudes are reactive given the way we come to understand our feelings, not by their targets. So long as a moral expectation or norm plays in the understanding of the feeling, it is a moral reactive attitude - we can be morally outraged at things done to us as well as to others, and we can be merely angry or disgusted at something one person does to another without supposing that there need be a moral component to that anger. I take this to more closely fit our emotional lives. 

This account is also able to avoid the need to propose a special space between ``accepting'' a norm (which amounts to consciously endorsing it) and ``internalizing'' a norm (which amounts to having a behavioral tendency to respond to it regardless of ones conscious endorsement.) Wallace discusses two ways that acceptance and internalization can come apart, which on his account creates a requirement for this third way of taking norms on. One way the two can come apart is in cases where we accept a norm consciously, but do not happen to have the feeling required by the norm we accept. This involves cases where we think we ought to be angry, but are not. Here, Wallace says the norm has simply not been internalized, though we still have the attitude that takes the norm to be appropriate \citep[p.41-5]{wallace1994}. Here, my account does not vary much from Wallace's. One simply evaluates one's lack of emotional response as somehow lacking or inappropriate, or perhaps simply reflects that others would find it so.

The more problematic case for Wallace is the case of the lapsed Catholic - a person who has internalized a norm about the wrongness of recreational sex through their Catholic upbringing that they do not consciously endorse, but still seem to respond to anyway. Wallace must claim that one is in some sense evaluating oneself as having done wrong (one ``holds oneself to the demand'') despite rejecting it because the feeling of guilt requires the quasi-evaluative content. 

The account I am offering is somewhat simpler. The feeling of irrational guilt is a feeling - shame, uncleanliness, etc - that one has been conditioned to feel in connection with recreational sex, coupled with a judgment that the feeling is inappropriate which cannot overcome the feeling itself (much like the case of irrational anger described previously). On this account, irrational guilt is not a reactive attitude. It is different than guilt, though it may feel similar, since the foundational ``feeling'' is going to be similar (a feeling of moral uncleanliness or shame). But simply having a defeating reason against the understanding of a feeling as legitimate guilt does not mean that we will automatically cease to feel guilty. Here, splitting the content of the reactive attitude from the feeling helps to isolate the difference in content between rational and irrational guilt - they get to be the same in the way that matters (how they feel), but different in what they mean (the content associated with them). 

This distinction between the feeling and our critical reflection on the feeling-response can also help us to more clearly understand the first condition mentioned on this account - that reactive responses have persons as their target. The reason that persons are essential as targets of the reactive attitudes is that our understanding of our emotional responses as reactive attitudes relies in an important way on the fact that persons are their target. Just what that understanding is and why it is important will be things that much of the rest of this dissertation will be concerned to elaborate and defend. At this point, however, a working definition is as follows: moral reactive attitudes are emotional responses understood in the light of their being directed at persons, and their being prescribed as appropriate (or inappropriate) by moral norms. This account is meant to capture both the sense that Nagel alludes to that moral responsibility seems to vanish as persons become things and acts become events - and connects with the peculiar change that Strawson alludes to when discussing our adopting of the objective stance (and foregoing the reactive attitudes) where we break off a certain degree of involvement with the person in question, and cease to treat them (or perhaps even see them) as a ``member of the moral community \citep[p.9]{strawsonp1974}.What this really means will require a deeper understanding of ``person'' as it operates here, and ultimately, whether moral responsibility grounded in reactive attitudes is defensible will depend heavily on whether or not persons, in the sense native to the reactive attitudes themselves, actually exist.

\section{Defeasibility and the Naturalistic Strategy}
Presuming that we will be able to generate an adequate account of persons to make sense of the reactive attitudes, how can the account of the reactive attitudes just given help the naturalistic strategy against worries about moral responsibility? Recall that the naturalistic strategy supposes that moral responsibility is grounded in our reactive attitudes which were in some sense ``naturally'' given and require no justification. However, Russell found one reading of this defense to be too weak (type naturalism) - defending reactive attitudes as a human capacity, and incapable of addressing worries that entertaining reactive attitudes might never be appropriate. Another reading he found to be too strong (token naturalism) - rendering our entertaining reactive attitudes utterly impervious to external critique, even in the face of the possibility that all human beings are morally incapacitated \citep[p.299]{russell1992}. Clearly there is some need to justify the reactive attitudes in the face of such challenges, and surrender them if the challenges are legitimate, but allowing for that seems to deflate the power of the naturalistic response which attempts to brush aside demands for justification.

If we view the reactive attitudes as emotional responses understood in the light of moral norms, we may be able to split the naturalistic defense along just those lines. The naturalistic strategy deflates the demand for a grounding justification for moral responsibility by grounding the initial commitment to moral responsibility in the reactive attitudes. No further justification is required for the fact that we have particular negative or positive feelings when we are poorly or well treated by others, or witness the same treatment of others. Anger does not require an ``external'' justification at all, because anger is a natural part of our phenotype. We don't need to justify our ability to get angry at things (type naturalism), nor the fact that we will in fact sometimes get angry at things (token naturalism) - even though we may still need to engage in discussions about whether or not anger was an appropriate response in this or that situation. We certainly don't need to have a general justification for anger in order to feel angry. We can see that the naturalistic strategy creates problems for the core skeptical argument (from chapter 1):

\begin{description}
\item[(1)] To hold persons morally responsible, they must have free will.
\item[(2)] x is incompatible with anyone having free will.
\item[(3)] x is true
\item[(c)] therefore we cannot hold persons morally responsible.
\end{description}

The problem with the core skeptical argument comes from an ambiguity in (1). If we understand (1) to mean that free will is the reason why we hold people morally responsible, then according to the naturalistic strategy, that is simply false. However, because our emotions are subject to reflective evaluation, we can still evaluate moral responsibility as appropriate or inappropriate. If we find that it is appropriate to hold a person morally responsible, then our response may turn into a reactive attitude - going from anger to resentment, for example. If the feeling is inappropriate, then the emotion is supposed to either diffuse or else be controlled. So rational reflection can (sometimes) defeat our emotions based on an evaluation of whether or not the emotion is appropriate. 

On this account, our engaged emotional responses (anger, joy, etc) are, in essence, our default setting. In the absence of rational reflection, the things that make us angry will make us angry, and those that make us happy will make us happy and their tendency to do this stands in no need of any general ``justification'' whatsoever. If being hit makes us angry, that is not because anger is justified by hitting any more than if seeing a clown makes us angry, the sight of clowns justifies anger. No justification is required in either case. However, importantly, we are ordinarily expected to reflect on our emotions, and, in essence, control them. And when evaluating my anger at being hit, I may come to judge that the emotion is appropriate because after all, people shouldn't hit others - and the anger becomes resentment. However, if I evaluate my anger at the sight of clowns, I might realize that anger is an inappropriate response to a clown, and perhaps that will diffuse my emotion, or perhaps I will try to control my anger, or work through it in therapy, or something of the like. 

The core truth in Strawson's naturalistic strategy is that it is a mistake to look for a general justification for the reactive attitudes in something like ``freedom of the will'' or even ``reasons responsiveness''. The general justification of the reactive attitudes is that they are just how we feel in certain situations, and feelings do not require that sort of general justification - neither as types, nor as tokens. However, Strawson did not leave enough room for the fact that we are also rational, social beings - and part of the participant stance is that we consider and provide ourselves and others with reasons to either forgo our emotional responses as inappropriate, or to acknowledge them as appropriate and good. When we reflect on our feelings, they are subject to the full range of rational analysis, discourse, and demand for justification - both the token naturalist response as Strawson gives it and the rationalistic strategy are far too strong, and as a result run roughshod over this feature of our moral lives. 

It is important to appreciate what the naturalistic strategy, properly reconfigured, does do. The naturalistic strategy sets the ``default'' response properly. Some arguments against moral responsibility suppose that the problem with moral responsibility is that we need some form of general justification to start feeling things like resentment, and in the absence of such justification we are disposed to treat people according to what Strawson called the objective stance. On this model, I feel resentful when Tom hits me because I believe Tom has free will, and because of his free will, Tom should have done other than choose to hit me. Skepticism then attacks moral responsibility by saying that we don't have any reason to blame Tom, if the causal thesis is true. I call this form of skepticism ``motivational skepticism'' because the doubt calls for a general justification for moral responsibility. The naturalistic strategy shows that no such justification is necessary - our reactive attitudes are motivated not by reasons or justifications at all, they are motivated by feelings of anger, joy, etc. I am angry because Tom hit me, full stop. In Strawson's language, we start out in the participant stance, not the objective stance. 

However, this does not mean that my feelings of anger are not subject to rational reflection, and it is here that I can be convinced not to be angry because my feelings are irrational, stay angry (because the feeling is neither particularly rational or irrational), or start to feel resentment. Given that I feel angry at Tom, I can be convinced that the anger is inappropriate - perhaps I am told that Tom was acting under hypnosis, and shouldn't be angry at Tom at all. This is a commonplace kind of argument and is a vital part of our moral lives. 

On this account, there may be some skeptical arguments that proceed by saying that there are no legitimate reasons for taking one's anger to be especially ``appropriate'' and turning it into resentment. For example, one might argue that one should never feel resentment at anyone, because for any time when a person angers you, they could not have done other than they did - a familiar skeptical worry, and if this is correct, then it will never be appropriate to feel resentment toward anyone, and the whole edifice of moral responsibility (at least that which involves reactive attitudes) is flawed and must be abandoned, revised, or replaced. In the light of this insight, we can reformulate the core skeptical argument from chapter 1 by removing the assumption that one must have free will to hold people morally responsible:

\begin{description}
\item[(1a] a person can be held morally responsible unless there is a reason not to hold them morally responsible. 
\item[(2a)] if x is true, then that always gives me reasons against holding anyone responsible. 
\item[(3a)] x is true. 
\item[(c)] therefore we cannot hold persons morally responsible.
\end{description}

This version of the argument (call it the core defeasibility argument) does not build free will or any other foundational belief into premise (1a), but our naturalistic commitment to holding other people to account has a built in rationality proviso, so that our unreflective emotional response is subject to revision. Premise (2a) introduces the possibility that there is a general defeating reason that will always defeat any reason to hold a person morally responsible and ``upgrade'' or emotional response to a reactive attitude. So general skeptical arguments are certainly still possible in the face of the naturalistic strategy. Perhaps the truth of the Causal Thesis does function as a defeating reason against moral responsibility. However, the skeptic now owes us some reason to believe that the premise in (2a) is true - why x is supposed to convince us not to hold people morally responsible. Why should I accept that the fact that Tom couldn't do other than hit me should defeat my anger, or should defeat the reason I have for resenting Tom (hitting me is morally wrong)? The naturalistic strategy does not overcome skeptical worries, but it does show that the skeptic has a good deal more work to do - they must explain why they are worried about things like the causal thesis, and convince us that we ought to be worried about them too.  

In the next chapter, I will investigate how a number of worries about moral responsibility could be understood as forms of defeasibility skepticism. This will involve understanding why someone might think that the causal thesis gives us reasons not to feel resentment, or ultimately hold people to be morally responsible (at least in the way we commonly do now). Both of the major kinds of objections against moral responsibility can be understood as defeasibility worries - the ``principle of alternate possibilities'' (PAP) and sourcehood. On this analysis, however, we will see that both of these concerns are motivated by the worry about the injustice or irrationality of moral responsibility. I will argue that this worry has its root in the fact that reactive attitudes invite us to understand a situation (including our response to it) differently in light of the fact that there are persons involved. In the next chapter, I will argue that source worries and alternative possibility worries are both compelling only insofar as they both challenge the belief that persons are special in any morally relevant way. 

\chapter{Potential Defeaters}

In the previous chapter, I argued that the naturalistic strategy offers only a partial defense of moral responsibility in the face of the Causal Thesis. The naturalistic strategy turns away objections to moral responsibility that demand a general justification for moral responsibility - moral responsibility is grounded in the emotional responses we have to different situations, and so our engagement and interest in responsibility has a naturalistic foundation. However, these responses can be questioned and challenged. We can challenge the legitimacy of a response within the norms and expectations that shape the reactive attitudes themselves - if I entertain a reactive attitude, my response is subject to challenges that distinguish between the right and wrong circumstances for entertaining that attitude as given by the norms of that attitude itself. If I am resentful toward someone who makes an honest and non-negligent mistake, my response can be challenged as illegitimate within the rules of the reactive attitudes. However, we saw that arguments can also be provided that might show that any sort of reactive response is inappropriate - giving a general defeating reason against moral responsibility. If we found that we were systematically morally incapacitated, for example, then this could be a reason not to ever entertain any of our reactive attitudes at all.

However, because of the naturalistic strategy, skeptics about moral responsibility will need to provide us with reasons to think that something like the Causal Thesis or determinism count as defeating reasons against moral responsibility. In this chapter, we will engage in a more detailed exploration of the specific problems that moral responsibility purportedly faces if the thesis of determinism or the Causal Thesis happen to be true. According to the analysis given thus far, if either of these are problems for moral responsibility, then it is because there is some feature of moral responsibility that one or both of these theses is supposed render problematic or make impossible. The outline of these features will be familiar - one of the features concerns the availability of alternate possibilities, and the other concerns whether or not we are the source of our actions. However, according to the naturalistic strategy, neither of these can function as defeating reasons by operating as a premise. We do not hold people responsible because we believe they have alternate possibilities, or because we believe they are the source of their actions - if either of these concerns are successful, it will be because the lack of alternate possibilities or sourcehood defeats reasons we might have for holding them responsible. 
Both possibilities have a prima facie plausibility at least. However, to understand this problem, we will need to have a closer analysis not of whether these things impact our judgments about moral responsibility, but why they do, and ultimately, whether they should. The latter question, however, cannot be decided without first being clear that each of these features may do a great deal more than the sparse functionality that J.J.C. Smart attributed to moral responsibility as reviewed in chapter 1. 

To do this, we will explore each of these features in more detail. First, we will examine the need for alternate possibilities in more depth. The belief that alternate possibilities are necessary for moral responsibility has two main parts, or so I will argue. The first is that the presence of alternate possibilities seems to be required to make moral judgments fair. The second is that the presence of alternate possibilities seems to be built into our sense of our own agency - our agentive phenomenology. Finding out that our judgments of moral responsibility are systematically unfair, or are somehow detached from our lived experience is would both be defeating reasons, and if the falsity if the PAP entails that our judgments are unfair and/or alien to lived experience, then the falsity of PAP gives us defeating reasons against moral responsibility. The second feature to be explored is the claim that we are the source of our actions. Interestingly, sourcehood arguments will have the same functions as alternate possibilities - namely in ensuring fairness, as well as adequacy to our agentive phenomenology. I will associate this feature primarily with the Causal Thesis, though ultimately, I think, these two distinct concerns are really just two ways of getting at the same basic expectations of moral responsibility - that it be fair and real. Finally, I will also argue that while the ``alternate possibilities'' track of argumentation cannot defeat moral responsibility by undermining fairness or our phenomenology, that source worries can and do present a serious challenge to moral responsibility. 

This chapter, then, will attempt to address four related problems - the claim that alternate possibilities are required for moral responsibility for fairness, the claim that alternate possibilities are required for moral responsibility on the basis of agentive phenomenology, the claim that sourcehood is required for moral responsibility for fairness, and the claim that sourcehood is required for moral responsibility on the basis of agentive phenomenology. While these four claims are deeply related, it will be clearest to treat each of them on their own, before exploring how there may be some interesting connections between them.

\section{Alternate Possibilities and Fairness}
The first challenge we will consider is the claim that if we do not have alternate possibilities open to us, then moral responsibility is unfair. On the face of it, this seems like an intuitive claim. If determinism is true, then the world is structured as described in Van Inwagen's \citeyearpar{inwagen1978} argument that was briefly reviewed in chapter 1:

\begin{quote}
The state of the world at time t plus the laws of nature entail any state of the world after t. 

Peter raises his hand at t1

Peter raising his hand at t1 is entailed by t+ the laws of nature

For Peter to refrain from raising his hand at t1, he would have to either change the state of the world at t, or change the laws of nature

No one (including Peter) can change the state of the world in the past, nor can anyone change the laws of nature

Therefore, Peter could not refrain from raising his hand at t
\end{quote}

At that time, we saw that determinism framed in this way seems to break down a distinction between acts that we freely choose to do, and those which we were forced to do. Ordinarily, we do not hold people morally responsible for things they could not but do - if we find that someone caused some harm, but they were forced, pushed, hypnotized, or else otherwise unable to refrain from that act, then in ordinary moral discourse, we tend to say that the person was not to blame. Van Inwagen's argument attempts to show that if determinism is true, then we have a ``one track'' history by arguing that the ability for Peter to have done otherwise (refrain from raising his hand) would require him to change the past or change the laws of nature, which are both outside of anyone's power. The incompatibilist will then argue that if determinism is true, everyone is in this position because there are no alternatives available to anyone for any choice whatsoever - and if we don't think it's right to hold people morally responsible in the first case, then we shouldn't think it would be right to do so if determinism is true. This argument appeals to our intuitions and behavior to argue that there is something wrong with moral responsibility if determinism is true.

As we have already seen, Harry Frankfurt initiated a popular line of criticism into this line of argument by attempting to show that people actually continue to hold people morally responsible even when it is shown that they lacked alternate possibilities. \citep{frankfurt1969}A great deal of the focus on alternative possibilities has been focused on showing why Frankfurt-style examples are or are not counterexamples to the principle of alternative possibilities. This debate, however, neither draws into question nor attempts to articulate reasons why alternate possibilities might be thought to be relevant to moral responsibility. Both sides of the Frankfurt-style example debate attempt to ground either criticism or defense on the way people actually behave. However, as these examples become increasingly esoteric and complex the connection to people's actual decision-making process becomes more and more tenuous. Appealing to these examples and pushing against intuitions they may exhibit does nothing to explain why alternative possibilities are (or are not) relevant for moral responsibility. It simply claims that they are or are not responsible, based on how people seem to intuitively respond to the cases. 

Instead of trying to decide whether people's judgments about moral responsibility depend on the presence of alternate possibilities, we will explore the basis of the fairness claim. Why is it that alternate responsibilities might be thought to be necessary for moral responsibility to be fair at all? And more importantly, is that claim really true?
In ``Responsibility and the Moral Sentiments'' R. J. Wallace focuses almost exclusively on the fairness objection. According to Wallace, fairness is a significant concern for moral responsibility because the practices around moral responsibility often have harmful or undesirable consequences for the target of those practices \citep[p.59-61]{wallace1994}. For Wallace, a proper account of moral responsibility will be one that makes it fair to hold people morally responsible, which is morally required because of the harmful consequences of holding people responsible. We have an obligation not to subject people to the harms associated with things like blame and especially punishment if moral responsibility is itself unfair. 

What is it about the truth of determinism that might make moral responsibility unfair? The primary possibility that Wallace considers is what he calls the generalization strategy. The generalization strategy uses the fact that there are clear cases in which we do not hold people morally responsible and argues that if determinism was true, we would be bound by pain of inconsistency to stop holding people morally responsible. Wallace considers two kinds of appeals that are typically called upon to mitigate or eliminate moral responsibility; excusing conditions in which we don't hold people responsible for an act due to considerations about the act itself, and excluding conditions, where we don't hold people responsible because of considerations concerning the person or people that we would be holding responsible. \citep[p.118-194]{wallace1994}

Wallace, drawing from the earlier work of Strawson and Austin says that there are different reasons why we exclude or excuse people. In legitimate cases of excusing a person, we do so because we find that a person really hasn't done anything wrong. \citep[p.119]{wallace1994}. In cases of accidents, we see that the result of a person's actions are not actually reflective of their will or intentions, and as such one did not actually violate the kind of expectation that underlies the reactive attitudes \citep[p.136-9]{wallace1994}. In cases of coercion, we have very much the same analysis - what a person does under duress or coercion does not undermine the fact that a person has proper moral regard - they are being forced to do something against their will, and given that for an act to be wrong it must express something of the quality of the will, then the person under duress doesn't themselves do anything wrong \citep[p.143-7]{wallace1994}. A similar analysis applies to cases of inadvertent bodily movements or physical constraint. On Wallace's analysis, determinism does not meet the criteria of an excusing condition because determinism does not rule out a person's having an inappropriate quality of will.

Excluding conditions are given a distinct analysis. According to Wallace, we exempt someone from moral responsibility because we find that they are not really a morally responsible agent at all, which on Wallace's analysis means that they do not have reflective self control consisting of the power to grasp and apply moral reasons, and to control their behavior by the light of such reasons \citep[p.156-7]{wallace1994}. After arguing that this exclusion applies to very young children, the insane, and the hopeless but unwilling addict, Wallace argues that there is nothing about determinism that shows that a person lacks an ability to grasp and apply moral reasons, and control their behavior by the light of such reasons \citep[p.180-2]{wallace1994}.
Here it is worth noting again that Wallace's analysis of unfairness tacitly assumes that what would make moral responsibility unfair is an inconsistency in our behavior. For Wallace, the incompatibilist objection runs something like the following: It is already clearly unfair to hold people responsible for accidents, or if they are children, insane, etc. However, the reasons for this generalize if determinism is true. Wallace's argument, as it stands, shows that a lack of alternate possibilities cannot be a defeating reason because the truth of determinism would not show that our existing practices of excusing and excluding people from moral responsibility are inconsistent. 

However, an incompatibilist may respond by claiming that having the power of rational self control is not itself a fair reason to hold people morally responsible, and that the truth of determinism actually makes the sorts of ``expectations of moral regard'' that are built into Wallace's account of the reactive attitudes unfair. If determinism is true, then any mismatch between our expectations and the way the world turns out is really inevitable, and to expect the world (or some person in it) to do other than what they are determined to do is unfair, because it expects the impossible. 
Wallace anticipates a response along these lines, and responds to the claim that it is not simply having a rational power that suffices for moral responsibility, but having the ability to exercise such a power \citep[p.184-5]{wallace1994}. He claims that the expectations that are built into the reactive attitudes need not distinguish between a general power and the ability to exercise that power. The reactive response is a response to the failure to exercise a general power at a time where it would be appropriate. One might have a general power to refrain from shoving people into puddles, even if at some particular time one could not help but push someone into a puddle (as opposed to having a mental illness that gave one an irresistible compulsion to shove someone into a puddle whenever the opportunity arose). The general power that Wallace has in mind would be something like the fact that I can respond to reasons, generally speaking, and so I am held responsible for when I don't respond to moral reasons. By showing that there is a clear distinction between things like being insane or doing something by accident on one hand, and the truth of determinism, Wallace has shown that there is nothing unfair about moral responsibility given the expectations built into the reactive attitudes. In other words, the form of fairness Wallace establishes is consistency in our exempting and excusing the sorts of cases we ordinarily do while continuing to hold people morally responsible even if determinism is true. Thus moral responsibility does not challenge fairness in this sense, and it is up to the incompatibilist to offer an alternate kind of fairness that moral responsibility does violate. \footnote{Wallace does consider one possible contender for an alternative principle of fairness - namely the principle of avoidability - but then argues that there are no concrete cases that need to appeal to the principle of avoidability, given the account he has given of excusing and excluding conditions, and no independent reason to take the principle of avoidability remains. (p.205-25).}

It is worth noting that Wallace's defense here is a defense that is entirely internal to the reactive attitudes. We can employ the naturalistic strategy from chapter 2 to show that the reactive attitudes are indeed given. However, Wallace's argument relies on the norms internal to the reactive attitudes to evaluate whether they are fair - fairness is evaluated according to an internal consistency. This may open Wallace up to the criticism Russell \citeyearpar[p.297-8]{russell1992} has against the reactive attitudes account of moral responsibility - namely that this account is guilty of a kind of ``token naturalism'' and puts the fairness or unfairness of the principles themselves beyond rational criticism by taking the principles themselves as given, rather than the reactive response alone. Wallace has effectively written out the possibility of a critique of the fairness of the reactive attitudes that comes from norms external to them. If it turns out that the reactive attitudes as a whole are unfair because, as it turns out, people are systematically morally incapacitated, this is something that cannot even be considered by Wallace's theory as given. 

This problem is a symptom of the system of expectations being a part of the reactive attitudes themselves. In chapter 2, I argued that we must treat the reactive response and the expectations that the response comes from as distinct elements, and this modification to the account can mitigate the force of this new objection as well. If we instead treat the response and the expectations as distinct elements, we can appeal to that distinction to mitigate the force of this objection. The emotional responses themselves are not subject to legitimate demands for justification - only the reasons and expectations that frame those responses are. This also means that the expectations can be defeated if they are shown to be unfair, but it also requires some explanation of what makes these expectations unfair in the first place. Wallace's defense shows that they are not unfair because they are inconsistent, and if they are unfair along some other criteria, the skeptic must supply that criteria and show both that it is one we ought to be concerned about, and that moral responsibility fails by that criteria. 

Thus, Wallace's account, slightly modified, is able to successfully defend against the claim that holding people morally responsible is inconsistent by blocking the generalization strategy. It is simply not true that the reasons we exempt or excuse people in ordinary moral discourse generalize to include determinism, so determinism cannot defeat moral responsibility on that line of argumentation. By splitting the reactive attitudes from the system of expectations, we also prevent Wallace's strategy from committing itself to token naturalism. Systematic moral incapacity can still defeat moral responsibility. However, without the generalization strategy, it is no longer clear how the lack of metaphysically open possibilities is supposed to entail systematic moral incapacity. The incompatibilist owes us another account of moral incapacity to motivate the fairness objection that reveal the apparent ordinary moral capacity we seem to have in everyday moral discourse to be somehow mistaken or illusory. 

\section{Alternate Possibilities and Phenomenology}
The next challenge to moral responsibility we will consider is the claim that our agentive phenomenology suggests that we have alternate possibilities, and if this is false, then moral responsibility is impossible, or at least requires significant modification. Why problems with our agentive phenomenology can generate problems for moral responsibility is less clear than the claim that if moral responsibility is problematic if it is unfair. Why is agentive phenomenology relevant at all? We can think of this as a challenge to determinism coming from agentive phenomenology coming in the following form: ``If determinism is true, then my experience of my own life is entirely wrong, It seems to me that whenever I make a choice, there are multiple possibilities before me, but if the thesis of determinism is true, then my experience of choice is wrong.'' But why might this make us rethink moral responsibility?

It is first important to be clear about what agentive phenomenology is. A fairly loose working definition is that agentive phenomenology is the ``what it's like'' experience of being an agent.  Horgan gives a helpful articulation of some of the elements of that phenomenology. It includes the experience of self as source (which we will examine later), purposiveness, voluntariness, and of being inclined, but not determined by reasons. \citep{horgan2007}. Importantly, agentive phenomenology has both a ``raw feel'' and intentional contents - these raw feels include information about what the content we  are experiencing (presentational content) is supposed to be about. The intentional content of the phenomenology of my seeing a cup on the desk includes a great deal, including certain expectations about things I cannot see (that the cup has a non-visible side, for example, what Husserl called the ``horizon''). For the purposes of this discussion, I will simply agree with Horgan that there is at least some phenomenology whose intentional contents are entirely determined by what is going on ``in the head,'' and which does not rely on external facts to determine what it is about, and that agentive phenomenology is like this. Thus, it is possible to examine the intentional content of at least some our phenomenology while bracketing or setting aside considerations about the external world altogether (as in Husserl's phenomenological epoche) \citep{Husserl1999}. 

We see that the argument that links the truth of determinism with the falsity of our agentive phenomenology relies on the claim that our agentive phenomenology has, as part of its intentional contents, the experience of genuine metaphysical alternatives. Furthermore, if this is wrong, then moral responsibility is threatened. Thus, this claim has two potential pressure points. The first is the claim that if our agentive phenomenology is wrong, then moral responsibility is under threat. The second is the claim that our agentive phenomenology is that of having metaphysically open alternatives at a given moment of choice. 

Why would a problem with our agentive phenomenology generate a challenge for moral responsibility? The falsity of our agentive phenomenology might provide a defeating reason by undermining our fundamental understanding of the moral environment we think we are interacting with. Moral responsibility is a natural reaction to the world as we experience it. We might think of this in epistemic terms: while a belief in the reality of the external world may be a presumption behind my reaching out to grab a cup, it would be odd to say that, in ordinary circumstances, my belief in the reality of the external world was one of my reasons for grasping the cup. The thesis of determinism ``raises the stakes'' on our ordinary behavior in the same way that the skeptical ``brain in vats'' scenario does for our belief in the veridicality of our ordinary experiences of the world, and it does so in much the same way - by showing us that there may be a difference between the way the world appears to us, and the way it really is. Another way of thinking about this is that the possible truth of determinism presents us with a reason to ask whether there is a distinction that Smith \citeyearpar{smith2007} describes as the difference between holding responsible and actually being responsible. Smith argues that judgments about whether we should entertain a particular reactive attitude are subject to considerations that are independent of two other considerations - considerations of whether an agent is open to moral appraisal, or open to legitimate moral criticism, and thus there really is such a difference.

It is possible that finding out that people do not have alternate possibilities can defeat my reactive attitudes - in this case, by showing that the whole system of reactive attitudes is inappropriate, because it is a response to a delusional picture of the world. Thus, if our agentive phenomenology is false, then our reactive attitudes are actually inappropriate, because people are not what we thought them to be. This shows that one of the things that we expect from moral responsibility is that when engaging in the practices around moral responsibility, we have something like a reality principle. We want our moral responses to be a response to the way the world really is, and not a response to an illusion or a lie. This is part of what makes Smilansky's account of moral responsibility, which defends moral responsibility as a kind of necessary illusion, unsatisfying for some. It fails to ground moral responsibility in the real world, however useful a lie it may end up being. \citep{smilansky2000}. 

However, it isn't immediately clear what the nature of the conflict is between the lack of alternate possibilities and our agentive phenomenology really is. It effectively raises the question of a reality/appearance distinction, but it does not immediately provide an answer as to whether there really is one. This shifts the question to whether our agentive phenomenology really does suggest that we have metaphysically open alternate possibilities, which we cannot have if the thesis of determinism is true. Unfortunately, it is not immediately clear how the phenomenological issue can be decisively settled - those engaged in these debates seem to offer varying reports on what their phenomenology of action actually is. My own sense is that while I have some sense that I have alternate possibilities, this is in a very metaphysically unambitious sense that what happens next in some circumstances depends on what I decide to do next. Others seem to have a more metaphysically robust phenomenology - Timothy O'Connor reports that his experience of choice: 

\begin{quote}
[...]seems for all the world to be up to me to decide which particular action I will undertake. The decision I make is no mere vector sum of internal and external forces acting upon me during the process of deliberation. \citep[p.173]{oconnor1995}
\end{quote}

All I can do is confess that my own phenomenology does not suggest that to me, but I do not know what ``being a vector sum of internal and external forces'' would seem like in the first place enough to judge whether my experience seemed like that or not. \footnote{Horgan makes the important distinction between phenomenology ``not presenting one's behavior as state-caused'' and ``Presenting one's behavior as not state-caused.'' in Horgan (2007) p.11. O'Connor seems to be conflating the two. While I agree that my phenomenology does not present my own behavior as state-caused, it is presenting something as not state-caused that I cannot imagine.}

Nahmias, Morris, Nadelhoffer, and Turner have suggested that part of the problem with settling the issue of what our agentive phenomenology really contains is that exposure to philosophical theory taints the results of phenomenological introspection by causing people to experience things in a different, theory-laden way \citep[p.172]{nahmias2004}.They suggest that empirical results on phenomenology are nevertheless possible, and suggest that focusing on laypersons, guided by psychologists trained in introspective methods, might be able to settle the phenomenological issues. Preliminary results reported by Nahmias, et. al. support the claim that the majority of phenomenology is compatibilist, including compatibilist with respect to the issue of alternate possibilities \citep[p.174]{nahmias2004}.

Though this result is helpful for my argument, I am less sure that we ought to be convinced that the project is really settling much of anything about the phenomenology, as there are numerous reasons to suppose that this project is beyond at least our current investigative capacities. The first of these is the concern that Nahmias, et. al. have acknowledged - that a person's phenomenology might be impacted by philosophical theory. While Nahmias, et. al. have attempted to control for the impact of philosophical theorizing by limiting the investigation to non-philosophers, it is really not clear that this is enough. Philosophical theory may well cause a person to understand their experience as compatibilist or not, and thus taint the data. Eliminating philosophers does not eliminate this problem because cultural influences other than philosophy might have the exact same effect, while being much more difficult to control for - including things like religious beliefs, beliefs about science, and other folk metaphysical beliefs. If Nahmias, et. al. are right to be worried about the impact of philosophical theory and if the objective is to get at a report of a ``pure'' phenomenology untainted by theory to get at what is pre-theoretically contained in the intentional contents of the phenomenology alone, then it seems unlikely that simply isolating philosophers from the general population will do the job. Such full isolation may not be possible, because it's not very likely that anyone in any linguistic community is free of some theory about the way the world or the people in it are that would impact their result.

There is also reason to be concerned with the fact that even among laypersons, there is a lack of consensus around what sort of metaphysical picture (if any) agentive phenomenology contains. To understand this concern, it is helpful to consider the process of phenomenological inquiry itself. One of the features of inquiry into phenomenology is that if we set aside certain assumptions about what the contents of phenomenology mean ``externally'' (a process that Husserl called epoche or bracketing), \citep{Husserl1999}, what remains is entirely transparent and immediate to us. We can, for example set aside the question of whether one's experience of the world corresponds to the way the world really is, and instead investigate that experience qua experience. At this level, the question of whether we are getting the apparent world itself ``right'' is absurd - there is no longer any distinction between appearance and reality to appeal to that might generate such a concern. This applies equally to intentional contents - there may be a question about whether the appearance of the cup ``as an object'' corresponds to reality (ie. whether there really is a cup). However, there is no sense to the question of whether the appearance of the cup ``as an object'' is really an appearance ``as an object'' - whatever the reality may be, the appearance is transparently that ``of an object.''

It is certainly not the case that people can never make any errors whatsoever about phenomenology - however, if there are errors, those errors must come at the level of interpretation of phenomenology. We can form mistaken beliefs about our experience ``I thought I saw a ghost'', and perhaps even project those beliefs onto the experience themselves ``I saw a ghost'' - but the level at which we cannot be mistaken is the raw phenomenology itself ``I had an experience as-of a ghost appearing.'' The satisfaction conditions for that claim are in the experience itself.  

If we turn this consideration toward agentive phenomenology, we can see that the same considerations apply. If we are examining the experience of agency, and whether there is some intentional content in that experience that indicates robust alternate possibilities, we are setting aside questions about whether that experience is correct and there actually are alternate possibilities available to us. Instead, the question is about whether agentive phenomenology is ``as of having metaphysically robust alternate possibilities.'' Given that this is our inquiry, it should also be clear that if any such thing is transparently available in phenomenology, then it should be utterly impossible to be mistaken about this - mistaken, that is, about how things seem to us to be. 

The preliminary studies that Nahmias, et.al, have done report a 62\% majority for compatibilist responses on survey testing the sense of having the ability to do otherwise (as opposed to 35\% offering the ``libertarian response''). However, when reporting on something that is supposed to be as transparent as phenomenology, what explains the fact that there is any variation whatsoever? If the answer to this question is genuinely phenomenological and clearly reportable, then there should be no way to be mistaken because the satisfaction conditions are   identical to the presentation. Thus, the variation in responses would have to be explained by saying that people actually have differing phenomenology. On the other hand, if the variation is occurring at the level of interpretation, then the reporting isn't on the raw phenomenology at all, but on a post-theoretic interpretation or judgment of the phenomenology.

Along these lines, Horgan \citeyearpar{horgan2007} argues the point that even if our agentive phenomenology is compatibilist or incompatibilist, whether it is or not may not be introspectively available to us. According to Horgan, to come to a decision requires not only introspective awareness, but judgments about the intentional contents of our phenomenology. If this is the case, then the empirical results that Nahmias, et. al. have gathered are of ambiguous value - they may tell us less about the intentional contents of agentive phenomenology, and more about the way people judge those contents. In particular, Horgan suggests that questions about phenomenology may shift our understanding of ideas like ``freedom'' themselves, which then impact the judgments we make about whether our agentive phenomenology suggests that we have freedom, which is now understood in a different way given contextual cues in the very asking of the question. 

There is, however, another shift in context that Horgan does not explicitly consider - that our agentive phenomenology says absolutely nothing at all about what our feeling of being free to act means. Perhaps our agentive phenomenology is such that the intentional contents are of our being free in certain contexts without included necessary or sufficient conditions as to what ``free'' actually means. One possibility is that the intentional content of my agentive phenomenology just tells me about what attitude I should take regarding making a decision. 

There is reason to believe that at least, Horgan's suggestion that the intentional contents of agentive phenomenology are at least judgments, and not straightforwardly ``read out'' of the phenomenology. As Nahmias, et. al. results show, there are multiple theoretic interpretations that can be taken from the raw phenomenology, which suggests that the raw phenomenology itself has to be neutral between them. As an example, if I can interpret my experience as ``of a ghost'' or as ``of a hanging sheet'', there is a common raw experience of a small white, billowing mass that could equally be described as ``appearing as of a ghost'' or ``appearing as of a hanging sheet'' depending on my theoretical inclinations. However, as a description of the appearance (and not whether I'm getting the facts about the world right), either description works equally well. If we consider agentive phenomenology in this light, if people alternately report that they feel ``really free'' or ``not really free'' in some scenario, then it is far more likely that the variation in their response comes from the theoretical model that they are reading into the experience, rather than the claim that they are getting different raw phenomenology from the experience. Otherwise, it has to be the case that people are reporting back incorrectly on their raw phenomenology while understanding the correct definitions for the report. This would be like someone who knows the appropriate uses of the word blue, having a blue experience, and yet somehow mistaking it for a red experience. The more plausible interpretation is that agentive phenomenology is neutral with respect to metaphysical nuance between the sense that deliberation influences what happens and our having robust access to distinct metaphysical possibilities. This gives reason to be suspicious of the claim that some idea of contracausal freedom is built into our experiences in reports that people give of their own experiences of free action. 

Finally, it's not clear how such high concept ideas such as ``contracausal freedom'' or ``agent causation'', or some description thereof could be part of our naive phenomenology in the first place, given that these ideas require a great deal of cultural and theoretic scaffolding to become available to us at all. The idea of an agent (as distinct from a thing) or a cause are both embedded in a particular cultural context. In his ``Philosophy and the Scientific Image of Man'', Wilfrid Sellars \citeyearpar{sellars1962} investigates some of the foundations of the more general conflict between our causal, scientific view of the world, and the natural ``manifest image'' of our folk psychology and it's theoretical refinements. According to Sellars, the scientific image, with it's notion of cause, and non-agentive ``things'', is actually grounded in changes in the manifest image, where once upon a time, we might have thought it a sensible question to ask why the wind blew down a house in the same way we might ask why a person knocked down the house. \citep[pXX]{sellars1962} Agency, I would argue, is an element of the manifest image, and is not there because we had an innate theoretical understanding of what agency meant that simply had to be analyzed out of agency. Theorizing around the idea of agency only really becomes necessary at all once we have the idea of an inert ``thing'' that operates according to causes to contrast it with. Indeed, giving a clear account of what agent causation is even supposed to be is notoriously difficult.

If this is correct, then there is a fundamental mistake in trying to find anything like ``agent cause'' or ``contracausal freedom'' in our phenomenology at all, because these ideas are not extracted from our phenomenology. Instead, they come about as we try to resolve our lived experience with our theoretical understanding of the world.  There is no sense in which these or other abstract concepts are embedded in our phenomenology, and there is for that reason no correct answer to what metaphysical concepts are embedded in phenomenology - even to ask the question is to make a mistake - the mistake of supposing that philosophy is going on at all in the core elements of our experience, and that the ideas in our experience have embedded in them a full conceptual analysis that we simply need to discover and analyze. I am skeptical that the results of conceptual analysis on our experience are embedded in our experience at all. 

I suspect that when you ask a person if they feel that some act they did was freely chosen and they say ``yes'' - that is the correct answer, phenomenologically speaking. Attempts to probe what people ``really mean'' when they say that they felt free invite confabulation and an attempt to theorize about an experience, rather than an attempt to extract the theoretical underpinnings of that experience. I do not know how to argue in favor of this suspicion except to say that the presumption that conceptual analysis of experience like this isn't simply confabulation seems to be ungrounded, and given the propensity to which human beings seem disposed toward confabulation, there should be a bit more concern about whether the suspicion I've suggested is correct (for a good philosophical overview of the challenges and extent of confabulation, see \citep{hirstein2005}). 

In light of all this, there is no strong reason to believe that an empirical account of agentive phenomenology is available to us, at least not with the current methods of investigation available. However, this does not mean that we are entirely at a loss. Indeed, it seems possible to bracket issues about whether or not people are describing their phenomenology correctly, and take the reports themselves as data about what people tend to report about their phenomenology, taking the data as what it most clearly and directly is - data about how people report on their phenomenology.

For my purposes, the fact that there is not a consensus on the metaphysical picture of agency suggested by phenomenology in the reports given about it indicates that these metaphysical pictures are not themselves contained in the phenomenology. If a clear metaphysical picture that included alternate possibilities was present in the phenomenology, then everyone should be able to see it, because the raw, pre-theoretical phenomenology is supposed to be transparent to us. It is possible that Horgan is correct; that our intentional contents do suggest either a compatibilist or incompatibilist picture of our actions, but that even if this is true, we cannot be immediately certain that a person's report of what their phenomenology suggests is correct. For the purposes of this discussion, it does not actually matter whether Horgan or I is correct, both possibilities create serious problems for the attempt to empirically probe our agentive phenomenology. 

How, then, do I account for the fact that people seem to believe that the lack of metaphysically open possibilities does conflict with their agentive phenomenology? How can the thesis of determinism act as a defeater despite the fact that it is not challenging some existing intentional feature of our agentive phenomenology that suggests the falsity of determinism? Why is moral responsibility ever taken to be challenged by the thesis of determinism?

My suggestion follows Horgan's own - that our judgments of what is required for us to be free, and thus morally responsible - may be influenced by context. The challenge seems to have a natural home in the not uncommon experience of one's sense of autonomy being challenged when first confronted with the thesis of determinism. When people are presented with the thesis of determinism, the falsity of determinism seems to be the sort of thing that is necessary for my agentive phenomenology to be veridical, and thus agentive phenomenology is understood as requiring the falsity of determinism. In this way, the thesis of determinism may present the hearer with a kind of false choice: either my experience is of a world where determinism is true, or it is of a world where determinism is not true. The hearer is pushed to choose one or the other. However, it is possible that the intentional content of agentive experience says nothing whatsoever about the truth or falsity of determinism, and the presence (or not) of metaphysically open alternate possibilities, and instead, it says something about our experience that seems similar to metaphysically open possibilities. In particular, it may simply be that my agentive phenomenology tells me that my choices matter in what I am going to do, and when I hear the thesis of determinism, I believe that my choices can only matter if I have metaphysically open possibilities. Thus, I can be duped into falsely believing that determinism comes into conflict with my agentive phenomenology. 

If all of this is correct, then we no longer have a reason to think that the lack of alternate possibilities serves as a defeater against our commitment to moral responsibility - not because phenomenology doesn't matter, but because our phenomenology does not require us to have metaphysically open possibilities, even though some falsely judge that it does. And given that the reactive attitudes are our default natural position - we start with an assumption that we are going to blame people, etc. unless we have a reason not to - then the Naturalistic strategy can continue to defend our commitment to moral responsibility. 

There are, however, other potential defeating reasons against compatibilism coming from causal-incompatibilists that are less worried about alternative possibilities, and more worried about ways in which the Causal Thesis shows that we are not the sources of any of our acts, and for that reason cannot be held responsible for them. It is to these concerns I now turn. 

\section{Self as Source, Fairness and Agentive Phenomenology}
A second line of objection against moral responsibility comes from the claim that if the Causal Thesis is correct and our actions are all caused, either deterministically or indeterminsitically by causal processes, then there is no sense in which we are rightly thought of as causes of our actions, and that there is something wrong with moral responsibility if we are not the sources of our actions. As with alternate possibilities, we will need to examine why this claim, if true, would defeat moral responsibility. We will consider two kinds of objections - again looking first at considerations of fairness, and then phenomenology.

We saw that the objection from fairness did not give good reasons why determinism should defeat moral responsibility on the basis of the PAP. Wallace's argument showed that even giving our existing practices of excusing and excluding people from moral responsibility, there is nothing inconsistent about moral responsibility if determinism is true. Furthermore, the naturalistic strategy provides a foundation for moral responsibility such that some reason is needed to believe that moral responsibility is somehow unfair in a way that defeats our reactive attitudes. One incompatibilist objection to Wallace's account that was briefly considered was the claim that the expectations embedded in moral responsibility are themselves unfair - Wallace's defense really only showed that there were no arguments internal to the reactive attitudes that showed them to be unfair - but the mere lack of alternate possibilities alone was unable to provide a clear account of why that is.

The Causal Thesis gives the fairness objection new teeth. The incompatibilist can agree with Wallace that there is nothing unfair or arbitrary about the kinds of cases in which we do and do not make exceptions or excuses. However, the objection that the expectations themselves are unfair can be grounded in the claim that it is arbitrary to hold a particular person responsible for some act if the causal thesis is true because if the Causal Theis is true, then  a person is not the source of their actions. 

One well-articulated form of this objection comes to us as Galen Strawson's \citeyearpar{strawsong2010} basic argument. According to the basic argument, if we consider a person at the moment of their birth, who they are is a result of environmental and genetic factors - it makes no sense to hold the person responsible for any of those things, because they are not in their control. From that moment, everything that person does will be the result of who they are and environmental factors. Similarly, who the person becomes will be the result of the same, and the history of such changes. If a person cannot be held responsible for who they are, and what they do or who they become is a result of who they are, then they cannot be held responsible for those things either. There is then nothing left for them to be held responsible for \citep[p24-7]{strawsong2010}.

We might question why the incompatibilist claims that this is unfair. Strawson seems to rely on intuition here, probably a fairly common intuition, that it is worth examining. The real concern raised by the basic argument is that if the Causal Thesis is true, we are simply intermediate steps in causal chains that eventually result in our behavior, and there is no sense in which we are the ultimate originating source of our actions. However, to hold someone morally responsible is to pick them out of a causal chain that extends past the beginning of their lives and out further beyond them and in essence single them out to answer for what has happened ``through'' them. The worry is that holding someone morally responsible is arbitrary, because unless the person is the ultimate source of their action, there is no principled way to settle on them, and not their parents, or perhaps even the big bang and hold them responsible instead.

If no response to this objection can be found, then this does show that moral responsibility is unfair if the causal thesis is true. However, the character of the response shows where we might expect a response to emerge - the charge that stopping at some individual agent in the causal chain is arbitrary. One key supposition in the basic argument is that the only way to have a non-arbitrary stopping point would be to isolate some interruption or new element in the chain of causation itself. One possibility we will explore in later chapters is whether there may be a way of stopping moral responsibility not on the basis of picking out some responsible agent as a unique kind of cause of the action in question, but rather bearing some other, non-arbitrary relationship to the act in question that still meets the test of ``fairness'' that we would require for moral responsibility.  However, this is not the only test that such an account would need to pass. 

The second objection to moral responsibility claims that moral responsibility is somehow based on the phenomenological sense I have that I am the source of my actions, alongside my tendency to project that same relationship of a person to their acts onto others. If the causal thesis is true, then I am fundamentally misled about my phenomenology, which is of my being an ``ultimate source'' of my actions. If this is correct, then moral responsibility does have an appearance/reality problem, and there may be an important difference between holding responsible and being responsible. However, a great deal hangs on whether ``ultimacy'' is indeed a part of pure phenomenology. There is certainly some reason to think that phenomenology is in some sense ``of my doing x'', and we might raise concerns about ultimacy being the result of enculturated or confabulated interpretations of phenomenology. However, for these concerns to be plausible, we would require another plausible interpretation of the phenomenology such that we are not systematically deceived about our experience. If our sense of self as source does not contain any implicit claims about our being an ``ultimate'' cause what alternate interpretation is available? 

We can see that concerns about fairness and about the self as a source both track back to a formulation of the problem raised in Nagel's ``Moral Luck'' \citep{nagel1979b}. For Nagel, the challenge for moral responsibility is that it is supposed to eliminate moral luck - anything we are held morally responsible for is supposed to be ``up to us'' in a deep sense, and there seem to be a number of ways in which that is simply not the case. The fourth, and most pressing kind of luck is ``constitutive moral luck'' - being lucky (or not) in having the will that one has given that the will is shaped by some combination of given biology, upbringing, and environment. Since everything that one is is ultimately reducible, on this account, to these conditions outside of the individual's control, then the quality of our will is itself something we are lucky (or not) to have. But according to Nagel:

\begin{quote}
Something in the ordinary idea of what someone does must explain how it can seem necessary to subtract from it anything that merely happens, even though the ultimate consequence of such subtraction is that nothing remains [...]. The problem arises, I believe, because the self which acts and is the object of moral judgment is threatened with dissolution by the absorption of its acts and impulses into the class of events. Moral judgment of a person is a judgment not of what happens to him, but of him. [...] something in the idea of agency is incompatible with actions being events, and people being things. [...] nothing remains which can be ascribed to the responsible self, and we are left with nothing but a portion of the larger sequence of events which can be deplored or celebrated, but not blamed or praised \citep{nagel1979b}.
\end{quote}

The Causal Thesis threatens moral responsibility because it threatens the very idea of  persons (as agents) - it reduces persons into things, and actions into events and moral responsibility requires that there be a distinction. If persons in some coherent and phenomenologically adequate sense do not really exist, then moral responsibility, which takes persons as its proper targets is defeated because nothing exists that qualifies as a proper target of praise or blame. If this is correct, however, then in order for this argument to really defeat moral responsibility, then it must be the case that the Causal Thesis really does show that persons in the sense needed to be coherent targets for moral responsibility do not really exist. In the next chapter, I will more closely examine the way that personhood is supposed to be undermined by the causal thesis. If we can generate a description of ``self as source'' that meets the criteria of fairness and phenomenological adequacy, then the Causal Thesis might no longer provide a defeating reason against moral responsibility.

\chapter{Responsibility and the Self}
In the previous chapter, we saw that the challenges to moral responsibility from concerns about fairness and phenomenological adequacy are both concerned with the way that the causal thesis seems to undermine a sense of ourselves as persons. The modified naturalistic argument has helped to clarify the question of moral responsibility. We do not need a reason to hold others morally responsible. Instead, moral responsibility is understood as grounded in rationally moderated and constrained emotional responses to things we do and do not approve of, and the emotional foundation that generates our commitment to responsibility is just a natural feature of human beings. However, these responses are still prone to being systematically defeated by reasons and the causal thesis may be able to systematically defeat our commitment by showing that if the Causal Thesis is true, then holding people morally responsible is unfair and that it is based on a misleading phenomenology. Both of these objections turn on the more fundamental claim that if the Causal Thesis is true, then moral responsibility takes human beings to be something that they are not: a self that is the ultimate cause of its actions. Thus, in both objections, moral responsibility is grounded in a mistaken view of ourselves and others, and both of these things serve as defeating reasons against moral responsibility, if the Causal Thesis is true. 

In this chapter, we will explore the idea of the self that underlies these objections. Both agent causal libertarians and hard incompatibilists have a common idea of the self that must obtain for us to be morally responsible, an agent cause that stands outside the regular causal order. The basic argument as given by Galen Strawson rigorously and clearly spells out an influential hard incompatibilist objection to moral responsibility, as does Derk Pereboom's four-stage argument. I will argue that both of these arguments are flawed because both premise that agent causes do not exist, yet continue to insist that we identify ourselves with the agent cause as who we ``truly'' are. Both make use of an identification with a thing they claim is nonexistent in the structure of their arguments (with the explicit result in Galen Strawson's argument that we do not ``truly'' exist). This incoherent commitment to the agent cause as ``true'' self is the core of the problem. This is a commitment we are not supposed to jettison on pain of ad hoc revisionism, since this idea of the self is what people are naturally supposed to be committed to.

\section{The Thin Self}
In chapter 3, I argued that two major challenges to moral responsibility really track back to concerns about the self that arise if the Causal Thesis is true. The objection from phenomenology argues that moral responsibility is grounded in our lived experience, which represents us as free sources of our actions. If the Causal Thesis is true, then that lived experience is mistaken, we are not free in the way our lived experience suggests that we are because we are not the kinds of beings that our experience of freedom leads us to believe. In the discussion leading up to his Basic Argument, Galen Strawson attempts to point to this sense of freedom:

``Some philosophers may insist that they still do not really understand what kind of freedom is in question. But if they do, they are being (tactically) disingenuous, for the freedom in question is a property, real or imagined, that nearly all adult human beings - in the West, at least, and not just in the West, believe themselves to possess. To say that one doesn't understand what it is to claim to lack the most basic understanding of the society one lives in, and such a claim is not believable.'' \citep[p.2]{strawsong2010}. 
Later, Strawson claims that our commitment to this idea of freedom is grounded in our experience of our own agency, which is integral to our understanding of ourselves as ``a self-determining planner and performer of action, someone who can create things, make a sacrifice, do a misdeed.'' \citep[p.95]{strawsong2010} The problem seems to be that the Causal Thesis casts this view of our lives as illusory, and makes it impossible to really understand ourselves as occupying the role we feel that we do in our own lives. According to Strawson, ``true moral responsibility'' is defined in terms of our being ``truly self-determining'', and we are only truly self-determining if we have ``true freedom.'' True freedom is understood as what we need to have to be ``truly morally responsible.'' While this definition is obviously circular, Strawson is not troubled by this, as the circle does help to clarify the mutual interdependence and relationship between these concepts. These three elements must all stand or fall together, and all are embedded in ordinary folk intuitions about freedom. 
The second objection mentioned in chapter 3 is the objection from fairness, and this argument also relates back to our idea of who we are. If the Causal Thesis is true, then to pick a person out as morally responsible is to arbitrarily pick them out from a chain of causation that started well before that person was even alive. If the Causal Thesis was false, then picking a person out as responsible would not be arbitrary, or so the reasoning might go, because a person as an agent cause initiates new forces in the world which can be traced back to their self as the origin. The Causal Thesis makes the self that our actions are supposed to be traced back to impossible. This set of concerns is captured in Galen Strawson's basic argument \citep[p24-5]{strawsong2010}:

\begin{description}
\item[(1)] Interested in free action, we are particularly interested in actions performed for reasons, intentional actions as opposed to reflex actions or mindlessly habitual actions. We wish to show that such actions can be free.
\item[(2)] How one acts when one acts intentionally for a reason is, necessarily, a function of or determined by, how one is, mentally speaking - in certain respects, at least.
\item[(3)] If, therefore, one is to be truly responsible for how one acts, one must be truly responsible for how one is, mentally speaking - in certain respects, at least.
\item[(4)] But to be truly responsible for how one is, mentally speaking, in certain respects, one must have chosen to be the way one is, mentally speaking, in certain respects. It is not merely that one must have caused oneself to be the way one is, mentally speaking; that is not sufficient for true responsibility. One must have consciously and explicitly chosen to be the way one is, mentally speaking, in certain respects, at least, and one must have succeeded in bringing it about that one is that way. 
\item[(5)] But one cannot really be said to choose the way one is, mentally speaking, in any respect at all, unless one already exists, mentally speaking, already equipped with principles of choice 'P1' - with preferences, values, pro-attitudes, ideals, whatever - in the light of which one chooses how to be.
\item[(6)] But then to be truly responsible on account of having chosen to be the way one is, mentally speaking, in certain respects, one must be truly responsible for having these principles of choice P1. 
\item[(7)] But for this to be so, one must have chosen them in turn.
\item[(8)] But for this, i.e. (7), to be so, one must have already had some principles of choice P2 in the light of which one chose P1.
\item[(9)] And so on. True self-determination is logically impossible because it requires the actual completion of an infinite regress of choices of principles of choice. 
\end{description}

The Basic Argument explicitly spells out the point that we cannot choose to be who we are, yet what we do is a result of who we are. Thus, everything we are, and thus everything we do is the result of things that we are not - what Derk Pereboom elsewhere calls ``alien-deterministic events''. \citep[p.48-50]{pereboom2001} 

One point implicit in the basic argument, but which Strawson puts less emphasis on is the fact that when we say that someone is responsible, the person must be responsible for something. On the other hand, Bennett specifically cites this relationship, saying that for accountability, someone's actions must relate ``in a certain way'' to their decisions.  \citep[p.16]{bennett1980} To say that ``Jack is responsible'' full stop, without an explicit or tacit understanding of what Jack is responsible for does not make much sense. On the other hand, to say ``Jack is responsible for a heinous act of violence'' makes sense, because to be responsible for something is to bear a special relationship to something else - something like ``is morally answerable for''. The key element, however, is what the nature of the relationship between Jack and his actions needs to be. 

Strawson's basic argument is centered on the claim that the requisite relationship is something he calls ``true responsibility'', which requires that a person must be responsible for the way they are, mentally speaking, in order to be responsible at all. Another way of thinking of the thrust of the basic argument is that ``true freedom'' requires persons to be causes of themselves, and the basic argument shows us that to be the cause of ourselves, we would need to actually complete an infinite regress. If you deny the existence of a person being a cause of themselves - which hard incompatibilists and compatibilists agree on, then Strawson claims that the kind of freedom that people unreflectively believe they have turns out to be impossible.  

However, there is a bit of an oddity in Strawson's argument, one which is partially obscured by the use of the impersonal pronoun ``one'' throughout the argument. The problem is that the basic argument treats the relation (or really, the relata) in ``is responsible for'' inconsistently - the argument makes a fundamental commitment to and denies the possibility of the existence of an agent cause such as the agent causal libertarian believes in. This inconsistency is not entirely unconscious. It arises in part from the claim that folk intuitions are being taken seriously. The collapse of the reasoning into inconsistency is taken to be a problem for the idea of the agent cause itself. However, once the exact nature of the inconsistency is tracked down, it will be difficult to attribute this to taking this purported folk intuition seriously - but rather, only half-seriously. Furthermore, it will be argued that attributing the agent-cause theory to our natural belief about free will is also a questionable step to take. To clarify this problem, we will apply the reasoning behind the basic argument to a particular individual, Jack.

Suppose Jack commits a heinous act of violence because Jack is a villainous, cowardly person. Jack is not ``truly'' responsible for the heinous act of violence because Jack did not choose to be a villainous, cowardly person - Jack was determined to be villainous and cowardly by his heredity and environment.

This argument claims that Jack is not morally answerable for the heinous act of violence because Jack is not answerable for his villainy and cowardice, since he (Jack) did not choose it. The first thing to notice is the nature of the responsibility relationship itself. In the responsibility relationship, we might think of the two sides of the relationship as being the object of responsibility (what is to be answered for, or perhaps ``the crime'') and the subject of responsibility (what is going to be doing the answering for - ``the defendant'').  In the case of Jack and the heinous act of violence, Jack is the subject held accountable for the object: the heinous act of villainy. When we are considering whether Jack is responsible for being villainous and cowardly, we are considering Jack's villainy and cowardice in the same way we consider Jack's heinous act of violence - as a potential object in the world that we might relate back to Jack by saying he is responsible for it. Jack as a subject is the person who might be morally evaluated for being villainous and cowardly. We are pushed into this split because Jack is not the ultimate metaphysical origin of the villainy and cowardice (he was born/raised to be that way), so the villainy and cowardice become objects of evaluation - part of the crime, rather than the defendant. These objects are distinct from the subject of evaluation, and the subject's relationship to those features is drawn into question - is Jack responsible for being that way? 

This process of reasoning objectifies Jack's character. It adopts a perspective where those features  are evaluated as contingently part of who Jack is. This form of objectification is fairly common and natural. Jack  might look at himself and think ``it would be better if I didn't have these characteristics of cowardice and villainy, and instead was brave and heroic.'' We might even imagine Jack attempting (and failing) to change these aspects of his character. All of this makes perfect sense from the perspective of the agent causal libertarian, who would see Jack as an agent, attempting to change his cowardice and villainy, and morally evaluating Jack-the-agent as either innocent or guilty of having failed to make that change. 

Strawson's basic argument, however, makes this impossible. Through the regress that his argument introduces, every aspect of Jack will eventually be objectified to account for Jack's cowardice and villainy (either being the basis for his cowardice and villainy, or the basis for that basis, etc) until they lead out to alien-deterministic events. The basic argument does away with moral responsibility not by doing away with the wrongness of the acts in question, but by doing away with the subject or defendant being considered altogether. All of Jack's personal characteristics, everything Jack is has been objectified, and there is no Jack-the-subject to stand trial.

This is fully in accord with Strawson's own understanding of his argument. Strawson proposes a thought experiment in which we are invited to spend a few moments regarding every action we take, and every thought we have as fully determined by something that is ``not us'' - or in other words, to live and experience our lives with a consistent occurent belief that everything we do is causally determined by alien-deterministic events. He tells us that there are two results from seriously engaging in this experience. The less engaged result is that we might maintain a sense of self, but have the feeling that there is nothing for the self to do. The more engaged result is that the ``mental someone'' that we are goes out of existence altogether. The full effect of this experiment is for it to dissolve the sense of self altogether - not only is there nothing for the self to do, there is nothing left for the self to be. \citep[p 2]{strawsong2010}. It is, however, interesting to note that elsewhere, Strawson expresses commitment to a mental self of some sort. In The Self \citep{strawsong1997}, Strawson argues that there is an existing self - a single distinct mental thing that is the subject of experience. However, this self need not be an agent, nor does it need to have unity over time. 

Here we see that in Strawson's three part circular definition of ``true moral responsibility'', ``true self determination'', and ``true freedom'', there was a missing piece: the ``true self.'' The true self is the lynchpin around which the others must turn. As I will argue, the reason why we don't have the ``true'' version of responsibility, freedom, or self-determination for Strawson is actually because we lack a true self.

When the agent-causal libertarian engages in objectification of the self - they are essentially analyzing Jack as an agent cause and noting that all of Jack's characteristics were contingently part of who Jack is - Jack-the-agent could have had other characteristics, and may be culpable or not for the choices that led him to be who he is, and he sits in the defendant's chair as the core ``self'' being accused of the crime, not only of a heinous act of violence, but of his cowardice and villainy. The crucial question for the agent-causal libertarian is whether Jack-the-agent is guilty or innocent of those actions and characteristics.

The problem the Causal Thesis raises for the agent-causal libertarian's approach is that when we look at the situation from the perspective of the basic argument, the accused disappears, and there is no one left to be guilty of the crime, so to speak. But at the same time, it no longer makes any sense to say that Jack didn't choose to be the way he is, because for Strawson's argument, crucially, the Jack that did or didn't choose to be the way he is doesn't exist in the first place.   That Jack has been eliminated by the disbelief in agent-causation, and everything else that Jack might be has been objectified - it is the substance of the crime itself, and not the accused. While Strawson might in one sense welcome this result, as he does not believe that self-as-agent exists in the first place, on the other hand, his argument makes use of that self in the structure of the objection as the thing it is supposed to be wrong to hold responsible. It is present in the argument as the thing that it is wrong to hold responsible for all of the things that Jack is. Essentially, the argument states that for Jack (the person-in-the-world) to be responsible for the heinous act of violence, it must be the case that Jack (the one Strawson says does not exist) chose to be cowardly and villainous. This happens because everything that Jack is has been objectified, and in so being objectified, is unfit to be the subject of responsibility. What remains in Strawson's argument is a thin self - a Jack that has no characteristics, no existence - he is a mere logical placeholder in the argument whose only role is to not be responsible for everything else about Jack, making holding Jack responsible problematic. 

We see this same erosion of the self in Derk Pereboom's four stage argument. In the four stage argument, Plum goes from being manipulated by local neuroscientists, to being preprogrammed by neuroscientists, to being rigorously trained to be a certain way, to being ordinarily determined by heredity and environment. In the first case, Plum qualifies as ``free'' under a number of major compatibilist descriptions of freedom, yet intuitively looks to be unfree in the first case (being directly manipulated by neuroscientists). The supposed explanation for the lack of freedom is changed from case to case, and yet the intuition still remains that Plum is not free, until we get to ordinary determinism \citep[p 112-6]{pereboom2001}. In each stage, we are presented with what is apparently a self being manipulated by something alien and not-self - in the first two cases, it is the neuroscientists, in the third, it is the brainwashing, and in the last, it is physical determinism. In all cases, we refuse to admit that Plum really is the ostensibly alien element. But all of the work in this argument is really done by the fact that determinism is being presented as an alien ``not-self'' interloper in the agency of the epiphenomenal true self - the ``not-I'' that I cannot rightly be held responsible for, and the plausibility of step 4 is really dependent on the implicit claim that all parts of the system of physical determinism are ``not-I''.

Applying the four stage argument to Jack, then in the first step, we say Jack is not responsible for the heinous act of violence because Jack is not responsible for the neuroscientists who made him act. In the second, Jack is not responsible because the neuroscientists made him villainous and cowardly, and in the third, because his training made him villainous and cowardly. In the fourth, it is because  determinism made him villainous and cowardly. However, in cases 2-4, we are once again faced with the fact that villainous and cowardly is really all there is for Jack to be. When we say that Jack is not responsible for the heinous act of violence, the Jack that is not responsible is thin - everything that Jack could be is contingent and an object of choice, and thus not ``true'' Jack.  

The four stage argument does help to illustrate some of the potential importance of the thin self - the cases of ``distant'' neuroscientist manipulation (case 2) and thorough brainwashing (case 3) highlight the fact that were it not for these manipulations, Jack might have ended up with qualities other than cowardice and villainy, and thus Jack's being cowardly and villainous is a result of his victimization at the hands of the neuroscientists or brainwashers. However, in these two cases, the Jack we have as a result of these manipulations is the only Jack that exists, cowardly and villainous - and  to say that Jack is or is not responsible for being who he is is to plead on behalf of a thin ``true Jack'' to say that he is not responsible for his cowardice and villainy. In both of these cases, the sense that there is a true Jack is made all the more plausible by the fact that the manipulations by the neuroscientists or brainwashers makes the Jack we have plausibly a ``false Jack.'' The neuroscientists and brainwashers are to blame, if anyone is, about how Jack is. 

The crucial question is whether this same reasoning transfers over to case four. The compatibilist can argue that ``true Jack'' in case 2 and 3 is Jack in the absence of such manipulation, and simply determined by ``normal'' causal determinism, whereas the incompatibilist maintains that the workings of causal determinism are just as inappropriate to be ``true Jack'' as are the machinations of neuroscientists or brainwashers. However, in this case, just as in Strawson's basic argument, this leaves us with an impossibly thin Jack whose role is essentially as epiphenomenal target for blame. 
The incompatibilist might argue that this seeming incoherence is really the result of our ordinary practices which have two incompatible commitments - to the causal thesis, and to the self as agent cause, and incompatibilist arguments only make explicit the result of these two commitments - an impossibly thin self has to serve as the morally responsible subject, yet cannot, because the thin self is incoherent. This shows that moral responsibility has a fatal problem, not that hard incompatibilism does. 

However, there are two telling problems with this analysis. The first problem is that it is not simply that holding responsible relies on the thin self, as the hard incompatibilist would have it. To argue that there is we should not hold a person morally responsible because of the causal thesis requires that there is some person who it would be wrong to hold morally responsible, and this is just what we lose in the thoroughgoing objectification that leads to the thin self. To say that Jack is not responsible for being cowardly and villainous is to make a distinction between Jack and his characteristics of cowardice and villainy because being cowardly and villainous are only contingently who Jack is. For the hard incompatibilist, however, any and all characteristics of Jack are only contingently who Jack is. But to say that Jack is not responsible for who he is holds onto an idea of Jack that is distinct from everything about him: the thin self.  Essential to the hard incompatibilist argument is the contingent relationship between Jack and his cowardice and villainy and everything else about him and the claim that things could have gone otherwise for Jack. The argument that we should not hold Jack responsible relies on this.

The alternative, of course, would be to say that the claim is really that we should not hold villainy and cowardice responsible for being villainous and cowardly (this must be thought of as distinct from saying that we should not hold a villainous and cowardly person responsible for being villainous and cowardly - the ``person'' re-introduces the contingency of the attributes). On the face of it, it is simply not clear what is wrong with blaming villainy and cowardice for being villainous and cowardly. However, there is more to this objection than appears - there is something to the idea that responsibility (rather than simply disapproval) requires contingency in some sense. However, as I will argue in chapter 5 - there is a way that we can make sense of contingency without appeal to the thin self, or without requiring anything like metaphysically free choice. 

The second problem is that both the basic argument and the four stage argument actually operate by claiming that all targets of moral responsibility are inappropriate targets, not that responsibility itself is incoherent. Pereboom's four-stage argument attempts to show that Plum is an inappropriate target for moral responsibility by showing that there is no real difference between obviously inappropriate cases of moral responsibility, and those we take to be appropriate. The Basic Argument attempts to undermine moral responsibility by showing that nothing could be a proper target of moral responsibility, because nothing could have chosen to be the way that it is. In both cases, it is not that moral responsibility is shown to be wrong, it is simply that the targets we pick are wrong, and moral responsibility is undermined overall by showing that for anything or anyone in the world, they are not an appropriate target. Showing that moral responsibility is ``broken'' operates, in both cases, by showing that there are no appropriate targets - blame is being shifted away from the target, and not simply dissolved. (Even in Galen Strawson's examples, blame dissolves only after it is shown to have no appropriate target, and then only in the best of cases). Both cases work by employing the idea of an appropriate target - what I am calling the thin self - that has powers that neither argument thinks are even coherent, much less possible, and which has no features of its own. Neither denies the coherence of responsibility itself, and so the question of who is morally responsible still comes up. 

The most pressing problem, however, is that as we saw in the modified naturalistic strategy, we do not need a reason for the question of moral responsibility to ``come up'' - the question emerges as we try to control and focus our emotional responses to things that other people do that we like or do not like. Belief in moral responsibility does not motivate the attitudes, the attitudes motivate the beliefs. The real lesson of the naturalistic strategy has been that the appropriate question for moral responsibility is not ``is it reasonable to hold people morally responsible?'' but ``is it unreasonable to?'' Responsibility cannot simply be denied, it has to be defeated. The question here is really whether these arguments can really defeat the natural inclination toward moral responsibility. 

In both the basic argument and the four stage argument, incompatibilist conclusions are motivated by the claim that it is wrong to hold Jack responsible. Both attempt to defeat moral responsibility by pointing out that someone is wronged. However, in crucial points in both arguments, there is a switch between the Jack about the world, who takes up space, has a particular personality, and engages in actions and a ``thin'' Jack. This ``thin'' Jack is distanced from Jack-in-the-world (and thus cannot be blamed for being Jack-in-the-world) while at the same time, his existence is denied. This is the result of the wholesale objectification of the self that regards all of Jack's characteristics and features as only contingently Jack is. In another way of putting it, the reference ``Jack'' is fixed to the agent cause for both the libertarian and the hard incompatibilist, but then the hard incompatibilist denies that the agent cause exists, and rather than deciding that perhaps the reference for Jack was fixed inappropriately, they simply accept that Jack (as a morally responsible agent and ``mental someone'') does not exist. 

Thus the prevailing tactic has been to try to defeat moral responsibility by showing that the targets of moral responsibility are not appropriate targets. However, this strategy trades on the idea of an appropriate target - the thin self or the agent cause - to do all of its work. The idea of an agent cause allows incompatibilist arguments to go through but only if we are agent causes, precisely because there is a meaningful contrast between the appropriate target (Jack as agent-cause) and inappropriate (Jack-in-the-world). The thin self is without content. It is a placeholder for an appropriate target that does nothing more than serve as a standard by which all other targets are judged as inappropriate. But if all that the thin self does is stand as a standard that nothing can actually meet, then why insist on it in the first place? 

One possible response to this line will be to invoke an alternative version of the self to do the work of bearing (wrongfully) the attributions of moral responsibility. We see a possible version of such a self, a non-agentive conscious self, at the end of Galen Strawson's \citeyearpar{strawsong1997} paper ``The Self''.  There, the self is a single distinct mental thing that is the subject of experience, but crucially it is not an agent. We might suppose that this experiencing conscious self is the thin self, and there is something legitimate about worrying about giving that self bad punishment experiences, because it did not choose to act or be the way it is. There is an ``innocent'' subject: a person understood only as a conscious being. We do want to protect this self from what the consequences of the actions of the self as object, in particular the various non-self factors that made it what it was. 

However, this objection cannot work. Strawson commits himself to materialism in this model of the self . Specifically, he claims that every thing has a non-mental being, including those with a mental, experiential character. I will follow Strawson is supposing that this is in the brain, and its activity. This is the same brain which is presumably some part of the cause of how we act. The problem here is that for Strawson, as a materialist, the experiencing self, the acting self, and the brain/body cannot be metaphysically distinct things. The self that experiences punishment is made up of the same ``stuff'' that the self that has engaged in the blameworthy behavior. If this is true, then there is no way to objectify the person as an agent without at the same time objectifying the person as a subject of experience. It cannot be unfair to hold the experiencing self responsible for what the acting self has done because they are the same thing. 

The hard incompatibilist may instead grant the incoherence of both holding the self innocent of responsibility while maintaining that it does not exist. However, here, they may argue that the standard against which moral responsibility is evaluated is not something that they have introduced - the standard comes from ordinary folk intuitions. The problem of the thin self that occurs in the basic argument and the four stage argument come from ordinary folk intuitions which attempt to maintain and argue from the position of the agent cause - which is required for moral responsibility - while at the same time admitting to the impossibility of the agent cause. The basic argument and the four stage argument both simply highlight this problem, and demonstrate the implications of the inconsistency in ordinary thought about moral responsibility. 

Of course, it is crucial to both of these arguments that the agent cause is actually required for moral responsibility, and in both cases, this is essentially built into the front of the argument rather than argued for. In the basic argument, we are compelled to eliminate anything else the person could be from the argument by the reasoning that in order to be responsible for choosing what we do, we must also be responsible for choosing who we are - the only way such a choice is possible would be if there were a point from outside everything we are to make that choice. The four stage argument similarly eliminates anything we are from the picture in the fourth step, where causal determinism stands in for the coercive elements (brainwashing and remote-control) that eliminated any sense in which the agent was responsible in previous steps of the argument. Both arguments have built into them the standard that nothing fully integrated into the causal chain of events in the world are fit sources of moral responsibility for us.

This is not as problematic as it might seem up front, because both of these moves are supposed to be intuitive. Basic folk intuitions easily assent to eliminating someone as morally responsible for something if the person's relationship to their action is as described in either the basic argument or the four stage argument. This also highlights the importance of the claim that Galen Strawson makes about ``true'' moral responsibility and ``true'' freedom - the forms of freedom that Strawson believes we are all familiar with, and that almost everyone (agent causal libertarians aside) admits does not exist. However, the weight of both of these arguments rests on the plausibility of both the claim that a commitment to the agent cause is actually what our ``natural'' intuitions really are, and the claim that it actually matters what our natural intuitions are. It is to both of these points that we now turn.

\section{The Importance of the Thin Self}
The basic argument and the four stage argument might both be criticized along two lines of attack. They could be criticized as wrongly supposing that our basic intuition includes an essential commitment to the disembodied perspective of the agent cause/thin self in order to maintain a commitment to moral responsibility. Alternately, it might be argued that even if our intuition does implicitly connect moral responsibility to the agent cause, that is so much the worse for intuition, not for moral responsibility. There has been a great deal of debate concerning the first argument, attempting to track what our intuition really is. However, before addressing that debate, it is first worth addressing the second challenge which says ``so much the worse for our intuition.''

In many ways, compatibilists and hard incompatibilists largely agree that if intuition and the realities of our moral situation come apart, then so much the worse for intuition. The major disagreement has really been about what and how much is lost if our intuition is wrong, and we are not agent causes. Compatibilists tend to argue for much less revision - though we see some revision may be needed in semi-compatibilism \citep{fischerravizza1998}. On the other side, Derk Pereboom has argued that much of the function and even the meaning of moral responsibility can be preserved - the only fundamental change is the loss of the use of the concept of desert, and even that is argued to have less impact that we might otherwise expect \citep{pereboom2001}. Strawson, similarly, fully expects compatibilists to accept the basic argument, and agree that the kind of freedom expected in that argument is impossible, and that it is what at least some people mean when they talk about freedom \citep{strawsong2010}. In some ways, the gap between hard incompatibilists and compatibilists has gotten smaller, and in some ways the differences between the positions are becoming less about substantive disagreements about what is or is not the case, but rather over our attitudes toward freedom and moral responsibility. 

This tracks well onto the terminology expressed in P. F. Strawson's seminal ``Freedom and Resentment'' which splits the field between two camps: optimists who feel that moral responsibility is not threatened by determinism, and pessimists, who feel that it is. We have seen a return to this terminology in Dennett \citeyearpar{dennett1984} as well as Russell \citeyearpar{russell2002}, and this return does capture an important aspect of this debate. Dennett frames his arguments as responses to incompatibilist ``bogeymen'' that frighten away our commitment to moral responsibility.

Russell cautions about the wholesale dismissal of incompatibilist ``bogeys'', and describes some of these fears as arising at different levels of ``distance'' and some of these fears - particularly those at the distance that Russell describes as being at the horizon - are legitimate. These horizon fears are focused on issues of ultimacy - issues about whether we have a real say about who we are and a final say in our own conduct \citep{russell2002}. It is at this horizon that both Strawson and Pereboom's arguments take us, and it is worth considering Dennett's own tactic of dismissing such fears - a form of optimism which says ``so much the worse for intuitions.''

In Elbow Room, Dennett gives us a naturalistic account of free will which is largely practically oriented and sticks to the common usage of most of our talk about free will. However, when reaching these horizon concerns about ultimacy, Dennett is willing to jettison what seems to be our ordinary talk about moral responsibility - what he calls the ``absolute-concept-of-guilt-before-the-eyes-of-God'' conception of guilt \citep[p.166]{dennett1984}. This is remarkably on track with the kind of moral responsibility Galen Strawson is interested in, which he describes as being the sort of responsibly we would need to have for it to make sense for us to be condemned to hell or rewarded with heaven \citep[p.2]{strawsong2010}. This form of responsibility - the kind Strawson refers to as ``true'' moral responsibility is to be simply set aside because it is impossible. According to Dennett, we should content ourselves with the kind of freedom and moral responsibility that is possible for us and which satisfies a great deal of what we are typically concerned with: the ability to control and decide our courses of action, and to have some control to choose other than the most immediate and ``desperate'' means to satisfy our desires. 

Russell, however, argues that this response is contrary to the spirit of most of the rest of Dennett's approach, shifting from a ``descriptive metaphysics'' approach, which attempts to describe the world as it actually appears in our practices, to a ``revisionary metaphysics'' approach, which attempts to prescribe a certain view of the world - a view absent the kind of moral responsibility that we may actually be committed to. According to Russell, Dennett's preferred view of moral responsibility is utilitarian and forward looking \citep{russell2002} - very much in the spirit of Smart's approach in chapter one, with, perhaps, a few more of the ``functions'' of moral responsibility filled out, so that moral responsibility does more of what we want it to. In essence, Dennett urges us to drop an impossible standard because it is impossible, and to continue on with the many elements of moral responsibility that we can defend as perfectly functional. 

According to Russell, one element which Dennett's account misses is precisely the sort of engagement in moral responsibility implicit in the reactive attitudes. However, even if we supplement Dennett's account with the reactive attitudes, then we still return once again to what I have called (in chapter 2) a ``defeating'' argument against the reactive attitudes in this case - namely that the lack of ultimacy gives us a reason not to hold people morally responsible \citep{russell2002}. As we have already seen, the problem for the reactive attitudes is the lack of a proper target - which must be a person in some robust sense which the idea of the agent-cause is meant to stand in for. Dennett's proposal, much like Smart's, is essentially to set aside the engagement of the reactive attitudes altogether, and salvage what remains of moral responsibility, which, as it turns out, is a great deal.

However, one might worry that Dennett (and other compatibilist accounts) are using the terms of ordinary moral discourse in slightly different ways - re-appropriating them to continue moral practices in a world that is quite different from the context in which those practices were formed. The emphasis that Galen Strawson puts on ``true'' moral responsibility and freedom relates to this worry about the re-appropriation of most of the mechanics of moral responsibility, and so for Strawson, Dennett's compatibilist moral responsibility is a kind of ersatz moral responsibility; perhaps good enough in a pinch, but not what we thought we had (Strawson more charitably calls the responsibility and freedom Dennett seems to be talking about ``basic''). The disagreement between Strawson and Dennett is really about what standard we should have for saying that the moral responsibility we are capable of is really connected to the kind of moral responsibility that most of us believe we have, and to which we orient our decisions and practices. Strawson resists dropping the standard because it involves moving the goalposts to make an account of freedom that affirms our ordinary moral practices easier to achieve. Dennett focuses on the practical and urges that the freedom and the moral responsibility we can achieve should be our focus. But just as we saw with Smart in chapter 1, Dennett's pragmatic approach is in danger of neglecting one of the potential functions of moral responsibility and ignores some of the important affective elements of moral responsibility. 

Part of what we may be looking for in an understanding of moral responsibility is an affirmation of our values and our lived experience. If the standard that Strawson argues for is really tied to our attitude toward moral responsibility and morality in general, then Dennett's account really does lose something important that an agent causal theory might provide, were it true. It would cut out the reactive attitudes which are responsible for our initial engagement in issues of moral responsibility in the first place. Dennett's utilitarian defense of moral responsibility transforms it from something we are engaged in personally, into a mere instrument of social control applied to ourselves and others. As such, its meaning changes, as do the opportunities for intrinsic motivation in moral responsibility. Both the basic argument and Pereboom's four stage argument adopt the perspective of the thin self to highlight the fact that if we are not agent causes, then something important is dropped out - not just practically, but existentially. Dennett is aware of the potential for ``existential funk'' in his position, but says simply that these moments usually pass, and like the golfer in a slump we need simply to ``keep our head down and follow through.'' \citep[p.168]{dennett1984} Existential engagement with moral responsibility becomes a dead end, and those with needs in those areas are advised simply to set them aside - cease inquiry and act. Other important questions must also be set aside - such as the difference between treating someone ``as if'' they were morally responsible, and responding to whether or not they actually are morally responsible.

Dennett's response to this problem leaves out some crucially important things that many people expect from an account of moral responsibility, and if moral responsibility is incapable of these things, then Dennett's optimism is clearly uncalled for. While it may not be the case that everyone is motivated to actually seek out the existential foundations of moral responsibility, it is certainly arguable that most engagement in the practice of moral responsibility assumes that those existential foundations are available even if not evident - often grounded past the veil of mystery in the will of God (despite more than two thousand years of questioning that very move). Similarly, a loss of that existential foundation (or belief that it is possible) often motivates very different attitudes toward moral responsibility - this certainly being one of the major motivating themes across Nietzsche's concern with nihilism, for example \citep{nietzsche2010}.

Therefore, setting aside the standard that Strawson and Pereboom show us in the basic argument and the four stage argument on the basis that the standard is unreachable is a costly response. Large and important features of moral responsibility are lost, some of which help to situate the high importance afforded to morality in general. Appealing to the unreasonableness of the standard set out by the basic argument and the four stage argument as a reason for rejecting it may ultimately have to be the tactic for compatibilists who wish to preserve something of moral responsibility, but in doing so, the fact that moral responsibility might not do everything we thought it did cannot simply be passed over in silence. 

\section{Objectification of the Self}
If we are to resist the challenge of the basic argument and the four stage argument, then another argument we can make is that the standards those arguments establish do not actually reflect the standards we actually have for moral responsibility. Pereboom establishes this standard through assent in his intuition-pumping four stage argument. Most people readily assent that the lack of freedom for Plum in each step of the argument is not mitigated by the compatibilist conditions built into each case, even when that case is shifted to straightforward determinism  \citep[p.116]{pereboom2001}. Galen Strawson builds the standard into the assent to the premises of the four stage argument - crucially to (4) and (5), which say that to be responsible for what one does, one must be responsible for how one is, and that one must already exist to be responsible for how one is. \citep[p.25]{strawsong2010}. Assent to these arguments seems to make the case that this is what people's intuitions actually are.

The battle of dueling intuitions has seemed to be the real battleground for this debate. However, for reasons reviewed in chapter 3, we should not be convinced that we currently have the methods to determine what ``our'' fundamental intuitions really are. Furthermore, it may also not be the case that a person's reports on the intentional content of their lived experience will always be certain and reliable reports of what the intentional content really is, since, as Horgan argues, reporting that agentive phenomenology is as of being an agent cause involves a theoretical judgment applied to that phenomenology, and involves a fairly abstract and specialized bit of knowledge (what it is like to be an agent cause, vs what it is like to not be one) \citep{horgan2007}. 

Horgan's distinction between transparent reporting on phenomenology, and judgment about phenomenology is important. If Horgan is correct about the fact that ``being an agent cause'' is a judgment, then there is a way of distinguishing between an affirmation of our lived experience in the form of the raw phenomenology (including the actual intentional contents of that phenomenology), and an affirmation of our judgments about that phenomenology. In this language, the basic argument and the four stage argument might be operating treating our phenomenology and our judgments about what that phenomenology means as a single thing. If that approach is correct, then if the causal thesis is true, then we are not agent-causes and our experience of ourselves and of others is deeply mistaken. If, following chapter 2, finding out that we are systematically deceived about the world might defeat many of our ordinary responses to the world, including thinking of ourselves and others as morally responsible. Responding to this aspect of these arguments is a matter of showing that our real commitment is to our phenomenology, and not the judgment that our phenomenology is ``of being an agent cause''. In other words, what is important in a theory of moral responsibility is that it affirm our lived experience, and not the story we tell ourselves to explain that lived experience. On this view, the problem with the basic argument and the four stage argument is that they take the explanation of our lived experience to be the lived experience itself, which artificially puts the theory of agent causation in a vitally important place. 

If this account is correct, then it would explain why anyone might think that the theory of agent causation was somehow vital to moral responsibility, and it would show how we could deny that claim, while still admitting that the lived experience of agents is of primary importance in understanding what it is to be morally responsible. In fact, however, I think that the picture is somewhat more complicated. I think that our primary commitment is a commitment to our lived experience, and not the judgment that that experience is of being an agent-cause. However there are also likely elements of morality and moral responsibility that have been influenced by the historical role that the theory of agent causation has played, though I suspect our commitment to those is less deep and has much more to do with our cultural understanding of moral responsibility than our embedded reactive attitudes.

In a work of this scope, it will unfortunately be impossible to give a full defense of this historical claim. A full account would need to clearly articulate the difference between the lived experience and the account of that lived experience - a task I am not certain is even possible (anything generated to explain lived experience would simply be a new account of the lived experience, and get us no further). Furthermore, a robust history of the theory of the agent cause would be needed. Such an account would show the historical, rather than the phenomenological foundation of the theory of the agent cause (which is likely not even a single theory, but one which has changed and developed over time, and had numerous versions at a single time).
However, if a historical account of the role of the agent-cause in moral responsibility is at least prima facie plausible, then this would be enough to unsettle the default assumption that the idea of the agent cause is embedded in our phenomenology. Simply insisting on this as an intuition, either of a single philosopher or a large sample of the ``folk'' should no longer be accepted as sufficient to establish that we ``really think'' we are agent causes, and this provides room for considering the objectification account as an alternative that is, at least, no worse grounded. While a full historical argument is outside the scope of this work, it will still be worthwhile to present an argument for the prima facie plausibility of a historical account for the role of the idea of the agent-cause in the debate about moral responsibility. 

The core claim of such an argument is that the idea of the agent cause has a historical origin - and the judgment that we are really agent causes is largely a response to developments in our understanding of the world that seemed to depict the rest of the universe as being very different from our lived experience. The historical developments in our understanding that contributed to the development of the theory of the agent cause are captured in the still ongoing attempt to merge what Wilfrid Sellars called the manifest image and the scientific image of human beings. 

In his ``Philosophy and the Scientific Image of Man'' Wilfrid Sellars discusses the problem of a conflict between two incompatible world-views that human beings are nevertheless committed to. The manifest image is a more or less refined explanation that takes our lived experience as its foundation, and whose core category of understanding is person. This is to say that everything is essentially understood as a kind of person, which in less refined versions is akin to animism, where rocks and wind might have intentions, wishes, etc. but which, in more refined versions tends to depersonalize most matter - rocks and wind are understood as ``mindless'' and ``intentionless'' - taking the category of person and stripping the fundamental features of persons away so that these things become persons-minus. Even in the more refined versions of the manifest image, persons remain the most basic element, and everything else can be understood according to how much or little it shares the features of personhood. Sellars classes most continental philosophy, as well as 'ordinary language' analytic philosophy as versions of the manifest image, and emphasizes that these refinements are themselves scientific. \citep[p.494-5]{sellars1962}.

The scientific image differs from the manifest image in that the familiar category of person is replaced by postulated imperceptible entities - subatomic particles at this point in our scientific development. While in the manifest image, everything is understood as more or less of a person, in the scientific image, everything is understood in terms of the existence and activities of these subatomic particles, built up into increasingly more complex systems. Persons, too, are understood in this framework, so that - ``the scientific image of man turns out to be that of a complex physical system.'' \citep{sellars1962}.

The source of the conflict here is that both of these images cannot be true - we can see the conflict embedded in simple questions about physical objects - is the table I see the real table, or is it really a system of imperceptible particles,  and mostly empty space? Sellars believes that there are a few possible responses to this conflict:
\begin{quote}
(1)Manifest objects are identical with systems of imperceptible particles in that simple sense in which a forest is identical with a number of trees.
(2)Manifest objects are what really exist; systems of imperceptible particles being 'symbolic' or 'abstract' ways of representing them
(3)Manifest objects are 'appearances' to human minds of a reality which is constituted by systems of imperceptible particles. \citep[p.506]{sellars1962}
\end{quote}

He gives arguments against (1), and to a lesser extent addresses (2), but concludes that (3) must ultimately be the truth, and thus, the scientific image seems clearly to be the true image. The problem, however, is that this does not mean we can simply abandon the manifest image. According to Sellars:
``[...]man is essentially that being which conceives of itself in terms of the image which the perennial philosophy [the manifest image] refines and endorses.'' \citep[p.495]{sellars1962}.

According to Sellars, if we were to suddenly stop seeing ourselves as persons according to the manifest image, then something essential to our way of being in the world would fundamentally change. We can see how readily this captures elements of the free will debate. Galen Strawson's thought experiment - where we regard ourselves as fully determined by alien-deterministic events - is supposed to leave one ultimately with the feeling that the ``mental someone'' that one normally identifies with simply does not exist, as everything that it is or does is understood as a part of the ongoing chain of causation. We also see this same concern expressed by Thomas Nagel, when he discusses the idea of persons and actions being slowly absorbed into the class of things and events. The manifest image of human beings is absorbed into the scientific image, and it is that absorption that seems to threaten moral responsibility. \citep{nagel1979b}

All of this maps fairly directly onto the challenge presented by the Causal Thesis, which is certainly part of what Sellars describes as the scientific image. \footnote{Sellars actually says that the idea of a cause is part of the manifest image - but there he is focused on a cause, and ignoring the fact that everything that happens happens as a result of everything in the physical environment. The billiard ball would not have moved if not for the ball that hit it, but also the table, the air, the earth's gravity, etc. This is fully consistent with the Causal Thesis.}  The Causal Thesis suggests that everything that we do is ultimately traced back to some external cause, and fully integrated into the overall operation of nature. The commitment to incompatibilism might be thought of as a commitment to the link between moral responsibility and the reality of the manifest image. If we are simply systems of particles, however complicated, then moral responsibility is somehow flawed. On this account, the idea of the agent cause is bound to the manifest image - it is part of the irreducible category of ``person'', and the theoretical idea of the agent cause has been drawn from our lived experience. 
Horgan's insight about the distinction between phenomenology and judgments about phenomenology should make us question this. It may be that our lived experience as of a person and as of an agent in some sense is essential to our personhood. However, the question is whether the theoretical picture of ``agent cause'' is part of the essential picture we have of ourselves, or whether it is simply a (failed) attempt to explain some aspect of our essential humanity. 

What reasons are there to suppose that the idea of an agent cause emerged out of this conflict? There are a few aspects to the idea of the agent cause which certainly suggest a link. The fact that most explanations of the agent cause primarily characterize it by what it is not (uncaused cause, non-alien-deterministic, having contracausal freedom) - and that the point of contrast is invariably with some aspect of the scientific image (causation, determinism) itself demonstrates both a reactive tendency in the definition, and an essential link to the scientific image. These claims understand the idea of an agent cause by the fact that it is an exception to the scientific image - and characterizing it as an exception to that image requires that image to exist in the first place.

This account alone is, of course, not an argument. However, some support for the prima facie plausibility of this account can be found by examining when and why such a distancing might have seemed necessary. Essential to this distancing is the prominence of objectification in the scientific image - a point of view which seems to threaten our ordinary sense of personhood, in both ourselves, and to a lesser extent, others. We can see the emergence of objectification and an early response to objectification in the work of Descartes - a figure whom Sellars picks out as engaging in an early and important attempt to see the manifest and scientific image together. 

According to Sellars, Descartes and other interpreters of the new insights of Physics took on an increasingly mechanistic view of the physical world - conceptually grounded in the atomism of Democratus. However, this atomism was not a thoroughgoing one. Features of the world that seemed to play no role in mechanical explanation, such as color, were relegated to the world of appearance and understood as a state of the perceiving mind rather than a state of the world itself, a move that forms one of the foundational aspects of Cartesian dualism. Sellars tells us:

\begin{quote}
The same considerations which led philosophers to deny the reality of perceptible things led them to a dualistic theory of man. If the body is a system of particles, the body cannot be the subject of thinking and feeling unless thinking and feeling are capable of interpretation as complex interactions of physical particles; unless, that is to say, the manifest framework of man as one being, a person, can be replaced without loss of descriptive and explanatory power by a postulational image in which he is a complex of physical particles, and all his activities a matter of the particles changing in state and relationship.'' \citep[p.507-8]{sellars1962}.
\end{quote}

In his ``Sources of the Self'' \citeyearpar{taylor1989} Charles Taylor argues that this split between appearance and reality was a major element in the development of ``objectification.'' Of course, the appearance/reality distinction is nearly as old as philosophy itself - however, the distinction made by Descartes was of a fundamentally different kind than what we inherited from Plato, for whom reality was essentially patterned on the world of ideas - appearance was simply to imperfectly apprehend the genuine order of ideas. According to Taylor:

\begin{quote}
For Descartes, in contrast, there is no such order of Ideas to turn to, and understanding physical reality in terms of such is precisely a paradigm example of the confusion between the soul and the material we must free ourselves from. Coming to a full realization of one's being as immaterial involves perceiving distinctly the ontological cleft between the two, and this involves grasping the material world as mere extension. The material world here includes the body, and coming to see the real distinction requires that we disengage from our usual embodied perspective, within which the ordinary person tends to see the objects around him as really qualified by color or sweetness or heat, and tends to think of the pain or tickle as in his tooth or foot. We have to objectify the world, including our own bodies, and that means to come to see them mechanistically and functionally, in the same  way that an uninvolved external observer would.'' \citep[p.145]{taylor1989}
\end{quote}

For Descartes, objectification involved personal disengagement; seeing things, even ones own bodily experiences, as split into objects and appearances. Objects really exist, whereas appearances were really about the way those objects impacted our senses, and the fundamental confusion to be cleared up was in misunderstanding those subjective appearances for objective reality. Of course, for Descartes, there is also still a mental substance - fundamentally different in kind from the objects of extension. The full appreciation of that gap is played out in the Meditations, where Descartes engages in a program of radical doubt which suspends belief in everything except itself as a thinking thing, and must then reconstruct, on its own terms, the world of extension as something other than mere appearance.
 
To understand the impact that this had on the idea of the self and on our understanding of our own lived experience, we need look no further than Descartes' comments on animals. In his Discourse on Method, Descartes denies that animals have a ``rational soul'' - what we would now think of as mind. He makes this claim while being fully aware that many animals have analogous organs to human beings, and even some analogous behaviors. This, however, is not a persuasive reason to suppose that animals have a rational soul because, in fact, animals are actually like automata. Descartes compares animal behavior to clockwork, with the organs of the animal analogous to the springs and gears of the clock, and the ``apparently'' rational behavior exhibited by animals as analogous to a clock's ability to tell time, even outstripping our own ability in all of our wisdom, yet still mindless. \citep[p.141]{descartes1984a}

Here we see that all of the physical world of extension is essentially mechanistic, regardless of its apparent rationality. One's own mind is real and known clearly and distinctly, and as we have seen is of a categorically different kind than anything in the world of extension. Other minds present a bit of a problem for Descartes, however, we see that Descartes retains one avenue to show that other minds actually exist - the presence of linguistic ability proves the presence of the rational soul \citep[p.140]{descartes1984a}. However, this ability only evidences the existence of a rational soul because it is, according to Descartes, far too complex for mere mechanism to accomplish, not because language is inherently mental. Linguistic ability functions as a kind of reverse Turing-test.

We can see that what Descartes seems to be doing here corresponds to the depersonalization of the manifest image described by Sellars - fundamental characteristics of persons (like mind and subjectivity) are being categorically denied of the world of extension altogether - both mentality and extension exist, but they are now split by a vast ontological gap. It is equally clear that lived experience puts personhood on the ``mind'' side of that gap.  Descartes argues that the mental substance cannot be derived from matter (an argument that does not survive, but which he refers to in the Discourse on Method part 5 \citep[p.141]{descartes1984a}. We see in the Meditations that mind can doubt the existence of extension altogether, so that as much as I am essentially a thinking thing, the world of objects is essentially not me. 

This process of objectification also operates in the realm of the will. According to Descartes, the will is a boundless and nearly unlimited faculty. It is worth looking at some of Descartes' comments on this in detail:

\begin{quote}
It is only the will, or freedom of choice, which I experience within me to be so great that the idea of any greater faculty is beyond my grasp; so much so that that it is above all in virtue of the will that I understand myself to bear in some way the image and likeness of God. For although God's will is incomparably greater than mine both in virtue of the knowledge and power that accompany it and make it more firm and efficacious, and also in virtue of its object, in that it ranges over a greater number of items, nevertheless it does not seem any greater than mine when considered as will in the essential and strict sense. This is because the will consists simply in our ability to do or not do something (that is, to affirm or deny, to pursue or avoid); or rather, it consists simply in the fact that when the intellect puts something forward for affirmation or denial or for pursuit or avoidance, our inclinations are such that we do not feel we are determined by any external force. In order for me to be free, there is no need for me to be inclined both ways; on the contrary, the more I incline in one direction - either because I clearly understand the reasons of truth and goodness that points that way, or because of a divinely produced disposition of my inner thoughts - the freer is my choice. \citep[p.40]{descartes1984c}
\end{quote}

A number of points are worth taking note of here. First among them is the unlimited nature of the will - in some ways, equal with God, and certainly something in which no greater faculty can even be imagined. More importantly, however, is the degree to which this reasoning instrumentalizes the passions, in Taylor's words. The will, for Descartes, simply consists in our ability to affirm, deny, pursue, or avoid, while things like desires are classified as passions - ``perceptions, sensations, or emotions of the soul which we refer particularly to it, and which are caused, maintained, and strengthened by the movement of the spirits.'' \citep[p.339]{descartes1984b} Descartes goes on to tell us explicitly that these are perceptions precisely in that they are thoughts which are not actions of the soul or volitions. The passions are instrumentalized because they become passive objects of choice for the will to assent or deny, pursue or avoid and, according to Taylor, exist only to reinforce responses from an organism that promote its survival or well being.  \citep[p.150]{taylor1989} Thus, the values and dispositions are objectified - they are objects of choice which are considered distinct from our assent or denial. The passions are part of the world of extension - captured in the movement of the animal spirits which link the mind and the world of extension. 
We can see a clear version of this objectification in the basic argument, embedded in the premise that supposes that in order for a choice to be free, it must come from desires, values, or propositional attitudes which were themselves freely chosen, in steps (4) and (5) of the argument \citep[p24-5]{strawsong2010}. We see this implicit in the previous example of Jack, who objectifies his villainy and cowardice, viewing them as contingent objects of choice that, as it turns out, he as an agent cannot be responsible for choosing. Those values and dispositions become things to be chosen, rather than constituting the subject doing the choosing. And just as in the case of the mind - the passions are objectified by being connected to the world of extension, and split away from the subjective power of assent or denial, which gets put into the ontological world of mind.

Built into this Cartesian division at a fundamental level is an ontological distinction between subject and object. Descartes' attempts to view the manifest image - of which persons, mentality, and will are parts - together with the scientific image as it emerges in his time as the world of extension. But this stereoscopic view is accomplished by means of this ontological split between mind and extension. We can clearly see in the unlimited nature of the will, its power to freely assent or deny the passions, and to exist apart from them, a clear precursor to the idea of the agent cause which retains all of these powers. But we also see that the separation between the ontological realm of mind from the realm of extension was necessitated, in part by the fact that the objectifying perspective was clearly not to be applied to ones' own mind. It was done from the position of one's mind, the clear and distinct idea of which was to serve as an Archimedian point from which the world of extension could first be truly understood.

However, as the scientific image progressed, the ontological division that Descartes made became more difficult to accommodate, until it has become fairly common to deny that anything really belongs on the ``mind'' side of the Cartesian divide. However, the fact remains that the fundamental features of our lived experience - not the least of which is the fact that we have a lived experience at all - continues to persist. These features have historically been associated with the mind side of the Cartesian divide, and as that division becomes less and less plausible, the option has seemed to be to abandon the category altogether, or to hold out the belief that it persists, and the human mind and human will still occupy that side of the divide. 

Resistance to allowing mind and will to cross the divide has a great deal to do with the original terms of the division, which took the lived experience of those things as the essential and defining characteristics of one side of that divide, and more importantly, perhaps, taking the extension ``side'' of the division to be essentially objectified - essentially non mental and non-volitional.  The negative definitions we have of agent causation - uncaused, contracausal, non-alien-deterministic, etc. all seem to be an explicit refusal to take on the ``objectified'' nature of the Cartesian world of extension, or, in the case of hard incompatibilists - a recognition that our own conception of our agency ultimately must refuse this objectification in order to be anything like what we thought it was. The concept of the agent cause, the epiphenomenal self are both, on this view, placeholders for the refusal of the full objectification of persons. But if this is true, then the agent cause is only essential if it is the only way to hold out against the objectification of Cartesian extension. 
Of course, what I have done here does not establish that the idea of the agent cause is not somehow implicitly contained in our phenomenological intentionality. However, it has given us at least a prima facie plausible alternative explanation  for the seeming importance that the position of the agent cause seems to have, and suggests that the idea of the agent cause may be an attempt to hold onto something essential while explaining how and why that idea might appear to be necessary to maintain our attitude toward moral responsibility. 

In this chapter, we have seen that the expectation of ultimacy in moral responsibility is connected to a particular conception of being a person that seems to be bound up with it. However, some of the challenges which seem to highlight this difficulty - in particular Strawson's basic argument, and Pereboom's four stage argument - seem to include an inconsistency in their commitment to the agent cause. Both arguments deny the existence of the agent cause, yet their objections against moral responsibility only make sense if one takes seriously the idea of an agent cause as actually existing in the first place, since in both arguments, the agent cause quietly reappears as the victim of the injustice of holding a person morally responsible - a victim that both arguments suppose cannot exist. From this the epiphenomenal self as a conceptual placeholder appears - keeping in place the standard that we actually have for moral responsibility, while maintaining that it is an impossible standard. The standard cannot simply be jettisoned without loss - in particular, the standard helps maintain the existential weight that we ordinarily afford to moral responsibility, which in turn supports the reactive attitudes which motivate engagement in moral responsibility. We saw, however, that the importance of the idea of the agent cause may be overstated - what may actually be essential is the denial of the objectification of persons, in which case the idea of an agent cause may simply be one, ultimately flawed attempt to block that objectification. In the next chapter, we will examine alternative attempts to deny the objectification of persons, while staying out of conflict with the Causal Thesis. 

\chapter{Self-Government and Responsibility}
In Chapter 4, we saw that a number of prominent hard incompatibilist objections to moral responsibility are grounded in an inconsistent commitment to the existence of the agent cause. Both the basic argument and the four stage argument deny the existence of the agent cause, and yet both maintain a commitment to a very thin notion of the agent cause, implicit in the expectation that a real agent should be able to choose to be an entirely different agent with a new set of motivations, drives, etc. The thin agent is the entirely featureless entity that is supposed to be able to choose to be a different person. I also suggested that this idea of the thin self might have a history - one connected to the objectification of the self that has slowly occurred since Descartes split the world into body and mind. The hard incompatibilist position relies on the claim that for moral responsibility (or some features of moral responsibility) to survive close scrutiny, the thin self would have to exist, and since the thin self does not, and could not really exist, then we cannot be morally responsible.

The hard incompatibilist's thin self is not really a commitment to a self, so much as it is a commitment to some transcendental remainder after considering the impact of biology and environment on who a person is. But just what is the source of this requirement? I have suggested that the hard incompatibilist views a person's character as objectified - a ``thing'' that one must be held responsible for or not. The thin self is simply the subject that is held responsible for the object. The problem with this line of thinking, however, is that it holds onto the thin self in some sense, by insisting that it would be wrong to hold someone responsible for their nature and character. These are things that they had no ultimate choice over. But who is this ``they'' supposed to be, other than a vestige of the agent cause that can choose or refrain from being a certain kind of person? The hard incompatibilist continues to need the thin self to motivate the claim that it is wrong to hold a person responsible for something that has happened to them. The claim that the person is not responsible for the negative aspects of their character is motivated by the objectification of all of their characteristics, treating them as something that has happened to them and is only contingently who they are. The ``victim'' is the thin self, and the continued commitment to this in the motivation of their argument reflects that the hard incompatibilist is not fully committed to the causal thesis. There is still an element, stripped of any characteristics or causal impact, that figures into their worries about moral responsibility. It is not that the agent cause doesn't exist but rather that it is entirely impotent, unable to do anything but suffer the outcomes of nature and nurture, passively carried along the causal stream. As argued in chapter 4, this thin self is a mere epiphenomena, yet one that the hard incompatibilist requires to be shielded from responsibility for the outcomes of nature and nurture. This thin self is necessary to make any sense of the worry that a person doesn't choose who they are - the ability to choose oneself requires a perspective from outside oneself to make any sense, and the worry that there is something wrong with the lack of an ability to choose the self requires the same. 

The next argument that the hard incompatibilist may make is that ``folk intuition'' requires the agent cause as the morally responsible agent - arguing that moral responsibility is bound up with the natural phenomenology of our agentive experience which suggests that we really are agent causes. However, there is reason to worry that our agentive phenomenology includes no such high level metaphysical claims at all, and even if there is a folk belief that our phenomenology is ``of the self as an agent cause'' that such a claim requires judgment that can go wrong. Furthermore, no argument is really given for the claim that the understanding we have of agentive phenomenology really is in any way natural or implicit except for a prima facie plausibility for the claim. However, it is possible to generate competing prima facie plausible claims that the identification of our agentive phenomenology with agent causation is a historical event, and one which may have been precipitated upon a large and culturally powerful error - Descartes' splitting of the human self into mind and extension. Before deciding on which of these stories are true, both will need to be investigated, and the hard incompatibilist is not entitled to presume the ``native phenomenology'' story if an alternative account of the self can be given. 

To proceed from this point, an alternate account of the self must be given that both fits within the causal thesis, and is satisfactory as the subject of moral responsibility, which will be the primary task of the next three chapters. However, at this point, it is not immediately clear what will decide whether an account of the self is satisfactory. It will need to be able to sustain judgments about moral responsibility, and will in some way need to be an adequate idea of what a self might be. But how will we know when we have accomplished either? 

\section{Agent Cause and Criteria for the Moral Self}
Part of the challenge of giving an account of moral responsibility is the variation of intuitions about what would count as sufficient for an account of moral responsibility, and an account of an agent or self that we could rightly hold morally responsible. As was mentioned in chapter 1, accounts of moral responsibility and agency all seem to accomplish different things. I suspect that many of the variations in intuition - whether an account is good enough, or not - really depend on what sorts of things the account can and cannot do, and how important the holders of these various intuitions feel that each element really is. J. J. C. Smart felt that abandoning both desert and the peculiar emotional charge that attaches to judgments of moral responsibility was an acceptable, perhaps even a trivial loss \citep{smart1961}, whereas many philosophers since then have not been so quick to relinquish the importance of those elements. We have already seen criteria outlined when reviewing how certain attempts at account of moral responsibility might fail. In chapter 2, the criteria of fairness and phenomenological adequacy were discussed - failing to be fair, or to be true to our lived experience were both taken to be potential defeaters against holding a person morally responsible. In chapter 4, a  third criteria was also mentioned - the meaningfulness of moral responsibility. All three of these criteria will be brought to bear on the account in this chapter, and as such are worth reviewing. 

In chapter 3, we addressed many of the concerns around whether moral responsibility could be fair through a slightly modified version of arguments originally given by R. J. Wallace \citep{wallace1994}. Wallace claimed that worries about fairness were not a problem because the fact that we excuse and exempt people from moral responsibility in certain circumstances does not entail that we must exempt or exclude them in all circumstances - blocking what he calls the generalization strategy by arguing that a ``general power'' of reflective self control gives us a way of differentiating between cases where we ordinarily exempt or excuse people, and those in which we hold responsible. However, we saw the notable exception to Wallace's defense that we could still be concerned about the fairness of moral responsibility on a systematic level if there were no principled way to pick out an agent as bearing some particular responsibility for some event we are holding them responsible for.  The worry here is that moral responsibility might be unfair because picking one person out of the web of causation is arbitrary. But what is the basis for this concern?

The concern seems to be due to an overly strong link being made between moral responsibility and causality. Certainly when holding a person responsible, it is a necessary condition that the agent be causally involved in the event in some way (or perhaps not involved when she should have been). We do not hold people responsible for things that they have no part in. Thus, the agent having a causal relationship with the event they are to be held responsible for is a natural point of focus for figuring out who is responsible - we look at an event, and look for what caused it, and the cause is responsible for the effect. In many ways, the question ``who is responsible for x?'' overlaps with ``who caused x?''

The causal thesis, however, complicates the task of finding who is responsible, because if we see the actions of persons as fully integrated into the web of causality, then we can see the person, and their decision to act in a particular way as an effect with its own cause. If we are reasoning from effect to cause to figure out who is responsible, it becomes difficult to understand how we choose a principled stopping point. In fact, the rules of when we decide to stop along the chain of causation to find the responsible person seem to shift in different circumstances. The worry here is that our ``stopping point''  where we stop seeing a person as an effect but instead hold them responsible as a cause is arbitrary and as such, is unfair. 

If persons are agent causes, however, this stopping problem is resolved fairly simply, because persons become first causes - at least some element of what a person does is not the effect of anything prior. Therefore, holding the agent responsible is not arbitrary, because stopping responsibility at the agent is a forced move. The buck stops at the agent and can go no further. The agent cause gives us an ontological stopping point. However, it is worth noting that a positive account of this ontological stopping point are notoriously difficult to provide. Event causal libertarianism does provide this to some extent - the indeterminacy that is involved in event causal accounts \citep{kane1999, balaguer2004} provides a metaphysical stopping point by introducing, in essence, an new cause without prior determination in the form of the ``random'' element in event causation. The indeterminacy is an uncaused change in the causal web that starts inside the agent's decision-making process, making it so that there are truly open possibilities, where both of the possible outcomes from these possibilities are something that an agent wills. In this way, the outcome is open, and consistent with the agent's will because the agent's will was split.  However, this indeterminacy also makes it more difficult to identify the agent with this stopping point - because the outcome of the decisions is (supposed to be) truly random, and thus independent of the agent's thoughts, dispositions, values, reasons, etc., even though that outcome will necessarily be something willed by the agent. On the other hand, an agent cause is supposed to provide this identification, though it tends to do so under a veil of mystery. In a sense, the agent cause is whatever actually works as a new cause, yet is also somehow connected to who the agent actually is. We get exactly what we want, at the expense of having no idea how it actually works.

The concern of phenomenological adequacy was also raised in chapter 3. There, some pressing questions were raised about whether our agentive phenomenology says anything at all about the metaphysics of agency, and especially whether it includes agent causation as part of the built-in intentional content of our experience. Even with these challenges questioned, there still remained a worry about moral responsibility that arose from agentive phenomenology. Whatever our agentive phenomenology might be, it seems that moral responsibility is deeply tied to our own idea of what it is to be an agent which is heavily informed by agentive phenomenology.  If the causal thesis is true, then the worry here is that the idea I have of myself and others that comes from agentive phenomenology is illusory. There is no fundamental difference between human action and any other event in the universe, all are effects, and the feeling of self-determination we have is an illusion. This could undermine moral responsibility by convincing us that the target of moral responsibility - the ``self'' that I feel myself and believe others to be - does not exist. 

The ontological stopping point that the agent cause provides also addresses the phenomenological objection. Our experience is that of being a person, with powers to cause things to happen, and of being able to freely choose in at least some cases what I am going to make happen. If we are agent causes, then our experience of ourselves as persons is enshrined in our ontology - We experience ourselves as free and self-determined because we are - because we stand apart from the tidal push and pull of the cosmos, free to resist or accommodate it. My experience of myself as a person it is not illusory because it has the idea of the agent cause to answer for it.

In chapter 4, I began to address the criteria of the meaningfulness of moral responsibility. On the most basic level, we might think of this as the strong affective charge that moral responsibility often has, combined with the belief that this strong affective charge is somehow appropriate, important, and even existentially significant, and that judgments about moral responsibility (along with judgments about morality in general) are incommensurable with other value judgments.

In some sense, the arguments of chapter 2 provide some defense for this - the modified naturalistic argument shows that the affective charge attached to moral responsibility may just be ``given'' and does not stand in need of defense because we naturally emotionally engage when things contribute to or frustrate us. However, as argued in chapter 2, this affective response is subsequently interpreted and processed by our reason, beliefs, and expectations in ways that can modify our response. We may feel angry or upset when someone causes us to feel pain in our foot - but when we understand that pain as the result of either an unavoidable mistake, carelessness, or deliberate malice from the person who caused the pain, the response changes. It is important to note that this is not (usually) simply a rational, detached evaluative shift - deciding to dispassionately hold something against someone's moral credit or not. These judgments are intimately bound with shifts in the affective character of our response. If we decided that someone hurt our foot our of malice, the typical response will not simply be to accuse the person of wrongdoing, but to feel differently toward the person - to become angry and outraged toward the person. To discover (and believe) that in fact the injury was an honest mistake will cause a drastic shift in both judgment and feeling and may even include embarrassment at the previous ``outraged'' response.

Why would the causal thesis undermine this response? Something similar to Wallace's generalization strategy (concerning fairness) applies here. Again, in chapter 2, I argued that the affective, person-targeting response may be our default response, which is to say when we are troubled or jostled by anything, our default response is to act and feel as though someone has wronged us. However, as the causal thesis has developed, we have come to understand much of our environment as an assortment of things - rocks, trees, weather, etc - which do not have attitudes toward us in any respect, and which do not, in fact, decide to do anything to us at all. Rocks do not lie in wait prepared to ambush our unsuspecting feet and so responding to rocks the way we respond to people is inappropriate. One can be displeased at stubbing one's toe on a rock, but it is not sensible to be angry at, or blame the rock. (Though it is worth considering the flashes of reactive response we can sometimes feel when inanimate objects hurt or frustrate us - before we dismiss the feelings as inappropriate). 

Instead, we begin to treat objects instrumentally. We attempt to use them to our benefit, and change or avoid situations in which they bring us harm. In essence, we adopt what Strawson called the Objective Stance \citep{strawsonp1974} toward objects. But, as persons become integrated into our causal understanding of the world, it begins to seem as though the attitudes we have toward people are equally unjustified. If persons are things, and the reactive responses we have do not properly apply to things, then there is something wrong with those reactive responses. We ought to adopt the kind of instrumental, objective stance toward people that we adopt toward other objects - treating them as candidates for ``manipulation or control'' rather than as persons, who can be targets of resentment, gratitude, etc. 

This worry is also deeper than Wallace's generalization strategy. Unlike in the case of fairness, the reactive response to persons is not simply based on the fact that there is a distinction between persons and objects. After all, if the problem was simply one of drawing a line to avoid generalization one can simply draw the line between person and thing along the line between biological and non-biological. Human beings are persons, and nothing else is (which does not preclude non-persons as objects of moral concern, eg. animals but just precludes them from being held morally responsible).

Drawing the line between the biological and non-biological only helps us to make the judgment that persons and objects are different. It defends the use of two different stances - the objective and the participant - against charges of arbitrariness because it tracks that difference onto a real difference in kind. It is still not clear, however, why the split between the biological and non-biological should matter. If the difference in treatment between persons and objects was simply instrumental (that it makes good sense to treat human beings as persons because that is a better way of getting what we want) then perhaps the biological distinction would be all that we need. However, the fact that we feel and act very differently toward persons and objects is not addressed in this approach. Those worried about the meaningfulness of moral responsibility seek a validation for the power and sway of those feelings. It is not simply that we feel upset or glad. We must feel indignant or thankful - feeling combined with some judgment about its appropriateness and importance. 

This shift in feeling is, I will argue, linked to another feature of moral responsibility: the participant stance where we regard someone as a full participant in the moral community, which Strawson contrasted with the objective stance. When we are in the participant stance we engage emotionally in richer and deeper ways with the person we regard. Strawson claimed that not only resentment, but complex attitudes and practices like love were connected to the participant stance - a position others, such as Wallace \citep[p.30-2]{wallace1994} have tried to split away from the reactive attitudes and the participant stance, but which I will argue are essential to it. One of the key features of the participant stance is that the way we are ``allowed'' to treat a person is in many ways heavily constrained compared to our treatment of objects. For example, to change a person's behavior by ``reprogramming'' them without their consent would be considered a terrible thing to do, and absolutely forbidden - the brainwashing/drugging examples in many of the thought experiments around moral responsibility are examples of this forbidden behavior. However, we have absolutely no such scruples around doing the same sort of thing to a computer, and many cases not even an animal (using conditioning to get a dog to respond how you want). 

But how would a person being an agent-cause help to support the existential import attached to moral responsibility? Once again, the agent cause sustains these features of moral responsibility by providing a metaphysical stopping point by simply denying that persons are things. Unlike objects, which are determined in what they do by other things, people are ``first causes'' - and it is in this special metaphysical status that human beings find the ground for their special moral status. The reactive attitudes are appropriate for human beings precisely because they are metaphysically different than mere objects. So too, is the special treatment that persons receive - the restraint in our treatment of them, even when they do things we do not like. We must reason and persuade, and not brainwash because human beings have a will that is metaphysically distinct and out not to be manipulated in the same way that mere objects are. The Causal thesis, by denying the existence of agent causes undermines this distinction.

Notably, though, there is no clear and unmysterious reason why the agent cause deserves this special treatment - the fact that there is a difference is supposed to be enough to justify this distinct kind of regard. The existential import of moral responsibility is that it connects to and helps to sustain the belief that we are different from the ``things'' of nature - that we are entitled to different treatment than mere things. The Causal thesis undermines this by stripping away the metaphysical difference that is supposed to exist between persons and things. According to the causal thesis, persons are fundamentally collections of particles, albeit a fairly complex and interesting collection of particles. \citep{sellars1962}. How is it that they can then deserve and be entitled to special treatment and regard? An instrumental justification - something along the lines that ``it would be good or advantageous to treat people as such because...'' undermines the very idea of ``specialness'' that underpins the existential distinction in the first place, a position that could be sustained as some sort of illusionism \citep{smilansky2000}. Besides which, it seems unlikely that an instrumental defense of the participant stance would even stand on it's own terms. Prima facie, it seems like the most instrumentally advantageous thing for a person to do would be to treat others instrumentally, while pretending and appearing to treat them as persons. 

Thus we see that fairness, the validation of our experience, and the existential import of moral responsibility are all supported by the same basic feature of the agent cause. The fact that the agent cause is supposed to be metaphysically distinct from the rest of the world draws a line between ourselves and others as persons and the encroachments of the objectification that has transformed our perspective on almost everything that is not a person. These may not be the only criteria available, but some combination of them do seem to underpin some of the most salient incompatibilist objections at play in the literature.  I suspect that much of the remaining resistance is actually connected to the failure of most compatibilist accounts to properly address the ``existential'' work that moral responsibility does for us. 

For a compatibilist account of moral responsibility to succeed on these terms, it will have to provide a defensible basis for how and why we can ``stop'' at a particular person in the web of causation and hold them responsible in such a way that continues to validate and fit with the felt experience of being a person, as well as make sense of the existential import we attach to persons as somehow distinct from a mere object. It must do all of this in a way that validates existing practices and attitudes toward moral responsibility, but all without the metaphysical foundation of the agent cause to do any of this work. 

In this chapter, I will argue that an alternative model, that sees responsible human beings as fully incorporated into the order of nature, can achieve the main tasks that the agent cause, if it exists, would be good at accomplishing - generating a ``fair'' focus of moral responsibility, while sustaining a coherent commitment to our lived experience and the sense of meaningfulness that attaches to moral responsibility. This model of the responsible self has already been well worked out in the literature, especially in models that understand our identity as connected to an understanding of ourselves through our deliberations and decisions (what Korsgaard \citeyearpar{korsgaard1996, korsgaard2009} calls a ``practical identity''). I will argue that these models are largely correct, though with something of a misplaced emphasis - in much of the literature, we are understood as being agents in this practical sense, and this is how it is we can be held responsible. I will argue that in fact, the relationship is reversed - it is by being responsible that we come to be agents of these kinds, and that the defense of moral responsibility really comes down to the value attached to being these kinds of (practical) agents. 

\section{The Kantian Will}
There have been a number of attempts to understand moral responsibility as a feature of practical reason, elaborating on how human beings are responsible in virtue of the Kantian idea of being ``a law unto ourselves.'' in a tradition that includes Frankfurt, Watson, Korsgaard, Bok, and Bratman. These philosophers have all identified something incredibly important for moral responsibility - a concept of our own agency. However, many of these accounts are excessively rationalistic, and as such, do not really describe moral responsibility as most would be familiar with it - as accounts of moral responsibility, they would be revisionary. These accounts are, by and large, missing the important role that the reactive attitudes play in moral responsibility, when they leave any room for them at all. 

Frankfurt, Watson, Korsgaard, and Bratman are primarily concerned with providing accounts of our agency or freedom. Each of these accounts is, I will argue, a variation of the same Kantian theme - a focus on our ability to be a ``law unto ourselves'' and this feature is rightly identified as the core feature of our agency. 

In the first formulation of the Categorical Imperative, Kant says that our actions must be understood as coming from a maxim which can at the same time be willed as a universal law. The first formulation of the categorical imperative tests the maxim against the standards of universal law, but in a way, this starts rather late in the game. One important feature of this formulation is the idea that human beings act according to a maxim at all, and that this maxim is where the moral worth of the action lies. A maxim is a principle of action, and it is the ability for the will to act on maxims at all that Kant identifies as the key element of practical reason. 

\begin{quote}
The will is thought as a capacity to determine itself to acting in conformity with the representation of certain laws. \citep[78]{kant1998}
\end{quote}

This is the understanding of the will as a law-giver. The laws it can give it does so because we can act on representations of principles of actions - maxims. For Kant, this ability to act on maxims makes it possible for us to formulate maxims of action that accord with the form of reason itself, which Kant argues is the Categorical Imperative. For Kant, the will is only autonomous if the principle of action it gives itself and acts upon is drawn from its own nature, which means acting from duty according to the categorical imperative. But we see in Kant's idea of the will an essential split between the will as having the capacity to act on principles of action, and the fact that it can also provide the content of the principle of action in the categorical imperative, acted on from a respect for duty alone. 

It is this Kantian ability to act on ``representations of principles of action'' that forms, I will argue, the kernel of our agency around which responsibility and personhood both revolve. The main problem for neo-Kantian accounts has been surrounding the other ``side'' of the Kantian will - the content of the principle of action that makes up the Kantian will. While Kant's own solution is problematic for a number of reasons (and ultimately, it is unacceptable) it will be worth considering what might be thought of as tempting about Kant's account of the will. We will need to investigate more closely what the Kantian approaches all get right for the purposes of moral responsibility: the ability ``for the will to determine itself to acting in conformity with the representation of certain laws.''

\section{Reflective Self-Evaluation}
The key element of the neo-Kantian will is the ability to act on the representations of laws - though there is something problematic in Kant's restriction of this ability as acting on laws specifically, which are an important example of a more general category that I will later follow Hilary Bok in calling a standard.  Standards, I will argue, are possible because of a capacity that Korsgaard has called ``reflective self-evaluation.'' This is the ability to consider and evaluate our own mental states, taking aspects of oneself as objects of evaluation. Doing this generates new possibilities in our thinking. Korsgaard's idea of reflective self-evaluation is most clearly expressed in her ``Sources of Normativity'':

\begin{quote}
But the human mind is self-conscious in the sense that it is essentially reflective. I'm not talking about being thoughtful, which of course is an individual property, but about the structure of our minds that makes thoughtfulness possible. A lower animal's attention is fixed on the world. Its perceptions are its beliefs and its desires are its will. It is engaged in conscious activities, but it is not conscious of them. That is, they are not objects of its attention. But we human animals turn our attention to our perceptions and desires themselves, on our own mental activities, and we are conscious of them. We can think about them. \citep[p.93]{korsgaard1996}
\end{quote}

According to Korsgaard, the essential feature of reflective self-evaluation is that it can take its own mental activities as objects of attention, as contrasted with perceptions and desires that are entirely focused on the world. It is this ability of the mind to attend to its own mental states that makes the ability that Kant identified as crucial to the will possible. The maxims that the will can give itself are representations of the mind's activities, such as judgments about what is good in the long run, which attempt to generalize into principles of action.

However, while Kant felt that the fact that the representation of the will as acting on its own laws was essential, Harry Frankfurt has helped to identify the fact that ``law-like-ness'' need not be a feature of the will. In his ``Freedom of the Will and the Concept of a Person'' Frankfurt attempts to explain freedom of the will through a distinction he makes between first order desires, which are focused on the world and which can conflict within the psychological space of a single person, and second order desires, which are preferences about what first order desire might prevail in those conflicts. \citep{frankfurt1969}. On the analysis I am proposing, Frankfurt's second order desires are important because they are about first order desires - they are representations not of the world itself, but of our own mental states, turning our own reasons for acting into objects of evaluation by first making them objects of attention. The formation of second-order desires are another way of realizing Korsgaard's more general capacity we have for reflective self evaluation. Kant might insist that evaluation happens when we subsume our actions under maxims, and then see if those maxims pass the test of the Categorical Imperative. For Frankfurt, having another desire about a first order desire is sufficient. Both attempts capture the same essential feature: our attention is turned to our own reasons for acting, rather than the thing in the world that our actions may be directed at, and the ability to take our own reasons for acting as objects of attention and evaluation is the first crucial feature of the Kantian will. 

To see why second-order mental states are so meaningful, it is worth considering Frankfurt's wanton - a being that is only capable of first-order desires.  We can imagine engaging in discussion with a wanton about a proposed course of action - e.g. robbing precious jewels from a museum. In such a discussion, normative questions can certainly arise for the wanton - what time of day she should commit the robbery, which glass cutters to use, or what the best way to disable the alarms might be. All of these practical questions are, however, internal to the desire to steal the jewels. Questions of practical reasoning that extend beyond the boundaries of that desire will fail to have the same sort of purchase for the wanton. The wanton can also be moved by conflicting desires - the desire to steal the jewels and the desire to be a good public citizen. However, these reasons will only impact the level of desire that the wanton has to steal the jewels. These reasons make stealing the jewels more or less attractive. What cannot make sense to the wanton is a criticism of those desires themselves, because the wanton cannot form a second order desire about which first order desire they prefer to prevail. Arguments saying that wanting to make oneself richer at the expense of the public good is a bad desire to have, for example, will make no sense. Once all of the factors have been taken into account about the desirability of sealing the jewels, and the practical question of how to do so, there is no further question for the wanton. 

When we introduce the ability to consider and evaluate first order desires through the capacity we have for reflective self-evaluation, an entirely new domain of discussion suddenly becomes possible. We can now ask which of our desires are good, and which desires ought to determine our behavior. We may feel the push and pull of the desire to be wealthy, and the desire to be a good public citizen - first order desires directed at the world, just like the wanton. However, we can now also think about those desires and be moved to consider them as good or bad on their own merits, or according to a more abstract principle, or as against some conception of ``law''. Here we see that the possibility for standards of preference between desires other than that of the strongest desire suddenly become possible. We may hope that our most noble desire wins when we experience an internal conflict. We can wish or hope that we might be a good citizen and refrain from theft even when presented with a powerful desire to steal the jewels. The strongest desire may still win, but the possibility of having standards for which desire ought to win other than ``whichever is strongest'' suddenly allows for the possibility of a disconnect between the desire we act on, and the desire we hope we would act on. 

Through reflective self-evaluation, we can raise the important question of what desire we are going to act on. These questions are an important shift because they are external to our immediate desires and are no longer appealing only to the strengths of those desires themselves. The wanton, who acts only according to their strongest first-order desire is, in an important sense governed by the objects desired, which was precisely Kant's worry about being determined by desires. What is right or wrong to do is determined by the conditions that will satisfy the desire in question. Because reflective self-evaluation both represents our desires to us and opens them up to evaluation, it also makes it possible that our desires can be measured against something else - evaluations of those desires themselves, as in Frankfurt's case, against our values for Watson, or as according to principles of action and measured against the standard of the Categorical Imperative for Kant. And if we think of the wanton as invariably governed by the object of his or her desire because it is the condition of the satisfaction of those desires that determines what the wanton shall do, then reflective self-evaluation opens the possibility that something other than those desires might ``govern'' us - something external to the strength of our first-order desires themselves. \footnote{These external considerations have already been identified as essential to our moral lives.  The failure to take account of and properly integrate these external questions was a problem we encountered when considering Strawson's naturalistic strategy in chapter 2. There, Russell gave us two potential readings of the naturalistic strategy - the strong reading (token naturalism) claimed that our reactive attitudes, because they were natural ``given'' elements of our moral lives that their role in our moral lives was outside of legitimate moral inquiry. One could ask whether some reactive response was legitimate given the rules of the reactive attitudes, but one could not question the application of the rules themselves. On the analysis given in chapter 2, that strategy was unsatisfying precisely because the ``external'' questions about our ethical lives are ruled out.  

Understood in this light, we can better understand why those external questions play this essential role in our moral lives. To question the legitimacy of the reactive attitudes is, in essence a ``second order'' question - a question not about how we should (ethically) feel about something, but rather a question about the legitimacy of ever acting on those reactive responses. Token naturalism accepts our reactive responses as ``given'' and subject only to internal criticisms, and so being a token naturalist would make us something similar to a wanton - not barring us from having desires about our desires, but rather barring us from evaluating whether or not we should act on our reactive responses. Token naturalism enshrines a feature of our moral lives (the reactive attitudes) and sets them in stone. Reflective self-evaluation gives us the ability to consider and evaluate our desires, beliefs, principles, and other elements of ourselves, which is a step beyond evaluating only the outcomes in what we do.  We can reflect on and evaluate our desires and motivations from a point of view which calls them into question.}

\section{Self-Governance}
The second important element of the Kantian will is self-governance. Reflective self-evaluation makes our will an object of evaluation, but this alone simply makes it so that we can consider our own actions as good or bad relative to a standard outside that of their own satisfaction. The second crucial element is that the external standard can also determine what we do.

Hilary Bok gives a thorough analysis of this ability, which she calls the capacity for self-government in her account of responsibility. Bok identifies our capacity for practical reasoning as the central feature of responsibility. We engage in practical reason whenever we decide what we are going to do according to standards that we hold, and we are engaging in self-government when we are actually determined to act through those standards. Our capacity for self-government comes about as a result of the particular problem of practical reasoning. It can be an open question to us about what we are going to do, and further, there is no fact of the matter about what we are going to do until we have decided what we are going to do \citep[p.105-6]{bok1998}. This holds true even if determinism is true (and would be false only if fatalism was true - that some result obtains inevitably and independently of what anyone chooses or could choose to do). Practical reason sets before us the question of what we are going to do, and our capacity for reflective self-evaluation sets before us the possibility of determining what we are going to do according to standards which we have reasons for believing are important. I think that this account of self-governance is essentially correct, and ultimately, it is this capacity that moral responsibility is concerned with.

However, there is a significant objection which has not been sufficiently dealt with, and can be understood if we consider Kant's understanding of self-government. Recall that in the analysis of Kant given here, self-government is split into two elements - the capacity for the will to act according to representations of law, and the particular content of the law as given in the Categorical Imperative. These can be understood as the understanding of the self as the lawgiver in the kingdom of ends on the one hand, because it is from the form of the will that the content of the law is given. On the other hand, the will is subject to those laws because it is able to act on representations of principles of actions, and when those representations are laws, then it is obeying those laws. Kant's account closely binds the content and ability to act. Frankfurt, on the other hand allows for the possibility of reflective self evaluation while making the standard of that evaluation a desire, rather than a law or a rational principle, splitting the ability for self-government that comes about through reflective self evaluation from the content of the principles of government, which can now be desire. This is a ``soft'' form of reflective evaluation which is reflective only insofar as it is conscious of our motives for acting, and other desires can provide the means by which our motives for acting are evaluated. This sharply contrasts with Kant's ``strong'' reflective evaluation, which, in joining the capacity to reflect and act upon our reflections with the content of the will (in the categorical imperative), making the will independent of our motivational structure. Frankfurt's soft reflective evaluation created space to break free of the Categorical Imperative as the only principle of self-government, and helps to make alternatives, like Bok's ``avowed principles'' and Watson's values coherent.

However, the split between ability and content of the will is problematic, and to understand this, we must consider the work that the content of the free will does for Kant. One of Kant's worries about the will is the condition he contrasts with autonomy - heteronomy. Heteronomy of the will is when the will is determined to act by maxims which ultimately have their origin in the contingent objects of the world. For Kant, when we act on a desire, then it is the object of that desire which determines reason. If I act out of a desire for money, then reason becomes an instrument of money, which I ``obey'' out of a desire for it. In Kant's famous case of the lying promise, when the maxim of my action is to secure a loan that I will not have to repay, that maxim is my maxim because of my desire for money, and I am in some sense, a slave to money. In conditions of heteronomy of the will, our buttons are being pushed from the outside, and we are not in a condition of self-government.

Importantly, Kant did not think that there were desires that were somehow intrinsically ``ours'' - since they were directed at and by things that were external to ourselves. Kant felt that if we were capable of freedom, then we must govern ourselves rather than being governed by things of the world. If our capacity for self-government came out of our ability to act as determined by representations of laws, which is itself a function of pure reason, then our autonomy comes about only if the content of the law can come from reason itself. Kant believed this was possible in the categorical imperative, whose content comes from the very nature of reason itself, which organizes things under general principles.

Whether or not it was successful, it is important to realize how Kant's conception of autonomy is related to the idea of self-government. In the case of heteronomy, I am not engaged in true self-government because I am not giving myself the maxims that determine my behavior - contingent features of the world are doing that. Acting from the Categorical Imperative is self-government because reason gives itself its own laws, and only when acting according to the Categorical Imperative is reason both the author and executor of the law. That is how, for Kant, self-government is really self-government. Our capacity to rule ourselves is only in service to reason itself. This gives us the motivation behind Kant's strong claim that it is only when we act according to the categorical imperative that we are free. 

One of the problems for accounts of agency that have departed from Kant's by doing away with the role of the Categorical Imperative has been in explaining and defending what makes the principles they have espoused ``self'' government. This problem has dogged Frankfurt's account and is directly related to the split he develops between the action of the will through reflective self evaluation, and the content of the will in the Categorical imperative. Gary Watson \citeyearpar{watson1986} initially criticized Frankfurt's account by arguing that Frankfurt does not convincingly explain why we should care especially about our second order volitions. The problem, according to Watson, is that we must be able to pick out the desires that truly represent who we are, and the mechanism that Frankfurt gives us to do this is simply this mechanism of reflective consciousness and evaluation. The desires I want are who I really am. However, one might ask what happens if there is a conflict between second order desires where the strongest simply wins out. According to Watson, this puts us at the same difficulty as with conflicts at the first order, and the only way to resolve the problem would be to ascend to a third order desire - a desire about the desire of which desire should be my will, with a potential regress problem. Watson finds Frankfurt's solution of ``resoundingness'' unconvincing - ending the regress by fiat - because what we really need, according to Watson, is a reason to identify the agent with one set of desires rather than another \citep[p 94]{watson1986}. Watson offers a competing account, which contrasts desires with values - which are rationally considered judgments about what it would be best to do, as opposed to a simple impulse to action, which is more akin to a desire. \citep[p.91]{watson1986} However, as Velleman \citeyearpar{velleman1992} points out, Watson's analysis is subject to the same problem of identification - why should we identify with our values, rather than our motivational system?

All of these cases are related to Kant's problem of heteronomy - they reject the claim that freedom involves the absolute autonomy of reason where reason provides both the principle of action, as well as the ability to act on it. This strong reflective evaluation is problematic precisely because it is unclear whether any sense can be made of pure reason providing distinct reasons for acting that are entirely distinct from our desires as motivational forces \citep{williams1981}. However, the elimination of the expectation that we must be autonomous in Kant's sense, that is free from determination of the will by anything heteronomous to it, does not eliminate the problem that Kant saw with heteronomy. Frankfurt, Watson, and Bok all admit a problem of heteronomy whenever we act as determined by external standards. Frankfurt's unwilling addict suffers from heteronomy of the will which comes about because they are not acting according to their second-order desires, but the force of Watson's objection is that we don't yet have a reason to identify the second order desires as the ones we must act in conformity with. The problem, ultimately, is that in the case of the unwilling addict - both the desire to take the drugs and refrain from taking the drugs are the addict's desires. We can add ``values'' or ``standards'' to the psychological profile - these are all interesting candidates for the ``self'' part of self-government, but none of them have the convincing power of 1:1 identification that Kant's standard had. The difficulty is not simply that none have offered a convincing way to distinguish between heteronomy and self-government, but none have offered a way to even make sense of that distinction.

Korsgaard takes a less dramatic step away from Kant's solution, but this does not solve the problem. In Self-Constitution \citeyearpar{korsgaard2009}, Korsgaard argues that we construct our identities through our actions. We are, in a sense, what we do. However, Kant's categorical imperative gives us a foundational limiting condition of our identity. For Korsgaard, our identity comes apart when the principles that we act upon cannot be resolved with one another and we are, in essence, split apart by internal conflict. The Categorical Imperative does not give us the content of our will, but it does give us the fundamental condition for our will having an identity at all. It describes the integrity of the will \citep{korsgaard2009}.

Korsgaard's approach does apply the Categorical Imperative while avoiding worries about whether or not reason can provide its own motivational principle for action. Instead, the Categorical imperative is a formal constraint on our reasons for acting and provides a principle for the integrity of the will. Fundamentally, our commitment to the categorical imperative comes through the fact that actions outside the Categorical Imperative cause a fragmentation of the integrity of the will. For Korsgaard, an act is essentially an act because it is based on a principle of action, and our principles of action must be unified if we are to have a unified will. For Korsgaard, the Categorical Imperative is the principle of unity that we need. Within this, we are able to act according to desires (hypothetical imperatives) which provide the motivational force we need to act. 

However, there are a number of objections that can be raised against this account. The place of the Categorical Imperative on this account is questionable in itself.  Presumably, any ``split'' in the self caused by violating the Categorical Imperative ought to be resolved by eliminating the part that is in violation - the part that obeys is the ``self'' that self-governs. But why should this be the case? For Kant, there is an identity relationship between the content of the will and the structure of the will. My respect for the Categorical Imperative in myself is because the content of the Categorical Imperative is my own reason's rule over itself. For Korsgaard, the content of the will is provided by desire which for Kant would be acting in conformity with the Categorical Imperative, but not ``from'' the Categorical Imperative.  For Korsgaard, the Categorical Imperative only gives us a principle of integrity for the will - using the rational coherence of the categorical imperative to give the will a rational integrity. Our commitment to the Categorical Imperative comes through the fact that it is only by acting according to a unified principle that our acts do not split us apart. \citep[p.179]{korsgaard2009}

Korsgaard does not give us a reason to suppose that rational integrity should be especially important to us, or even that integrity should be. Reflective self-evaluation and the ability to act according to a principle is really what makes our behavior into acts - distinctly human events that are tied to our reflective self-awareness. The unity of those acts (that they do not conflict with one another) is what is supposed to make us into ``good'' and whole selves, while acts in conflict are defective. In the first place, it is not clear why we would want or need to be good in this way. Even if we grant that we want our lives to be unified, and this unity must be reflected in our acts in some sense, we might consider other ways of securing an integrated identity - perhaps narrative identity \citep{schechtman1996} could provide what we need. On a deeper level, we could also set aside concerns with integrity altogether, and countenance splits in our identity within the whole of our lives. Korsgaard gives us a reason to ``take a side'' in the internal conflicts of desire, but ultimately cannot tell us why acting in conformity with one part of our psychology is ``self governance'' but acting according to another is not. 

Bok does not address the problem directly at all, but does identify standards as ``avowed principles for action'' \citep[p.129 fn4]{bok1998} and have their pride of place in her account in virtue of the fact that they are the principles by which reason evaluates itself. This idea of being ``avowed by reason'' is a more general way of addressing the same principle by which Frankfurt, Watson, and Korsgaard all attempt to address the problem of heteronomy. Despite the problems, I think that the approach that focuses on our ability to act according to representations of principles of action is fundamentally correct, and following Korsgaard, I also agree that the ``self'' in our self-government is constituted through our principles for action. However, I do not think the Categorical Imperative need be implicated in our self-constitution and that the standards by which we constitute ourselves are not subject to the formal constraint of the Categorical Imperative. Instead, the ``self'' that is responsible is a self in virtue of being responsible, and to be a self in this sense is to be a person. In the next chapter, we will explore what being a person in this way amounts to. 

\chapter{Responsibility, Dignity, and Personhood}

In the previous chapter, I began to articulate a practical reason-based account of persons, which identifies reflective self-evaluation and self-government as key features of persons, both features identified as essential in Kant's account of moral personhood. However, we saw that on Kant's account, the capacity for self-government that we have - our ability to reflectively endorse particular actions and then act on that endorsement - is tied to the content of the moral law in the Categorical Imperative. Other accounts in the Kantian tradition attempted to disconnect the capacity of self-government from the content of the moral law, but in doing so they encounter a problem in resolving the difficulty that Kant found with the heteronomy of the will. In identifying the capacity for self-government with the form of the moral law, Kant is able to identify the will with pure reason in both content and capacity. Other accounts, like Frankfurt, Watson, and Bok all encounter a problem because they identify something else as the content of the will, desires for Frankfurt, values for Watson, and standards for Bok. Against each of these, it is possible to question why our capacity for self-government - our will - should be especially committed to any one of them against the other desires we might have. An account of persons as practical identity will need to make sense of this challenge.

In this chapter, I will argue that Bok's account of responsibility, which is based on standards, is largely correct, but not as an account of responsibility, but rather as an account of personhood. I will address some of the challenges that her account faces as an account of responsibility, and then show how it may be used and extended as an account of personhood. Among one of the key extensions is the recognition of the role of what Kant called dignity - a special respect that we are supposed to have for the capacity for self-government that is the key feature of persons. 

\section{Responsibility}
On the model I am proposing here, personhood and responsibility are opposite sides of the same coin whose center is the extension of our deliberative capacity made possible by reflective self-evaluation and the self-government. The key element of this extension is the ability we have developed to act on higher-level considerations, and evaluate how we should act according to standards that are the result of considered standards rather than the immediate strength of our desires. 

In order to understand this shift more clearly, consider the difference that Nagel \citeyearpar{nagel1979b} points out in the difference between mere ``ranking'' and moral evaluation. For Nagel, there is a fairly fundamental difference in the evaluations we make of objects, and that of people. An object may be something that we strongly like or dislike, and we may be able to give reasons indicating why the thing is likable or not. When we evaluate events, we can understand them as fortunate or unfortunate - strongly perhaps - a volcano exploding and destroying a town is immensely unfortunate, for example. However, when evaluating persons, there is more than approval or disapproval present. We accompany these approvals or disapprovals with a judgment about what happens to them as especially apt or not given our approval. If someone is evil, it is not only that they do things I disapprove of, but perhaps there is something satisfying and appropriate about their being punished. And though being intentionally shoved is far less unfortunate than a volcano exploding and destroying a town, there is an additional moral evaluation attached to the shoving, not only that it was bad, but that in some sense it is wrong that it happened, and someone is accountable for that wrongness. 

One way we can make sense of these differences is through the shift that second-order evaluation makes possible. When we consider the explosions of a volcano, we are considering only elements out in the world - the volcano, and the events that transpire through its explosion. We do not evaluate whether the volcano's decision to explode was a good one because we do not think that the volcano makes decisions. When we consider the actions of a person, however, it starts to become coherent, at least, to say that a person is bad - because they acted on desires that they ought not have acted on, for example. The person becomes a target because their own principles of action are now targets of evaluation. Because we can think about our principles of action, we can start thinking of ourselves as persons. The fact that we have principles of action that have been rationally chosen on and can be used to understand and explain our actions make us ``capable of self-government''. We are acting on standards that are the result of our own deliberation, and we are able to articulate and have differences between our desires, our will (what we actually do), and our standards. 

Thus, we can see that reflective self-evaluation makes the idea of having a will meaningful. The alignment between our desires, our standards, and our actions becomes meaningful because it is possible for them to be misaligned in various ways. We can act contrary to our desires but aligned with our standards, or vice-versa, and this capacity for various forms of alienation make the lack of alienation meaningful. And unlike in the case of a wanton, it is possible to imagine that I might be motivated to act by something other than what actually does or will motivate me, and then somehow judge that one or the other of those wills would be better. When we think about and form opinions or judgments about our desires, we start to make a distinction between ``my will'' and everything else - including any of the desires we are evaluating as potential candidates for endorsement by (or in) the will, and we come to identify with the endorsement - the will reflects what ``we'' really want, even if our desires threaten to overwhelm us.

What we see here is the development of a distinction between what a person's will is, as distinct from their strongest desire. Their will is really the thing that they endorse on reflective evaluation. However, we already saw that Watson and Velleman raise a question about exactly why some particular feature (second order desires, values, or what have you) get to lay claim as having authority - why do these markers make some particular motivational element a person's true will? In Frankfurt's case of the unwilling addict with conflicting desires - why is it that it is the desire not to take the drug, vs. the desire to take the drug that gets our endorsement as what the person's will really is, never mind how they act? Why does having a second order desire (and an unsuccessful one at that) settle the question of who the person really is? 

This objection brings out a very important point - when the unwilling addict takes the drug, they are doing, in some sense, exactly what they most want to do. That act is their will even though they wish it wasn't. Under an important description it is something they are doing freely, just so long as they are unimpeded. That desire, though the person wishes they didn't have it, or it wasn't as strong as it is, is actually their desire, and it is part of who they are. In this sense, the unwilling addict just is all of their desires without regard to their coherence or compatibility with one another. As a human being, we are all of our psychological states - no matter how we feel about them. The claim that our second order desires or values, or the roles or identities we accept are who we really are is, on this understanding, simply false. All of our desires and values are who we really are, on this view of the self, which I will refer to as the extended self.

However, in our second order desires, our values, and our roles, we are in another sense taking sides in the many subterranean struggles that go on in our extended selves - between conflicting desires, as well as between different desires that might be in competition for resources, roles in our decision making, etc. In this process, we come to endorse, accept, and identify with certain desires against others, or above others, and out of this process emerges a different kind of self, carved out of the extended self through a process of identification with certain elements of the extended self, and, importantly, alienation from others. 

The extended self accounts for the objection that has been brought against Frankfurt, Watson, and Korsgaard by conceding the point - on one understanding there is no sense in which acting against the standards each of them espouse is a form of self-alienation, so long as they are acting on their own desires. Heteronomy in the extended self is simply impossible. However, when we are making appeals to moral responsibility, blaming, or praising, then we are not really addressing a person as an extended self. The proper targets of this kind of talk are persons, and they are carved out of the various struggles within the extended self, where we attempt to govern ourselves according to our standards. It is through the attempt at self-government according to standards - any standards - that we come to be persons. 

The understanding of persons being developed here is an alternative to the ``thin self'' model of personhood. On this model, persons are not metaphysically distinct agent causes as agent causal libertarians expect, nor should they be, as hard incompatibilists maintain. Instead, ``person'' is a way of capturing certain aspects of the ``extended self'' that are of particular interest to us - motivations we want to protect and standards we want to maintain, all according to a vision of who we are, what projects we are engaged in, or what values we truly care about. This particular interest is related to our deliberative capacities - the fact that our deliberations can impact our decisions, and ultimately what we end up doing. On this way of understanding personhood, a person is better understood as a ``thing'' in the way a Bishop in chess might be - it is defined not as much by what it is, but by what it does, and what it cannot do. Personhood is a practical category, not an ontological category.

Hilary Bok's account of moral responsibility is very similar to the account of persons I am developing here. One of the main problems for Bok's account is she takes her account to be about responsibility, rather than being about persons, and this confusion leads to a number of difficulties for the account as an account of moral responsibility. As her account does contain the essential mechanisms behind personhood, however, it is worth considering it in closer detail. 

On Bok's account, responsibility is situated in practical reason, which is reasoning about what we ought to do - what ends our efforts will be directed at, and what we will do to bring about those ends. Our capacity for doing so comes through our capacity for self-government, which corresponds to the Kantian insight that our decisions can be determined by principled reasons (what Kant calls the representation of law), rather than by the mere blunt force of the strength of our desires. For Bok, we are capable of self-government because we are free in a practical sense, our freedom consists in the fact that we can make decisions, and there are things about our lives that are not determined until we make a decision. Our standards have a practical ability to determine what we do, and because we are interested in self-government, we are interested in how well our practical reason determines our behavior such that we continue to act in the interest of the standards we avow. Responsibility, for Bok, is primarily a way of differentiating between when we didn't act on our standard because of a flaw in our will (which she understands as our ability to decide and act in favor of our standards) or whether our failure was due to some extraneous factor that practical reason could not have reasonably avoided \citep{bok1998}. The key difference for Bok are cases where a greater commitment to our standards might have changed behavior (where we might have given into temptation to do something against our standards), or when a greater commitment could not have changed our standards (we were drugged, hypnotized, etc). Moral responsibility is a mechanism for identifying cases where we might profitably work on our will (or avoid tempting situations, or avoid some mistakes) so to better meet our standards, as opposed to cases where no such work would make a difference (because the failure was not a failure of our ability to choose what to do correctly). 

One crucial feature of Bok's account is the distinction she makes between practical reason and theoretical reason, which sits at its foundation. For Bok, freedom and responsibility are concepts that are internal to a particular kind of reasoning - reasoning about what we are going to do. This is contrasted with theoretical reason, whose ``aim is to describe and explain the world insofar as this can be done without engaging in practical reasoning.'' \citep[p.63]{bok1998} Because of these distinct ends, theoretical and practical reasoning cannot establish genuinely conflicting claims. The apparent metaphysical elements involved in moral responsibility end up being constructs we have constructed to ``perform the operations involved in practical reasoning [...]'' \citep[p.72]{bok1998}. The essential point is that the concepts of practical reasoning are located within a practice - just like the way a bishop moves is located in the practice of chess, or the way division works in the practice of mathematics. The problem that incompatibilists have, on this understanding, is tantamount to the objection that the way a Bishop moves is described by the laws of physics, and not by the rules of chess. The objection isn't exactly wrong, but it gravely misunderstands the question and context. To belabor the metaphor, a ``chess incompatibilist'' either attempts to claim that the rules of chess are additions or exceptions to the laws of physics, or else concludes that chess and bishops cannot exist. 

Before developing this theory of personhood and seeing how it is related to moral responsibility, it is worth considering that strategies that attempt to locate moral responsibility in practical reasoning have received criticism before.  Galen Strawson \citeyearpar[p 276]{strawsong2010} has also considered the possibility that moral responsibility might, at bottom, be somehow related to a the distinction between practical reason and theoretical reason, or involved in ``human truths'' rather than theoretical truths, and has found this way of resolving the problem wanting. According to Strawson, the best results lead to a situation where we must either sacrifice our belief in the overall coherence of our understanding of the universe and a belief in the unity of truth. Ultimately, according to Strawson, this is too steep a price to pay, but he also recognizes that deciding whether we ought to abandon our belief in the unity of truth or the belief in the ``human'' way of being is a distinct issue that cannot be easily decided in a general way.

For Strawson, the fact that these two forms of life have radically different commitments at their foundations creates the problem. These different commitments are not spelled out, but are likely similar to the contrast we have seen between the scientific image and its commitment to the abstract entities of physics, vs the manifest image and its commitment to persons as described by Sellars \citep{sellars1962}. In Sellars' language, Strawson is concerned that we cannot see the two images together, and in order to maintain a commitment to both images, we must abandon our belief in the unity of truth.

We can also imagine this objection being insisted upon even in light of the way Bok makes the distinction between theoretical and practical reasoning, and would be tantamount to simply insisting that the ``practice'' of practical reasoning is not real, nor are the concepts that are employed in the exercise of practical reason, while the commitments in theoretical reason are (akin to the claim that chess is not real). Furthermore, when we engage in practical reasoning, we falsely believe that responsibility is real in the same way that bricks or tables are real. The crisis brought about by the Causal Thesis is the insight that it gives us that responsibility and persons are not really real - that they are mere fictions, and incapable of being ``seen together'' under a unified image of the world. They would need to be so unified for us to continue our engagement in the practices associated with them. 

This problem, however, really comes about because of the tendency to see the theoretical and practical, the scientific and the manifest as ``images'' in the first place. This way of thinking and speaking about these forms of life sets up an expectation that the unity of the world that they are about must be reflected in the image we generate. A unified image reflects a unified world, and an image that lacks unity reflects two more different worlds. On this view, if the images are not unified - if there are different and incompatible theoretical and practical reason based images - then there is an implicit commitment to two different and incompatible worlds. As Strawson suggests, this would be a very high price to pay to maintain a commitment to moral responsibility. 

However, Strawson himself discusses the theoretical and practical/moral as ``forms of life'', which suggests that they should not really be understood as images at all. Instead, they are both names for very generic kinds of activities we can engage in. In the theoretical, we are trying to do things like predict, control, and so on, and we are doing this both by speaking about the world, making predictions about what will happen, as well as many other kinds of activities. In the practical form of life, we are attempting to do something else: deciding what we are going to do, or discussing value, or other such activities. Neither activity is absolutely neutral - there is some human interest involved in either form of living, and neither activity can really be understood without taking into account that interest. The theoretical way of life is evaluated as successful or not depending on whether such talk is actually able to do what we are trying to do in that form of life (eg. predict the outcome of this or that physical system's activity). 

If we understand the theoretical and practical form of life as engaging in different activity with different kinds of success conditions (accurate prediction vs. making happy, for example), then we can understand the disunity between the ways of life as being the result of the fact that each is attempting to do something different in the world, rather than as a commitment to two (or more) different worlds. The so called metaphysical commitments - to subatomic particles and persons - are not two different ``pictures'' of the world. Our talk of subatomic particles is a way of focusing on particular features of the world we are interested in for the purposes of prediction or some other theoretical task, and our talk of persons focuses on different features of the world we are interested in for the purposes of decision-making. Both talk about the same world, but the talk is always going to be inflected by our intentions in speaking about it. On this understanding, it is fairly easy to maintain a commitment to the unity of ``truth'', if by truth we mean the way things actually are in the world, rather than a commitment in the unity of the ways we talk about the world. 

This is an unabashedly pragmatic way of thinking about truth, and a discussion of the details, and a discussion of  relative merits and flaws of a pragmatic vs. a correspondence theory of truth is far outside the scope of this dissertation. However, we can see that Strawson's objection to the ``practical'' form of life approach to understanding moral responsibility does not have the consequence that we abandon belief in the unity of truth - it actually only requires that we abandon belief in a particular way of understanding the unity of truth (as some form of correspondence theory). We can maintain a full commitment to the ``human form of life'', and the unity of the real world, if we sacrifice the belief that all of our talk about the world must ultimately fit together, independent of any considerations about what the talk was trying to accomplish. This may still be too steep a price for some - but it is certainly less than Strawson represents, and if the correspondence theory of truth commits us to abandoning personhood, then this might legitimately be taken as a failure of the theory of truth in question, rather than a failure in our historically successful and useful way of talking about persons.

Understanding responsibility and personhood as tied to practical reason does not commit us to the belief that they are any shade of ``unreal'' - but Bok's account faces more serious challenges than this. We have seen that responsibility, for Bok, is a concept of practical reason whose role is to help us diagnose and improve failures in our practical reason, but this is where her account begins to run into serious difficulties. The definition and justification for the concept of responsibility essentially come through its role in our first-person practical reasoning about what to do. It is practically useful in that it helps us to maintain our standards and make us good at self-government which we are taken to have an intrinsic interest in.  This is itself tied to the fact that from the first person perspective, there are many situations which are not determined until the individual who is deliberating comes to and acts on a decision they have made, and responsibility is meant to help us pick out when those moments happen, and if we have reasoned and acted well in those moments. All of this works fairly well from the first person perspective. There are, however, serious challenges when we consider the shift to holding others responsible, which Bok attempts to address, as well as when others hold us responsible, which Bok does not differentiate from holding others responsible. 

Bok's justification for holding others responsible comes in two steps - we must regard others as morally responsible insofar as we regard them as capable of self-government, and we must regard others as capable of self-government. There are serious problems for both of these steps, though Bok spends a good deal more time focused on the claim that if we regard others as capable of self-government, then we must regard them as morally responsible. 

Holding others morally responsible starts from a regard we have for them as capable of self-government. If we do regard others in this way, we hold them morally responsible out of a sort of parallel bit of reasoning: we are capable of self-government and therefore are responsible, so if others are capable of self-government, then we ought to hold them responsible. Even if we grant that we ought to regard others as capable of self-government - an assumption which we will question - this argument is still problematic because it does not take into account the way we come to regard ourselves as responsible. On Bok's account, we hold ourselves responsible as an instrumental strategy in making us more effective at self-government. We distinguish between things we have done and might do that can be changed by the effective application of practical reason, vs. those that are not, and it is only the latter that we are responsible for.

Responsibility, in this sense is an attempt to figure out what could be changed in what we do by changing the way we reason, or how we act upon our reasons. This is something we must do, since we are consistently in situations where we are employing practical reason to decide what to do, despite the fact that whatever we do decide to do will have been what we were determined to decide and act upon. In the first person perspective, responsibility is legitimate because we are tuning the effectiveness of practical reason, and we must shift away from the perspective that regards ourselves as determined. We cannot passively wait to see whether we will have chicken or fish for dinner - at some point, we have to decide. By recognizing our responsibility in those kinds of choices, we can recognize when we have made a poor choice and improve our decision-making the next time around. We hold ourselves responsible in order to identify when we should try harder to live according to our standards. 

When regarding others, however, we are no longer in a situation where we are deciding what to do. We may recognize that they are in a similar situation, and that they for the same reasons have reason to hold themselves morally responsible, assuming they, too, have standards they are interested in fulfilling through practical reason. However, we also have reason to regard them as fully determined physical systems, because that, too, is what they are. More importantly, that is what they are ``for us'' precisely because we are not in their shoes and as such, do not have to make the decisions they are making. We simply do not have the same reasons for holding them responsible as we have for holding ourselves responsible, even though we may be able to understand why they might hold themselves responsible.

Bok claims that our reasons for regarding others as morally responsible are derived from ``our interest in acknowledging their ability to govern themselves and in the conclusions about themselves that they have reason to arrive at in the course of exercising this ability.'' \citep{bok1998}. But she has not shown that we have this interest - since the interest we have in the responsibility of other cannot be the same as our interest in our own responsibility. We are differently situated with respect to ourselves as opposed to others.

This problem leads to a deeper problem, however. Bok's account has depended heavily on the idea of self-government according to standards, and when we hold ourselves responsible, we do so out of an interest in being able to conform to our own standards. These standards, however, can be anything - Bok herself gives the example of someone adopting standards that would make them an excellent assassin. So our interest in our standards is intrinsic - these are our own standards for ourselves - and as such they are self-government. We hold ourselves responsible, for Bok, because we care about our ability to conform to these standards - responsibility is depicted as a kind of self-discipline in the interest of a higher vision for one's own life. And surely, responsibility is that sort of thing.

However, responsibility also covers things that are very different than that. In the ordinary operation of society, we are held responsible for standards that are imposed onto us by force, by shame, by coercion, and by cultural conditioning. These are standards we may not have really reflected on, and might not endorse if we did. It is not clear what interest we have in upholding these standards, and so, what interest we have in holding ourselves responsible in these cases. We also hold others responsible for standards they might not endorse, and we do so without a particular interest in ``acknowledging their ability to govern themselves'' or even in reasoning with them about what they ought to do. When we hold the assassin responsible for killing the politician, we are not trying to hold them to their own standards, or show them that really, assassinating the politician was against their own standards. We are trying to subvert their standards and impose our own onto them. So too, is the despot who holds people responsible for withholding anything they have from him.

Bok says very little about the impact that variations in standards have on her account, though at times, she seems to assume that these standards will eventually converge. She claims, for example, that in practical reason we are interested in arriving at justifications for our action that ``all persons in similar circumstances would regard as justifiable were they to reason correctly.'' \citep[p.193]{bok1998}. Depending on whether ``people in similar circumstances'' means that they hold the same standards or not, this is either vacuous, or a highly controversial argument that would require some support. It is not clear, for example, that the assassin should be interested in what the non-assassin reasons about in her situation, since they are aiming at very different standards than the assassin. But whatever her assumptions about convergence, her account does not address the fact that moral responsibility occurs across standards.

This connects to another issue for Bok's account - its psychological implausibility. The motivation that Bok gives for holding others responsible, an interest in their capacity for self-government, is a fairly benevolent, rationalistic, and Kantian view of our motivation for moral responsibility. But how can this account be reconciled with the way discussions and enactments of holding others responsible play out in our world? Holding others responsible is sometimes vindictive, emotionally charged, and not at all concerned with respect for the other person. When we ``hold someone responsible'' for cutting us off in traffic, it is rarely with a thought of their dignity as a self governing human being, and may, in fact, be accompanied by very undignified and disrespectful words and gestures. The emotions behind responsibility can also get far darker and have much more severe ways of being acted out in the world. As Strawson pointed out - responsibility has a deep connection to our emotional lives, and this is a connection that seems to play no role in Bok's account whatsoever. At best, there is room for the emotions as accompanying our judgments of moral responsibility after the fact. 

In fact, it is this emotional charge to moral responsibility that helps to motivate many of the incompatibilist concerns about moral responsibility, and which takes us to the fact that on Bok's account, her defense of moral responsibility all premise that we must regard others as capable of self-government. Bok argues that the question of why we have an interest in viewing others as capable of self-governance reduces to the question of what reason we have to regard others as persons rather than things, and claims that this is ``essentially [...] the question why we should not be sociopaths, and that is not a question we need to appeal to philosophy to answer.'' \citep[p.190]{bok1998}.  While I think that this analysis of self governance is correct - as has been argued extensively here, moral responsibility does depend on whether or not it is right to treat ourselves or others as persons - Bok claims that this question is outside the scope of her discussion, though she offers a pair of quick arguments in support of this requirement.  

In the first attempt, Bok argues that in all moral systems, there is a moral demand to treat others as persons. However, this claim may trade on some equivocation on the word ``persons''. As Bok has been using the term, the essential feature of persons is that they are capable of self governance. Treating people as persons in this sense is certainly a requirement of Kantian morality, but it would require further argument to show that this is also a feature of consequentialist morality or virtue ethics, and for many versions of consequentialist ethics, it seems unlikely that such arguments would be successful.

Further, she argues that our practical interest in whether our principles are justified also present us with a reason to treat others as persons. Bok says that a claim is justifiable when all persons in similar circumstances would regard it as justifiable were they to reason correctly. However, it is not clear that practical reason requires justification in this sense. I may reason about what I should do in order to rob a bank successfully - my only interest in ``justification'' there might be in whether or not my plans are likely to make my endeavor successful, which would decide whether I ``should'' work in daytime or at night, for example. Neither argument offered really shows why we ought to treat others as persons.

Perhaps most problematic, however, is her unsubstantiated claim that not treating others as persons amounts to sociopathy. There have been arguments raised that claim the opposite - that treating others as ``morally responsible'' and capable of self-government is harmful, and that abandoning this kind of treatment would be more humane and compassionate. Many of these arguments, such as from Sommers \citeyearpar{sommers2007} and Strawson \citeyearpar{strawsong2010} focus on the fact that viewing others as determined makes better sense of their behavior. It takes into account that when a person does something bad, it was because they were determined to do so, and anyone would have done the same had they been subject to the exact same conditions (genetic, environmental, and historical) that the other person was subject to. \citep{sommers2007}. Furthermore, the harm that comes about from the emotions connected to moral responsibility - and in particular ``resentment'' - are shown to be harmful and often problematic, and would be seen as such if we were to stop regarding people as ``persons'' in the sense Bok is using the term - capable of self-government, and instead shifted to the objective stance. That Bok has missed these concerns with moral responsibility is understandable, because her account does not take into account the role of the emotions in moral responsibility. However, this makes her account either implausible, if it means to defend responsibility as we currently use the term, or else highly revisionary. All of these problems are connected to the fact that Bok attempts to project our own inward interest in our own responsibility outward. 

This tendency, however, can be traced back to an assumption that was identified and argued against in chapter 2 - the belief that in order to hold people morally responsible, we must in some way be convinced or rationally motivated to do so. In chapter two, I argued for a model of moral responsibility grounded in our emotional lives - the motivation comes from raw emotional responses which are shaped and constrained by reason. Ultimately, what is needed is not an account of why we hold people morally responsible, but rather to account for the limits and rules according to which we do not hold people responsible. What is mysterious is not why we hold a person responsible when they have wronged us, but why we ever make exceptions or excuses for people so that we do not hold them responsible even when they have done something that we do not like. What Bok has provided is not a model of responsibility, but it is a very good model of the ways in which we constrain and control our emotional responses - our responses are informed and shaped by our understanding of others as persons.

\section{Personhood}
To see an individual as a person is to take seriously their capacity for self-government according to standards. A person attempts to guide their life in relation to standards of some sort - who they think of themselves as being, who they aspire to be, as engaged in some long term project, or according to values that they hold. In other words, being a person involves taking sides among the various desires and psychological states that make up what I have called the extended self. To better understand this, we will need to look more closely at the interaction between the extended self and the person, and how such a thing would work. 

As we have already seen, the extended self really covers all of our behaviors, and all of the psychological states that underlie those behaviors, regardless of our feelings about any of those things. The extended self can both desire something and desire an incompatible opposite - an extended self can both love and hate doing the same activity at the same time. The only coherence of the extended self comes from the fact that all of these things are occurring in and through the same body, and the same body is limited in what it can do (eg. the body can eat cake or not eat cake, but cannot do both at the same time). Frankfurt's example of the wanton is a kind of extended self whose behavior is determined by the strongest desire it has - the only coherence the wanton has is from the fact that it only has one body with which to act, and presumably, from moment to moment, different and conflicting desires can prevail. A wanton may now most strongly desire to eat lots of cake, and later most strongly desire to be healthy and in good shape, and though these desires may be in conflict, a body can pursue one or the other at different times, so there is nothing ``wrong'' with a wanton pursuing either desire at different times, even though one desire may consistently defeat the other. There are other ways that motivations may conflict - aside from intrinsic conflict (eat lots of cake vs. maintain my diet), they may also compete for our immediate attention (I want to see a movie at 4:00 and I want to go jogging at 4:00), or may compete for resources (I would like to major in Philosophy, I would like to major in Economics). For the wanton, these competitions will always be resolved in the exact same way - we will do whatever our strongest desire is at the time.

Importantly, all of these conflicts are different elements weighing in on the same essential question - what am I to do? Our capacity for reflective self-evaluation changes the situation significantly, because we understand that we can decide what we are going to do, and do so according to different standards we might hold. Our decision need no longer be determined only by what we most want in the world here-and-now, but can also be informed by the way we value more abstract principles like values, or be informed by an understanding of how things might impact longer-term goals, or be informed by the image of who we are, or the person we want to be. This may or may not change the fact that our strongest desire will still determine our behavior, but it can certainly change what our strongest desire ends up being, all things considered, by considering our behavior against longer term goals or in light of an understanding of who we do and do not want to be. 

Against Frankfurt, Watson, Bratman, and Korsgaard, I do not think that any one particular set of standards or way of ``holding ourselves together'' need be privileged. Bratman \citep{bratman2007} suggests, for example, that this essential holding together is connected to long-term planning, and Korsgaard \citep{korsgaard1996, korsgaard2009} suggests that it must be under some conception of identity, my suspicion is that in the range of human experience, these, as well as other principles (such as Watson's \citeyearpar{watson1986} values) actually pay a role in holding us ``together'' as persons. Following Bratman, our ``coming together'' as persons need not be around some self consciously understood idea of our identity. On the other hand, as we come to have unified projects, these things will, through our active commitment to them in our actions, start to generate ``who we are'' - making us from extended selves into persons. 

This is because our standards are not ``who we really are'' at all - their special status does not come because any of these things are preordained as the self that governs. In essence, what makes a person a person is that they are not wantons - determined by unreflected-upon desires, but rather that at least some of their desires and motivations have at the very least come before the tribunal of practical reason, and the question ``is this good?'' has been put to them and the objects that they aspire to. This ``good'' must be understood in the thinnest possible sense. It is simply the question of whether or not the person approves of the desire or course of action, and why they do. Persons undergo that process at least some of the time, and that is what makes them persons. 

The other essential part of personhood is the ``taking seriously'' of this essential capacity, and this is the work that Bok's concept of responsibility is really capable of explaining. Taking our capacity for self-government seriously means more than having standards against which we evaluate. It is coupled with an expectation that when it is ``up to us'' - meaning that effective practical deliberation can lead us to making a decision in favor of standards - then we will actually do so. Bok's concept of responsibility can and does track the difference between cases where failure to live up to standards is a case of insufficient commitment to a set of standards, as opposed to when it is the result of some other factor (being bumped into, drugged, mentally ill, etc). This is essential because it establishes the boundaries of our ``personhood'' as the object of moral address. 

Extending beyond Bok, however we see that responsibility is not only concerned with establishing the limits of moral address, but also involves the fact that we are in numerous ways ``bound'' to those standards by expectations, and by measures both internal and external which reinforce our commitment to those standards. Our standards are supposed to have a decisive role in what we do, and when they do not, we may feel things like regret, or shame, or disappointment with ourselves. These feelings express, even in cases of failure, our internal commitment to the standards that we hold. On the other hand, when our actions meet or exceed our commitments to these standards, we may feel proud or joyful. As positive and negative feelings, these emotions reinforce or deter us from engaging in activities depending on whether they support or run against the standards we hold. Shame, guilt, and pride are also the core self-reactive attitudes - the emotional force that is involved in holding ourselves responsible. The kernel of our personhood is maintained by holding ourselves responsible, and this illustrates the way that responsibility and personhood are connected. We become persons through responsibility which both sets the boundaries of when we can be expected to maintain standards, and reinforces those standards through our self reactive attitudes. 

Unlike Bok, however, it is a mistake to think that the self-reactive attitudes, or the phenomenon of holding ourselves responsible are the historical origin of personhood, nor even the origin of a human being's life as a person. It is far more likely that personhood develops not at first through our own commitment to our own standards, but rather the expectations that others have that we will obey both cultural and moral expectations. Nietzsche's genealogical story about the development of morality links the development of consciousness and the soul to the forced imposition of undesirable standards of behavior upon the weak. These ``slaves'' were incapable of realizing their own ``will to power'' because of the presence of the masters \citep[p.16-7]{nietzsche2010}. Indeed, Nietzsche discusses how this repression made the slaves ``interesting'' and capable of ``depth'' \citep[p.57]{nietzsche2010}. This story may simply be an illustrative fable, but it is one that emphasizes the same relationship between personhood and the imposition of standards that have come as the fruit of reflective self awareness. However, we can also consider that in the contemporary moral training of children, we can expect children to live according to external standards - moral standards, as well as standards related to conceptions of their role in society, social position, etc - without resorting to violent repression (though reward and punishment of some form will usually be a part of that training). Children are held to standards by the expectations of those around them, and slowly come to internalize them.

In this sense, as persons we are shaped by our society in significant ways, and we need not have the sort of internal commitment to our standards that Bok starts with beyond a simple internalization of the external standards. That internalization may just as well come about by force or brute indoctrination, and among the external standards we can come to be committed to are the standards of conventional morality - the set of codes and expectations that are carried along culturally, and may not even be ethical in any robust sense. We may internalize expectations based on our social position, role, sex, caste, or any other element of ourselves to which standards of expectation might attach. 

However, we can also come to reflect on and evaluate the standards we are given by society, change them, attempt to make various standards internally coherent, and coherent with one another, and evaluate them on their own according to how we feel about them, etc. This is to ask whether the standards by which we govern ourselves are really ``good'' in a deeper and more robust sense. This is the model of the person that appears in Plato's Republic, whose central issue can be understood to be the issue of effective and proper self-government, and the question of whether our unreflected-upon, immediate desires ought to rule us, or whether we ought to be ruled by a considered vision of what is good arrived at through focused philosophical contemplation on the essential nature of the Good. Many of the worries that come up in the Republic make sense here, and indeed, the model of unfreedom that emerges in the Republic - the tyrannical man - is simply a description of a person ruled by first-order desires that are not aimed at the good, but which are too strong for a reasoned understanding of the good to rule over. The tyrannical man is an addict, and if reason has a voice and an insight into what is ``really good'' on considered reflection, then they are Frankfurt's paradigm case of the unwilling addict.. \citep[p.1180-99]{plato1997}

The model of personhood described here - a recognition of our ability to govern ourselves according to standards is connected to responsibility because it tells us what the boundary conditions are for the expectation that we actually will live according to those standards. These ``expectations'' are not meant to be predictive - instead, they shape when and how we should respond to the person ``as a person'', or what Strawson called the participant stance, where our reactive attitudes are fully engaged, as opposed to the objective stance, where we forgo the attitudes and stop seeing the individual as a person. However, there is more to personhood than seeing the person as a proper target of the reactive attitudes and holding them responsible. Intimately linked to personhood and the participant stance is a concept of a person as having what I will follow Kant in calling ``dignity'' - and an understanding of this feature is essential in understanding both responsibility and personhood. 

\section{Dignity}
The expectation that we are capable of living according to rational standards is what subjects us to the reactive attitudes of others, and makes us targets for being held responsible. However, there is also a related value attributed to our ability to act according to standards in which our capacity for self-government is afforded special respect, which is manifest in various ways, but is part and parcel of the overall package which Strawson called ``the participant stance.'' This respect is manifest in our behavior in a number of different ways, but can perhaps be most directly understood through the fact that treating people as persons implies a fairly restrictive set of expectations around how they can be treated, the most central of these is the expectation that any attempts to influence their behavior be brought before their own capacity for practical reason. This is essentially an expectation that we not attempt to subvert another person's capacity for self-government by disabling or circumventing their capacity to live according to standards that they accept, or, in other words, to treat them as ``an object of social policy; as a subject of what, in a wide range of sense, might be called treatment; as something certainly to be taken into account, perhaps precautionary account of; to be managed or handled or cured or trained [...]'' \citep[p.9]{strawsonp1974}.

This is the idea that persons are supposed to be free from certain forms of interference. Persons should not be manipulated, or treated as objects, and their own ends should be respected in any dealings with them. They may be reasoned with and persuaded, but not coerced, tricked, or manipulated to act in ways that attempt to bypass or overpower their deliberative capacity. These expectations around treatment are an essential part of what it is to treat someone as a person. For Strawson, adopting the objective stance toward a person withdraws the expectations that an individual live according to standards, but also withdraws the corresponding restrictions against manipulation and attempts to control them in our dealings with them. Both of these shifts involve no longer taking their ``personhood'' seriously, because the two are deeply connected. The respect we have for people as having their own ends just is the respect we have for their ability to live according to standards. Responsibility and personhood are two different ways of thinking about the same thing - a life lived according to standards. 

We can illustrate this connection by considering the different responses to a situation we might entertain when adopting the objective vs. the participant stance. Consider the case of the drug addict's addiction to their drug of choice. Suppose that there existed a low-risk medical procedure that would eliminate the addiction with no other side effects. After this surgery, the patient would be glad that they had the procedure because part of the addiction-breaking procedure makes the person glad that they received the procedure. Further, suppose this procedure always makes people happier as a result of doing it and finally, it is possible to do this procedure in a person's sleep without their being aware of it. The question to be considered is whether we should do this procedure to a willing addict, that is, to a person who is addicted to a drug and endorses this fact about themselves (they have a second order desire in favor of their addiction) though they also have a desire, which they do not endorse, against having the addiction (basically the exact reverse case of the unwilling addict)? 

If we consider the addict according to the objective stance, then it is fairly clear that we should do the procedure to the addict, which, in the end, changes their desires around and makes them happy about it, with the further benefit that it would help to remove the scourge of drug addiction. The only factors against the procedure would be the strength of desire of the addict for their addiction, but actually doing the procedure changes this fact about them, since one of the features of the treatment is that the addict is made glad to be free of their addiction. All the treatment does is to change something about the addict, in a way that benefits both the addict (considered as an ``extended self'' and as an object in the environment) as well as everyone else. 

The situation changes significantly if we regard the person in the participant stance. To respect someone as a person means that one accepts restrictions on what can be done to them. Of particular concern here is that any attempts to change the willing addict must fall under the scrutiny of his reflective self-awareness. The willing addict, on reflection, endorses his addiction - that addiction is a part of the person he is because maintaining that addiction is within his standards and vision for his life. The treatment bypasses that commitment entirely, removing both the addiction and the desire for the addiction. If we are treating the willing addict in the participant stance, respecting them as persons, then doing the procedure is wrong precisely because it does not respect them as persons. 

It might not be immediately clear whether doing this procedure to the willing addict is actually wrong, or what is wrong with it. Indeed, there is room to disagree about what it would be right to do. However, the decision to do one rather than the other reflects very different commitments to personhood as a form of life - if one endorses the procedure, then one does not think that we really owe personhood commitment (or perhaps we do, but only to the point it doesn't interfere with pleasure or utility or some other measure of good), because the procedure does violence to the personhood of the willing addict. Interestingly, one of the key good-making features of the procedure in the objective stance - that it alters the addict such that they endorse the procedure after they have it actually makes the procedure far more transgressive in the participant stance. To understand both the wrongness of the procedure in the participant stance, and the special problem with the ``endorsement shifting'' aspect of the procedure, it is first necessary to get clearer about what ``respecting personhood'' actually is. 

On the view of personhood I am arguing for here, personhood is tied to the narrow self as reflectively endorsed according to the standards that the person holds, as opposed to the extended self, which includes all psychological states of the person regardless of whether they reflectively endorse them. To respect the individual as a person is to take their capacity for self-government seriously, and taking that capacity seriously means, first and foremost, not trying to bypass it. This means that when we come into a conflict with a person such that we want to change their behavior, and that behavior is tied to their standards (meaning it is behavior that they reflectively endorse, such as the addiction of the willing addict), then we must directly address the individual as a person by offering them reasons and enticements for them to deliberate upon and choose. The only acceptable procedure for changing the willing addict is to attempt to convince them to renounce their addiction or change their addictive behavior. The restriction that we must appeal to a person through their capacity for self-government is one of the core restrictions involved in treating an individual as a person. 

This restriction is central not simply because we believe that it is central, or because we just happen to want it to be central. The restriction is one of the core rules of behavior that constitute the practice of treating someone as a person. To fail to do this is to stop engaging in a crucial kind of person-responsive behavior which, at its core, restricts us to giving reasons (in a fairly loose sense of ``reasons'') to one another when we want to influence one another. In cases where we attempt to bypass the capacity for self-government, it is not simply that we are not giving someone their due as a person - it is that we are no longer treating them as a person at all.

The giving of reasons plays a similar constitutive role when we engage in theoretical reasoning about what we ought to believe with one another. Suppose we consider an atheist and a theist attempting to convince one another that their position is correct. In the ordinary course of things, each will offer reasons for what the other ought to believe, and the shift from theist to atheist, or atheist to theist will happen because they are moved by reasons from one position into the other. Suppose, however, there is a treatment to change someone's beliefs analogous to the drug treatment above - it bypasses the reasons they have and simply implants them with new beliefs, and new attitudes about those beliefs. Suppose, further, that the atheist uses this treatment to shift the theist's beliefs to atheism. Suppose further that all of the new beliefs that the theist comes to have, making him into an atheist are rational, and well grounded. \footnote{Thanks to Paul Russell, who suggested the parallel epistemic case}

In this case, the shift from theist to atheist was done in the wrong way, because it was done without an appeal to reasons, but rather through a medical ``belief-editing'' procedure. We attach value to the procedure of reason-giving in forming and changing our beliefs, and it is not difficult to understand why. Appeals to reasons help our beliefs to be rational ones, since we will have come to believe what we believe according to a procedure that itself appeals to reasons. By this procedure, shifts in belief are done for reasons, and through reasons. We want to protect and privilege the rational procedure for a number of reasons. There are certainly good practical reasons, if we suppose that reasoning about things is more likely to lead us to true beliefs (and arguably, there is also the fact that appealing to reasons addresses an individual as a person and respects their authority to decide what they believe, in a way that mirrors the respect in procedures of practical reasoning). 

In the epistemic case, being convinced for and through reasons is a central part of the rules of the epistemic procedure that we engage in when we discuss what we believe. It is part of that procedure because doing things in that way helps us to form rational beliefs, which is something that we value. But this way of doing things is intrinsic to the practice of rational belief formation because it is itself a success condition of rational belief formation. It is not, on the other hand, intrinsic or necessary to the wider phenomenon of belief formation. One need not strain one's imagination much to imagine a world where many beliefs are the result of irrational methods of belief manipulation. Indeed, this is exactly what can be chilling about things like advertising - where preferences for products and purchasing decisions are manipulated using techniques meant to bypass a person's rational decision making process, and which cause them to have beliefs about the products based on other things (associating the product in question with sex, or prosperity, even if there is no real connection between the two as a basic example). Appealing to reasons is a necessary aspect of engaging a person as a rational belief-former. One can depart from this constraint, but to do so is simply to stop engaging with the individual as a rational being, and instead treat them and the beliefs they hold instrumentally. 

In the case of the willing addict, the problem is the same - we can attempt to convince the addict that their addiction is bad, but if we are to respect the addict as a person, the only way we can do this is by appealing to the addict's reasons in an attempt to get him to shift his standards. The medical procedure bypasses the decision making procedure, and in so doing manipulates the addict at a level beneath and behind their standards. The ``addiction cure'' medical procedure does not engage with the willing addict as a person because it attempts to shift the standards and actions of the willing addict by bypassing their decision-making procedures altogether. To do this is to fail to treat the willing addict as a person because engaging with an individual through their rational capacity to decide what they want to do for this or that reason is just what it means to treat them as a person at all. 

In both the epistemic and deliberative case, the standards are in place to ensure that their reasons (or standards) have a primary place in coming to a decision about what to believe, and what to do, and to treat anything as a person is to concede that primary place to their own reasons. Importantly, their reasons may themselves be irrational. It is difficult to think of a well grounded reason that might justify the willing addict's commitment to his addiction - but nevertheless, it is a result of the standards and desires for life that the addict reflectively endorses. What is important for personhood is not that the reasons for our decisions and beliefs are rational (as Kant would hold, and which gives the Categorical Imperative it's pride of place as our only ``true'' will) but simply that they are reflectively endorsed. The object of that reflective endorsement can be a belief, a desire, or an abstract commitment to a formal moral principle - a person can decide on a wide variety of things to commit themselves to. Once they have, the only way to shift those standards and activities is to give the addict reasons to change their behavior.

Just as in the epistemic case, where we value the rational procedure because we think that doing so makes us more likely to have true, rational beliefs, we have reasons to value the rational belief-shifting procedure in the case of practical reasoning. In this case, we value our ability to deliberate and decide to do things for higher-order reasons. In essence, we value our capacity for self-government, and our capacity for self-government entirely relies upon our ability to reflect on what we would like to do according to standards, and then attempt to act according to those standards. This reveals what is especially problematic about the endorsement-shifting aspect of the addiction-curing procedure. When the procedure changes the willing addict's endorsement of their addiction such that they instead endorse their new lack of addiction, the change has struck at the core of their personhood, because it has struck at their standards themselves. Changing a person's standards through an irrational procedure like the medical treatment in some sense destroys (or at least damages) the person they were, and in a sense replaces them with a new person in the same way that brainwashing a person destroys who they were and replaces them with someone else. This isn't a ``metaphysical'' destruction of their being, but rather the shift from the willing addict to the happy non-addict has occurred according to a procedure that radically disrupts the ordinary rational continuity between who a person is and who they come to be. The willing addict ceases to exist as a person because he cannot be properly connected to the happy non-addict as a person. 

It is also important to note that this problem of discontinuity is not connected to the fact that the procedure artificially alters our psychology, which becomes clear if we consider a slightly different scenario. We can consider the identical addiction-cure procedure under the participant stance, except now the individual in question is an unwilling addict - they have a second order desire not to act on their addiction, but their addiction consistently overpowers them. In this case, we can apply the cure, and it really is a cure because the individual's addiction is not part of who they are as a person since they reject the desire on reflective self evaluation, however powerless they may be to act consistently with that rejection. Rather than being a part of who they are, the addiction is treated, in a sense, as alien to the individual as a person, even though it is still part of their extended self. 

This shows that personhood is intimately linked to those aspects of ourselves that we reflectively endorse. Responsibility plays an important role in personhood because it takes those standards seriously. It backs them with emotional, social, and sometimes even physical punishments and rewards, as well as through the changed ``rules'' for interacting with a person that require that interactions all go ``through'' our personhood - any attempts to influence behavior are supposed to be subject to reflective evaluation and conscious deliberation. Thus, the treating of oneself and others as a person is made possible in part by responsibility - it is through being responsible we come to be persons.

The restriction against overriding a person's capacity for self-government can also be thought of as respecting them as ``ends in themselves'' - it is the recognition that as a person, the individual is a ``giver of law'' to themselves. For Kant, of course, this law must be the categorical imperative itself, because Kant understood desires as being alien to reason, directed at objects in the world, and thus a case of being ruled from without. However, if we understand personhood as being particular sets of limitations and endorsements set against the extended self, then the ``rules'' of personhood are not additions or exceptions from our nature, but instead are a particular subset of our nature that we take a particular interest in given the standards we hold. The ``rules of persons'' are not incompatible with our nature as biological human beings, but neither are they identical to them in the same way that the rules for moving a bishop in chess are compatible with, but not identifiable as the laws of physics. 

These restrictions which are in place as a form of respect for the capacity for a person's self-governance  are primarily what I understand as ``dignity.'' Much like responsibility, dignity may have a spotty history in which it looks significantly different in its origins as compared to now. Nietzsche's ``respect between equals'' \citep[p.23]{nietzsche2010} is certainly a candidate for a conceptual origin, and throughout history, standards have been imposed on others without a corresponding recognition of those so imposed as ``ends in themselves''. Nothing in the account I am suggesting supposes that Dignity has always been coupled with responsibility - however, it does come coupled with any notion of responsibility which begins to expect from others an ability to live according to standards that is intrinsically motivated. My suspicion is that dignity, personhood, and responsibility have become more closely bound as the imposition of standards shifted from being merely conventional, cultural, and imposed by force to being linked to voluntary involvement in pursuit of a good that in one way or another is supposed to transcend provincial mores, either through a vision of a transcendental, objective good, or as standards that would be voluntarily accepted by rational agents negotiating toward a common good. In other words, the three become bound as our standards become ethical, and as responsibility becomes moral responsibility. It is this sort of ground that becomes fertile for discussion of things like human rights. 

However, one might argue that the connection between dignity and responsibility are fairly tenuous even in contemporary ethical life. A paradigm example of cases like this might be where we punish a criminal, and deliberately attempt to impose our standards on them. The thief may be living according to their standards quite successfully, and when we hold the thief responsible, we are often trying to subvert those standards and replace them with standards we prefer. 

However, whether this is actually a violation of a respect for dignity will depend on the specifics of the punishment, because punishment can be directed at an individual ``as a person'' but also more widely address the individual as an extended self. Therapeutic punishment which treats the standards of the thief as an illness to be cured - symptoms of a poor upbringing or genetic disposition - may, in fact, violate the dignity of the thief and treat them as an object of treatment. It does not take the thief's commitment to their standards seriously, nor the results of having those standards as really attributable to the thief as a person, rather than seeing the thief's actions as the unfortunate confluence of the various causal factors which we must intervene in to bring about a better world. Much like in the case of the anti-addiction therapy, this will depend on whether the thief rationally endorses the standards of their life as a thief. Acting against those rationally endorsed standards and deliberately changing them is a form of brainwashing, which is a form of violence against the thief as a person. It is possible to evaluate this as the right thing to do on balance - however, again, it will depend on how seriously one takes the thief's ability to govern themselves according to their standards seriously.

However, we might also consider retributive punishment which seeks simply to make the thief suffer some harm as a consequence of their bad actions (beyond restoring the stolen property to the rightful owner). This form of punishment also attempts to subvert the thief's standards, but it does so in a different way. In this form of punishment, actions like theft are said to have additional social consequences that are made to artificially attach to the violation of standards that the society means to defend or impose. Provided the punishment is not cruel or excessive in a way that entirely overwhelms the thief's rational capacities, retributive punishment can be understood to respect the dignity of the thief because it essentially places additional reasons to be considered before the practical reason of the thief being punished. Such punishment does not necessarily respect the standards of the thief, but in attaching additional consequences to living according to those standards, it is still appealing to the thief's capacity for self-government.

The connection between responsibility, dignity, and personhood help to explain P. F. Strawson's mostly unexplained claim that a thoroughgoing shift to the objective attitude would make certain other human relationships impossible, including Love. Indeed, Strawson specifically states that one of the things that becomes impossible when engaging with a person in the objective stance is ``reasoning with him'' - though negotiation is still possible. On the account given here, I understand reasoning with someone as engaging with a person in a way that takes their capacity for self-government seriously, in which case shifting to the objective stance makes reasoning with someone impossible by definition. One may appeal to reasons, but in adopting the objective stance, that is no longer a privileged path of access to persuasion, and more importantly, things that circumvent our capacity for self-government are no longer off limits. 

Furthermore, we need not suppose that the line between appealing to practical reason or attempting to subvert or circumvent it is hard, fast, and easy to objectively locate. Something like seduction or a direct enticement of the appetite probably respects a person's capacity for self-governance, unless, perhaps, the person has taken a vow of celibacy as part of a religious order, or perhaps, as has become increasingly discussed in the media, claims that they are addicted to the pulls of the appetite in question. We can even understand the claim that something is an addiction as a way of signaling that some desire that a person is acting upon is a compulsion and ought to be treated therapeutically, rather than supposing that actions driven by the addiction are part of who the ``person'' is and should be held responsible for.  There is a gray and indistinct border between appeals to the person and attempts to circumvent the person - but this border tracks well on the kinds of everyday disagreements that actually take place on the question of whether or not a person is responsible for something. 

On the account of personhood developed thus far, we can make sense of some of the challenges that have confronted practical identity approaches to responsibility and personhood. The core challenge is a result of the splitting of the Kantian connection between the capacity for self-government, and the content of the principles of self-government, which for Kant were both ``pure reason''. Splitting the content from the capacity left practical accounts with a problem - how to identify anything as being who the person ``really'' is in the case of an internal conflict between motivations, desires, or other reasons for acting. This approach misunderstands what is going on by making the issue a question of who the person truly is - an approach essentially inherited from Kant. Instead, we must grant that both sides are who the person really is as an extended self, and the distinction comes not because of a metaphysical distinction, but a procedural one. Certain reasons, desires, and motivations are prioritized because they have been reflectively endorsed and adopted as standards according to which a person wants to live and guide their lives. Our commitment to these standards is not because they are who the person really is, but because it is who they want to be and we value the ability for someone to attempt to live their life according to standards that they have endorsed. This reflectively endorsed form of life is personhood. Responsibility is key to personhood because it is a way of taking those standards seriously, but taking these standards seriously is also linked to what Kant called dignity - a forbearance on certain kinds of treatment such that we act in accord with respect for the capacity for self-government that persons have. 

One question that is still left before us is whether ``personhood'' in the sense described here is able to support an understanding of responsibility that can meet or exceed the expectations of fairness, phenomenological adequacy, and existential weight that have been identified as essential issues it must address in the light of the Causal Thesis. We turn to this question in the next chapter. 

\chapter{Possible Objections}

The account of personhood and responsibility that has been given thus far links personhood and responsibility as two aspects of the same practice - expecting of all those engaged in the practice that they live according to standards and a related set of expectations that take the capacity to live according to standards seriously. The question still remains as to whether this account accurately tells us what is going on when we go about our lives holding ourselves and others responsible, and treating persons as ``special,'' and different than mere things. Importantly, this account does not claim to describe what people think they are doing when engaging in these practices. Instead, the mark of success is whether the account of responsibility that is generated is able to support the things that responsibility is supposed to be able to do, and not the explanations often given about how responsibility is supposed to be able to do it. 

Responsibility is the result of the interplay between our emotional responses to what happens to ourselves and others subject to the restrictions and amplifications of an understanding of the targets of those feelings as persons, in the sense described in this chapter. And as we have seen, the causal thesis made three expectations for responsibility problematic - that it should be fair, that it fit our lived experience, and that it have the appropriate existential weight. In order to support this account of responsibility, we will need to see whether it can meet these expectations. 

\section{Fairness and Phenomenological Adequacy }
Two of the objections - fairness and phenomenological adequacy - are both related to the apparent vanishing of the self in the light of the causal thesis. In chapter 3, we saw that moral responsibility was unfair because it was unfair to zero-in on one particular link on the chain of causation as responsible, and phenomenologically adequate because in our phenomenology, we seem to be selves that choose to do things. We now have an alternate account of persons based on personhood as a way of living, rather than as a distinct metaphysical category. Whether responsibility can meet the objections about fairness and phenomenological adequacy depend on whether this account of persons can provide an adequate ``stopping point'' for responsibility, and whether it fits with our lived experience. 

If the argument in chapter 3 is successful, then phenomenological adequacy is the easiest defeater to address, because our phenomenology gives us so little of a theory of persons in the first place.  Chapter 3 argues that it is implausible that the intentional content of our phenomenological experience contains any robust metaphysical claims like agent causation, and that a more plausible explanation for the connection between phenomenology and agent causation is to be found in the history of our response to scientific progress. However, the relaxation of phenomenological worries need not depend on that historical suggestion. All that needs to be shown is that our phenomenology does not mislead us into thinking we are something that we are not. We can see that this theory of persons is not misleading because there are aspects of the theory that answer to key phenomenological features of agentive experience. The feeling of choosing could correspond with the reflective endorsement of some motivational force which is then enacted - it becomes our will. This is, in a sense, the feeling of internal harmony described by Frankfurt when we ``do what we want to do, because we want to do it.'' The self that chooses can be understood as the aspects of our extended self that we identify with and endorse. Our phenomenology, of course, suggests none of this, because our phenomenology still does not contain an implicit theory of the self as a person, any more than it contains the theory of the self as an agent cause. However, the theory of persons here is at least prima facie compatible with our phenomenology. Thus, our phenomenology cannot be said to be misleading us if the theory of personhood I have set out accurately describes how we actually are persons. 

Worries about fairness are, perhaps, more pressing. Objections that the truth of the causal thesis would make responsibility unfair are also related to the apparent breakdown of the category ``person'' within the causal thesis. As already discussed, the worry is that if persons do not exist, then picking out a particular human being as responsible is arbitrary. If a person is an agent cause, then picking them out as responsible is not arbitrary because persons are ``first causes''. The concern is not simply having some principle to pick persons out as responsible, but that the principle used to do so makes sense. The logic of the agent cause makes sense as a way of picking out the person because it maps on to an important relationship (causation), and identifying the agent as the source of the action in question, responsibility attaches because of the sourcehood of the agent. Responsibility involves saying that a source is good or bad, and the question of fairness on this understanding is the question of whether we have picked out the right ``cause'' to hold responsible. This way of reasoning about responsibility is not entirely without criticism. As an example, there are questions about how using causation to track responsibility makes sense of instances where we hold people responsible for things they did not do (holding someone responsible for killing a plant because they did not water it). 

The alternate account of responsibility developed here also shifts the question of fairness - the central question is no longer ``did I pick out the right cause?'' but rather ``was the person sufficiently responsive to the demands of the standards they are being held to?'' The fairness of holding a person responsible will depend on how we understand ``sufficiently responsive''. In the case of moral responsibility, this is really a question about whether the demands of morality played enough of a role in determining their behavior, and this understanding of the question can help explain a number of familiar moral situations that are actually more puzzling under the agent causal model.

One area where the agent causal model is puzzling has to do with instances where a person is coerced.  Suppose Black is coerced to rob a bank by Smith, who has strapped a bomb that he will detonate on noncompliance. If Black robs the bank on the causal model, is he responsible for doing so? We can anticipate a variety of responses here - Black is not responsible, Black is responsible but not blameworthy for doing so, or Black is both responsible and blameworthy for robbing the bank. Black is being coerced to do something that he wouldn't otherwise do. If we analyze this scenario according to the agent causal model, then Black as an agent cause certainly ``caused'' the bank to be robbed, but because Smith coerced him, we want to blame Smith, in part or whole, for Black's robbing the bank. 

Furthermore, we can also imagine how judgments would change if the situation was changed. Black is now coerced by Smith to rob a bank, who has taken Black's lunch, a delicious ham sandwich, out of the office refrigerator and is threatening to eat it on noncompliance. In this situation, Black is also being coerced to do something he wouldn't otherwise do, though most would hold Black responsible in this instance - even though nothing has really changed about the relationship of various ``causes'' on the outcome. What has changed, obviously, is the penalty.

If we think about these scenarios on the model of moral responsibility given here, where we hold people responsible because they have failed to give sufficient weight to moral standards, then both situations make a great deal more sense. In the first case, whether we think Black is responsible will depend entirely on whether we think the standard ``one ought not rob a bank'' weighs against the valuation of one's own life. Those who say that Black is not responsible are those who think that he obviously ought to look after his own life above refraining from bank robbery, and those that say he is responsible are those that say the demands of morality still hold no matter the penalty. On this account, there is no sense in distinguishing ``blameworthy'' from ``responsible'' - those who take that position under the first model are split between these two models - responsibility tracks Black's causal role, and blameworthiness tracks the evaluation of proper responsiveness to moral demands. When we shift from Black's life to Black's delicious ham sandwich under threat, the evaluation of blameworthiness shifts because there are few who think that Black's concern for his ham sandwich ought to outweigh his dedication to moral standards.

On this model, ``sufficiently responsive'' is understood in terms of the weight that the standard in question has on our deliberation and ultimately, our decision of what to do. We are able to be responsive to these standards because our deliberation, into which our standards can play a role, can be and often does determine our behavior - this is freedom in Bok's sense. However, there is a fairly ready incompatibilist objection here - ``sufficiently responsive'' is just the same old compatibilist canard dressed up in new clothing. The incompatibilist can grant that being morally responsible involves being ``sufficiently responsive'', but maintain that if the causal thesis is true, then no one is sufficiently responsive, because whether someone is going to adhere to the standards in question is determined by their biology and environment. Therefore, whether someone is sufficiently responsive is determined outside of themselves, and it is never fair to hold them responsible for that. 

However, it is important to understand what is going on between this account and the objection being made to it - they are operating with entirely different understandings of ``fair.'' For the incompatibilist, holding responsible is only fair when people are agent causes because the right way to track responsibility is by going back to the source of an action and holding responsible at that point. On this model, holding responsible is fair relative to whether a person was sufficiently responsive to the standards in question because being responsible just is adhering to standards in our deliberative practices. None of this is to claim that there is not a substantive disagreement here - there certainly is - but the disagreement really isn't about whether responsibility is fair. Instead, the issue is whether it is okay to treat human beings under the special rules and restrictions and expectations we have for persons - interacting with them as moral participants, or whether we must instead recognize that human beings are just particularly complicated things, and must be treated always in the objective stance. This debate is a debate about two different ways of being and interacting with human beings.

The Causal Thesis has challenged the practice of personhood by trying to show that persons don't fit in the natural order as a metaphysical category (or that whatever persons are, they lack freedom, metaphysically understood), and that being responsible presupposes that metaphysical category. However, on the account of personhood and responsibility given here, that is not what is going on. Personhood is embedded in a practice that comes out of our ability to reflectively self evaluate and deliberate about what we are going to do. If we understand personhood as a practice - something we are doing and a set of expectations about how we treat each other - then an objection can only work if it shows that the practice is incoherent or somehow ``bad''. 

The incompatibilist may attempt to resist understanding this shifted understanding of the debate as a practical question of what to do (``should we hold people responsible or not?'') It is difficult to understand how that resistance would be organized however. If the incompatibilist says that persons and responsibility aren't real because they do not fit into the metaphysics of the causal thesis and so holding people responsible is not fair, this is a thinly disguised demand that we not hold people responsible except when their actions fall outside ordinary causality, which shifts back to the question of what we ought to do. If instead there is a claim and argument that persons do not exist, responsibility does not exist, or anything of the like, then one can simply point to a person being treated as responsible in a way that is consistent and fair (under the practical description of moral responsibility) saying ``thus I refute.'' On this account, personhood and responsibility emerge out of the practice of treating human beings as persons and holding them responsible. Arguing  that those categories, while functional, are not real takes us back to the first objection - that we ought not act according to anything that falls outside the causal thesis. Overall, it is telling that these apparently ``theoretical'' debates about whether or not responsibility or persons exist are usually taken to have practical, normative consequences about how we should or should not treat each other. 

The incompatibilist may attempt to argue that the practical account of persons is incoherent on its own terms - but in order to do this, the incompatibilist will have to provisionally accept the understanding of terms like ``fair'' as they appear internally to the account. The traditional worries about ``fairness'' and the analysis of ``can'' will need to be expressed in terms internal to the compatibilist position in order to show the position is incoherent, for example. Given the history of the debate, this is not an incredibly promising strategy - both the compatibilist and incompatibilist positions have been fairly well defended from ``within''.

Alternately, the incompatibilist may attempt to show that responsibility and personhood don't work - that they do not function in the way we expect them to. This amounts to an attempt to defend one of the defeaters against moral responsibility - phenomenological adequacy, fairness, or existential import. The first two defeaters have already been discussed, and the third will be considered and an attempt will be made to defend against it shortly. Alternately, a new defeater might be identified, which would amount to showing that there is something else moral responsibility is supposed to be able to do, and showing that it is unable to do it. 

Finally, one may attempt to show that the objective stance - treating people as complex kinds of things is somehow better than the participant stance where we hold people to be persons - directly engaging the issue as a normative one. However, P. F. Strawson has suggested that a shift away from the participant stance to a thoroughgoing commitment to the objective stance would be a great loss. Valuable human interactions - things like love, pride in accomplishment, and other kinds of human relationships are lost, according to Strawson, if we were to shift to the objective stance. However, a number of philosophers have argued that there would be no great loss.  Tamler Sommers \citeyearpar{sommers2007}provides an admirable defense of a wholesale shift to the objective stance as humane and as including substantial potential benefits for human life. Sommers argues that adopting a thoroughgoing commitment to the objective stance would be a good thing, arguing that many of the concerns about losses to human life raised against the objective stance are groundless, and further, that adopting the objective stance would actually be a good thing. Pereboom \citeyearpar{pereboom2001} argues a similar point, conceding that there may be some differences in our lives with the absence of the reactive attitudes, though these would be largely be either neutral or beneficial. 

According to both Sommers and Pereboom, there are indeed certain things about our lives that would need to change significantly if we were to adopt the objective stance. Resentment and indignation would be out of place, though they question whether the loss of these negative emotions would actually diminish human life. The threat to gratitude is also overstated, gratitude would change, but this would largely involve a shift to appreciation.  Both Sommers and Pereboom also argue that the threat to love identified by Strawson and others as endangered by the objective stance is not sufficiently argued for and provides an account of love as a fairly complicated form of highly positive regard, different in degree and complexity but not kind from other forms of positive regard. Finally, adopting the objective stance might lead to greater tolerance and even compassion as we are forced to accept, in some sense, those we disagree with rather than resenting them. Both philosophers argue that by abandoning the participant stance, not much of value is lost, and that there is a gain in compassion and an overall ability to get along brought about by the loss of a central feature of the participant stance - resentment. 

The account of personhood and responsibility developed here make it a bit easier to articulate what is lost in shifting away from the participant stance. This will not ultimately decide whether we ought to abandon the participant stance and personhood - but it can make a bit clearer what is at stake. 

The claim that the loss of resentment would not be a bad thing is, perhaps, the most difficult claim to criticize. Resentment and indignation do not immediately present themselves as especially defensible features of human life. According to Sommers, resentment is often about petty things - losing a parking space, or other such trivialities, and yet takes a very serious toll on the quality of human relationships \citep{sommers2007}, and Pereboom says that resentment as an attitude seems to do more harm than good. \citep[p.200]{pereboom2001}. 

Resentment, however, is a fairly complicated emotion. On the account of the reactive attitudes first developed in chapter 2, resentment is a negative feeling informed by the understanding that the target of that feeling is a person who has violated certain expectations that they ought not to have violated. The normative dimension of the feeling intensifies the negative affect - it changes it from a feeling of disapproval or upset that the event happened to anger because the event should not have happened. It focuses on the person qua person as a target of the negative emotion. In the objective stance, the intensification and targeting caused by the understanding of the object of the anger as a person does not happen. Sommers gives the example of Sally regarding a thief in the objective stance. She will want the thief caught and perhaps punished, but only for the instrumental ends of protecting others and deterring further crime. There is not targeting and intensification causing greater upset for Sally, or motivating her to do anything needlessly harmful to the thief. \citep[p.327-8]{sommers2007}

Both the targeting and the intensification, however, occur as a result of regarding the person as special in virtue of being the sort of being that can live according to standards, even though they didn't. It is a heightened expectation, and, within the ``person'' paradigm, a part of what it means to respect them. As we have seen, this respect is the same respect that affords their projects, their identity, and their values special treatment from us, and is supposed to limit all of our interactions with them as going through their rationally deliberative capacities. Despite the ugliness of the way this often manifests - to resent someone is still to afford them a certain degree and kind of respect, even though we may act disrespectfully or worse toward them. It still takes their capacity for self-government seriously and as such, maintains a fundamental respect for them as persons. 


Furthermore, the pettiness and ill-feeling that Sommers and Pereboom identify with resentment need not be what resentment looks like. Further considering Sommers' example of the thief in the participant stance, we might look at their action, feel negatively toward the robbery, take the burglar as a person seriously and have that targeting intensify the emotion of anger into resentment, bringing about the usual bad effects that come with resentment. However, one can also feel the negative emotion, consider the thief as a person, and then attempt to forgo the intensification of emotion while continuing to maintain respect for them as a person by forgiving them. Here, along with Sommers' Sally, we might hope that the thief still receive appropriate punishment for the same reasons Sally does. However, alternately, the forgiveness might involve disavowing the punishment, outwardly displaying a deeper respect for the personhood of the thief in the hope that an outward show of respect for their personhood would inspire an internal respect for the personhood of others and reinforce their ability to live according to the standards they ought to. 

Forgiveness, on this account, must go through the same mechanism as resentment - it is an alternative response to the same situation, and arguably one that has great value. Because this implicitly involves treating the thief as a person as understood in the participant stance, this is a good that is lost in the objective stance. This forgiveness would also lead to the greater good of heightened compassion that Sommers takes to be a good connected to the objective stance. Of course, increasing our forgiveness and respect toward each other as persons is not easy to do, though Sommers himself admits that a thoroughgoing shift to the objective stance is not easy either. 

The loss of the special status and respect for persons is also connected to the other losses that Sommers and Pereboom both attempt to respond to - the loss in gratitude and in Love. According to Pereboom, there may indeed be some change to gratitude, but its essential features are entirely compatible with gratitude - namely thankfulness and joy at being benefited by another. The only feature lost is a belief that the person we are grateful to is praiseworthy for what they have done. \citep[p201-2]{pereboom2001}. However, Pereboom is missing an essential aspect of gratitude that is connected to the additional respect that is supposed to be afforded to persons, and which impact both the thankfulness and joy at being a beneficiary that Pereboom acknowledges. The additional ``shine'' that gratitude has over merely being pleased and regarding a person positively has to do with exceeding standards. One of the sentiments expressed in gratitude is the claim ``you didn't have to...'' an acknowledgment that you have treated me better than I expected, had a right to expect, could have expected, etc. When responding with gratitude in the objective stance, the importance of the standards themselves is fairly small. They are simply psychological facts about what we anticipate. They may have some normative force, but only instrumentally so, and provide a mere point of measurement against which what actually happens can be compared. Evaluation is pointless, since what will be, will be. 

In the participant stance, we take our ability to live according to standards seriously - they are the guideposts of our personhood. Exceeding them takes on the additional shine because they are the evaluative benchmarks that our personhood depends upon. Thus, the standards that have been exceeded aren't simply anticipations of what to expect. It is joy not merely in getting more than I expected - it is joy in the way another has treated their relationship to me as a person, which marks a difference in kind between being benefited by another person, and the joy at being benefited by a random event like a lottery win. The two are different because they are connected in the first case to a person, and in the second to a thing. 

Love is, perhaps, a more complicated case to provide an analysis for, since love has a much more complex and varied history, and the term is itself used in many different ways by different groups of people. Pereboom rightly states that Strawson's claim that hard incompatibilism would undermine love requires further argument than Strawson provides. Pereboom argues that in most cases of love, such as the love of parents for their children, love does not seem to depend on the belief that those children are free, nor does love seem to depend on some conception of a person as deserving it because they have freely chosen to adopt some particular moral characteristics \citep[p202]{pereboom2001}. I freely concede these points - there is certainly no self evident ethical content common to all cases of love at all. Pereboom further considers Kane's argument that we might want to be loved as a result of another person's freely willed choice. For Pereboom, the essential feature of that desire may simply be that love be willed, a kind of Kierkergaardian ``continued commitment'' over time. Pereboom cannot understand why the continued commitment of love being ``freely'' willed would add anything to it. \citep[p203-4]{pereboom2001}. Apart from that, for Pereboom, love is a ``wishing well for the other, taking on many of the aims and desires of the other as ones own, and the desire to be together with the other.'' \citep[p202]{pereboom2001}. He claims that none of these are threatened by hard compatibilism.

I think, however, that Pereboom misidentifies the source of concern that hard compatibilism raises for love.  Instead of being worried about the moral quality of a person's will, I think that love is threatened when we fail to have the special respect for the beloved made possible when we see them as a person. In love under the participant stance, we can see that love might be connected to the Kantian idea of seeing the beloved as a person with special status - as a ``source of value'' and an end in themselves. 

Seeing another as a source of value brings to the fore one of Pereboom's claims about love, that it involves taking on the aims and desires of the other for one's own. In one sense, hard compatibilism  does not make this impossible, insofar as we can easily see ourselves taking on a commitment to additional desires because they are desires our beloved has. But in the case of love, whether or not we take on the desires of another as our own surely depend on what those desires are. For example, we might suppose that one loves an addict, and the addict will, at times, strongly desire the drug they are addicted to. If one loves the addict, this will probably not involve taking on the desire that one's beloved get what they desire, and may involve exactly the opposite. Navigating the desires of the beloved that one is committed to will involve some vision of which of those desires the person is committed to as a vision of the good in their own life, as opposed to desires that they might rather not have at all. It involves a commitment and response to the person as a person because it involves differentiating their standards from their desires. 

A hard incompatibilist can appear to do something very similar, insofar as they also seek to thwart their beloved addict from getting the object of their addiction. However, for the hard incompatibilist, the reason for doing this will be somewhat different. They may see, rightly, that the addict is better off without the object of their addiction, because it is bad for them, etc. But for the hard incompatibilist, this is going to start to take on the shine of paternalism, since this standard of the good is motivated by the hard incompatibilists own idea of the good, and perhaps very much against what their beloved wants.

However, if we see the beloved addict as a person, the situation becomes somewhat more recognizable as love. In the best of cases, the addict has renounced their addiction, and so the idea of an addiction-free life is a commitment that the addict themselves has. The person who loves the addict and yet acts against their desire for the drug can then be understood to upholding a commitment to the addict's own standards, even when the addict cannot maintain their own commitment. In darker cases, the addict's commitment may not be as clear. Perhaps there are competing sets of standards at different times. Cases like this would be much more fraught and perilous with the person who loves the addict, in essence, acting as an advocate for one kind of life to emerge from the conflict, and resulting in a blurring of the line between love and paternalism. In such a case, we might even say that the person who loves the addict is acting from love on behalf of one ``part'' of the addict, and paternalistically against another part. Finally, we can imagine someone who loves the willing addict, and having to confront the difficult choice of either acting paternalistically toward the addict, trying to force them to break their addiction, or else making the painful decision that they cannot be committed to the ``standards'' of the addict, and often, breaking off the relationship with them as a result, hoping that they come to desire sobriety at some future point. 

The increased complexity in the case of love comes through the fact that certain sets of the addict's desires are seen as part of who they are as a person, and thus severely restricting the way we may attempt to intervene in their lives. Further, we can imagine the hard incompatibilist being offered the chance to give the willing addict that they love the addiction-removing treatment described earlier, and perhaps not understanding why one would choose to do anything but give the treatment. However, if we view the willing addict as a person, then the decision is, at the very least, not so clear. This is also not to say that one or the other of these decisions would be clearly the right one to make. However, the difference between the reasons that weigh on the decision track onto the real difference between Pereboom's ``hard incompatibilist'' version of love, and the view of love connected to the account of personhood given here. 

We can also see here a suggestion as to why Strawson might have linked love to the reactive attitudes in the first place. Both the reactive attitudes and love are identified by Strawson as connected to the participant stance, though he does not clearly explain the connection between the three. In this account of persons, we can see why, contra Wallace, we would want to link the reactive attitudes to love and other forms of ``participation''. All of them are linked by the fact that they require the category ``person'' as their target, and are undermined or changed if persons do not really exist. 

On the basis of these arguments, we can say that, the shift to the objective stance actually does involve some fairly significant losses to human life. Resentment does have some intrinsic value, both in itself as connected to the respect for another as a person, and in the deeper value that it has in underlying genuine forgiveness. Gratitude and love also both lose something, not from the loss of freedom, but from the erosion of the boundary between person and thing occurs as a result of adopting the objective stance. Furthermore, the gains of greater compassion and goodwill as a result of shifting to the objective stance also seem to be possible within the participant stance. If this is correct, then Sommers and Pereboom have not shown that a shift to the objective stance is without loss, or that we would be better off in a thoroughgoing shift to the objective stance.  If this is right, then there can be no objection from fairness or from phenomenological adequacy against the account of persons described thus far as a foundation for moral responsibility. Personhood structured around our capacity for and commitment to standards is not unfair, and it can fit within the felt experience of life, insofar as phenomenology offers constraints on how we must understand our lived experience. 

\section{Responsibility, Persons, and Meaningfulness}
The question of the overall value of adopting the participant stance and treating people as responsible persons is closely related to the third criteria - that an account of moral responsibility must be existentially satisfying. What this means or looks like is difficult to articulate, but its failure is easier to detect. We see it in expressions of despair, ``existential funk'', and meaninglessness that are often expressed when people consider the possibility that we might not have free will. It is the sense that life is somehow drained of meaning - entirely, or perhaps it just has less meaning than we had hoped, if the causal thesis is true. As discussed in chapter 4, this criteria is one of the least consistently addressed by compatibilists, but this may well be because different people have different existential expectations. Dennett and Smart both express fairly low expectations around this issue, and a kind of stoic dismissal of the desire for anything deeper which is of very little help to those who entertain those deeper expectations. These expectations are, as argued in chapter 4, connected to the reactive attitudes and show up here as the felt ``seriousness'' we attach to moral responsibility - the importance that different forms of responsibility have in our lives, and the idea that our values, our identities, and our projects are not only good, but they confer meaning to our existence. 

There is a puzzle about how or why anything confers meaning, however, which makes it difficult to address how the truth of the Causal Thesis threatens the meaning of moral responsibility, and further, what would be needed to restore that meaning. This difficulty may make the demand for greater meaning easier to dismiss for those that do not feel it, and more difficult to insist on for those that have those deeper expectations. Thomas Nagel does attempt to directly address the question of meaningfulness in ``The Absurd'' \citep{nagel1979}, and starts by questioning some of the language used to express the need for meaning through expressions of the negation of meaning. Nagel investigates how we often feel that if something ends or will end, it loses meaning, and that when something is shown to be small in scope, it loses meaning - it becomes absurd. According to Nagel:
\begin{quote}
In ordinary life, a situation is absurd when it includes a conspicuous discrepancy between pretension or aspiration and reality [...] This condition is supplied, I shall argue, by the collision between the seriousness with which we take our lives and the perpetual possibility of regarding everything about which we are serious as arbitrary, or open to doubt.  \citep{nagel1979}
\end{quote}
It is not immediately clear why the Causal Thesis and the result that we are not agent causes should generate a threat to the seriousness which we take our lives. Perhaps our metaphysical status as agent causes is supposed to separate our actions and agendas entirely from the clockwork world of mechanism, and that clockwork world somehow makes the things that go on within it alone meaningless. In such a case, finding out that we, too, were clockwork beings would be absurd, perhaps. 

Nagel's explanation of absurdity makes perfect sense of our response to the Causal thesis on the model of personhood proposed here, however. A central feature of our personhood is the taking of our standards seriously - of seeing them as higher, of adopting restrictions around them, traditions of honoring them, and developing hierarchies to understand their relative importance. The Causal Thesis invites us to view ourselves and others from the objective stance - as extended selves and not persons, in which these hierarchies are viewed as contingent human conventions, as a ``mere'' practice, rather than as a part of the real world, a decision to take seriously in a kind of willful ignorance of the continuing, thoughtless mechanism of nature. Our seriousness is arbitrary because it is not reflected in what truly ``is'' - the world as it appears in the objective stance. 

This account of personhood might be accused of failing to be existentially satisfying because it in some sense concedes everything that worries about absurdity are afraid of. It freely admits that our personhood is not metaphysically enshrined. What we take seriously is, in some sense, our choice, but not as a ``radical choice'' that Sartre appeals to in order to make meaning. We are not blank slates. What we take seriously is itself ``given'' by a clockwork universe that does not itself take those things seriously, as is our success and failure to actually live according to those standards. This account might get our situation correctly, but it turns out to be very disappointing. The affective charge we attach to our standards through the reactive attitudes, the seriousness with which we take our own personhood are rendered absurd by the understanding of a universe that does not care about any of these things. Our seriousness is silly, and we are unable to do anything about that fact. There is a full and fatal collision between aspiration and fact. 

However, it is worth considering whether there is another foundation for our seriousness apart from being metaphysically enshrined, particularly given that the language that Nagel cites to express concerns with absurdity both have to do not with metaphysics, but with scope - duration and expansiveness. We express a loss of meaning when we see that the things we care about will eventually end, or that they are very small when compared to the size of the universe. Nagel attributes the loss of seriousness to the fact that we can step into wider perspectives in which our enterprises lose the meaning we attribute to them in our more narrow perspective because we can question the criteria which confer seriousness themselves. He also identifies the fact that the way we tend to confer meaning is by situating what we are doing in something ``larger than ourselves'' - service to society, science, or God, among other possibilities. However, the problem is that stepping into that perspective, we can also call those criteria into question, and there does not seem to be any reason to find ``ultimate justification'' for what we are doing in this higher step than the step before it \citep[p.23]{nagel1979}.

Nagel's investigation is focused on the loss of meaning, and as such does not investigate why taking the step backward into the larger is supposed to confer meaning in the first place. For Nagel, this ``stepping back'' problem is really the source of the absurdity. What seems to confer meaning - stepping back into a larger perspective from which we can see the meaning - does so only in virtue of the fact that we stop needing reasons after a certain point. For Nagel, the search for meaning is really the flight from meaninglessness into the ever-larger, in an attempt to find a framework which renders what is within it as meaningful and justified. However, since no framework is not itself subject to the same sorts of questions (``why should this confer meaning?) that flight goes on for as long as we have questions, and that duration is itself arbitrary.

The problem here is remarkably similar to the regress problem that faced Frankfurt's account of second order desires. Second order desires are supposed to confer meaning by endorsing first order desires, but those second order desires may themselves come into conflict, needing a decision from an order of desire still higher. In fact, Frankfurt's desires and Nagel's meaning are both subsumed into what are called standards in the account of personhood developed here, and has already been noted, responsibility is, at core, taking our standards seriously. For Personhood, the puzzle is really the question of why we take our standards seriously - why we hold ourselves ``responsible'' to them at all.

The similarity between this problem and Frankfurt's is suggestive, however. We can take a different view of the connection between meaningfulness and ``largeness'', which we can explore by going back to Frankfurt's example of the wanton, and exploring the meaning that a wanton would be capable of. For the wanton, what is worth pursuing is simply what their strongest occurrent first order desire happens to be - that is the only thing that could or would make sense to the wanton as important, then and there, and past desires and anticipated desires could be understood s meaningful in the same way. However, with the introduction of standards - our values, our commitment to longer term projects, or certain conceptions of our identities and our lives - there is an expansion in scope that can be both or either temporal or spatial. The expansion is temporal as we consider longer term projects, and it is spatial as we consider more of the universe in our decision making (and deeper, as we consider more abstract, universal principles and values). Duration, expanse, and depth are also all metaphors, that, in English, are used to confer the idea of meaning.

Furthermore, on this model, our personhood is a product of the shift to thinking according to standards and taking those standards seriously, which can help us understand the shift from our standards being arbitrarily important, to having deep existential import. Our standards have existential import because they are behind our existence as persons. Threatening those standards and their importance is, in that way, a threat to our existence, insofar as we understand ourselves to be persons. 

When thinking of persons and meaning in this way, it is clear that Nagel has made the same mistake that those who have accused Frankfurt of a regress, or ``arbitrariness'' at stopping at second order desires have made. Frankfurt's accusers, and to some extent perhaps even Frankfurt himself have taken the problem to be getting to a principled stopping point. Since third-, fourth- and n-th- order desires seem coherent, Frankfurt is taken to have a regress problem. Why is the second order desire so important when there are possible higher order desires? Nagel, in believing that the source of meaning comes from the (nonexistent) end of the regress is making the same mistake. In both cases, the foundation is really the first step - not in reaching the higher-order considerations, but in the ability to ascend to higher order considerations at all. Meaning comes not from the end of the regress, but from the beginning, because personhood as a form of life becomes possible from the shift from the immediate to the adoption of standards.

Those higher-order standards can be brought into question, and doing so can confer deeper meaning to them, and the enterprise of deepening meaning has been one of the most beautiful human endeavors across philosophy, art, literature, and science. It is also an understandable mistake that identifies the seeking of greater depth with the need for the ultimate justification of meaning, because the initial generation of meaning came from the first shift from the immediate to standards and involved calling the immediate into question. But the justification and foundation of meaning, on this account, is simply in the asking of the question that makes the form of life ``person'' possible. 

The Causal thesis threatens meaning from the incompatibilist perspective because the agent cause is an artificial means to provide a metaphysical stopping point and a bottom depth in our search for meaning. If that stopping point is removed, then the Causal Thesis makes the enterprise of meaning absurd only under the supposition that it receives its justification from the bottom of the depths, rather than from the shift in our lives that it makes possible in the transition from mere wanton to person. 

The involvement of responsibility in our existence as persons can account for the existential import of moral responsibility, though one can still imagine the objection raised by some that this is not enough. Personhood is merely conventional which makes the existential weight that this has merely conventional as well. This account of the existential weight of moral responsibility may help to explain why moral responsibility has the sort of existential expectations we attach to it, but it also shows that those expectations cannot be met. On this objection, only what I have called ``metaphysical enshrinement'' constitutes an acceptable foundation for the existential import that moral responsibility is supposed to have. If we are not agent causes, we simply do not have what we need for our feelings about responsibility to be justified.

To this objection, there is little left I can say but to express puzzlement at the need for metaphysical enshrinement, which I find just as puzzling for empirical matters as I do for practical matters like meaningfulness. I do not feel the pull of these reasons in the slightest, which leaves me at a loss about how to address their need. All I can do is raise the question of why or how finding a metaphysical basis is supposed to confer existential weight. Why should we suppose that meaning is only real if it is received as an objective ``fact'' from the universe, rather than being something we generate through the activity of living and valuing? Persons as a form of life certainly exist. We cannot seriously deny that we do value things, that we do hold ourselves and other persons according to different ``rules'' than we do non-persons. The continuing need for a metaphysical foundation for this activity seems to me to be an expression of the claim that all of this activity is a mistake, because from a certain perspective we are capable of adopting, ``personhood'' has no explanatory value and therefore does not exist. To this, I can only insist that expecting something like explanatory value from ``person'' drastically misunderstands what a person is, and there are more activities in life than explaining and predicting, and personhood belongs to some of those other activities: valuing and deciding.

\section{Summary}
On the account given here, responsibility and personhood are two sides of the same feature of human life. Responsibility is the product of our affective responses understood and acted upon in light of the expectations we have of ourselves and others which generate the reactive attitudes. Considered from the other ``side of the coin'' we are persons through our commitment to our ability to live according to standards - be they values, projects, or particular conceptions of our identities, a commitment lived out through responsibility.

In chapter 2, we saw that the problem for moral responsibility was not in providing a reason why we hold people morally responsible - that is well provided by our affective responses and is a natural feature of human beings. As such, it required no justification. However, we saw that moral responsibility faced potential defeaters that suggested that our responses might never be appropriate.  In chapter 3, we considered two of those defeaters - fairness and phenomenological adequacy - and saw that while it was possible to respond to versions of both of those worries, there was also a common worry that each connected to - that the causal thesis rendered impossible the essential element of any evaluation of responsibility: the person. 

In chapter 4, we considered this concern more closely, particularly as it underpinned two strong incompatibilist arguments: Galen Strawson's basic argument, and Derk Pereboom's four-stage argument. In both cases, we saw that these arguments were connected to a commitment to the idea of an agent cause as being the only possible target of moral responsibility, and a denial that such a being was possible. However, this left concerns about moral responsibility in an uncertain state, as it was not entirely clear who or what would be wronged if we held a person responsible in the face of the Causal Thesis. If was further argued that the assumed commitment to the agent cause that underpins both arguments is itself uncertain - its grounding in ``folk'' intuition is itself questionable, and further, even if such intuitions were found to be common, this may just as easily come from a historically grounded and utterly contingent misunderstanding about our agency. If an alternate account of persons, that allowed moral responsibility to do what it is supposed to do even in the face of the truth of the Causal thesis might then be preferable. 

In chapters 5 and 6, we developed an account of persons which took as central the capacity to reflectively evaluate our current and future actions according to standards, and then choose what we would do based on those deliberations, rather than on whatever desire happened to be strongest at the time, as a wanton might. Practical reason based accounts such as this have suffered from concerns about the nature of our commitment to standards, or values, or projects as arbitrary (the problem of heteronomy). This concern was traced back to the splitting of Kant's link between the capacity to act on reasons, and the content of reason which were unified, for Kant, in the Categorical Imperative. Responding to this challenge involved denying the claim of heteronomy - a person's desires were not entirely alien to them, but rather became alienated by fiat, as a person came to identify themselves within or connected to certain of those desires, values, or standards as set above others. This special concern with standards that can (or try to) act on through our capacity for self-government was also connected to certain restrictions on the treatment of self and others that prevent most forms of interference with that capacity such that the only acceptable appeal must be through reasons that appeal in one way or another to that deliberative capacity. 

In chapter 7, we examined whether this linked account of responsibility and personhood were able to address the concerns of fairness, phenomenological adequacy, and existential weight that were expected of a successful account of moral responsibility. In arguing this, we saw the link between responsibility, personhood, and meaning and this helped us to understand the deep commitment we have to responsibility, as responsibility is linked to our existence as persons. 

\bibliographystyle{apalike}

\bibliography{dissbib}

\end{document}
